\subsection{暑假的到来}

「哇塞——!我们的暑假终于到来啦!!!」

「你怎么那么激动啊」\\

七月过半,休业式和通知注意事项的班会结束后,学生们获得了自由,正和气地聊着暑假的安排。\\

在班会结束的瞬间,树就变得兴奋起来,而周看着那样的他,只是不由得感觉闷热。\\

「这不是理所当然的吗,地狱一般的课堂结束后就是天堂……前方就是乐园……!」

「那只是因为你不喜欢学习。我又不讨厌」

「学霸给我闭嘴。周和椎名卿卿我我的时间也能增加吧」

「居然说卿卿我我……我说啊,我们也不是二十四小时都在卿卿我我的啊」\\

倒不如说是互不讲话,各过各的时间比较多。\\

在同一个空间度过时,两人常常一起学习,分担家务,并不是一直都在卿卿我我。

对真昼而言,学习自不用说,为了健康与美容,她还有做运动,保养身体。周也会和她一起跑步锻炼身体,若是认为两人老是黏在一起可就大错特错了。\\

「……我就直说吧,你们只是有意识地卿卿我我难度太高,其实一直在无意识地卿卿我我」

「哪有」

「我估计你们偶尔会视线交汇笑一笑啊,搂住胳膊啊,牵牵手之类的」\\

周无法否定。\\

他虽然不怎么和真昼拥抱,但是这种琐碎的肢体接触倒是经常会做。

卿卿我我的标准很难界定。周不把这些当作卿卿我我,但是在常人眼里这似乎就是卿卿我我。\\

「你看吧。你们的卿卿我我可是让人光是看着就会感到热起来啊。对吧,优太」

「啊哈哈,是啊。我看着都会觉得害羞」

「连门胁你也……」

「不过多亏这样,想要阻碍你们的人变少了,所以我并不觉得这是坏事」\\

的确,至少在同年级里,没有多少男生像预想中那样骚扰他、找他的茬,或者做出抢夺真昼的行为。

之所以会这样,很大程度上是由于真昼毫不掩饰自己喜欢周吧。因为她不关心别人,所以那些人似乎就放弃了。\\

即使如此,周也做好了被抱怨和找茬的心理准备,但是班上的同学不知为何甚至营造出了守望的氛围。老实说,周觉得摸不着头脑。\\

「讲真的,周什么事都没有,也是多亏了椎名的压力」

「压力?」

「或者说是牵制?看到体育节时椎名那副样子,他们还能做什么啊。要是周被做了什么,椎名肯定会发火的」

「真昼发火啊……我想象不出」

「我也想象不出,但她绝对会生气吧。椎名相貌端正自然不用说,老师也非常信赖她,要是和她敌对的话会很可怕」\\

树补充说「让一直都那么温柔的人生气的话绝对会很恐怖」,周也对此默默同意。\\

(她大概是不能惹火的类型吧)\\

周也讲过,他不太能想象得出真昼生气的样子。\\

但是,周明白若是惹她生气就会很不妙。\\

真昼脸上一直挂着温和的笑容,不会因为一些小事而生气,但是周觉得,一旦跨线,她便会面带笑容,讲道理讲得对方无法反驳。考虑到体育节时发生的事情,这也不是不可能。\\

周不打算惹火真昼,况且若他做了什么,比起生气真昼更先会感觉到悲伤。周下定决心,尽量让真昼保持内心平静。\\

「……是打算要让我生气吗?」\\

周在内心发誓的时候,真昼和千岁一起走到了他这边。\\

「椎名,不是在说我,我们在聊的是如果周被做了什么的话,你大概会生气」

「那是肯定的……但我不会太发火的。我会当面交谈到对方能理解为止」\\

看着真昼的微笑,树身体略有发颤。\\

真昼大概会和刚刚宣言的一样,用尽话语来让对方理解吧。估计她会以笑容和讲道理为武器,步步紧逼,从而使对方同意。从这方面来讲,真昼还是让人不想与她敌对。

周希望自己不会和她为敌。\\

「周,你可不能惹火昼儿啊?」

「我不可能做让她生气的事情吧,倒不如说到底做什么才会让她生气啊」

「……花心之类的?」

「你觉得我会吗」

「我是觉得不会啦?以周的性格来说是不可能的吧。周敞开一次心扉之后就会珍惜对方」

「……谢谢夸奖」\\

由于得到当面表扬,周感觉有些羞耻。\\

「不过太过珍视都变得胆小了,比如说只做到亲脸这一步」

「真昼」

「不、不是,我没有感到不满……那个,她问了我印迹的事情」

「好了忘掉吧」\\

既然真昼被问后讲出了事情的始末,周觉得还是不要谈起这个话题比较好。\\

「啊,那个果然是亲……」

「树」

「好好,我的朋友阁下真爱害羞。明明只有那点事情,我们都平常地做的」\\

树喊了声「是吧千岁」然后和她卿卿我我起来。周则在内心里嘟哝了一句「我又不像你们两个已经登上了大人的阶梯」。

他们交往已有两年,当然已经到达了周和真昼尚未到达的阶段,况且周也常听树讲这些,所以周没什么吃惊的,只不过会隐隐约约感觉有点羞耻。\\

真昼也一样,或许是从千岁那儿听说了,脸噗的一下变得很红。她和周想象的事情大概都是一样的。\\

(……大概还需要很久吧)\\

亲嘴都没有过,身体的结合恐怕还遥不可及。周也没有现在就要做的欲望,所以只好以两人的步伐慢慢走近。\\

周与真昼对上视线后,她的脸变得更红,低下了头。周顿觉无比害羞,把眼神从真昼身上移开了。
