\subsection{回家后的谈话}

「总觉得很抱歉」\\

与门胁分别后,两人又在游戏厅玩了一会便回家了。在这之后,周向着坐在沙发上休息的真昼做出了道歉。

真昼已经解开、打理好了头发。听到周这么说,她呆呆地睁圆了眼睛。\\

「为什么突然这么说」

「你看……给门胁知道了」

「那是不可抗力吧,再说我觉得结果还算好,姑且是得到了理解……」\\

真要这么说倒也没错,但即使如此,真昼恐怕也会因为对方怀疑「真的没在交往吗」而烦恼吧。\\

虽然幸运的是门胁理解了这边的情况,比较爽快地离开了,但听到真昼如此坚决的否定,周还是会感到心痛。\\

「而且,我也并不是抱着绝对不会暴露的想法才出门的。这种情况我也有考虑到,另外我也觉得还好是暴露给门胁不是别人」

「倒也是,门胁各方面都理解了我们,并且还很替我们着想,真的是个好人」\\

还好暴露的对象是门胁。

尽管日后不免遭到追问,但想到以后在学校不用再有隐瞒着门胁的罪恶感,或许暴露给他才是正确的。\\

虽然周觉得自己对真昼的心意也似乎也让门胁知道了,但只要门胁不告诉真昼本人的话就没问题。\\

在卡拉OK的时候周或许会被捉弄,但门胁和树都明白分寸,应该是不会做得太过分的。\\

「……周君很欣赏门胁呢」

「嗯?啊,算是吧。交谈的机会变多之后,我就明白那家伙果然是个好人,会那么受欢迎也能够理解。他很厉害,不管是内在还是外在都很帅」

「你挺信赖他啊」

「嗯,我觉得他是值得信赖的人」\\

周是看重择友的这类人,这一点他自己也有所认识。\\

如果对方人品不好,周就不会想要去靠近他,也不会允许他接近自己。周养成了无论如何都会警戒他人的习惯,而这个警报器对门胁没有起反应。\\

在本能上,周就隐约觉得门胁是个好人。正因如此,就算暴露给他,周也没觉得有多少焦虑,并且还觉得这是正确的。\\

「那么,是物以类聚这么一回事呢」

「我倒不知道哪里和他算是同类了……」

「周君又在自卑了……门胁是觉得周君的人品好才会想要和周君打好关系的吧?不就和周君对门胁这么想是一样的道理吗。周君觉得可以信赖的门胁都认可了周君,所以周君要更有自信才行」\\

真昼坚定地断言后,用手指轻轻地戳着周的脸颊。周便暗自苦笑起来。\\

周觉得自己果然是敌不过真昼;或者说,每次否定自己之后,她都会毫无保留地肯定周,这样的存在让周非常感激。

「请抱有自信」真昼进入了说教模式,而周抖动着肩膀轻笑,向她表达谢意。\\

「真昼一直都在表扬我啊」

「这是正当的称赞。都是一直自我否定的周君不好」

「那是习惯了」

「为什么你会养成那种习惯啊,真是的」\\

真昼傻眼地嘟囔道。\\

被这么一问「为什么」,周感到难以回答。

周姑且知道自己会变成这样的原因,对此也有认识,但有些事情就算明白道理也无济于事。\\

那是非常辛酸的回忆。

尽管在周的心里那件事情已经算是过去了,但是总还有些后遗症。

即使周明白那是坏习惯,他却还没能完全改正。周觉得以后必须得在真昼说起前就提起注意。虽然说这个坏习惯大概不会那么简单就改得过来。\\

「嗯,我会注意的……我说真昼」

「嗯?」

「……谢谢」\\

能够与在自我否定时肯定自己的人相遇,对周来说就是最大的幸福。\\

周仅仅是坦率说出了自己心中所想,真昼却向周投去惊讶的视线……然后靠在了周的肩膀上。\\

「周君,你真的是……」

「真的是?」

「没什么」\\

真昼说着「笨蛋」,将额头压在周的肩膀转来转去。周尽管感到不解,但还是放任真昼这么做了。
