% !TeX program = LuaLaTeX
% !TeX encoding = UTF-8

\documentclass{article}

\usepackage{luatexja}
\usepackage{luatexja-ruby}
\usepackage[no-math]{luatexja-fontspec}
% \usepackage[ZhongYi]{luatexja-zhfonts}
\usepackage[a5paper]{geometry}
\usepackage{float}
\usepackage{titlesec}
\usepackage{setspace}
% \usepackage{newunicodechar}
\usepackage[yyyymmdd]{datetime}
\usepackage{indentfirst}
% \usepackage{tipa}  % for \textsubdot
\usepackage[perpage]{footmisc}
\usepackage[unicode,hidelinks]{hyperref}
\usepackage[numbered]{bookmark}

% 思源黑体
\newjfontfamily{\jpfont}[
  BoldFont=SourceHanSansJP-Bold,
  YokoFeatures={JFM=prop}
]{SourceHanSansJP-Normal}
% 方正兰亭黑系列
\setmainjfont[
  BoldFont=FZLTZHK--GBK1-0,
  YokoFeatures={JFM=prop}
]{FZLTXIHK--GBK1-0}
\renewcommand{\familydefault}{\sfdefault}

\ltjsetparameter{prebreakpenalty={`—, 10000}}

\onehalfspacing
\titlespacing*{\subsection}{0pt}{8.9ex}{3.4ex}

\def\two@digits#1{\ifnum#1<10 0\fi\number#1}
\renewcommand{\thesubsection}{\two@digits{\value{subsection}}}
\renewcommand{\abstractname}{简介}
\renewcommand{\dateseparator}{.}

\counterwithout{subsection}{section}
\newcommand*\sectotoc[1]{\section*{#1}\addcontentsline{toc}{section}{#1}}

% DO NOT remove the following line! It is processed by a script
% \setcounter{subsection}{<N8440FE_SUBSECTION_COUNTER>}

\title{关于邻家的天使大人不知不觉把我惯成了废人这档子事}
\author{
    \begin{tabular}{rl}
        作者 & {\jpfont 佐伯さん} \tabularnewline
        翻译 & taroxd, tongyuantongyu, 浪人, kae, 冰川镜华 \tabularnewline
        校对 & taroxd
    \end{tabular}
}

\begin{document}

\maketitle

\begin{abstract}
藤宫\ruby{周}{\jpfont あまね}住的公寓邻家住着一位学校人气第一的可爱天使。

拥有被唤作天使的美貌,优秀少女——椎名真昼。周是并无特别出彩之处的普通学生,他曾经以为,尽管她是邻居,自己在过去和未来都不会和她扯上关系。

直到他遇见雨中湿透的天使。

「人情我会还的。说起来,房间最好整理一下。简直看不下去」

「要你多管」

天使大人说话有些严格,两人的关系从把伞硬塞给她之后开始。

周感冒时前来照料,指责周不爱护身体而来帮忙做饭,两人进行共同作业(打扫房间),一起出门……

最初冷淡而后逐渐变得开始撒娇的真昼,和一开始是怕麻烦的消极主义却不知何时敞开了心扉的周——

这是不坦率的两人逐渐走近的故事。
\end{abstract}

\tableofcontents
\newpage

\sectotoc{第一章}

\subsection[天使大人是娇嫩欲滴的女人]{天使大人是娇嫩欲滴的女人\footnote{原文为 {\jpfont 水も滴るいい女},有因为下雨而脸上滴着水的双关含义。}}

「你在干啥」\\

藤宫\ruby{周}{\jpfont あまね}和她——椎名真昼第一次说话,是在连绵不绝的雨中看到她坐在公园的秋千上的时候。\\

\vspace{2\baselineskip}

周今年升到高一,并同时开始了独居生活。而在他公寓的右邻住着一位天使。\\

天使当然只是一个比喻。然而椎名真昼那么美丽可爱,使得这个比喻仿佛像是真的一样。\\

她那亚麻色的直发一直都顺滑有光泽,透明般的乳白色皮肤保持着没有一丝粗糙的光滑。端正的鼻梁,还有长睫毛下面的一双大眼,合在一起体现出了娃娃般纤细的美。\\

周和她在同一所高中,并且是同年级,所以经常听到别人对真昼的评价。其中大半都是「文武双全的美少女」。\\

事实上,她在定期考试中始终保持着第一名。体育课上也总是首屈一指的活跃。\\

因为班级不同,周对她的了解并不详细。如果传闻没有错的话,那她简直就好像完美超人一样。

她并没有什么像是缺点的缺点,而且眉清目秀、成绩优秀,同时性格谦虚老实,会受欢迎也是难怪的。\\

有这样的美少女住在边上,这个环境应该让一部分男生垂涎欲滴吧。\\

但就算这么说,周也不打算对她做什么,更不觉得自己能做到。\\

当然,周的心里也认为椎名真昼这个少女非常有魅力。\\

然而,两人的立场不过是邻居。周既没有机会和她说话,也没有打算和她扯上关系。

要是这么做了,周恐怕会受到男生的嫉妒。说到底,如果只是住在旁边就能关系亲近的话,恋慕她的男生们也不会那么辛苦了吧。\\

再提一句的话,异性魅力和恋爱感情并不总是等同的。在周的认识里,真昼是最适合远观的鉴赏用美少女。\\

因为这个,周完全没有对酸酸甜甜的关系有所期待,也自然不可能和她扯上关系。周只是住在真昼的旁边,甚至不曾与她接触。\\

于是,看到她不撑伞独自呆在雨中的样子时,周心里想着「在干啥啊」并且露出了看待可疑人物一样的眼神。\\

\vspace{2\baselineskip}

雨大到让所有人都径直奔回自己家里,然而她却在学校和公寓之间的一处公园里,一个人坐在秋千上。\\

(在雨中干啥呢)\\

周围有些昏暗,雨也让视野变得模糊。然而,那显眼的亚麻色头发和校服,使得周一眼就能看出她是真昼。

只是,周不明白她为什么不打伞,呆在那里任由雨淋。\\

看样子,她似乎并不是在等着别人。对于淋湿她也没有抗拒,只是心不在焉地朝着某个方向望着。

稍稍上仰的,那原本就缺少色素的脸,现在气色很差甚至显得苍白。\\

这是闹不好没一会儿就能感冒的状态。然而即使如此,真昼还是静静坐在那里。\\

既然连回家的打算都没有,这应该是她自己想要这么做的吧。既然是本人的意愿,或许不该由其他人来插嘴操心。\\

周这么心想着,正准备横穿公园——但最后却看到真昼那扭曲得仿佛要哭出來似的表情,周挠了挠自己的头。\\

他并没有想要和她扯上关系这类的动机。\\

只是,如果放着露出这种表情的人不管,会让他良心感到疼痛。仅此而已。\\

「……你在干什么呢」\\

为了表示自己没有别的意思,周用尽可能冷淡的声音向她搭了话。她摇着吸收了水分仿佛变沉的一头长发看向周这里。\\

她的脸蛋还是一如既往的漂亮。

即使被雨打湿了,她的光芒也没有被掩盖住,反而连雨都像是小物件一样更加衬托出了她的美丽。正所谓娇艳「雨」滴吧。\\

水灵的一双眼睛看着这里。\\

真昼姑且也知道周是邻居吧,毕竟两人早上偶尔会遇到。\\

只是,因为突然被搭话、因为至今为止完全不相关的人前来接触,她暗褐色的眼睛渗出了薄薄的警戒。\\

「藤宫。找我有什么事?」\\

周心里产生了「啊原来记住了我的姓啊」这种微妙的感慨,同时也看出来,这个警戒是不可能会放松的吧。

虽说不是完全没见过,但两人是陌生人。被搭话后会增强防备也是理所当然的。\\

说起来,时常有各个年级的男生向她告白或者接近她,所以她或许是不太想和异性有什么接触。大概她是觉得男生的动机不纯吧。\\

「没什么事。只是见到你雨中一个人呆在这里,肯定会在意的吧」

「是吗。谢谢你的关心,不过我是自己想呆在这里的,请不用管我」\\

淡泊的语气上不是明显的警戒,而是虽然柔和却完全不想让别人深入询问的感觉。\\

(嘛,果然是这个结果啊)\\

她明显是有什么隐情。对于她这一「不要管我」这一拒绝的态度,周也没有深挖的打算。\\

周原本就是心血来潮去搭话的。询问原因只是自然的发展,并不是周有多么在意。\\

如果她想要呆在这里的话,那样也没什么问题。

倒是真昼,在心里应该产生了「为什么要来搭话」这种感情吧。\\

她以柔弱的美貌猜疑地看着周。于是,周仅仅回答了一句「是这样啊」。\\

继续搭话下去的话必然会被厌恶,所以是时候该撤退了。

幸好,不管真昼对周的印象是好是坏,两人之间都没有什么关系。于是,周爽快地做出了决断,不管她自己回家。\\

不过,知道有个少女独自一人在这里淋成落汤鸡,心情上也是不太舒服的。\\

「要感冒的,打着伞回去吧。不用你还了」\\

所以,周在最后稍微多管了一点闲事。

要是她感冒了,周也没法睡得香。抱着这种想法,周把打在自己头上的伞递给了她。\\

周让她接过了伞,准确地说是把伞硬塞给了她。趁她还没说话,周转过身去。\\

周快步离开之后,背后传来了真昼的声音。

然而,声音小到几乎被雨声盖了过去。所以周并没有在意,而是迅速穿过了公园。\\

周的心思只是希望她别感冒,才把伞硬塞给了她。或许是因为有做这件事,一开始想要无视她走过的这种罪恶感少许减轻了一些。\\

既然她拒绝对话,周也不打算再和她有什么关系。\\

既然无缘,就此别过吧。\\

再次走上归途的周抱着上述的想法。至少在这时周还是这么想的。

\subsection{天使大人的提案}

「周,你鼻子好吵」

「你才吵」\\

第二天,感冒的是周。\\

赤泽\ruby{树}{\jpfont いつき}这位同班同学,或者说是损友,指出这一点后,周想要发出哼声表达不满却失败了。

相对的,周用鼻子呼吸的时候就会发出鼻涕的声音,这在这个意义上倒是有哼哼的意思。\\

周的身体非常不舒服。不知是因为鼻塞还是感冒本身,脑袋里一直感到有刺刺的疼。

他姑且是喝了药店卖的药水,然而症状并没能完全抑制下来,结果就是这个模样。\\

看到因鼻塞而扭曲着脸,和纸巾打着交道的周,树的眼神比起担心更多的是吃惊。\\

「昨天你还好好的吧」

「淋了雨」

「没事吧。话说你昨天没带伞吗」

「……给别人了」\\

在学校,周自然是不可能说是给真昼了,所以只能蒙混过关。\\

顺带一提,在学校瞧见真昼的时候,她脸色不差,表现得也很精神。只有把伞交出去的自己得了感冒,这状况也只能说是可笑了。

不过,原因是没有好好洗个澡暖暖身子,其实是周自作自受。\\

「那么大的雨你还把伞借出去,你人也太好了吧?」

「没办法吧,反正就是给别人了」

「感冒的风险都顶上了,你是给谁了啊」

「……路过的迷路小孩?」\\

虽然说比起小孩身材要好多了。不如说其实根本是同一个年纪。\\

(……啊这样啊,她的表情就像迷路了一样啊)\\

自己表达出来之后,终于理解了。

当时真昼的表情,简直就和迷路的小孩寻找父母的时候一模一样。\\

「你可真是好心」\\

树并不知道周现在想起昨天的真昼是怎样的心情,像是调戏一般地笑了。\\

「不过啊,不管是借了伞还是怎么着的,你之后随便擦擦身体就不管了吧。感觉那才是原因」

「……你怎么知道的啊」

「你那不爱护自己身体的样子,去了你家谁都知道」\\

所以才会感冒啊笨蛋——被这么若无其事损了一句之后,周不得不闭上了嘴。\\

正如树说的一样,周基本上不太把自己的事情放在心上。

再补充一些的话,周不擅长整理收拾,所以房间乱成一团。吃的东西也是便利店的便当、补品或者在外面吃。

树为此还无语地表示「真亏你敢说是独居啊」。\\

从这个角度来看,周生活太过随便,也难怪会得感冒吧。\\

「今天赶紧回家早点休息吧。还有一个周末,赶紧给治好了」

「嗯……」

「至少有个女朋友来照料该多好啊」

「吵死了。有女朋友的人给我闭嘴」\\

树有些自豪地笑着。看到那副样子,周不爽地用手指扣进眼前的纸巾盒。\\

\vspace{2\baselineskip}

随着时间过去,周的身体状况变得越来越差。

原本感冒的症状还只是头痛和鼻涕,现在还加上了喉咙疼痛和倦怠无力,一起支配住了周的身体。\\

放学后,周尽管是紧盯着前方尽快往前走,然而感冒的影响比想象中还要大,让周的步履变得非常沉重。\\

即便如此,周还是到达了公寓的入口。周拖着沉重的脚步,走进电梯之后,把身体靠上电梯的墙上。\\

周的呼吸比往常更急更热。\\

在学校,周似乎还忍受住了。然而,或许是快到家而大意的原因,周的身体一下子变得难受起来。

电梯里独特的失重感,平时他是无所谓的,而现在就化成了痛苦。\\

即使如此,马上也要到家了。\\

电梯停在了自己住的那层。周缓缓走出电梯,朝着自己房间所在的走廊迈出脚步——却先愣住了。\\

周的视线前方,是他以为不会再有什么话可说的,飘扬着亚麻色头发的少女。\\

就外观而言,她可爱的容貌上充满了活力,皮肤上的气色也很不错。

怎么想都是她更可能感冒,但事实上她却活蹦乱跳的。或许是平时她就注意身体,这一差距如实地体现出来。\\

真昼的手上,握着前些日子硬塞给她的,整整齐齐折好的伞。

这是明明说好不用还,她也依然跑来还了吧。\\

「……明明不用还的」

「有借有还是天经地……?」\\

她话说到一半停了下来。不如说,停下来是因为看到了周的脸。\\

「那个……有发热吧……?」

「……和你没关系吧」\\

在最差的时机撞上了。周这么想着,皱了皱眉。\\

说极端点,伞爱还不还,都无所谓。

然而,两人在这个时机遇到并不是好事。她这么聪明,很快就能推断出周感冒的理由吧。\\

「可是,这是我借了伞的原因……」

「那是我自说自话,和你没关系吧」

「有关系。因为我在那里所以你才会感冒的」

「无所谓了,不用你担心」\\

周心里是觉得,不想要自己为了自我满足而做的事情反而让人担心。\\

然而,真昼却没有就这么老实放过他的样子。端正的美貌上露出了焦急的样子。\\

「……已经够了。再见吧」\\

有问有答也让周不太惬意,于是周决定强行逃离真昼的追问和担心。\\

摇摇晃晃地,周随手接过了伞,从口袋里拿出了钥匙……到这里还没有问题。\\

周略有迟缓地打开自家门的瞬间,身体上就失去了力气。\\

或许是终于要走进家门而感到安心的原因,周的身体摇摆着倒向了后方的墙壁。\\

虽然心里觉得不妙,不过走廊上护栏很结实,只是撞一下并不怕撞坏。护栏高度也够,所以不可能落到外面去。只是撞疼的话,那也没办法……周已经做好了疼痛的心理准备。\\

然而,周的手臂被猛地一拉,强行让周恢复成了原本的姿势。\\

「……你这样实在不能放着不管」\\

细小的声音传到了周有些模糊的意识里。\\

「人情,我会还的」\\

或许发热上头了,脑袋模模糊糊的,没有能够理解她所说的事情。

因为,在理解之前,真昼支撑起周无力的身体,打开了周的家门。\\

「我进去了,这是无奈之举,请原谅」\\

真昼声音静静的,但却坚定得不容分说。

感冒的周也没力气抵抗,就被这么拖着,和第一次和同年代的女性一起回家了。\\

生病时,周虽然没有女朋友来照料,不过似乎有一个天使会来照料他。

\subsection{天使大人照料病人}

因为发热,周很晚才想起来自家的现状——不如说是亲眼看到实际情况之后,周才后悔把真昼放进来。\\

周住的公寓是一房一厅,有厨房和储藏室。\\

客厅面积挺大,有寝室,还有储藏室,这对独居生活而言已经相当奢侈了。因为父母还算富裕,考虑到安全和交通,所以最后选择了这里。\\

要求独居住在这里的是父母,所以周并不打算提什么意见。不过,周心里觉得,不用花那么多钱也没关系。一个人住这么大的家吗,实在应付不过来。\\

先不说这些。周这个人是独自居住,并且不擅长整理收拾。\\

当然,别说客厅了,连寝室都是乱糟糟的。\\

「真是看不下去」\\

天使大人,或者说救世主大人,尽管外表可爱却给周开门见山地奉送上了这么一句直白的话。

因为事实上确实看不下去,周也无法反驳。如果知道要让别人进来,周多多少少还会移开点东西。但事到如今说这些也晚了。\\

真昼光泽的嘴唇中发出一口叹气。即使如此她也没有回去,而是把周搬到了寝室。

在中途,两人差点摔倒。把屋子弄得这么乱的本人痛切地感受到,再不认真收拾或许就太不妙了。\\

「总之,我先出去一下,在我回来之前换好衣服。没问题吧」

「……你还会回来啊」

「把病人放在床上不管,我会睡不好觉的」\\

真昼似乎和周上次看到她那时候是一样的想法,所以周也不好多说什么。

真昼离开房间之后,周就老老实实按真昼说的,换上了室内的便服。\\

「……乱的到处都是,不如说没地方落脚……这种地方你是怎么生活的啊……」\\

在换衣服的时候,传来了小小的困惑的声音,让周感到非常抱歉。\\

\vspace{2\baselineskip}

换好衣服躺下之后,周似乎是不知不觉中睡着了。他使劲睁开沉重的眼睛,首先映入眼帘的是亚麻色的头发。\\

沿着头发往上方看去,只见真昼静静站着看着周。看来刚刚发生的一切都不是做梦。\\

「……现在几点」

「晚上七点了。你睡了有几个小时吧」\\

真昼淡淡地回答之后,配合着周坐起身体,把倒好在杯子里的运动饮料递了过来。

周心怀感激地接过杯子喝了一些,总算能够把视线转向周围了。\\

或许是因为睡了一觉,周感觉身体稍微好转了一点点。\\

周注意到了自己脑袋上凉凉的。摸了摸,指尖上传来了像是布一样,有些硬硬的感觉。\\

周的头上贴着家里不可能有的冷敷贴。注意到这一点,周便抬头看向真昼。真昼则直截了当地回答说「从家里带来的」。

周的家里既没有冷敷贴,也没有运动饮料。所以,运动饮料也是她拿过来的吧。\\

「……谢谢你特意拿来了」

「不用谢」\\

这冷淡的回答让周只得苦笑。

她提出来照料应该只是由于罪恶感,而并不是想和周聊天。说到底,在几乎只是一面之交的男人家里两人独处,这种状态下想来也不可能亲近地说话。\\

「总之,我把桌子上的药拿过来了。这药最好不要空腹食用,你现在有食欲吗」

「嗯,还算有吧」

「这样吗。那我烧了点粥,你先喝吧」

「……诶,椎名亲手烧的?」

「除了我还能有谁。不想喝的话我喝掉好了」

「啊不要我会喝的请给我喝吧」\\

周完全没想到真昼除了照料自己之外,还会为自己准备好粥,所以惊慌失措了一瞬间。

说实话,真昼的料理水平是未知数。不过,周并没有听说过家庭科的课上真昼有失败这种类型的传闻,至少应该不会太糟糕吧。\\

周立刻就低头回答说要喝,让真昼露出了有些无语的眼神。不过,真昼还是点了头,并把床头柜上的温度计递了过来。\\

「我去把粥端来,你先量个体温」

「嗯」\\

周照着真昼说的,敞开了衬衫前面,拿出了温度计。这时,真昼慌忙撇开了脸。\\

「等我不在房间里了再量啊」\\

真昼的声音微微有些慌张。周往她那儿看过去,只见她的脸蛋有淡淡的红晕。\\

周的心里觉得,男人的胸板不像女孩子一样需要藏起来,所以对真昼的反应感到不可思议。不过,真昼可能是对皮肤的颜色没有免疫力,周只是敞开了前面就让她明显慌了起来。\\

真昼白色的脸颊染上了淡淡的蔷薇色,脸依旧朝着别处微微发着抖。不知是否是错觉,感觉她的耳朵也染上了颜色。真昼的害羞程度可见一斑。\\

(……啊,感觉有点明白她周围的男生为什么会说可爱可爱的了)\\

周觉得真昼确实是美少女,但并没有更多的想法。美丽、可爱,这是不假的,但也不过如此。

也许该说是人造品的美——真昼给周的印象就和艺术品差不多。\\

然而,现在的真昼露出了微微的害羞,她那慌张的样子更有人类的感觉,因此有种不可思议的可爱。\\

「……那你赶紧去把粥拿来不就好了?」

「不用你说我也会去的」\\

只是,两人的关系并没要好到周能老老实实夸她可爱的程度。要是说了出来,肯定会让她对周有奇怪的评价。于是,周把感想咽了下去。\\

周没兴致地那么一说之后,真昼就啪嗒啪嗒地快步往房间外面走去。

她的动作多少有些慢,是因为动摇呢,还是因为房间太乱呢。恐怕是后者吧。\\

迷迷糊糊地目送她离开之后,周再次小小叹息了一声,心里想着为什么会变成这样。\\

(……嘛,大概是因为责任感和罪恶感吧)\\

一般来说,人是不会跑进不怎么熟的男生家里照料病人的吧。要是被侵犯了事就大了。\\

真昼带着这个风险依然做出了这样的选择,看得出她内心非常愧疚吧。再加上周的态度明显对她没有兴趣,这说不定也是让她安心的原因。\\

不管怎么说,真昼是没有其他办法才来照料的,这一点应该不会有错吧。\\

「……我拿来了」\\

周用有些发热的脑子想着这些事情的时候,真昼有所顾虑地敲了敲门。

真昼似乎是担心着衣服有没有穿好而不打算立刻进来。这时,周才想起来,把衣服弄松是为了量体温啊。\\

「体温还没量」

「趁我不在还不量好啊……」

「抱歉,我走神了」\\

周老老实实道了歉,把温度计夹在腋下。没过多久就听到了有些闷闷的电子声。

周拿起温度计,只见屏幕上显示着38.3°C。虽然不至于去医院,但这个体温还是挺高的。\\

周整理好着装,告诉还不打算进来的真昼「已经好了」。真昼这才端着放着一锅粥的托盘小心地进来了。

她这么明显地放下心来,是因为周把衣服整理好了吧。\\

「多少度啊?」

「38.3°C。喝点药睡一觉就会好的」

「……药店的药都只是针对症状,不能消灭病毒本身。要好好休息,激活身体的免疫功能啊」\\

尽管被责备了,不过周知道这是真昼在担心,所以总觉得心里痒痒的。\\

说着「真拿你没办法」,真昼叹了口气,把锅连着托盘一起放到床头柜上,打开了锅盖。\\

锅里是放了梅干的粥。考虑到对胃的负担,粥比较稀,大概一份米七份水吧。

里面放了梅干,应该不是为了味道,而是因为据说这样对感冒好吧。\\

锅上没有冒出热气,然而却传来了温暖的感觉。这种感觉大概是表明,这锅粥不是现做的,而是故意凉下来的吧。\\

真昼无视了盯着粥看的周,而是麻利地把粥盛到碗里。梅干细细地散在粥里面,而里面的籽都被细心地挑掉了,红色的果肉淡淡地混合到了白色里面去。\\

「喝吧。应该不烫了」

「嗯,Thank you」\\

周接过了粥,不过还是握住勺子盯着粥看。真昼也感到了纳闷。\\

「……干什么,是想让我喂吗。那种服务我是不接受的」

「我才没那么说……只是觉得原来你还会做饭啊」

「一个人独居这是肯定得会的」\\

对于还不能好好独立生活的周,刚刚那句话还是挺痛心的。\\

「……你在做饭之前,最好先把房间收拾收拾」

「您说得是」\\

真昼好像大概知道了周在想着什么,赶紧打了一剂预防针。周轻轻嘟囔着,舀了一勺粥放到嘴里来把这事糊弄过去。\\

舌头上粘稠的粥味,不出所料地充分体现出了米的原汁原味。盐放得很少。

不过,碎开的梅干带来柔和的酸味和咸味都非常入味,形成了绝妙的平衡。\\

周并不是特别喜欢吃咸梅干,不过却很喜欢这温和的酸味中带有微甜的感觉。如果身体健康的话,他希望直接把这些梅干浇到米饭上,做出这个味道的茶泡饭。\\

「好吃」

「谢谢夸奖。不过只是煮粥,谁煮都差不多的」\\

真昼一副若无其事的表情回答说。不过她脸上露出了微微一笑。

这和学校偶能见到的对外笑容不一样,是流露出安心的微笑。这让周下意识地凝视过去。\\

「……藤宫?」

「啊,没事」\\

柔和的笑容只露出了一瞬就很快消失了。周感觉有些可惜。\\

周尽管心里这么想,但没有说出口,而是再次一口一口舀着粥吃试图蒙混过关。

\subsection{天使大人的粥与现况}

「……总之今天你就静养吧,也请记得好好补充水分。另外如果要擦汗用这个。脸盆里水已经准备好了,把毛巾放进去然后拧干来擦就好了」\\

饭后,真昼勤快地拿来了瓶新的运动饮料、倒好水的脸盆、毛巾,还有备用的冷敷贴,一起摆在了床头柜上。\\

再怎么说以一面之交的关系在异性家里留宿也不合适,而且这样周也会觉得不自在,便接受了对方的行动。\\

在周的注视下,真昼检查着有没有遗漏的事情。\\

(……以发自义务感来说这还真是服务周到啊)\\

虽说嘴上有点毒,但她做起事来却十分认真卖力。周露出无奈的苦笑,感觉逐渐习惯了这样的真昼。\\

(往后就算是两清了,谢谢照顾啦)\\

估计,之后就不会再和她有什么关连了吧。因为,她不过是偶然照看了自己一次。\\

那么,既然往后就不再接触了,就趁这个机会问下在意的事情吧。

大概是药起作用了吧,周虽说还是一副昏昏欲睡的感觉,但烧似乎已经退了一些,头脑也比睡前清晰了不少。\\

「那个,可以问你件事么」

「什么事」\\

安排好必要的东西之后,真昼看向了周这边。\\

「那时候你为啥淋着雨坐在秋千上啊。跟男朋友吵架了么」\\

周还是对昨天那导致如今事态的开端耿耿于怀。\\

那时,真昼正淋着雨坐在秋千上摇着。她为什么会在那里呢。

说起来还是因为当时看见她那如同迷路了的孩子一般的眼神,对此感到在意才把伞硬塞过去的。

可周却不知道她为什么会露出那样的表情。\\

看着那像是在等谁一般的样子,周猜测她是不是跟男朋友吵了一架什么的,但真昼却一脸无奈地看向了周这边。\\

「很抱歉,我没有男朋友,也没有交男朋友的打算」

「啊,为啥?」

「不如说为什么会觉得我有呢」

「看你那么受欢迎的样子,怎么说也有一个两个吧」\\

对正与真昼交谈着的周来说,她更像是一个性格比较强势的,但挺有人味的普通少女,但在周围人的眼里却并非如此。\\

在其他人看来,她不但是清纯可爱,乖巧谦虚的美少女,更有着令人一见钟情的,天使般的美貌。

再加上年级第一,体育全能,还有刚刚才见识的厨艺。那样的话,想必她的人气一定很高吧。\\

周瞧见过别人跟真昼套近乎,同时也知道自己的同学有不少对真昼有意思的。\\

都这么一个随便挑随便选的状况了,哪能想到她没有和任何一个人交往。\\

周这么想着,便用了「一个两个」这么一个词,可真昼听到这个词,脸上却突然僵了一下,然后露出了稍稍扭曲的表情。

「没有,我也不记得我是那种跟好几个人交往的没节操的人。绝对不可能」\\

眼神突然变得冷淡的真昼,淡淡地做出了否定。周立刻就明白了,自己是踩到地雷了。

或许是感冒的原因,周突然感到一阵恶寒。不知是不是错觉,感觉连屋子里都冷了三分。\\

「啊,我不是那个意思。抱歉」

「……不,是我这边头脑发热了,对不起」\\

不过,在周低头道歉之后,冰冷的氛围便瞬间散去了。\\

周感觉与其说她头脑发热,不如说空气被她影响到像在暴风雪中一样冷。但周并没有把这话说出口。\\

「……总之,那时候不是这个原因,只不过是想让大脑冷静一下罢了……搞得让你担心我,甚至还害你感冒了,实在是抱歉」

「没事的。反正也是我这边自作多情。所以也不希望你那边有什么罪恶感。这么一来,和椎名你的关系也算是到此为止了吧」\\

真昼果然是受罪恶感驱使而照顾周的吧。她听了周的后半句话眨了眨眼,惊奇地看着周。

这是对关系到此为止的说法有点在意吗。\\

「我们也没什么特别的共同点,关系到此为止也是当然的吧。就算你是年级第一的美少女啊才女啊天使啊什么的,我也没打算想入非非啊。欠我一份人情真是幸运啊嘿嘿什么的,你以为我是这么想的么?」\\

看着真昼尴尬地稍稍移开视线,周苦笑着想道果然如此啊。\\

这应该不是本人意识过剩,而是曾经确实发生过这样的事情。\\

给美少女卖人情,借此拉进关系,这算是可行的手法。\\

真昼似乎是经历了好几次这种事情,也难怪那个雨天会那么警戒。既然是为了自卫,那也不好怪罪什么。\\

「对你来说也很麻烦吧。跟不喜欢的男人扯上关系」

「这倒是」

「是吧」\\

听见了本人的肯定,周反而感到一丝有趣。\\

以乖巧的好学生可爱的天使闻名在外的她,也有喜欢、讨厌和烦恼的事情。这让周稍稍有了一些亲近感。\\

对真昼来说或许是不慎说漏了嘴,她稍带怨恨地看了一眼诱使她失言的周。

真昼也是个有着感情的人类,而这就是最好的证明。\\

「其实也没关系的吧?不如说我倒是安心了。天使也和人一样有这种困扰呢」

「……请不要再那样叫我了」\\

看起来似乎叫她天使会很害羞的样子,真昼持续露出着不满的眼神。

由于这样也挺有趣,周又一次笑了出来。\\

「嘛,也没啥要紧的事情,没理由特意去麻烦你啊」\\

周下了这样的断言,真昼听了则惊讶得睁大了眼睛,然后微微露出苦笑的表情。\\

\vspace{2\baselineskip}

回想起真昼认真地低头行礼之后离开的场景,周躺在床上呆呆地望着天花板。\\

尽管药起了效,但身体还是很累。一旦放松,睡意便会涌上来吧。\\

周闭上眼睛,回味着今天的事情。\\

被天使(毒舌系)照看了什么的,说给谁听都不可能被相信吧。而且也不是什么值得说的事情。\\

今天的事情,是周和真昼两人间的秘密。\\

秘密,用这个词心里会有种痒痒的感觉。明明只是因为麻烦懒得跟别人说才这么决定的。\\

明天,就是一面之缘的陌生人了。\\

周这么默念着,渐渐地沉入了梦乡。

\subsection{寒空下的相遇}

如周所言,周与真昼的关系,回到了一面之缘的陌生人的程度。\\

受真昼照顾,第二天周便康复了。在便利店里买东西时两人正好遇到,但也没有什么特别的交流。不过,看见精神好的周,真昼稍稍露出了安心的表情。\\

周一上学的日子,也并无二致。不过他人。

非要说变化的话,仅仅只是上学时碰上了会低头示意的程度吧。\\

「哦——周你没事了啦」

「受你照顾呐」\\

上周回家的时候,看见周那半截入土的样子,树也很是担心,今天便早早地在楼梯口等周。树周末还发了条『没死吧』的短信。\\

就算周回了一条说自己没问题的短信,树还是半信半疑,今天看见本人活蹦乱跳的样子才做出夸张的动作舒了一口长气。\\

「哎呀——你当时都那个样子了我也会担心的啦。嘛,虽说这回病好了,你还是放点心思在生活上吧。比如整理屋子什么的」

「你咋跟哪里的谁一样啊」

「嗯?」

「啊没啥。……只是上周知道了该整理了。近日我会整理的」\\

树立马吐槽「给我现在就去整理啊」,而周则无视了这条吐槽。

那堆东西要整理只花个半天根本搞不定。\\

周无所谓地扭开了头,树尽管没追究下去,但是一脸的无奈。\\

「嘛反正是你家,随你便了。下次我去的时候记得整条道出来啊」

「……我看着办」\\

周苦着个脸在门口换好鞋往教室走去,听见旁边的教室的喧闹,不禁扭头看过去。\\

窗户里映着美貌一如既往的真昼不分男女地吸引着同学的场景。

面对搭话,用安静的微笑应对的真昼的身影,与前几日记忆的她,完全是两个人的样子啊。周这么想着,不自觉地露出了苦笑。\\

看见露出这样表情的周,树也把视线移了过去,看见真昼之后,一脸明白了的表情。\\

「啊啊椎名啊。还是老样子人气爆棚啊。毕竟是美少女呢」

「毕竟是天使大人嘛。……树也觉得椎名可爱么」

「你这么问那当然啦。嘛我的话有千在了,感觉不过是鉴赏用的感觉」

「你能别秀了么」\\

树的话,有一个叫\ruby{千}{\jpfont ちぃ},准确来说是叫千岁的女友。

这可是对关系超好的情侣,两人一起的时候可是能秀到周眼瞎的程度。\\

「要秀滚开别在这秀」周一边说着一边伸手赶着树,但树却不以为然。毕竟已经是经常的事了,「这家伙真没劲」树笑着回道。\\

「说回来啊,倒是周你不觉得椎名很可爱么?」

「挺漂亮的。就这样」

「真是淡泊呢」

「反正是我们高攀不上的高岭之花呗。也搭不上关系,那不就静静欣赏得了」

「是啊」\\

虽说前几天七七八八地发生了照看自己的事情,但本来就是次元不同的存在。\\

周跟真昼好上什么的,这种未来不可能存在。优秀的人之间才会互相吸引吧。

对自己的没用有自觉的周,跟不但可爱还全能的真昼之间,发生些什么有的没的首先就不可能吧。\\

扯上关系什么的,已经不会再有了。周是这么想的。\\

\vspace{2\baselineskip}

「……你在吃什么呢」

周的想法被颠覆,是周在阳台上吸着果冻饮料看着窗外的时候。

\subsection{名为赠物的天降之恩}

周连跑便利店都觉得麻烦,所以一边吸着家里常备的果冻饮料,一边靠在栏杆上呼吸着屋外的空气,结果真昼却恰好在这时走到了阳台上。

真昼看到了周后,跟周一样把头稍微探出阳台的护栏,接着注意到了周正吸着的果冻饮料,稍稍皱了皱眉。\\

周完全没想到自己会被搭话,结果就呆呆地愣住了一会儿。\\

「看了就明白吧。花不了一分钟便能补充能量的果冻」

「……你晚饭不会就是这个吧?」

「那还能是啥」

「……明明是个食欲旺盛的男高中生就吃这点?」

「多管闲事」\\

平常的话周是靠着便利店便当就着点配菜过活的,不至于就这么简单。不过今天,周懒得去弄晚饭,又没心情吃杯面,便靠着果冻饮料应付过去。

周估摸着这点量也是不够,等会可能还要来点零食之类的东西。\\

「……不做饭么」

「不做也不会做。你不也是知道的么」

「……而且还不会打理卫生,真亏你能一个人活下来呢……」

「啰嗦。跟你没关系吧」\\

周确实是被戳到了痛处,所以皱着眉头把剩下的果冻饮料吸完了。\\

关于扫除,周前几天已经吃了亏,本就打算处理一下的,但天天说来说去反而搞得人不想干了。\\

周反而非常好奇真昼为何总是这么啰啰嗦嗦的。而真昼盯着这样的周,然后轻轻叹了一口气。\\

「……请稍等一会」\\

话音刚落,真昼就从阳台走回了房间。\\

听着隔壁阳台关上窗户的声音,周小声嘟哝了一句「到底怎么回事」。

光说让人等着,是要等什么啊。\\

周疑惑地看向真昼的家里,但理所当然地没有回应。\\

(差不多也凉下来了,我想要回屋啊)\\

虽说周也照对方所说正在等着,但秋天的晚上比预想的还要冷。一件汗衫实在是不够。\\

不如说周也不知道为什么自己会就这么乖乖地等着。\\

外面的气温眼看就要降到呼吸会有白雾的程度。周长吐了一口气,这时从玄关传来了一阵电子音。

听见这来客的门铃声,周扭过了头。\\

周能想到的来客只有一位。\\

周实在不知她为何会来。避开散乱一地的衣服和杂志,他走到了门口。

即便不看猫眼,也知道来的是谁。周用脚把拖鞋撩到门口,解开防盗栓打开门——不出所料,比周眼睛稍低处,摆动着一丛亚麻色的头发。\\

「……你干啥呢」

「你过得太不像样,让我都看不下去了……虽然是剩下的,你拿去吧」\\

真昼语气冷淡地说着,手往前伸了过来。

比周小上一圈的娇小的手,端着一个特百惠的饭盒。半透明的盒盖模糊地透出里边煮食的影子。\\

或许是因为里面的东西还有些温度,盖子上起了一层水雾,虽然看不清,但里面应该是煮食没错吧。\\

周连着眨了几次眼睛。真昼似是理解了周的眼神里那询问原因的意思,深深地叹了口气。\\

「还不是因为你不好好吃饭。补品只是辅助,当不了主食」

「你是我妈么」

「我自认为自己的主张算是很普通的。另外你房间该收拾收拾了吧?现在这样连落脚都难」\\

真昼瞄了一眼周的身后,一副受不了的样子明显地眯起了眼睛,让周无言以对。\\

「……走还是能走的嘛」

「根本没有啊。一般来说衣服就不该掉在地上」

「就是会掉」

「洗好晾干叠好收起来就不会掉。读完了的杂志也请打包收拾好。踩到滑倒了可就是大问题了」\\

虽说话里略微带刺,但周明白真昼不知为何是纯粹在关心他,自然也不能一一回嘴。\\

确实上次来照看他的时候,两人就差点因为房间太乱而摔倒了。被这样说也是理所当然。

周听得一脸苦涩但是回不了嘴,只好从抿着嘴的真昼那接过饭盒。

慢慢扩散到手掌上的温度,在这渐渐转凉的天气里,让人很是心暖。\\

「……那,我可以吃这个么」

「你不要的话我只好倒掉了」

「别别我吃我吃。天使大人亲自做的晚餐一般可是吃不到的啊」

「……能别那么叫我么,真的」\\

周怀着报复的念头试着用了下学校里的外号,结果真昼的脸明显地染上了红晕。\\

也许对本人来说这个外号实在太羞耻了。站在她的角度,周也肯定会觉得不舒服,这倒是理所当然的。\\

真昼脸上泛起红晕,甚至还有点哭相地瞪着周。周看到这副样子不禁笑了出来。\\

「抱歉,以后不这么叫了」\\

再这么叫显然会真的坏了对方心情,所以开太多玩笑是不合适的。再说双方的关系也没好到那种能随便开玩笑的程度,事情不好做太过分吧。\\

真昼似乎也不想再被这么叫了,清了清嗓子,表示自己重新振作了一下精神。

然而她脸上微微泛红,看上去和刚才并没有多大区别。\\

「那这个,我就满怀感激地收下了。话说,你也别再对我生病那事过意不去了」

「那倒没,反正照顾了下生病的你也算扯平了。这个只是我的自我满足……嗯,只是我看你过着这样的废人生活,感到在意而已」

「是是是」\\

周在被真昼看见的时候都是一副邋遢样子,从某种意义上来说,她做出这样的判断或许是理所当然的。

就连现在周身后的走廊也是乱七八糟的,而且在真昼照看他那时就已经被看了个光,事到如今想瞒也瞒不住了。\\

「……要好好吃饭,保持规律作息哦?」

「真要当我妈啊」\\

看着一本正经说着的真昼,周一脸疲倦地吐槽道。\\

\vspace{2\baselineskip}

周端着分到的赠物回到家里,摸出一双超市拿的一次性筷子,坐在了客厅的沙发上。\\

真昼强塞过来的这东西,味道究竟如何呢。

周觉得,上次的粥很好吃。虽说舌头由于感冒而有些不灵敏,但那个从生米认真煮出来的粥,是一点一点温和地渗透到胃里的味道。\\

从那次的经验来看的话,真昼的手艺应该不错。那么实际上是怎样的呢。\\

周怀着几分期待,又有几分犹豫地打开了盒盖。淡淡飘出的无疑是煮菜的香味。

这是几种根菜和鸡肉煮成的。汤的颜色略淡,清楚地映出了鲜艳的胡萝卜的颜色以及旁边点缀的扁豆。\\

各色食物全部都切成了可以一口吃下的尺寸,强烈地刺激着只吃了点果冻的周的食欲。\\

周迅速掰开一次性筷子,首先夹起了一块萝卜。\\

「好吃」\\

口味如何,迅速见了分晓。\\

味道清淡,有高汤的风味,十分有注重健康的真昼的风格。而且,这还不是买来的那种颗粒调味料,而是拿鲣鱼和海带认真煮出来的汤吧。这样做出来的汤,美味完全不同。\\

周细细地咀嚼,享受着在嘴里渐渐散开的高汤、调味料、以及蔬菜本身的味道。

面对这不但充分发挥了蔬菜本来鲜味,而且还彻底入了味的煮菜,即使周并不爱吃蔬菜,也同样可以大快朵颐。\\

里面的鸡肉不多,好像在表示要好好吃蔬菜一样。那些鸡肉吃起来也十分鲜嫩,毫无干柴感,除了量以外无可挑剔。\\

以女高中生的料理来说,菜式的选择很朴素,但完全体现出了制作者的水准。

可以说,这味道和刚刚学会做菜的人做出的东西有着天和地的差别。\\

要是再来点饭啊味噌啊酱油啊啥的那就更棒了——虽然周这么想,但不巧的是他没去烧饭……不如说家里的米甚至都用完了,这点小愿望也无法实现。

虽说事到如今再说这些也晚了,但周还是后悔着,早知道就去买两包速食饭包回来了。\\

「天使还真是厉害啊」\\

周用着这本人听了怕是又要不高兴的叫法,称赞着学习运动加上家务样样全能的真昼,并一刻不停地享受着这味道理想的煮根菜。

\subsection{天使大人是老妈一说}

「还你这个。很好吃」\\

第二天晚上,周拿着借给自己的饭盒去了真昼家。\\

虽说周是真的不会干家务,但洗点东西还是没问题的。从礼节上讲也该好好洗干净还回去。周抱着这样的想法,带去了认真洗干净的饭盒。

虽说洗的时候的大费周章这事肯定没法跟真昼说了。\\

也许是听到门铃声便预料到了是周,真昼并没有看是谁便走出门外。\\

真昼身着一袭酒红色针织连衣裙。她看见周,微微眯起了眼。

她瞥了一眼饭盒确认之后,说着「好好洗干净了呢,真了不起」这样夸小孩子的话,让周不由得眉头微皱。\\

「为了我特意这样,谢谢了。这个给你」\\

真昼拿回了饭盒——到此为止都还没有问题,可接着真昼却拿出了另一个饭盒递给了周。\\

该说是果然吧,饭盒摸起来温温的。\\

里面的应该是茄子炒肉吧。冷下来温度没让盖子起雾,透过盖子可以清晰地看清里面的东西:茄子、熟猪肉,以及撒在上面的胡椒。

从颜色上看,肉上面的酱汁应该是味噌味的。稍显焦色的茄子与泛着光泽的猪肉,看着就很有食欲。\\

周觉得,这菜看上去很美味。

然而,周却不明白她为什么又拿来了料理。\\

「……啊那个,我是来还饭盒的来着」

「这是今天的晚饭」

「这我知道,但是啊」

「先问一句,没有过敏反应吧?挑食我就不管了」

「那个倒没有?但是再拿你东西就有点……」\\

连续两天晚饭都得到真昼的晚饭,结果会如何呢。\\

营养不全面的周对此求之不得,更重要的是真昼的料理水平远高于同龄女生,味道肯定不会差。\\

这饭盒里的东西肯定也很好吃。\\

不过这要是被学校的同学看见了那可就要变成超级悲剧了。当然,是在周学生生活没法安宁的意义上。

这栋公寓虽然设计上是供一人居住,但考虑到设施和地理位置等原因,租金不便宜。虽然这附近没有见到过真昼以外的同校同学,不需要担心被他们目击,但扯上这样的关系还是让周多少有点犹豫。\\

「做的量一个人吃有点多了,你收下我这边也很高兴」

「……这样的话那我就收下好了。不过一般做这种事情可是会被对方误解成对自己有意思的哦」

「你这么觉得?」

「哪有哪有」\\

毕竟对方一副你笨蛋吗的眼神,周没理由想到那边去。

再说了,才貌双全的真昼会对最近被看到的净是没用的地方的周有好意,简直就难以想像。\\

确实从可爱的邻居那得到晚饭这种事,简直就像恋爱喜剧漫画里的剧情,但两个人之间,却毫无恋爱喜剧的要素。恋爱当然是没有的,对话中也难寻喜剧成分。顺带一提周家里也没有米。\footnote{喜剧在日语中为{\jpfont コメ},与米在日语中的读音{\jpfont こめ}相同。}

有的只是天使大人的毒舌和对周的可怜罢了。\\

「那就没问题了……反正你本来也就打算靠着便利店便当和超市的小菜对付过去的吧」

「你怎么知道的」

「厨房怎么看都没有好好使用的痕迹,桌上还放着好多把超市和便利店里的一次性筷子。再说看你这样子不用动脑都猜得到。而且还一脸不健康的样子」\\

只来了家里一次就全都被看光了,周脸上不禁抽搐。可这也是事实,周也只得默默听着。\\

「……那就这样,我回去了」\\

该说的说完了该给的也给了,真昼关上了门回到了家里。\\

喀哒传来一声拴上防盗链的声音,同时周看向手上的饭盒。

手里端着一盆温温的晚饭,周轻叹了口气回到了自己家。\\

放了花椒和味噌的茄子炒肉果然美味,让周产生了强烈的想要吃米饭的心情。\\

\vspace{2\baselineskip}

结果,由于每天都拿着空饭盒换来装着晚饭的饭盒,周的饮食得到了戏剧性的改善。\\

真昼做的菜虽然调味并不浓郁,但却都很下饭。于是,周每晚都准备好速食米饭,和这些菜一起吃了。

真昼的料理每天都不一样,不论和食还是西餐,抑或是中餐,各种菜系都有出现,而且都非常美味,引得周食欲大发,很是难办。\\

每天都给搞得周都有点期待起来了。尽管周觉得这样很不好意思,但最近周就像是被饵料驯化了一样,不吃反而觉得怀念了。

天使的料理或许有依赖性。虽然心里想着不好,但周还是乖乖收下了饭盒,吃得津津有味。\\

「……最近脸色不错啊。反思了自己的饮食习惯了?」\\

也许是晚饭补充了不少营养吧,周的脸色好了不少。午饭时,树盯起了周的脸看。

周正吃着食堂里乌冬面。他面对一如往常感觉敏锐的树,流下了一点点冷汗。\\

「树,我觉得你有点吓人啊」

「咋了啊。还真说中啦?」

「不……嗯……该说是不得不反思吧」\\

每次跟真昼在公寓楼里遇见都会被说教几句,晚上还能得到晚饭,生活质量本身自然是会变好的。\\

虽然周想要对天使表达感谢,不过也有一点觉得她多管闲事的感情。\\

周稍稍含糊地肯定道,树则愉快地呵呵笑了起来。\\

「嘛那还不是。你那一脸病怏怏的样子肯定过得超胡来的吧」

「啰嗦」

「不过你咋就改正了啊?」

「……被迫的?」

「哈哈,被你老妈知道啦?」

「……虽不中亦不远矣」\\

真昼那语调真的跟老妈似的。

虽然叫成老妈又太年轻太可爱了,然而周却并不想拒绝这不知为何爱来照料的真昼。\\

「……我说啊树。我看上去真有那么不健康么?」

「嗯。原本主要是因为气色太白吧。身子很高但是那么瘦弱,脸上也一副没劲的表情,感觉这脸长得就不太健康」

「脸是天生的」

「知道。要的是更加有活力的脸啊」

「那怎么可能做得到……这样啊,一副死了的脸么……」\\

周不怎么照着镜子盯着自己看,所以不大清楚自己的样子。不过在别人眼里,他似乎是一副病怏怏的样子。

或许是因为周平常的表情看上去就像死了一样,真昼才会担心起来。\\

「周你也该多注意一下外表了啊。稍稍整一整也不至于这样啊」

「你这是若无其事地就损了我啊」

「是说你不肯穿好点一身寒酸还一脸死相有什么办法啊」\\

树借此机会劝周注意健康同时也注意一下自己的仪表,周则回了一句「多管闲事」扭开了头。

\subsection{天使大人既环保又平民}

「啊」\\

背后传来了一声银铃般的嗓音。\\

虽说周最近已经听惯了这声音,但这里并不是公寓,而是附近超市的零食区。\\

毕竟是有人在的地方,周没有料到真昼会对他有所反应,困惑地回过头去,看到的却是真昼瞪圆了眼站在那里。\\

她手上提着个超市的购物篮,里面放着萝卜、豆腐、鸡腿肉和牛奶这些晚餐的食材。

大概是她在路过零食区的时候遇上了周这么一个情况吧。\\

「先说好,是偶然啊。我可没跟踪你」

「我知道。毕竟超市就是这里最近了,这种事情我还是明白的」\\

面对周的抢先声明,真昼小声感叹着「倒是你为啥会想到那边去啊」,看向了手里拿着的记事本。

认认真真记下需要的东西,还真像做事一丝不苟的真昼的作风。\\

真昼仔细看了一遍写在那有着可爱花纹的记事本上的内容,没有看向零食区而是朝着另一边的调味料货架望过去。

「酱油和甜酒」,真昼发出了这样惹人怜爱的声音,寻找着家庭用品。她的姿态,尽管确实可爱,但更让周觉得有些不可思议。\\

「甜酒在这边,喏」

「不是那个,是甜酒味的调料。甜酒未成年人买不了的」

「这个也算酒啊」

「甜酒是当作甜味的酒来对待的。料酒因为加了盐没法直接喝,所以未成年人倒也可以买」\\

周拿起甜酒想要递给真昼,真昼却摇了摇头,把甜酒味的调料放进了购物篮。

几乎不做家务的周第一次听到这种事情,不由得应和了一句,同时盯着她麻利行动的背影。\\

真昼仔细看着摆着酱油的货架,留意到了写着价格的标牌,低声嘟哝着皱起了眉。\\

「……大特价一人仅限一瓶……」\\

真昼似乎是想再买一瓶备用。她惋惜地叹了口气,看向了这边。\\

「……那我也买一瓶咯」

「能懂我的意思真的谢谢了」\\

周察觉到真昼话中之意,苦笑着拿起了瓶酱油,而真昼则满意地把嘴唇弯出一道微微的弧线。\\

「……意外地节约啊」

「节约……不如说是能省就省吧。冤枉钱不该花吧」

「怎么说呢,真是有日本人的气质……不过,过着拿父母钱的生活的话一般都会这样的吧」」\\

周虽说是独自生活,但经济上还是要依靠父母。

周生在比较富裕的家庭,因此才能够住在那样整洁安全的公寓里。因为这样,周真的非常感谢父母。由于他不仅要交学费,生活费也要花掉不少,所以无谓的花费他都尽量避免着。\\

「……嗯。毕竟经济上还不独立,还是保持节制为好」\\

真昼淡淡地回答之后,整理起了篮子里的东西。她的声音里毫无热度。\\

真昼突然声音变得毫无起伏,有点吓人,但等再抬起头时,她已经回到了原先的表情。

刹那间瞥见的那黯淡的眼神,已经看不见了。\\

「……话说,你要买这个?」\\

如同改变话题一般,真昼看向周提着的篮子里的速食米饭和土豆沙拉问道。\\

从真昼那得到的晚饭确实好吃,但就凭那点量是不够的。所以,周平时就像这样,买来米饭当主食并来点沙拉当配菜。\\

「晚饭嘛」

「不健康」

「别啰嗦。不是买了沙拉么」

「虽然是土豆沙拉呢。……为什么过着这样的生活都搞不坏身体呢……」

「你太关心别人了」\\

如同说着你该多吃点蔬菜般,真昼眯细了眼睛朝着周放出了无言的压力,而周则扭开了头当作耳旁风。\\

这样那样地说了一堆话的时候,周结完了账,用塑料袋装好了东西,而真昼却从包里拿出环保袋装起了东西。

真是个爱护环境的平民天使大人啊。\\

可是虽然东西都装进去了,但那东西的量还是让周略微感到担心。

牛奶、酱油加上甜酒味调料,这就已经有三升了,虽说和水的密度有差异但少说也有三公斤了。这之上还买了食材,特别是那根大萝卜,想必是很重了。\\

虽说好好地装好扎好了口,但是就这么提回公寓怎么说都还是件体力活。\\

(结果上看因为有我才让她多买了这些调料和食材啊)\\

估计她是做了比平时还多的量,然后分给自己的吧。一直以来,周得到的已经接近一个人的份量了。虽然她说是只是因为做多了,但最近她应该是特意多做的。\\

结果上看,周给她填了相当多的麻烦。此时什么都不干的话,作为男人就太丢脸了吧。\\

看到真昼扎好了袋子,周试着提了提,虽说对自己而言不算太重,但是让女孩子提那就是挺重的负担了。

真昼虽然擅长运动,但是纯粹比力气又是另一回事了。不如说隔着衣服都猜得到那双细手不大可能有多大的力气。\\

看见周的行动,真昼眨了眨焦茶色的眼睛。

比起惊讶,更多的是感谢的氛围。\\

「……不是要抢你的啦」

「我不是担心那个。……就这点我还是拿得动的哦?」

「这种情况是直接接受的人比较可爱哦」

「说的跟人家不可爱一样」

「你看看你学校什么样,对我又是什么样」\\

也许是真昼对此也有认识,周看到她稍稍退缩了一点。\\

她在学校里表现出的那个所有人都认可的友善、温柔而谦虚的样子,在周面前却没有体现出来。\\

准确来说,她对周倒也很温柔,就是说话比较直白罢了。她似乎是对周根本没留委婉的余地,说话一直是有啥说啥。

这总比说谎要好得多,所以周倒是并没有多在意。\\

看见真昼默不作声,周认为机会正好,便提起了装满了食材的环保袋和自己的袋子,快步走向了出口。

背后似乎有些慌乱的动静,但周并未搭理,也不管两人拉开的距离,径直向前走去。\\

周并没有调整脚步等待真昼。

本来超市里就已经呆在一起了,要是两人并排着一起回家被谁看见了,恐怕要变成麻烦事吧。

互相来说,这个距离才是最理想的。\\

周装作两人没有关系,提着大袋大袋的东西匆匆往前走着。这时,他仿佛听见了后面传来的一句「……谢谢你了」的声音。

\subsection{天使大人的扫除大作战}

对周来说,家务活全都是不擅长的事,而其中打扫卫生则是最为令他苦恼的。\\

至于下厨,如果以受伤为前提,并且忽视外观与味道,倒也不是做不出来。\\

热一热塞进胃里不就得了——如果是这种理念下的,不但没有品相,口味也不太行的东西,周并非完全做不出。

当然这样的东西周既不想做也不想吃,也就自然不会去做。\\

洗衣服这事要是不会那生活估计要寸步难行所以没有问题。

实在不行也还有洗衣店,只要普通地把衣服扔进洗衣机,加点洗衣液和水一起转一转就可以,所以洗衣服是可以顺利完成的。\\

不过,只有打扫卫生这一件事,周是实在无能为力。\\

「这怎么办呢」\\

周末,周被树和真昼两人催着整理屋子,总算是下定决心要开始打扫,可面对这一片狼藉却无从下手。\\

周心里也清楚是自己的不对,但是东西这么多,他并不知道怎么收拾才好。\\

总之,周先洗了床单并晒了晒被子。

再往下该怎么打扫呢。

衣服跟杂志扔得到处都是,几乎找不到可以落脚的地方。\\

不幸中的万幸是,由于吃的东西味道很大,周都会及时扔掉,因此没有什么散发着异味或是沾着油污之类的过分的状况。不过是在地上扔得乱七八糟罢了。

虽然说,周苦恼的就是这乱成一团的东西。\\

正当周叹着气时,门口传来了门铃声。\\

周不禁漏出了啊的一声。\\

已经习惯了的来访者,不如说是送完东西就走的上天恩惠的快递员般的存在,如今却如同救世主一般。\\

周快步走向门口,结果脚下一滑,手直接扑向墙上,打开了门。\\

「打扰了,来拿下昨天的饭盒……你在干什么呢」

「……正准备打扫卫生」\\

真昼看到周站不稳的姿势和他的神情,眼神里微妙地透出了无奈。\\

「刚才好像传来挺大一声」

「……差点滑倒了」

「就知道是这样。打扫还没开始吧?」

「无从下手」

「就知道」\\

「乱得这么过分,确实没法下手」,真昼一如既往地发表着毫无顾虑的言论,令周表情抽搐但无从否定。

再说要是周非得跟她吵个胜负,那甚至都没法与她讨论如何开始打扫了。\\

不过说回来,该怎么问她呢。\\

周是准备问她扫除的窍门,但说起来她会给建议吗……周以略带犹豫的眼神看向真昼,而真昼则看着周背后散乱的走廊。

真昼看着身后的惨状,眼神仿佛说着「呜哇」一般。在真昼看来,走廊乱得真是相当过分吧。\\

「真是的。……这屋子,我来打扫吧」

「啊?」\\

周本来是觉得让真昼来帮忙这样的请求实在是厚脸皮,只打算问一下打扫方法的。

没想到,真昼却直接提出要帮忙。\\

「隔壁屋子这么脏想想也难受」\\

真昼的话一直都很过分,周也已经习惯,没什么生气的了,更何况真昼说的也是事实,没有办法反驳。\\

「连家务都不会做居然一个人生活,你是在闹着玩吗。猜也猜得到是抱着总会习惯的乐观想法混日子,结果至今还是什么都不会,你稍稍反省一下怎么样」\\

周完全无话可说。

妈妈也经常说着勤打扫就很轻松,而周却一直放置不管,结果就成了这样。周自己也认识到,这完全就是自食其果。\\

「本来的话,明明要是平常就勤打扫的话根本就不会变成这样的。这就是平日里怠慢的后果啊,真是的」

「……你说的对」\\

被说到这个地步周还不生气,一方面是真昼经常照顾周,让周没脸面对她,更何况真昼也确实说中了周的心情和过去的行动。\\

就是因为以为放着总能解决,没有重视,结果才变成这个样子。周已经只能对着真昼说的话默默点头了。\\

「这屋子可以让我来打扫么」

「……可以拜托你吗」

「既然是我提出来的那当然可以了。还有,我先去做点准备,你要是有什么隐私物品或者是贵重品就放到储藏间锁好门吧」

「这个不用担心」\\

虽说真昼说话很直,但是她那么亲切地来帮忙了,哪有理由去担心她偷东西啊。\\

再说,这么遵守常识还爱管别人事情的真昼,根本不可能加害于人吧。\\

「……你不会担心吗?」

「你也不是会做那种事的人吧」

「不是……是说你不担心我看见你作为男性想要藏起来的那些东西吗?」

「很抱歉没那种东西」

「那样的话倒是没问题。那我就去换身衣服,拿打扫卫生的工具过来。……这回大扫除,可要彻底扫干净了」\\

真昼耸耸肩,先回了自己屋子一趟,而周苦笑着望着她的背影。

\subsection{天使大人是(垃圾)扫荡大作战总指挥官}

返回周屋子的真昼与刚才相比换了一副打扮,穿着白色的长T恤和干草色的宽松长裤。

紧紧贴着身体的T恤,勾勒出窈窕而凹凸有致的身体曲线。

真昼把长发盘成丸子扎了起来,露出的雪白后颈让周微妙地有些不自在。\\

平常只看过她穿连衣裙或是短裙,对周来说这样的真昼有些新鲜。\\

本想着这样男孩子气的衣服会不会不适合真昼,但看来是多虑了。

周深刻地体会到,美人真是穿什么都好看。\\

不过,虽然这身衣服看起来确实比较容易活动,但却是可以外出的打扮。不知道她这身衣服弄脏有没有关系。\\

「那个,弄脏了没事么」

「反正是过段时间就打算扔掉的,弄脏了也没关系」\\

真昼边说着边穿好围裙,接着再次望向周房间里的惨状,轻轻叹了口气。\\

「先说好,要彻底打扫哦?」

「……知道啦」

「知道了的话那就赶紧开始吧。我可是不会放水,也不会让你妥协的」\\

被真昼「可以吧?」这样不由分说的语气一问,周只得乖乖地顺从。\\

如此这般,由天使发起的扫荡大作战拉开了帷幕。\\

\vspace{2\baselineskip}

「总之衣服先放到洗衣篮里面吧。原本打扫卫生应该要从上到下的,但要用吸尘器首先得解决这些衣服。地板都要被埋起来了。衣服这么多,要洗的话就分成几个部分来。还有穿过的和没穿过的分开来了吗,可以全部洗掉吗」

「啊啊都照你说的来就好……」\\

理所当然地,就算有吸尘器,也得先老老实实地从清理地面这堆东西开始。\\

「……地上没有扔着内衣吧?」

「再怎么说那些也会收进柜子里啦」

「那就好。总之衣服过一会再来处理,就算现在洗干净晒干,等下还是会被打扫扬起的灰尘弄脏。而且也没有那么多地方晒。不着急的话就等打扫完再去洗吧」

「是」

「……然后,杂志的话基本上扔掉吧。你要是有在收集的话另说,但扔成这样想必也不是。特别想留存的页面就先剪下来,然后处理掉剩下那些,扎好之后拿给废品回收站吧」\\

真昼迅速开始进行了打扫,一边指示着周把衣服都收拾到洗衣篮里,一边把杂志一本一本叠了起来。\\

虽说她让周看看有没有特别要留下的杂志,但其实周并不在意这个,于是摇摇头向真昼表示否定。真昼看到之后,便用自带的塑料绳麻利地扎好了杂志。\\

「衣服收拾好了的话就过来分辨下其他的杂物哪些是要的吧。扔在地上的杂物也一样,把需要的和不要的分开,然后把不要的扔掉。可以吧?」

「……哦」

「要是有意见就赶紧说」

「呃,这倒没有……只是觉得好有条理啊」

「不这样时间不够啊。也不想想你的房间有多乱」

「您说的是」\\

虽说是周末,但时间仍然是有限的。考虑到吸尘器的噪音对邻居的影响,只能在白天使用。\\

而光是使用吸尘器之前的工作就十分费事,真昼明白这点,才会尽可能地抓紧时间先收拾。\\

虽然想着劳烦真昼到这个地步真的很过意不去,不过多亏了真昼,转眼间就形成了越来越多的落脚之地,因此周的心里也非常佩服。\\

「椎名教官……」

「既然叫了老师还不赶紧学。你的私人物品这些我没法判断,所以你就自己把需要的东西好好地挑拣出来吧」

「Yes, sir!」

「别搞得我是个男生似的」\\

随口吐槽的天使大人,一脸严肃地用灵巧的双手收拾、处理着她能判断的东西。\\

由于周的习惯是什么东西都想存着,所以他很感谢、很羡慕真昼的干脆和果断。\\

尽管是别人的房间,真昼还是收拾得毫不客气。她的举止看起来十分有家庭感,简直就像是家庭主妇一样。

真昼那有条不紊的动作,就仿佛自己一个人就能轻松收拾完这间房间似的。\\

不过,她大概是行动太急,所以没有注意到脚下。\\

接下来这事毫无疑问是周的错。真昼踩到了地上的衣服,然后就这么失去了平衡。\\

当真昼从嘴里发出「啊」的瞬间,周下意识地滑到了真昼将要摔倒在的地板上。\\

接着,周感受到了柔软的触感与香甜的气味。其中还稍稍混有灰尘味,大概是由于周的慌乱而扑腾起的尘埃所致吧。\\

由于屁股着地,周感到一阵钝痛,不过还在忍受范围内,只有因为感觉到疼而叫了一声。同时周也感受着靠在自己身上的真昼的重量。

情急之中还能把她接住,算是很幸运了吧。\\

「藤宫……」\\

真昼抬起头,以微微发愣的视线看向周。尽管她看起来没有生气,不过似乎有很多话想说。\\

「摔倒了是我的不对,但就是因为会发生这种事情才要整理屋子的啊」

「真的十分抱歉,我在反省了……没受伤吧」

「没事。谢谢你特意来接住我。我才该说对不起」

「不,毕竟这都是我的错……」\\

周本来就已经得到了晚饭,现在甚至连打扫卫生都获得了帮忙,要是因为这个而让真昼受伤,实在是说不过去。

不如说,周感到非常抱歉,以至于连脸都不敢对上了。\\

如果真昼愿意的话,周甚至考虑了下跪,但是真昼似乎没有因为摔倒而责怪周的意思。

因此就算真昼用着可爱的声音小小地抱怨「真是的」,周也觉得是情有可原。\\

「收拾的目的可就是为了防止这种事情哦?」

「我知道。真的太抱歉了」

「……呃其实没必要这样道歉,毕竟我也是擅自过来帮忙的」\\

真昼似乎有些慌张地抬头看着周这边。

意外地被真昼以这样紧贴的姿势在极近距离下,用微微不安的眼神仰头看着,这让周非常难以冷静。\\

对和女性没什么缘分的周来说,光是这个距离就对心脏够不好的了,更别说他还正与美少女紧密接触着。

虽说双方都没有恋爱感情,但周觉得这样非常不合适。\\

而真昼似乎是没有意识到这个姿势,于是周轻轻抓住她的肩膀把两人分开,在羞耻心泛到脸上之前站起了身。\\

「……那就,继续吧」

「说的也是」\\

幸运的是真昼似乎并未注意到周的动摇,抓住周伸出来的手站了起来。

真昼似乎对两人的身体接触完全不在意,脸上的表情和平时一样。

而周则以「像真昼这样,被许多男性报以好意的少女,应该不会因此就动摇」这样的想法接受了现状。\\

周一脸苦笑地看着平静的真昼,觉得全让真昼帮忙很不好意思,于是打起劲来重新开始了打扫卫生。\\

「……吓了我一跳」\\

周应付这不熟悉的打扫卫生的工作也是十分头疼的吧。\\

因此,真昼那句小声的感叹,以及隐藏在淡色的秀发下那微微发红的耳朵,并没有被周注意到。

\subsection{天使大人,第一次的}

「……呼,终于变干净了」\\

结果,为了把周的屋子打扫干净花了一整天的时间。\\

整理地上的私人物品花了几个小时,然后还有洗衣服、清理灯具、擦窗户、弄吸尘器这一大把事情。等到全部弄完,已经是太阳落山的时间了。\\

真昼过来的时候还能见到的太阳现在已经完全沉了下去,由此可见两人到底忙活了多长的时间。\\

不过,也正因如此周的屋子才变得焕然一新。\\

地上打扫得十分干净,没有扔得到处都是的东西;窗户和窗沿也没有脏污;灯具也清理掉了灰尘,变得比以前更加亮堂。

周的房间也经过了打扫,所以地板上没有散落着东西,屋子里可以舒畅地休息了。\\

「居然花掉了一整天啊」

「毕竟乱成那样嘛……」

「那是你搞成那样的」

「您说的是」\\

面对天使大人兼救世主大人,周简直没法抬起头来,只得以毕恭毕敬的态度(想要下跪但被拒绝了)看向帮自己帮到这个地步的真昼。\\

而特意花费了一个宝贵周末来帮周打扫房间的真昼一边感叹了句「真是的」,一边扎好了垃圾袋。

虽然嘴上很毒,但她并没有显露出不悦,反而能看得出成就感。不过,她也有一点面露疲惫之色。毕竟让她义务劳动了一天,感到疲劳也是肯定的吧。\\

还让这样的她再去做晚餐就有些说不过去。

且不论晚餐有没有自己的那份,这种状态再让真昼干活就真的对不太住她了。\\

「我已经不想出门买菜了,晚饭干脆就点个披萨吧。至少今天就让我出钱吧,毕竟平常拿了你那么多东西」

「可是」

「不想和我一起吃的话,你自己带一块回去吃也可以」\\

真昼要是不想和周一起吃的话,那也没办法,让她自己带一块回去吃就行。

比起和真昼一起吃,周更多的是想要慰劳和感谢真昼,所以就算自己一个人吃也没关系。\\

「……不是这个意思啦。只是,披萨以前没有点过,所以有点吃惊」

「咦,没有点过么」

「……毕竟是一个人住所以没有点过……虽然有做过」

「居然会想要自己去做,太厉害了吧」\\

正常来讲,想吃披萨时都是点外卖或者出去吃二选一的。

特意从面团开始制作,会做这种费功夫的事情的人除了真昼应该没多少。\\

这真是擅长料理的人才会有的想法啊,周产生了这样的感想。\\

「点外卖什么的很平常的吧,我就经常点。你是那种连家庭餐馆都不会一个人去的类型吗?」

「根本没有去过」

「像你这样的很少见啊。我的话就算一个人也会去,爸妈懒得做饭的时候也会去。你爸妈不喜欢出去吃饭吗?」

「……我家的保姆会给我们做饭」

「还请保姆啊,挺有钱呐」\\

有钱人家的话那倒是可以理解。\\

真昼的举止很优雅,衣服和随身物品看着也很高级。

从这有品位的氛围和有教养的举止来看的话,不如说是那样也并不奇怪。\\

而她本人听到周这么说后,露出了薄薄的微笑。\\

「是呢,应该算是比较富裕吧」\\

真昼脸上的笑容,既非高兴,也非自豪,反而更近似于自虐,或者说是自嘲的表情。看见真昼这种笑容,周开始后悔自己的多嘴了。\\

以前提及父母的事情的时候她的回答也很冷淡,或许她跟父母关系并不是很好吧。

看上去,这是她不太想被提及的事情,所以周并不打算刨根问底。\\

每个人都会有那么一两件不想被知道、提起的事情。不多过问,也是面对没那么亲近的人的一种礼仪吧。\\

「嗯,也能当作一次经验嘛。喏,挑你喜欢的」\\

周不再继续关于父母的话题,而是把披萨的广告拿给真昼看。\\

这家是周常点的店,也是周所知道的范围内,所有提供外卖服务的店里面味道最好的一家。\\

虽然肯定是比不上用专门的石炉烤制的,但可选的配料从标准的到小孩子喜欢的,各种各样的都有,想必肯定会有能对上真昼口味的吧。\\

顺着话题转换,真昼接过菜单,视线迅速地扫了过去。

带有通透感的焦茶色的眼睛,钉在了各种各样披萨的照片上。\\

平时不怎么浮现出感情的双眼,现在看起来却仿佛闪耀着活力。\\

(……难道说,她其实挺期待的)\\

不知是否是周多心了,真昼看起来好像有点兴奋,在看了一会儿菜单后,指着一种一般聚会时点的可以体验四种味道的披萨,告诉周「那就这个吧」。\\

真昼像是在窥视一样看着周这里。在周同意后,她的眼里微微泛出了光亮。\\

见到她那略显喜悦的表情,周带着一点点苦笑,用一只手拿起手机拨通了广告上写着的电话号码。\\

\vspace{2\baselineskip}

大约过了一个小时,披萨送到了,真昼便立刻开动起来。\\

因为有四种口味,她似乎先烦恼了一小会到底要从哪种口味下手,最后决定从培根和香肠的披萨开始品尝。

不算意外地,真昼暴露出了大小姐的一面,小口地咬起了披萨。\\

虽然她是用手抓着吃,但进食的动作还是隐隐约约地流露出一股优雅风度,这恐怕是教育的成果吧。

但与此同时,周却又感觉真昼的举止透出一股小动物般的可爱感。\\

她细细地眯着眼看着拉出丝的芝士,放松的脸上带着笑意。这副样子看上去有种微妙的可爱。\\

平时的真昼看起来十分成熟,也让人感觉很稳重,而现在的真昼则表现出符合年龄的气质。

看着吭哧吭哧地小口吃着披萨的真昼,周产生了一股想要摸她头的强烈冲动。\\

「……怎么了?」

「呃,只是看你吃得津津有味」

「……别老盯着我看啊」\\

不过,她不满地皱着眉头的表情一点都不可爱。\\

「……怎么说呢,你还真是不可爱」

「不可爱也无所谓吧,不如说,如果我现在还是学校里的那副样子你也只会不舒服吧」

「那倒是。比起学校里的你,还是更习惯现在这样」\\

周跟真昼在学校既没什么接触,也没说过一句话。

只不过是偶尔能够看见那对每个人都同样和蔼的,天衣无缝的美丽笑容罢了。

而相对地,现在在眼前的她却不那么顾虑他人。\\

估计这才是真昼本来的样子,她在学校时则是进入了外出模式吧。\\

「对我来说倒是这边的样子更不容易累呐」

「不可爱的样子么」

「别记仇啊你……怎么说呢,学校里的你啊,完全不知道在想着什么」

「主要是晚饭做什么和上课的内容吧」

「你还会装傻说相声啊」\\

周想表达的意思是真昼像是有什么隐情一样。然而真昼却照着字面意思回答了。

她本人似乎没有装傻的意思,用有些不悦的眼神看向了周。\\

「不是那个意思,是说看不到你的内心啦。所以说,比起在学校那样,不知道你在想什么的样子,还是现在这样,即便有些不大友善,但是能直率地表达自己的感情的样子比较容易相处」

「……学校的举止不行吗?」

「这是你的处世方法,我并没有觉得不行。但我在想,你这样难道不会累吗」

「没有。反正从小就这样了」

「根深蒂固啊」\\

若是从小的习惯的话,做出那样的举止也算是能够理解,然而这也表示,她是有意识地要作出『理想的好孩子』的样子,并且别无选择只能这么做。\\

只不过,她隐隐约约透露出来的这些家庭环境相关的问题,周实在是无法去追问。\\

「……不过,有个能松口气的地方不也不错嘛?以结果来说是我帮你纾解压力了吧」

「……看着你那让人放不下心的样子我还真没办法放松」

「那可真是抱歉」\\

周动作夸张地耸耸肩,真昼则是有些开心地微微笑了出来。

\subsection{友人的来访}

自那次打扫卫生以来,周与真昼之间隔着的障壁似乎有略微变薄的迹象。只是两人的距离也并没有因此而靠近多少。\\

两人在学校依然像是陌生人,即使是在校外也顶多是分享晚餐的时候会寒暄几句而已。

就在前几天,周才被真昼提醒过要好好维持家中的整洁。虽说语气有些严厉,但周还是深切地感受到了真昼是位多么喜欢照顾人的少女。\\

也多亏了真昼的提醒以及她顺便给出的打扫建议,周的家里一直保持着刚扫除完毕时的整洁。\\

\vspace{2\baselineskip}

「哟,还真变干净多了啊」\\

树一听说周的屋子变干净,就在周末跑了过来。而看见这焕然一新的屋子,他不禁发出了感叹的声音。\\

「真没想到会变得这么干净啊,明明以前乱成那鬼样。记得之前我也帮你收拾过,结果没两天又脏了呢」

「啰嗦」

「不是我要说你,你想想自己保持地板没有乱丢东西的最长时间是几天」

「你放心,破了纪录,保持两周有了」

「两周就算破纪录你能有点羞耻心不?」\\

听到树说着些「一般不会把东西丢在地上」的大道理,周微妙地皱起了眉头,但碍于树只是在亲切地阐述常识,周又不好拒绝他的好意。

说到底,在请真昼帮忙之前,周也给树添过麻烦,因此在这方面周也没法变得强硬。\\

看到周憋着说不出话,树愉快地笑了起来。\\

「不过既然变得这么干净,真想把小千也带来啊」

「别,为啥我得在自家都要看你俩撒狗粮啊」

「跟我客气什么嘛」

「别拿我家当约会地点」\\

自己是有多悲哀才非得看友人情侣狂秀恩爱不可啊。

一直看着这公认的笨蛋情侣秀恩爱,周希望他们也能够为设身处得为自己着想一下。\\

虽然知道树是在开玩笑,但作为天天看着那俩在眼前放闪的人,周还真有点笑不出来。

这种事情,希望他们俩在自己家里做就好了。\\

「玩笑先不提。都搞得这么干净,应该就不会再弄脏了吧?」

「我有在妥善处理」

「所以说你这家伙啊……算了随你吧。但还是养成把拿出来的东西放回原本位置的习惯比较好哦」

「你是我妈么……」

「周你也真是的,屋子不经常打扫可不行哦?」

「不但令人恶心口气还真挺像我妈,你好可怕啊」\\

树做作地捏着嗓子发出的假声让周直起鸡皮疙瘩。\\

明明没见过周的妈妈,树演得却还挺像,简直吓人。

况且树一个大男人还搞出一副娘娘腔的德性真的恶心得周想让他立刻停手。\\

看着周吐出舌头装做要呕的样子,树愉快地笑个没停。\\

「周你妈原来是这样的吗。我妈则是根本不管事的那种啊」

「要说我还羡慕你嘞。我妈是事情一件一件没个完的那种」

「关心儿子的妈不好么」

「那样只会让孩子没法独立……」

「但你的情况是因为你太不像话了,你妈没法不多管管吧……」

「啰嗦。就算这样老妈她也太爱管我闲事了」\\

大概是因为周是独生子的缘故,所以周的母亲十分关心他。

与溺爱不同,她是那种事事都要插手,什么都要操心的类型。周虽然谈不上讨厌,但是应付起来有些头疼。\\

为了上高中方便而决定一个人住在学校旁边的时候,她也是这个那个的说了一堆,时不时还会跑来突击检查,实在有够麻烦。\\

「好啦,这不也说明周被看得多重要吗?」

「这份爱好沉重」

「你就放弃吧。总有一天你会知道这有多么珍贵」

「明明自己就是个标准的反抗期孩子,亏你还能摆出一副经验之谈的口气啊」

「哈哈哈。毕竟牵扯到小千,我也没辙啊」

树因为女朋友的事情跟父亲有不少争执,由他来说这些实在是缺乏说服力。但说的内容本身也算是有几分道理,周也就姑且听了一听。

这家伙也有自己的问题要解决啊,周这么想着长叹了一口气,而树本人则一脸乐观,表情上完全没显出辛劳。虽说树以前也说过「敢妨碍我跟小千的都被马踢死算了」这种有些吓人的话吧。\\

「不管怎样,老爸那边我会想办法的啦。总之周你要好好过日子哦?」\\

看着树爽朗的笑容,周面带几分烦躁地回了句「不用你说我也知道」,同时想到树说的内容跟某人简直是一模一样,而轻轻苦笑起来。\\

\vspace{2\baselineskip}

然而观看周的生活状况——其实并非树跑来的本来目的,他只是单纯想找周玩罢了。因此关于房间的话题很快便结束,两人一起玩起了游戏。

虽然当初是为了一周后的考试学习,但不知不觉间目的就变成了游戏。\\

「喂,你瞎浪费回复道具到时候可要没得用了啊」

「没事没事总归有办法的嘛」

「不是,说什么总归有办法啊你等级又没上去,到时候要出问题的……」\\

周烦恼着该怎么吐槽喜欢追求刺激玩法的树,而就在这时门铃突然响了起来,顿时令周产生了别的烦恼。\\

「嗯?客人?」\\

树也把游戏调进菜单画面之后抬起了头。\\

他知道周很少告诉别人自己家在这里,因此会来的朋友也几乎没有。再说就算有客人应该也会因为被入口的门禁拦住而按响对讲机才对。\\

「我也不清楚,大概是邻居吧?应该是有什么消息要传阅之类的」

「这样啊」

「我稍微出去下」\\

周好歹抑制住了自己抽动的脸部肌肉,瞒过了树后快步走向门口。\\

她按完门铃之后没有出声可真是万幸。\\

周也不做确认,直接伸手开门。为了避免让树看见,他把门打开了一个小缝钻了出去,然后顺手关上了门。\\

真昼一如预想站在门外,看着周一反常态的举动而连连眨起眼睛。于是周竖起食指对她做出「嘘」的动作。\\

「……拜托小声点。树来了我家」

「树?」

「我朋友。过来玩的」

「啊,原来是这样」\\

真昼明白了周一副鬼鬼祟祟的样子是怎么回事,点了点头便不再追究这个问题,而是和往常一样把饭盒递给了周。

看样子她从早上就开始准备了。里面装着的关东煮,在这个天气渐渐转凉的季节是再适合不过的菜色。\\

周心怀感激地收下之后,看着以一副理所当然的样子递来饭盒的真昼,轻轻叹了一口气。\\

「……呃那个,我一直都很感谢你对我的照顾,不过一直都没找到时间来表达我的谢意。抱歉」

「我也不是为了被你感谢才这么做的……不错嘛,屋子还能收拾到能招待朋友的地步」

「要我下跪磕头以表感谢吗」

「不是不是。千万别」\\

真昼一副像是在说「别弄得我像个坏女人啊」的无奈眼神,令周露出了苦笑。

毕竟面对她是真的没法抬起头,周说这话也是有几分认真的。受了她那么多照顾,就算下跪磕头也不为过。\\

而且从她那拿的晚餐量很可观,再这么当伸手党未免过意不去,所以周也打算找机会商量一下晚饭钱的问题。\\

「……那就这样,朋友来了你也不好说太久吧。我就先走了」

「……一直都谢谢了。树那边我不会说是你的」

「请务必这样」

「不过,就算我说实话他也不会信的吧」

「我想也是」\\

然而被这么坦率地肯定,也让周有点心情复杂。不过换做是周站在树的立场也绝对不会相信真昼会给周做饭这种事,只会怀疑这是周自己的妄想。

毕竟天使大人就是这种等级的高岭之花。\\

如果是高富帅姑且不论,自己这种又挫又懒的人能让天使亲自下厨招待,一般来说就算太阳从西边出来都不可能吧。\\

「……可以问你一件事么?」

「什么?」

「每天这样给我做晚饭你图什么」\\

一般来说劳力也是要算钱的,免费给晚饭什么的根本不可能。要是立场反过来周肯定也不会做的吧。

尽管周并没有期待「她对自己有意思」这种概率不到万分之一的事件,却也好奇得无法自已。\\

听到周的问题,真昼少许抬起头做出思考的样子,接着表情不变地回答道「只是我的自我满足」。\\

「这也并不是多困难的事哦。对我来说做两个人的份比做一个人的更轻松,也可能单纯是我喜欢招待他人吧」

「意思是你喜欢做饭?」

「这或许也是一个原因。而且你不会产生麻烦的误会,只会单纯地表达感想,这让我感到很轻松。再说你那饮食习惯我看着就难受,所以到最后还是我的自我满足」

「……是这样吗?」

「就是这样。所以说,你不必感到不好意思,就当作是天上掉下来的幸运就好了」

「好的好的」\\

真昼似乎也不想再说下去,彬彬有礼地致意之后留下一句「那我就先走了」便往自己的家走回去了。\\

(……真的是这样吗)\\

感觉这理由不至于免费提供晚饭啊,周这么嘟哝了一句后,也同样回到了自己的家里。\\

\vspace{2\baselineskip}

「谁?」

「认识的邻居,分了点吃的过来。我去放到冰箱里,游戏先别往下打啊」

「啊,抱歉BOSS我干掉了」

「喂你过分了啊」

\subsection{天使大人与老套展开}

周和真昼初次发生对话的那座公园,就在放学回家时会经过的路上。

周所住的公寓比较适合较少的人数,要容纳整个家庭有点勉强,所以公寓里小孩很少。附近的公寓也都是大同小异。

建在离此不远处的这个小小的公园,也因此酝酿出一股寂寥的氛围。\\

正是在这样一个小孩子们不会来玩的、冷清的地方——周看见了大概是放学后正在回来路上的真昼。\\

「你在这干什么呢」

「……没什么」\\

真昼在长椅上一动不动端坐着。看到周的身影,她微微眯细了眼。\\

这回与上次不同,因为互相认识,周也好搭话,但这么做之后真昼的回答却很僵硬。

她的口气并不像是在警戒,而更像是有什么难言之隐的样子。\\

「没什么的话就不要露出一副走投无路的样子坐在那里啊。发生什么事了吗?」

「……没有……」\\

尽管周很在意真昼那如同身处困境的表情,可真昼却并未道出个中缘由。\\

虽说和真昼有约定,出了家门就不扯上关系,但现在周看着真昼一副困扰的样子,便不自觉地搭话了。

真昼或许是不太希望周来多管闲事吧。\\

既然不想说就算了——周这么想着,看向一脸僵硬的表情的真昼,突然发现她上衣沾着几根白线——准确来说,是几根白毛。\\

「话说你的校服上有毛啊。是跟狗还是猫玩了么」

「才没有在玩,不过是救了一只在树上进退两难的猫罢了」

「这么老套的吗……啊我明白了」

「嗯?」

「在那坐着,绝对不要动啊」\\

周听完真昼的解释后,总算搞清楚了为什么她要一直坐在长凳上。他深深叹了口气,暂时离开了那里。\\

真昼一定会乖乖地停在那不动吧。

不如说她是动不了才更加准确。\\

这家伙总是在奇怪的地方逞强——周一边感叹着一边去附近的药店买来了湿布、绷带,再去便利店买来了咖啡用的冰块。回到方才真昼所在的地方,发现她果然还在那里一动不动。\\

「椎名,把裤袜脱了」

「啊?」\\

周直截了当地说完之后,真昼发出了冰冷到极点的声音。\\

「呃就算你发出那种声音……这样吧,我会转过去不看你的,你就拿我的衣服盖着脱。总之先冷却下伤处再贴上湿布」\\

周摇了摇手上提着的购物袋,顺带表明自己没有看人脱裤袜的奇怪癖好,而真昼的表情则明显僵住了。

看来是猜中了。\\

「……为什么会知道啊」

「只有一只脚鞋子半脱着,而且两边的脚踝大小还有微妙的区别,另外还一直不打算站起来。跑去救猫却把自己的脚扭了,真是老套」

「啰嗦」

「好好好。行啦把裤袜脱了,脚伸出来」\\

虽然是一看便知的事,但真昼似乎是没料到周会发现,所以露出了不满的表情。

不过,她老实地接过了衣服盖在了膝盖上,应该是打算按周说的做吧。\\

周转过身去不看真昼,把从便利店买来的冰块放进塑料袋里,并往里面灌进了水。

他扎上了口子不让水漏出来,然后从书包里掏出毛巾包上,现场做出了冰袋,接着慢慢地把身子转回来。\\

真昼则照着周所说的脱去裤袜露出裸足。

无论是没有多余脂肪、紧致而柔软的光滑脚部曲线,还是脚踝那不自然的肿胀,统统都一览无遗。\\

「看起来肿得不算严重,但乱动的话估计要恶化啊。总之,先冰一冰受伤的地方吧,虽然可能会觉得有些冷。等不那么痛了再给你贴上湿布,你好好静养」

「……谢谢」

「下次的话一开始就老实地拜托别人嘛。我也不是想着卖人情才帮你」\\

不如说是周这边想帮忙解决几件事,才能多少还上些那日积月累的人情。

真昼把脚放上长凳,冷敷着脚踝。虽然她脸上的表情没变,但已经没了拒绝周的意思,老老实实地坐着。\\

「不那么痛了吗?」

「……算是好了一点」

「那就给你盖上湿布吧,……别把我当成变态啊痴汉啥的生气啊?」

「我才不会对恩人说那么失礼的话」

「那就好」\\

周再次强调了自己没有不好的想法之后,蹲到真昼脚的位置,把湿布贴在红肿的脚踝上。

姑且问了一下有多痛之后,她说能站起来也能走路,但为了伤情不恶化才老实地坐着。总之还算是在轻伤的范畴吧。\\

周贴上湿布,用一起买的胶带固定好后,突然发现真昼正低头看着自己。

「意外地还挺能干的嘛」

「嗯,处理受伤还是可以的。虽然做饭不行」\\

周稍微开玩笑地耸耸肩,真昼则微微地笑出了声。

从刚才开始,她就一直是一脸僵硬。如果这样能让她稍稍放松下就好了。\\

看着态度稍稍缓和的真昼,周松了口气,从包里取出一条校服裤子。\\

「给」

「嗯?」

「别用那种表情啊。腿都露在外面了吧。又不能贴着湿布穿裤袜。这条我没穿过,你放心」\\

缠上了胶布的脚踝大了一圈,就这样穿上裤袜有些不太好,再说看上去也有些不自然。为了避免着凉和内裤走光,还是穿条裤子比较好。

真昼似乎是明白了周没有别的意思,便坦率地接过了裤子。\\

确认真昼穿好了裤子之后,周拿过刚才借出去的校服上衣,脱下现在穿在衬衫外面的大衣递给了真昼。\\

「给,穿上这个」

「所以为什么啊」

「你想让人看见你被我背着的样子吗」\\

再怎么说也不能让受了伤的人自己走回去,而且周最开始便是这么打算。

反正要回的地方也基本是一样的,周来带真昼回去既有效率也对伤情有益。\\

「啊,抱歉,能背着我的包么。毕竟背着包就没法背你了」

「就没有不背我的选项吗?」

「我说,扭着脚了就老实点咯。要是没人就算了,既然这里刚好有双好脚就好好利用下啊」

「脚吗」

「怎么,比较喜欢用手吗。你是想要我把你横抱着回去?」

「你有抱着我回家的力气吗」

「你是在小看我么……虽说确实没有自信」\\

抱起真昼本身是没什么问题,但要抱回公寓就实在有点吃不消。再说这么做太容易引人注目了,能不做还是不做为好。

周也明白真昼只是开个小玩笑,所以没有因为被看不起而生气,而是笑着想既然真昼还有精神开这种玩笑,应该没什么问题。\\

「穿好了就戴上帽子背上包。还有,你的包等我背起你来之后再提上,我要背着你没法拿」

「……麻烦你了」

「没事啦。作为一个男人,我还没丢脸到丢着受伤的人不管回家的地步」\\

周弯下腰把背对着真昼,真昼便小心翼翼地将身体挪到周的背上。

即便套上了大衣,穿了那么多层衣服,碰触到的真昼身体仍感觉十分纤细娇弱。\\

周确认真昼的两手已经以不至于勒住自己的程度抓紧后,慢慢地背着真昼站了起来。\\

该说是果然吧,真昼的身子很轻巧。

尽管真昼总是对周说这个说那个的,但她身体却纤细得让人担心有没有好好吃饭。不过也有可能是因为她的个子本来就比较娇小吧。\\

微微传来的甘甜香味,和着真昼不安地紧紧抱住自己的状况,令周不禁浮想联翩,但他拼命保持着不为所动的样子踏上了归途。\\

背着人这事本身多少还是会吸引路人的目光,但多亏真昼埋着头遮起了自己的脸,周并没有太受注目,算是得救了吧。\\

「好嘞,就到这儿吧」\\

把真昼背到了家门口放了下来后,周打算到此为止,于是很爽快地走开了。

真昼能扶着墙自己站起来,受的伤应该不算太重吧。所幸从明天开始都是休息日,静养几天的话应该能恢复到走路不成问题的程度。\\

「今天就不用管我的晚饭了好好静养吧。要不你也靠营养补品对付过去?」

「不用了。还有之前做好的剩着」

「那就好。再见」\\

不必担心饭的问题真是幸运。她能够不需要走动就再好不过了。

看着真昼掏出钥匙打开房门,周也摸出了自家的钥匙。\\

「……那个」

「嗯?」\\

周因为突然被搭话而看向真昼,真昼则是紧紧地抱着自己的包,怯生生地仰视着周。

那双微微摇晃的眼瞳让周感到有点疑惑。真昼的视线徘徊着仿佛有些为难的样子,但最终还是好像下定了决心似地直直注视着周。\\

「……今天,真的是非常感谢。帮大忙了」

「没事啦,反正是我自己想做的。那么,照看好自己啊」\\

对周来说要是真昼太过介意的话也会有些困扰,所以周轻轻地带过了话题。看见真昼低头行了一礼后,他就打开了自家的门锁。\\

突然,周发现自己的大衣和裤子还在真昼那,但转念一想反正真昼过几天就会还回来的吧,便没有回头走进了家门里。

\subsection{天使大人与班上的王子大人}

「喂,你小子咋成了这种全年短裤的元气系啊」\\

周一的体育课上,周感到了忧郁。原因之一是周不擅长运动,而另一个原因就是在这冻人的天气里,他落得了穿着膝盖长的运动衫的结局。

到了这个季节,主流已经是长袖运动衫了。而周的膝盖下面都露了出来,所以在周围人中间有些显眼。\\

「才不是嘞。忘带了而已」

「你还真是个笨——」

「啰嗦」\\

在周末,周没有遇上真昼,因而没有拿回自己的长裤,最后才成了这样。然而,周没法跟树这么说,只能说是忘了。

被笑话周还能忍着,但树嘿嘿笑着啪啪地拍他背时,他还是还手了。\\

听着树没新意的喊疼,周无奈地轻叹一口气,看向了别处。\\

刚刚他们正在操场上跳高,不过女生也是在上着需要使用操场的体育课,操场上有女生的身影。再者还是两班合上,操场上的人相当多。

那边是在进行田径类的竞技,她们正在等候时间中,所以看着周这边的体育课。\\

「门胁君加油——!」\\

体育课男女上课地点一般是分开的,现在女生在使得男生这边嘈杂起来……而女生们看着的是周的同班同学,有名的帅哥门胁优太。\\

周没怎么跟他说过话,但周知道他待人和气、学习也好,而且一年级就成了田径社的王牌选手,在女生间十分有人气。

对周来说不过是想着上天也会造完人啊的想法,但是对其他男生来说这就不那么有趣,有不少男生露出微妙的苦脸。\\

「哦哦——那边好厉害耶」

「是呢」

「没兴趣么」

「反正实际上没啥关系不是么。就算是同学但也没怎么说过话。怎样都好啦」\\

周觉得,反正也没有伤到自己,既然互相没关系,老实说怎么样都无所谓。\\

周尽管理解自己的想法是少数派,不过还是不至于跟其他男生一样到嫉妒的程度。

不如说他什么都那么完美,周反而觉得连嫉妒都没有意义。\\

「周你一直是这样嫉妒心很淡呐」

「怎么,要我说『如此受欢迎真是羡煞我也』么」

「你不是那种设定吧」\\

周斜看着咯咯笑着的树,望向一脸沐浴在女生声援里一脸爽朗笑容的门胁。\\

以男性的视点看,门胁也是体型均整、相貌帅气,简直就像是王子大人。因为一眼望去找不到可以算作缺点的缺点,实际上他的外号就叫王子。\\

面对女生们热情的眼神和高亢的声音,优太微笑着招手回应,周甚至钦佩地觉得他是个善于社交的。\\

「哎呀,王子大人这么熟练真是厉害啊」

「是呐。那笑法我可做不出来」

「女生们也很兴奋啊」\\

树的话有爱得不行的女朋友千岁在,对其他女生基本就是毫无兴趣的样子。

千岁看起来也是对门胁毫无兴趣的样子,想必树对门胁也没什么想法吧。\\

(王子啦天使啦,这学校还真是不少外号超羞耻的人在啊)\\

说起来天使大人,也即是真昼,她有好好静养么。

周末似乎没有出门了的样子,应该是安心养伤了,就是不知道伤好得怎样了。\\

正好另一个班就是真昼的班,周朝那边扫视过去,在操场边上望见了即便人很多但还是容貌很显眼的少女。\\

她没换上体操服,也不在上课的人群里,大概是在观摩吧。

真昼静静地站着,吸引了许多男生的目光。\\

虽然距离很远,但周和真昼对上了眼神,周尴尬地偏开了眼神,而真昼则在嘴角微微露出了笑容。\\

而因真昼这笑容对着周——不如说是对着男生们的集团,同学们说着「刚才她对我笑了?」「才怪是对我吧」骚乱了起来。\\

「这可是好机会啊,得向椎名同学展现自己吸引她」

「哪能让王子把好处全占了」\\

微微一笑便引起这么多反应,该说是厉害还是说只是他们太单纯了呢。\\

「……真是单纯啊」\\

树似乎也想着同样的事情,嘟囔了一句。周也不禁笑了出来。\\

「既然关系到学分,我们也差不多努力下吧」

「咋了,周你被天使大人看了也要展现展现么」

「才不是啦。不是说了没兴趣么」

「嘛,也是。你还真是对啥都没兴趣啊」\\

看到树又开始嘚瑟地说起女朋友的好,周敷衍几句之后,再次看向真昼那边露出了苦笑。

\subsection{天使大人的慈悲?}

「前几天真的谢谢你了。这是你借给我的大衣和运动衫」\\

这一天,真昼一如往常送来晚饭,除了饭盒以外她还带来了一个纸袋。

纸袋里隐约能看见的是上周五借给真昼的大衣和运动衫吧,是好好叠整齐放进去的。\\

「嗯。伤怎么样了?」

「已经基本上不痛了。现在痊愈之前暂时不会做剧烈运动」

「那就好。体育课记得也是在旁边看着的吧」

「是的」\\

以防万一,真昼体育课在旁边观摩,这应该是正确的做法吧。虽然她看上去不怎么痛了,但走路的姿势还是微微在照顾那只脚,可见应该还并没有完全痊愈。\\

周认同地点了点头,突然回想起体育课的场景笑出了声。\\

「嘛不过说起来啊,天使大人的人气还真是高啊。一个微笑就能让全体男生干劲高涨呢」

「所以不是说了不要那样叫我啊……。我是很困扰啦,他们至于这么高兴么」

「美人朝着自己露出笑容当然是这样了啊。你想门胁朝女生招手的时候,她们不也一样咿咿呀呀叫起来了么」

「……门胁……啊啊,那个很受欢迎的人吗」\\

真昼一脸没什么兴趣的样子——不如说确实不感兴趣,她光听到名字还记不起来,听周的解释才想起这个人来。\\

虽然不及天使大人,但门胁在年级里也算是十分有名的男生,因此真昼光听到名字不知道是谁让周很是意外。\\

「你没有兴趣吗?」

「没啊。毕竟班级不同,也没有什么能扯上关系的事情」

「诶。其他女生倒是挺感兴趣的来着,天天说着什么好帅啊」

「毕竟长得好看嘛。我没跟他说过话也没有跟他有什么关系,所以无所谓了」

「你对这种事情倒是很无所谓嘛」

「要是就因为美丑而对对方产生好感的话,你怎么对我没点好感?」

「哦哦,原来知道自己长得很可爱啊」\\

真昼的话确实在理。

长得好看可以是个产生好感的理由,但并非是只因为好看便会喜欢上对方。\\

周同意这个观点,也承认真昼是个美少女。虽说周有点意外的是她本人自己有认识并且还肯定了这一点。\\

「身边那么不得安宁的,我再怎么说也知道了。而且,客观看我也明白自己长得还算可以,打理自己也没有偷懒」\\

真昼理所当然地说着,但她完全没有表现出自大的态度。\\

实际上真昼也为了保持美貌而用尽了手段吧。\\

真昼的容貌本身便很端正,但她并没有就此满足。

头发似乎真若天使这一外号般几乎能看见光环。肌肤也十分完美,毫无褶皱黑斑。尽管在做家务,双手也并未因此变得粗糙,而且指甲也好好修剪过。

当出则出,当隐则隐,能把身体打理得如此匀称,想必不是一朝一夕的功夫就练成的。\\

「确实啊。你平淡地说着的都是事实,我是不会觉得不舒服啦。不过这样的话,你被夸奖的时候也不会害羞吧」

「别人太过奉承的话,我只会先觉得烦人哦」

「还真是麻烦啊,当个美人」

「也有相应的回报,所以也并非都是不好的事情呢」

「……真是一副事不关己的语气啊」

「怎么,难道要我害羞地回答『没那回事』更好么」

「别别别,这跟你平常画风不一样违和感太强了」

「就是说嘛。我也觉得,对你摆出那副样子没什么意义」

「是啊」\\

真昼这毫不掩饰的说话风格,就算现在改掉周也只会觉得为难,要是真昼照着在学校里的样子对待周,周觉得自己会稍微有点鸡皮疙瘩,所以还是希望她保持这样为好。\\

习惯可真是恐怖啊。要是学校的天使大人举止像天使一样,周反而会感到奇怪了。

周意识中的真昼已经完全是现在面前的这个真昼而非学校里的那个了。\\

两人得出的结论是现在这样就好,于是周看向了递给自己的饭盒。

比平时更大的饭盒里装着好几样的各式菜品。比起分赠已经更像是送来一份便当了。\\

「今天挺豪华的啊」

「毕竟受了你的关照」

「不是说了不用在意的么……哦哦,居然还有可乐饼啊」\\

可别小看这可乐饼。

虽说作为下饭菜可乐饼是十分常见,但自己做起来却十分麻烦,在家庭料理里面算是最麻烦的。\\

把土豆蒸完炒好,配上牛肉啦洋葱啦做出形状,冷却之后再裹上面衣下锅炸……如此这般,有一堆普通却很麻烦的工序。

基本不做饭的周只是看见母亲做这个的工序便觉得绝对很麻烦。

因此,让母亲做的时候她也经常不情不愿的。\\

「虽说只是把做好冷冻起来的东西炸了一下」

「所以顺带做了炸鸡吗」

「是呢」\\

油炸食品方面,周在独居时只吃过店里买的那种小菜,所以能得到这样的手制品简直是感激不尽。

要是再贪心一点的话,周还想要在刚炸出来面皮还酥脆的时候和饭一起吃。\\

「……偶尔也想吃点刚出锅的呢」\\

真昼考虑到卫生上的原因,都是把东西先凉一凉再放进饭盒里,因此吃之前必然要再次加热。虽说用烤箱可以还原面皮的酥脆感,但还是比不上刚出锅的味道。

当然这样也十分美味,但刚出锅的还是要好上不少吧。\\

周不过是把愿望说漏嘴了,没什么别的意思,但可能是自言自语说得太清楚,真昼听见之后稍微皱起了眉头。\\

「你意思是让我去你家?」

「我可没那么说啊,都已经分我饭菜了,再那样也太没分寸了」\\

周为了摆脱莫须有的怀疑,耸耸肩明白地表达否定,真昼则用手撑着下巴低着头往下看。

她似乎是在考虑着什么,没有和周对上眼。\\

「……一半」

「嗯?」

「伙食费各出一半,我可以考虑在你家做饭」\\

真昼终于开口,她说出的话威力之大,让周目瞪口呆。\\

虽然只是玩笑或者说是不小心说漏嘴的念头,结果真昼认真考虑之后还答应了,让周不由得不知所措。\\

一般来说,会有人想跑去关系不算亲密的男性家里做饭吗。

就算这样更有效率,对方毕竟是异性,而且关系并不至于亲密无间。按理肯定会觉得不安的。\\

「各出一半与其说正和我愿不如说我这边得到的太多了所以完全没问题……你不觉得危险么?」

「要是被做了什么锤烂就好了。在物理上,烂到再起不能」

「哎哟好怕,瑟瑟发抖」

「再说,就算我不那么做,你从风险上考虑,也不会做任何事情吧。你很清楚我在学校里的立场吧?」

「如果做了什么我肯定完蛋了啊」\\

周与真昼压倒性的人望差距加上真昼作为柔弱女性的事实,要是真昼说自己要被周做些什么那周就百分之百没法去学校了。\\

周还没有那么愚蠢和没有节操,以至于即使知道社会性死亡的结局还去做什么。

不如说周自己本身就没这个打算。\\

「再说了」

「嗯?」

「你看上去对我这种类型也没啥兴趣」\\

真昼一脸认真的断言令周不禁苦笑。\\

「要是我就喜欢你这样的呢?」

「那你估计就会不厌其烦地跟我搭话,然后我就会跟你拉开距离吧」

「那我算是被认可了咯」

「嘛,至少觉得你很安全」

「那还真是谢谢了」\\

这样就行了么——尽管周这么想着,但他完全没有打算对真昼做什么,所以没有否定。\\

周自然不会放过这千载难逢的能够享受新鲜的极上美味的机会。他在收下「无害的男人」称号的同时,获得了共进晚餐的权利。

\subsection{围裙与手制料理是男人的浪漫}

真昼同意在周家里做饭的同时,提出了如下条件。

\begin{itemize}
    \item 周出材料费的半数加上若干人工费。
    \item 如果有事不能一起吃饭至少提前一天通知对方。
    \item 食材的采购和饭后的处理由两人分担。
\end{itemize}

关于第一条中的人工费,是周不好意思占用真昼的时间所以才提出的。在这一点上真昼做出了让步,而其它部分则没有发生什么争执,顺利地决定了下来。\\

至于让真昼来做饭这一点,由于早就是既定事项,所以并没有什么可烦恼的。\\

于是在这么决定好的第二天,真昼便早早地拎着——准确来说是两只手抱着购物袋来到周的家,做起了下厨的准备。\\

「……还真的都新到几乎没有使用痕迹呢……」

「啰嗦」\\

家中有一位穿着围裙的女性。周明明身处于这种好似男人浪漫的具现一般的状况,却不知为何感到如坐针毡。\\

之所以会这样,理由之一是将头发扎成一束的真昼带来的新鲜感,但主要原因还是在于厨房基本就没使用过这点被真昼再次指出所造成的尴尬。\\

「明明有这么多好东西却放着吃灰」

「你能用上的话那不就不吃灰了么」

「那只是结果论。这么好的厨具都因为怀才不遇哭出来了」

「那就用你拿手的厨艺让它们破涕为笑吧」\\

周干脆地表达自己不行,真昼则一脸无语地看着他,但也许是料到如此,她只是叹了口气而并没有抱怨什么。\\

「那么,有做饭用的调味料吗」

「有啊,你当我傻吗。保存方法和保质期也都没问题」

「哎呀真是意外」

「因为都没开封」\\

大部分调味品都以未开封状态被放在阴凉避光的地方,所以应该不必担心吧。

明明都买来了,这些东西却得不到展现自己的机会。实际上由于周基本没下过厨房,所以压根就没动过它们。对调味料来说,能被真昼这位厨师使用,应该也算是物尽其用了。\\

「这可不是什么好自豪的事情。不过,要是不够的话我回家拿来用就好」

「帮大忙了」

「总之既然有基本的调味料,那应该多少能做出点东西。啊,今天的菜单我擅自定下来了,没关系吧」

「反正我不太清楚这些东西,能吃的话什么都行。我也不怎么挑食」

「这样啊。那我就动手了……请告诉我一下调味料放的地方」

「都放在这个篮子里」

「……还真的都没开封呢……」\\

真昼瞄了一眼塞满调味料的篮子,无语地皱了皱眉,不过因为周事先说过,她马上便恢复到原先的表情,到水龙头旁边洗起手来了。\\

「那我就开始做饭了。你就在客厅或者房间里等着就好」

「行。反正我也帮不上忙」

「还真是干脆……不过也好,要是你不会料理还晃悠来晃悠去的我也很难办」

「你也很直接啊」

「毕竟是事实。跟你也没有必要拐弯抹角的吧」\\

正如真昼所说,自己显然是个累赘,于是周老实地走回客厅观察起真昼的背影。\\

真昼洗完手后就迅速投入到了调理工作中。\\

虽然不知道她要做什么,但从准备好的材料看应该是日式餐点。

能在自己家让真昼做出那些美味的料理,周不禁感到有些不可思议,甚至怀疑自己是否在做梦。然而他看到真昼摇晃起背后扎成一束的秀发处理着食材,就知道了一切都是现实。\\

(……怎么说呢,感觉就跟有了老婆一样)\\

尽管两个人彼此都没有这样的感情,但眼前的状态看上去实在像自己已成了家一样,让周不由得心生联想。\\

周自然是对真昼没有一丝一毫的非分之想,不过有个美少女在自家厨房,这状况本身就足够让人浮想联翩了。\\

果然,不论是否抱有好感,可爱的少女愿意为自己做饭这一场景,都足以让周的胸口产生一丝悸动。\\

「……你不会在想些乱七八糟的事吧?」

「别瞎猜啊」\\

真昼头也不回的突然发问让周差点面部抽筋,但也幸亏真昼没有回头才让此事不至于败露。\\

「这家伙还真是敏锐啊」周心生佩服、感到背脊发凉的同时,也收起了微微涌出但还尚未形成邪念的男人心,观察起了真昼的背影。

\subsection{天使大人与至高的料理}

大约过了一个小时,饭桌上开始排起了一盘盘料理。

由于是真昼定下的菜单,故而桌上的都是符合真昼健康追求的和食。\\

「这边的厨具和调料也算挺够用,看来是不用我回家取了。明天开始还能做更精致一些的菜」

「你肯为我做饭就让我感激不尽啦」\\

或许是因为真昼不清楚有多少厨具和调料能用,所以比起精致的菜肴更多是简单的东西。但即便如此,色彩和摆盘也堪称完美。\\

青菜煮鱼、味噌煎蛋等等,各种对周来说连想都不敢想要去做的和式菜色并排摆在桌上。\\

尽管周不怎么挑食,但他基本上还是喜欢和食。看到真昼稍稍抱有歉意的样子,周甚至都想告诉她说自己想吃的就是这个了。\\

「……看上去超好吃的」

「这么夸我让我很高兴。赶快趁热开动吧」\\

真昼边说着边坐上了椅子,于是周也坐在了正对方向的椅子上。\\

单人生活准备的餐桌尺寸偏小,不论怎么坐两人靠得都很近。

幸运的是家里姑且有准备两把给客人用的椅子,但面前坐着一位美少女还是让周产生了一种难以名状的感觉。\\

不过,一旦开始品尝料理,真昼的美貌什么的也都无所谓了。\\

例行示意开动之后,周首先尝了一口味噌汁。

在嘴唇碰上碗沿那一刻,周一边享受着味噌与高汤的香气,一边慢慢地将其含进嘴里,然后与那香气相称的味噌与高汤的风味便在舌尖散开。

这种与速食味噌汤完全不同的柔和口味,肯定是经过了精心计算和调整才得到的吧。\\

味噌不太浓,咸淡上也保留住了高汤的风味。\\

第一口略显清淡,应该是因为真昼考虑到了味噌要和其他料理一同食用,这样的味道在喝完的时候恰好会觉得浓淡适中吧。\\

与其说是有什么不足,不如说是让人安心的、会引起品尝米饭和其他菜品欲望的味道。\\

「好吃」

「谢谢夸奖」\\

周坦率地表达出自己的感想,而真昼则放下了心,微微眯细了眼。

尽管周平常一直在夸她做的菜好吃,但是当面说出感想还是会让她紧张的吧。\\

看着刚刚一直在关心这边反应的真昼开始吃了起来,周也向着菜品伸出了筷子。\\

把桌上的菜全部尝了一遍后,周觉得真昼的料理果然非常美味。\\

煮鱼非常入味,同时还保持了肉中的水分。

为了做到入味而长时间加热的话,水分就会流失,使得肉的口感变得干巴巴的。但真昼做的煮鱼肉质却十分鲜嫩,口感很好。\\

煎蛋卷的调味则是正中周的喜好。

在表面鲜艳的金黄色引诱下,周尝了一口,舌尖传来的果然是高汤那柔和的风味。

煎蛋卷有加糖或者除了盐什么都不放等等的各种各样的派系,真昼做的则是加入了高汤略带甜味的蛋卷。\\

隐约而柔和的甜味,或许是蜂蜜吧?

放的量应该并不是很多,但留有余韵的甜味增加了味道的深度。\\

当然不论是甜味的还是咸味的煎蛋卷,周都不讨厌。

不过,周最喜欢的还是这种加入了高汤略带甜味的调味精致的煎蛋卷。如今吃到这理想中的蛋卷,周甚至有些感动。\\

「好吃」周自言自语地感叹了一句,然后又吃了一口。

火候的调整也是绝佳。周咀嚼着这饱含高汤、口感鲜嫩的煎蛋卷,静静地享受着这美味。\\

周一边默默想着「确实比我妈做的还要好吃啊」这种对不在现场的母亲有些失礼的事情,一边幸福地大快朵颐。接着,周注意到真昼正盯着自己在看。\\

「……看起来吃得很香呢」

「实际上也很好吃嘛。面对美味应该要抱有敬意不是么」

「嗯,这倒是」

「而且,比起板着个脸吃,还是这样坦率地表达好吃,我们两边都开心吧?」\\

就算料理十分好吃,不从表情上表达出来的话,制作者也会感到不安和在意。板着个脸的话就算说好吃也会让人怀疑到底是不是真话。\\

比起那样,不如坦率地把自己的感受表现在脸上,对双方都有好处。毕竟不管是感谢还是被感谢的人,都喜欢有个好心情。\\

「……是呢」\\

真昼似乎是接受了周的解释,微微露出了笑容。

如同松了口气般的、表达着安心的柔和笑容,其可爱程度,甚至让周的大脑有一瞬间变得一片空白。\\

「……藤宫?」

「啊,……呃没什么」\\

看得入迷了——这话自然是说不出口。周压抑住渐渐涌起的羞耻感,为了不被真昼发现而继续吃起了晚饭。\\

\vspace{2\baselineskip}

「……我吃饱了」

「喜欢吃就好」\\

周将摆在桌上的饭菜一扫而空,满足地表示自己吃饱了,而真昼则淡淡地回应了他。

不过,真昼表情很柔和,应该是因为看到周这样将饭一点不剩的吃完而感到喜悦吧。\\

「很好吃啊」

「看你的样子就知道了哦」

「比我妈做的还好吃呢」

「把女孩子亲手做的料理跟妈妈做的比较好像是禁忌哦」

「那不是贬低的时候的说法么?话说你很在意?」

「我倒是不在意呢」

「那不就得了。反正好吃的事实也不会变」\\

真昼的厨艺可不是光靠一点点下厨的经验便可以达到的程度。\\

周的母亲虽然和真昼相比有着更丰富的下厨经验,但她调味的喜好不同,而且还很随便,自然比不过真昼那精心计算调整的调味了。\\

不如说在做饭上连父亲都比母亲更加擅长,更不用说和真昼比了。\\

「……哎呀感觉我是不是太幸福了啊。毕竟每天都能吃到啊」

「我们都没事的时候是这样吧」

「……话说,每天一起吃饭真的好吗」

「不好的话我也不会这么提议了」

「话是这么说啦」\\

周也十分清楚像真昼这种直率的人,要是不喜欢的话一开始就不会这么提议,但即使如此他还是会烦恼这样到底好不好。

虽说周付了一半的材料费加上人工费,但还是不禁担心真昼的负担会不会太大。\\

「……我说,一般来讲,你会给谈不上喜欢的男的做饭吗?」

「还不是因为你生活太不健康了吗。再说,我很享受做饭这件事本身,也并不讨厌看你吃得津津有味的样子」

「但是啊」

「……要是你这么在意的话,我其实不给你做也无所谓的哦?」

「别别别还请你务必做上我的份」\\

周立刻反射性地回答,这也代表真昼的料理对周来说就是如此必要和符合喜好。\\

事到如今,要是真昼真的不再做料理,对周来说那可就真的算得上是性命攸关的问题。

虽说周对自己的胃已经被抓住一事早有认识,但现在的问题是真昼的料理实在太过美味。这样下去,一旦回到小菜就饭的日子,生活就会变得无滋无味,想想就可怕。\\

听到周那好懂的回答,真昼那有些无奈的脸上露出了似是苦笑的表情。\\

「那就请你老实收下吧」

「……哦」\\

想到与这大慈大悲的天使大人共进晚餐的日子还将继续,周带着喜悦、期待和罪恶感,不得不叹了一口气。

\subsection{天使大人成绩也很完美}

「周~考得怎么样?」\\

期末考试终于结束,总算熬过了考试地狱的学生们,比平常更加兴奋地在教室里聚成了几团。

周和树也是一样因为考试结束而松了一口气,评判着自己这次的发挥。\\

「嗯?一般吧。差不多还行」\\

听到树的发问,周虽然做出了回答,但其实并没有什么可说的。题目都在考试范围内,只要平时做好复习的话这场考试并不算难。

这次写起题来的手感跟以前并没有什么不同,所以周也没有什么特别的感想。\\

周虽然是个怕麻烦的人,但复习还是基本不会落下。

上课学的内容他大致都懂了,考试也发挥正常。虽然满分还是有些难,但考个八九十分还是没问题的。\\

「然后你年级前三十稳了是吧……你个学霸」

「靠平时习惯啦」

「就你那平时习惯你还有脸吹?」

「再怎么样也轮不到你这个天天秀恩爱不读书的家伙讲」\\

树和周的差距,与其说是头脑,不如说是树在女朋友的身上花了太多时间造成的。

树脑袋也不笨,要是认真起来的话应该也能拿个挺不错的名次。只可惜树把时间都优先花在了千岁身上,结果成绩就比不上周了。\\

「……女朋友可是个好东西哦?」

「对对对对」

「我说啊周,你也去找个咯」

「想有就能有那这世上男儿们也不会流下血泪了啊」\\

这世上想要女朋友而求之不得的人比比皆是,对某些人来说树这句无心之语听上去想必是十分扎心。\\

不过周倒是并没打算对树生气,说到底周现在根本也没有想要个恋人的欲望,于是只管听过便算了。\\

「再说,女朋友咋找啊」

「来个双重约会——」

「然后我和那个幻想的女朋友就会被你俩秀到闪瞎吧」

「那你们也秀啊!」

「你觉得我这性格能干出那种事吗」

「……看样子不行」

「嗯哼」\\

周也对自己这淡泊的性格有所认识。

周的性格怕麻烦而且说话直来直去,有些人可能会觉得冷淡,因而给人的印象不算很好。这种性格根本没法找到女朋友。\\

万一真的有了女朋友,关系想必也会很平淡,至少不可能像树那样大庭广众狂撒狗粮。\\

「不是我说,周你至少该找个喜欢的人咯。话说啊,周你要是剪掉点刘海,弄清爽点,整整发型,背挺直了,女生们绝对会刮目相看的」\\

周自认为对自己有正确的评价,即便达不到门胁那种帅哥等级或是树那种稍显轻薄的端整外表,周也觉得自己的外表绝对谈不上丑。

要是周好好打理打理自己的仪表和形象的话,也是有不输同龄高中男生的水准的。\\

不过,周即使好好打扮,他也没有能耐对接近他的人献殷勤。\\

「光凭外表就来套近乎的可都不是什么好货色哦」

「说是这么说,可要是对方对你没兴趣,你也没法了解对方的性格吧?」

「……就算是那样,我现在也没找女朋友的心思」\\

就算找到了女朋友,看见周平常的样子肯定也会幻想破灭吧。

周这人生活不能自立,日子过得邋遢,而且对人还不友好。甚至周自己也苦笑道「不如说要是有女孩子对自己感兴趣我倒还真想看看」。\\

毕竟周嫌与人相处麻烦,性格上就不适合和人交往,因而并没有想要女朋友的想法。\\

而且,现在真昼在自家做着晚饭,万一交了女朋友说不定会酿成惨剧。虽然周完全没有找女朋友的打算,并不会对此感到不安,但是单从这个理由上来说,周也不会想去找一个。\\

周心目中的优先级是真昼的料理>还没找着的女朋友,而且这个优先级恐怕没法轻易改变吧。\\

「真是个没欲求的家伙……要不让小千给你介绍几个朋友也行哦?」

「你可别瞎操这闲心。千岁她朋友都是群吵闹的家伙吧,光是当朋友怕就够让我头疼的了」

「毕竟周你是个阴暗角色嘛」

「是啦咋地」

「嘛,你要这么说那暂且就算了吧。不过啊,美妙的高中生活,连女朋友都没有,一个人空虚度日,不难受么?」

「不需要,而且感觉很麻烦」\\

虽然周并没有「你把学校生活当什么了」这种较真的思考,但反正女朋友这东西不是非要不可,所以周也没有想着去找一个。

再说了,喜欢的人既不那么好找,也不容易产生结果。\\

「……可惜了啊」

「是是是」

「不过啊,周你要是有了喜欢的人一定会变的哦?」

「你哪来的自信啊」

「就是你这样的家伙,宠起女朋友才会不要不要的」

「好好好你说啥都对」\\

周既认为自己绝无可能变成那种甜的发腻的人,也想像不出自己变成那样的情况,于是把树的话当成耳旁风就这么吹过去了。\\

树一脸无奈地看着周……接着,他忽然移开了视线,表情也舒缓了下来。\\

「\ruby{阿树~}{\jpfont いっくーん},回家吧?」

「哦,小千啊」\\

正好,树的女朋友千岁过来了,两人似乎是约好了一起回家。刚刚周和树聊了这么久,都是在陪他等着千岁。\\

周回过头,便看见一位一头亮茶色短发,带着男孩子气的少女,正满脸笑容地朝着这边——准确来说,是朝着树招手。

那活泼的气氛和明快的笑容,甚至让看着的周感到有些耀眼。她的性格也正如外表,为人友善、活泼明快,好也好坏也好,她都负责着炒热气氛,是个与真昼风格不同的美少女。\\

她跑到这边来之后,露出了笑嘻嘻的表情。

周希望她能就那样别说话,因为,千岁一说起来基本上周都会被欺负。\\

「小千你说是不,周这样的家伙,肯定会宠女朋友的」

「别多嘴」

「唉?什么?周有女朋友!?」

「有个毛啊」

「哎,什么嘛。有的话我还想打好关系呢~」\\

「切」的一声,千岁瘪着嘴一脸失望。\\

「我那幻想的女朋友要遭一波你那美其名曰打好关系的过分身体接触的罪还真挺可怜」

「唉,原来你有虚拟女友吗?」

「我是说假如有的话好吧!?」

「玩笑啦玩笑」

「应付你可真够累人的……」

「只是周你体力不足吧」

「是体力连着精神力全被你消耗掉了啦……」\\

比起体力,感觉累的还是精神。\\

本来周平常过着的就是除了熟悉的人以外基本上不说话、不起眼、没精神的学生生活,要被迫跟千岁这种全天精神高涨的生物对话,实在是艰难。\\

即便周的回应有些刻薄,千岁也毫不在意,对着一脸疲劳的周说着「真是不像样呢」,十分愉快地笑着。\\

树也同样笑着给出了「你赶紧习惯啦」这样随意的建议,因此周除了累得叹口长气以外毫无办法。\\

\vspace{2\baselineskip}

「……在干什么呢?」\\

周回到家吃完真昼亲手做的晚饭之后,洗碗回来就看到真昼在客厅摊开了试卷。\\

洗碗这事是轮班,但周为了不给真昼添负担抢先去洗了,因而这段时间真昼便在客厅里待着。她说是因为如果就这样把事情全部扔给周自己回去,会有些过意不去。\\

「给卷子算分」

「嗯,看得出来」\\

大概是在检查答案,真昼似乎正对着课本确认有没有写错。\\

「话说结果怎么样」

「如果答题纸上我没有写错的话就是满分了呢」

「只能说不愧是你啊」\\

真昼满分的回答太过平淡,让周也没有什么太大的反应。\\

毕竟已经好多次在月考排名上看见真昼那雷打不动的年级第一,周也不吃惊了。\\

本来周就觉得真昼说不定能做到,因此他听见满分也只有果不其然之类的想法。\\

「学习我不讨厌啊。再说我已经提前一年把要学的东西全部学过一遍,所以只要复习就足够了」

「呜哇,太可怕了。不愧是学神……」

「藤宫你学习不也挺上心的么」

「你还知道我成绩啊」

「名次能上榜的话,我都有点印象」\\

看来在搭话之前她就已经在一定程度上知道周这么一个人了。

本以为排不到个位数的人根本就进不了她的眼,不过真昼却不假思索地说出了周上次的排名,看来她还挺关心成绩表的。\\

周会花上一定的功夫学习,其原因,并不在于「学习是学生的本分……」这种较真的脑回路,而只不过是家里给出的条件罢了。\\

「毕竟是让我独居的条件嘛,保持成绩这事」\\

家里同意周一个人住的时候,提出了要保证成绩不下滑的要求。

另外还有半年回家一次这个条件,不过关于这一条在放长假时回一趟就行,所以基本上只要保持住成绩家里就不会多指手画脚。\\

「我的成绩也就保持不会造成自己的麻烦的程度而已,比不上你。你是超努力的吧」

「……我的话,是因为不努力不行」\\

真昼轻声嘟哝了一句,低下了头。

虽然她的表情被刘海遮住而看不太清,但肯定不怎么开心吧。\\

不过,真昼很快便抬起头恢复了平常的表情,所以周就错过了指出这事的机会。

就算是没错过,周也不会去问吧。毕竟那氛围,就像是在忍耐着痛苦一般。\\

时不时地,真昼就会露出这样的表情。

虽然真昼从来不会说自己正因为什么而感到痛苦和厌恶,但她给人的印象便是被一些事物所束缚,挣扎于其中的样子。\\

不难想象,变成这样的原因是家庭环境。\\

因此,周来插嘴干预是不合适的。

周十分明白这是自己这个局外人不应踏入的区域,因而一直保持着作为邻居的适度距离感。\\

周同样有不想被他人提及的东西。\\

他也常常切身体会到,别人干涉私事是件很没礼貌的事情,反而是装做浑然不知时自己会比较感谢。\\

真昼隐藏起刚才的情绪,以平日里清爽的声音说道「我差不多也该告辞了」,接着开始把课本和试卷收进包里。\\

周也不打算挽留,「噢」地简单应着,望向收拾着东西的真昼。

正当真昼把拿出来的东西全部收拾好,从座位上站起来的时候,周突然注意到,在空杯子的阴影处,放着一件不属于周的东西。\\

周伸手拿起来,发现这是每个学生都有的装着学生证的塑料套。

估计是她连着课本一起拿了出来,整理的时候却忘记了吧。\\

周看着这印着正面照、姓名、学号、出生日期和血型这些简单信息的学生证,喊住了正在门口穿着鞋子打算回去的真昼。\\

「落下了哦」

「啊,抱歉让你特意送过来。那么,晚安」

「晚安」\\

真昼礼貌地弯腰行礼之后离开了周的屋子。周目送着她,轻轻地叹了一口气。\\

回忆起刚才看见的学生证上写着的出生年月日——特别是月和日的部分,周扶住了额。\\

「……这不就在四天后嘛」\\

要是周没看到学生证的话,恐怕他永远都不会知道真昼的生日。想着要是早些知道的话就好了,周再次深深地叹了一口气。

\subsection{天使大人想要的东西}

「话说,你有什么想要的东西没」\\

第二天,周心想事不宜迟,就在晚饭时试着朝真昼提起了这个话题。\\

说是生日礼物,但周其实也没有什么特别的意思,只是想着平常受了那么她多照顾,所以决定给她送礼物当作回礼。\\

不过,周的问法毫无疑问很可疑吧。

连周自己也感觉这问法既不委婉还很粗神经,开始有点后悔,而真昼却一脸诧异地看了过来。\\

「为什么突然问这个」

「总觉得你一副无欲无求的样子,有些好奇所以问问」

「还是很突然……」\\

虽然周自己也觉得还有更好一点的糊弄方式,但说出去的话已经没法收回来了。\\

不知算不算幸运,真昼看上去并没有注意到是生日的事情。

说到底,真昼肯定觉得周不可能知道自己的生日,所以根本就没往那边想吧。\\

「嗯,需要的东西啊……要说现在想要的」

「想要的?」

「磨刀石呢」

「……磨刀石?」\\

因为得到的回答完全超出了预想,所以周下意识地就追问了一句。\\

不如说,恐怕谁也料不到问女高中生想要的东西会得到这样的回答吧。\\

一般来说,她们渴望的应该都是化妆品啊装饰品啊包啊这类的东西才是。周实在是没法预料到,居然会在这里听到想要研磨金属用的工具。\\

「嗯。磨刀石。虽然我已经有几块了,但果然还是想要目数细一些的,结尾时用的磨刀石呢」

「喂我说现役女高中生」

「请不要在我身上要求普通的女高中生」\\

听她这么说,周一时语塞。\\

即使客气点说,真昼也称不上是个普通的女高中生。

光是被称作天使就已经可见一斑,她不但文武双全,甚至连做饭和家务都不在话下。\\

真昼这般那般照顾着生活邋遢的周,那个勤劳的样子都能让人以为是主妇了。虽说她嘴上有点不饶人就是。\\

(就算这样,谁能想到居然会是磨刀石啊)\\

想要的东西居然是磨刀石,这样的女高中生感觉也就真昼一个。\\

「……你不自己买吗」

「其实不是不能买。不过,基本用不大上,而且价格还不便宜,所以才没有买而已。再说其实我已经有目数比想要的粗一点点但也能完成收尾的磨刀石了,所以也并不是很必要吧」\\

真昼随意地说着自己手上有好几块的事情,真无法想象她将来会是什么样子。\\

「……自己磨菜刀的女高中生啊这实在是」

「其实这样的人也是有的哦」

「就算有,我认识的人里也就只有你一个,而且会想要磨刀石的也就只有你了」

「听起来很稀有,挺不错的嘛」

「到底哪不错了啊……」\\

因为实在太稀有,所以周完全没搞清楚她的喜好和想要的东西。\\

周已是黔驴技穷,而真昼则歪着脑袋,一脸的不可思议。\\

\vspace{2\baselineskip}

「我说树啊」\\

因为周对真昼想要哪方面的东西一无所知,只好出此下策,跑去问问树来当参考。\\

按周的预想,既然树有千岁这个女朋友,那么他也应该搞得明白女孩子的心思,像是一般女孩子喜欢的东西想必有个大概的把握。

虽说周不知道真昼算不算是普通,但他推测女孩子会喜欢的东西真昼应该也不至于讨厌。\\

「咋啦」

「树你给千岁送礼物的时候都送了些啥」\\

周想着从树给女朋友赠送的礼物开始问应该可以,便这么发问,可树却向他投以吃惊的眼神。\\

「哎,你小子对谁有意思所以想要送礼吗」

「你看我像会做那种事么」

「不像」

「那不就得了」

「那你干嘛问啊」

「认识的人过生日,参考下」\\

别说是参考,周甚至都想照着就去买了,不过周也没打算明说。\\

「哼~嗯。要说的话还是对方想要的东西最好啊。话说你平常就该调查下啊这事情,这可是关系美满的秘诀啊」

「不是说了不是女朋友么」\\

假想下真昼成了自己的女朋友,周就能感觉到很多危险(主要是身边的杀气)。再说这事本身也是癞蛤蟆想吃天鹅肉。\\

确实真昼在身边感觉很自在,但那只是两个无欲无求的人感到志同道合罢了,完全谈不上恋爱感情。

当然周觉得她很可爱,但并没有打算跟她发展到这样那样的关系。这才是周对真昼怀有的感情。\\

「想要的东西啊……要是不大清楚呢?」

「那就看关系了。要是关系好,送些饰品也不错,但如果关系没那么亲近还是送些小物件或者消耗品比较稳当。要是送花的话应该会开心,不过经常也有收到很难办的情况」

「……你还真了解啊」

「毕竟多少学习过」\\

树和千岁并不是一开始就相思相爱,好像是初中的时候开始慢慢拉近关系的。周和他们不是同一所初中,所以不清楚详细的情况,但据说两人是克服了不少难关最后才发展成交往的样子。到现在树秀恩爱的时候周仍然能听到这些事情。

给千岁送礼的时候,树似乎也烦恼了不少,所以看得出他给出的选项都是费了不少心思的。\\

「另外,护手霜应该也还不错」

「护手霜?」\\

听见意外的选项,周开始了思索,而树则一脸得意地笑着解释。\\

「不管哪个年龄层都用得到啦。学生的话上课天天碰课本手容易干,工作的人打字吹空调手干也是常事,家庭主妇的话手泡在水里干活也容易变粗糙。作为礼物总能派上用场」

「嗯……你怎么知道这么清楚好恶心」

「还不是你来问我的」\\

啪的一声,周的背被树拍了下,但由于只是个玩笑,两人便互相一笑置之。\\

(护手霜么)\\

确实,这东西的话应该不会给她添麻烦吧。

虽说晚饭后洗碗的工作周自发地全部包了,不过真昼在自己家肯定还会洗东西,难说手会不会变粗糙。

看她那滑嫩的双手,想必平常也都会保养,那样的话送给她这些护肤品应该不坏。\\

「行,我会参考的」

「对了一会你也去问问小千她吧。有些着眼点应该只有同性才有」

「……哎」

「好啦你差不多也该习惯啦」\\

尽管谈不上讨厌,但周应付不来千岁这样的人。想到要去找她,周就微妙地提不起劲,显得不太情愿,而树则是愉悦地笑着,轻轻拍了拍周的后背。

\subsection{天使大人与生日}

周向树和千岁寻求完建议后,总算选好了礼物,在真昼的生日当天以一副紧张的表情看着她的背影。\\

以车站前的可丽饼屋卖的特制可丽饼(冬日限定非常莓果特辑)为报酬,周说动了千岁帮自己忙买了个东西,并把这东西也加进了礼物……可现在周却苦恼着该在什么时候把这礼物送出去。\\

而那过生日的本人,正和往常一样做着晚饭。\\

虽然周不清楚菜单,不过真昼似乎是在做和食的样子,但怎么看她都没有什么特别的感觉,表现得跟平时一样自然。

从当事人身上完全感受不到生日的氛围。不如说那淡定程度,简直让人觉得她是不是根本就不记得这回事。\\

甚至到了晚饭端出来后也没有发生变化。两人在餐桌上虽有对话,但进餐还是一如往常。\\

周真的拿不定主意该什么时候把东西交给真昼,于是看向藏在沙发后面那放着礼物的纸袋,皱起了眉。\\

总之周先收拾好了餐桌。等他回到客厅的时候,发现真昼正坐在那刚好两人位的沙发上看着似乎是自己带来的书。

就连看书的模样也美如画作,到底是不虚天使之名。\\

虽说周对要不要坐在真昼旁边有些微妙的犹豫……但一直退缩也不是个办法,于是周提起放在那里的纸袋,坐到真昼身旁。\\

真昼突然抬起了头。

大概是注意到了周的气息和纸袋摩擦的声音,真昼那焦糖色的双眼看向了周,然后又移向了周拿着的纸袋。

真昼的表情似乎有些不解。看来,都到了这个地步,她还没有注意到自己生日的事情。\\

「嗯,给你的」\\

周把纸袋推出去放在了真昼膝上,使真昼脸上更加茫然。\\

「这是什么」

「今天不是你的生日吗」

「是倒是……话说为什么你会知道。我可不记得自己有跟谁说过这回事」\\

真昼的眼里微微露出警戒的态度,但听到周说「你上次把学生证落屋子里了吧」之后,或许是接受了这个解释,便恢复了平时的表情。\\

「其实,没必要在意的。反正我也不过生日」\\

那冷淡而透出排斥感的声音,应该不是周听错了吧。

真昼那眼神,如同对生日这词汇本身抱着忌讳感一样。\\

周明白了,原来如此。\\

明明是生日,她的态度却毫无变化,其原因,并非是不记得生日的事情。

因为生日很烦人所以故意忘掉的——应该说是这么回事。

若非如此,她也不会用那种语调吧。\\

「啊这样啊。那就当作是平常受你照顾的回礼吧。权当我一厢情愿想要报恩」\\

周以「你不过生日也没关系,但作为感谢平日照顾的回礼是另一回事。这就当作我表达感激的心意而不是生日礼物」这个说法把礼物塞了过去。

每天都吃着这么好吃的饭,偶尔还来帮忙打扫屋子,虽然都是小事,但也实在是受照顾了。即便只是一点一点的,周也想要回报真昼。\\

周虽然很轻易地就退让了,但却执意要把礼物送过来,这让真昼有些混乱。尽管她有些困扰的皱着眉,不过还是接过了礼物。

真昼的视线,集中在纸袋里面用另一层袋子包装的东西上面。\\

「我可以现在打开吗?」

「嗯」\\

看见周点头,真昼紧张地从纸袋里把盒子拿出来,小心地打开包装纸解开缎带。

看着别人在自己面前慢慢打开礼物,让周感到格外的紧张。\\

里面放着的是树推荐的护手霜。因为是套装一起卖的,所以这个大盒子里还附带着一点小点心。\\

顺带一提这并不是那种带有香味的时尚品,而是以没有香味、适合家务、亲和肌肤、滋润保湿为卖点的东西。

周也确认过网上的评价,效果应该是不用担心的。\\

「抱歉,不是什么值钱东西。看你干家务手应该会干吧。虽然也有带香味的,不过那种你估计有了所以没买。听说这东西对皮肤好而且挺有效的」

「实用品呢」

「你的话更看重实用性吧」

「是呢。谢谢你了」

看着真昼微微露出笑容,似是在说「你还挺了解我的嘛」,周也稍稍放松了嘴角。

看来印象不坏。\\

之后虽然还有一件东西……但要当面打开周还是觉得有些害羞,如果可以的话周还是想真昼回到家再发现那个东西。\\

可事不如愿,在把护手霜放回纸袋里的时候,真昼似乎注意到了纸袋里还有一件东西,于是眨了眨眼。\\

「……是还有一件东西吗?」

「啊。呃,那个,怎么说。就是个来自于独断和偏见的附赠」

「附赠?」

「……附赠」\\

周撇开视线,只回答了这么一句。真昼歪着头搞不明白周的意思,但她觉得不如直接打开来得快,便从纸袋里把那东西拿了出来。\\

为了让那东西尽可能不起眼,周用了跟纸袋一个颜色的包装,还将其塞在了最底下,但果然这个大小还是很显眼。不如说真亏能在打开护手霜的盒子之前都没让她发现。\\

那东西的包装并非盒子,而是聚酯塑料袋。其大小,正好够真昼双手抱住。

看着她把那深蓝色的丝带小心地解开,周想着「我要不要先离开一下」的时候——真昼正好把里面的东西取了出来。\\

她用两只手小心地把里面的东西提了起来,相当意外地眨巴着她那两颗大眼珠子。\\

「……熊?」\\

真昼说着的,便是那东西的原型。\\

那是一个不算太大,大概小学生抱着大小正适合的布偶。

布偶身上的软毛颜色很淡,与真昼的发色很相近。它的脸上透出天真的感觉,上面缝着一双乌黑、光亮、圆润的眼睛,眼睛里正映着真昼的身姿。\\

「都高中生了还玩偶啊」她说不定会这么想。\\

尽管如此,听了千岁「女孩子不论长到多大都会喜欢可爱的东西」的建议后,周就选择了这个。\\

再怎么说男的一个人跑去买这东西实在是非常害羞,周便以车站前的可丽饼为报酬让千岁陪着自己去买了。

结果从挑选到打包周一直在被千岁笑嘻嘻地看着,说不定其实一个人去买羞耻感还会少一点。\\

「……觉得女孩子会喜欢这个吧所以」\\

周挠着头,不知是在跟谁解释般嘟哝了一句。\\

这种事周实在是不擅长。

不如说给异性送礼这件事,除了小时候送妈妈的以外就没有干过,周甚至没有想到自己会去做这种事情。\\

从男的那里收到这么可爱的玩偶会不会让她受不了啊……周偷偷瞄了眼真昼,看到她正紧紧盯着熊的脸不放。

也不知是高兴还是不高兴,真昼只是呆呆地望着布偶熊。\\

「嗯,不喜欢的话扔了也行」\\

「如果不喜欢的话那也没办法」周想着,玩笑般地说了一句,结果真昼却皱着眉刷地把头扭了过来。\\

「那种事不会做的!」

「嗯、嗯。看椎名的性格我想应该也不会的」\\

真昼的否定比预想要强烈,令周一边退缩一边点了点头。而真昼则再次看向手中的熊布偶。\\

「……我不会做,那么过分的事情的。会好好珍惜」\\

真昼纤细的手腕,像是要将其拥入怀中般,紧紧抱着熊布偶。

那姿态看上去,既像是孩子不愿喜欢的玩具被拿走,又像是母亲慈爱的拥抱。

一句话来形容,便是极为珍重地抱着布偶。\\

仿佛能配以啾的音效一样,真昼紧紧把布偶抱在怀中,并稍稍垂下眼帘看着它。\\

那脸上的表情,既不是平常的那种冷淡的表情,也不是被周的脱线惊呆时的表情,而是心安、柔和、泛着慈爱的、爱惜的表情。

还有她那天真无邪的纯洁微笑,美丽又惹人怜爱,让周不禁屏息。\\

(——不该看的)

望见这样的表情,周不由自主地会对此产生意识。\\

让顶级的美少女露出了这样的表情,还被自己看见了,这一事实,就算没有恋爱上的喜欢,也足以让周心跳加速了。\\

真昼那珍惜地抱着布偶,露出淡淡微笑的姿态,已经可爱到不论谁看见了大概都会着迷的程度吧。就算是自认为无欲无求的周也差点入了迷。\\

为了确认自己脸上积蓄了多少热度,周伸手捂了下自己的脸。手上传来了比平时更加明显的热感。

由于自己害羞得太过明显,所以周用真昼听不到的声音骂了一句「……靠」。\\

幸好,真昼正紧紧抱着熊布偶,把半张脸埋进里面,因而并未注意到周。

那副模样也是一样地可爱,让周好不容易才忍住了发出怪声的冲动。\\

「……这么喜欢的话,我也就心满意足了」\\

周想着说些什么,于是挤出了这么一句,真昼则稍稍把眼睛露了出来。\\

「……我是第一次,收到这种东西」

「咦,以你的人气这算是日常贡品吧……」

「你把我当成什么了……」\\

这带有稍许无奈的声音与表情,反而让周安心了下来。这大概是因为不再需要直视那样的表情了吧。\\

「……我没有告诉过别人过我的生日。因为不喜欢生日,所以我从来都不说」\\

「不喜欢」真昼在断言之后将视线移向了布偶熊。

真昼看着布偶的眼神很安详,与嘴中的话语截然相反,不知为何却让周觉得不大自在。\\

「一般,不认识的,或者没什么关系的人送我礼物我也觉得可怕所以不会收」

「我送的倒是收了啊」

「……藤宫同学又不是不认识的人」\\

真昼小声地回答着,然后把脸埋进布偶里仰头看向了周。周则开始后悔自己直视了她这件事。\\

她那无意中向上看的眼神,还有放松下来的、与年龄相应的天真感流露出来的表情,实话说,相当令人怜爱。

那可爱令人不自觉地产生了想要摸摸头的冲动,于是周在不经意间把手伸向了真昼的头,然后慌忙用力收了回来。\\

「……怎么了吗」

「没、没什么」\\

不知是注意到了周一瞬间动了的手,还是察觉到了周那几近爆发的心痒感,真昼咚地歪了歪头。

仅是这样,周的视线便差点被夺走了。美少女这种生物还真是可怕。\\

但是再怎么说,直接回答因为可爱所以看呆了,周还是会感觉羞耻,而且就算说了周也确信真昼只可能回答「啊?」。\\

而且,如果那样说的话周在各种意义上都会死亡,所以还是决定把这个冲动深藏于心。\\

「……谢谢你,藤宫」\\

周撇开了脸,而真昼纤细的声音再一次传进了他的耳中。

\psline

※娇化输出 20\%

\subsection{友人的窥探}

「我说周啊,和送礼那位咋样了?」\\

东西是一起去买的,要说当然也是当然,第二天千岁就来嘿嘿地笑着窥探起周的八卦了。\\

在别的班级的千岁放学后跑到了周的班里来,这还没问题。可是这张笑脸周实在不想应付,巴不得现在就跟他们说句拜拜。\\

「既不是你想象的那种关系也绝对没有那种展开」\\

至少周并没有怀着恋爱感情,也不是因为有什么想法才送的礼物。

真昼收到礼物是很高兴没错,然而根本不存在千岁期待的那种展开。\\

「哎呀你想想,就没什么人能让你那么上心吧。这么看关系肯定不浅,还是女的,八卦八卦咋了啊」

「我们没什么见不得人的关系」\\

树也帮着千岁说话,周没办法只能矢口否认。

真昼开心是开心了,可还有这些麻烦事儿,所以周才想尽量不跟人商量的。\\

周可不愿意填饱这俩人的好奇心,所以回答得很冷淡。树把手撑在嘴边,好像在思索着什么一样。\\

「……嗯。我说啊周」

「咋了啊」

「你是送你邻居了?」\\

虽然树情商高、直觉准,可这种时候还真是觉得麻烦。\\

「……你为啥这么觉得」

「你活动范围里认识的人,还说是受了照顾,那只能是邻居了吧。想想你又不是当地人,又和女生没什么交流。最近人还给你饭吃了,感觉你就是感恩了不是」

「你说是就是咯」

「唔唔……我说周,感觉你最近脸色好得不行啊」

「啊,我也发现了」

「那人给你送饭是不是挺频繁的啊。所以你就送个生日礼物感谢一下?」\\

因为说得太准,周拼了命才稳住自己的表情。

一串推测准得简直就像在现场看到一样,搞得周有时都怕了。树虽然看上去轻浮,实际上很认真而且观察细致,其实还挺受欢迎的。不过真希望他这些优点能只对千岁发挥出来。\\

「你还真敢这么乱猜啊」

「又不知道真相,不就只能脑补了。所以,到底是啥情况?」

「你就猜吧」

「小气的家伙」

「小气——」

「啰嗦」\\

不管他们说什么,周都不准备老实交代。\\

万一说漏嘴了一点点,那最后要是不彻底交代清楚的话——树先不说,现役女高中生这种热爱八卦的生物是不会停止追问的吧。

因为世界上存在着这种没有恋爱都能硬扯上恋爱的神奇生物,所以麻烦麻烦真是麻烦。\\

「简直了」周叹了口气,收拾东西背起包准备回家。

这是战略性撤退,也是为了回避他们的烧心攻击。\\

「拜拜了,你们就甭管别人闲事秀你们的恩爱去吧」

「不用你说也会的哦?」

「……\ruby{阿树}{\jpfont いっくん},我们去跟踪看他和那女的见面吧……」

「哪有你那样在人面前说的,再说压根没你们想的那码子事,跟来也顶多跟到大门口」

「切」\\

虽然千岁嘟着嘴唇很可爱,但是眼神却一副认真样。\\

看千岁这样子,不开玩笑她真做得出来。周瑟瑟发抖地丢下这俩人快步离开了教室。\\

\vspace{2\baselineskip}

「……好危险」

「什么危险?」\\

周回到家不由得感叹了一句,接着真昼很好奇地问道。

现在这时间要做晚饭还太早,真昼买完菜来了周家里,所以两人正一起稍微休息着。周的自言自语好像是被真昼听到了。\\

顺带一说,今天的她和以前一样。

昨天那笑容是半点都见不到了。她这平时的表情让人怀疑昨天的事情是不是在做梦。

这样才是普通的,不如说周希望她这样。要是再让她摆出昨天那种表情,周是感觉自己心脏要疼的。\\

「啊,怎么说,就是礼物这事,让树他们八卦了」\\

周补上一句「因为之前找树他们商量的」,叹了口气。大概是记住了树的名字的真昼,像是全都明白了一样吐了口气说「啊原来如此」。\\

「嘛,毕竟是藤宫看上去就不会买的东西」

「虽然说不是这个意思啊」\\

周想送女性礼物,这件事本身似乎就让他们觉得不可能是周会做的事情,所以才会有恋爱云云的怀疑吧。

实际上,双方都没有感受到酸酸甜甜苦苦的这种伴随恋爱的味道和感情。\\

「是我这边的事情。真是的,他们瞎想个什么劲」\\

确实,真昼那么可爱,那时是有想摸摸的欲望。这一点周不否认。\\

然而周觉得是个青少年都会这样,说到底周只是再次体会到真昼是个超级美少女然后心跳了几下而已,哪可能是恋爱感情。

就算喜欢她的人格,周也没想过要和她成为这样那样的关系这种夸张的事情。\\

悄悄瞄上一眼,还是一如既往端整的美貌。

然而,并没有昨晚那样的悸动。周再次确认自己并不是喜欢上了她,轻轻叹了口气。\\

要是让真昼知道了周在看她,不知道她会说什么。于是周把视线移回到手机上,忽然看见聊天App的图标上已经攒了几个表示未读消息的数字。\\

心想着这大概是树吧,周打开App,结果新消息那儿的名字在周的预料之外。\\

看到志保子这个名字,周皱起了眉头。

这是周为数不多的三位女性联系人之一。

具体来说,就是千岁、真昼,还有——母亲。\\

有什么事啊,周想着打开了她的私聊界面。上面写着周不擅长对付的兴致高涨的文章,内容大概是考试怎么样、生活有没有什么困难之类的这种事情。\\

周不擅长应付千岁,就是因为家里人有个和千岁差不多的……不如说是感觉千岁年纪大了大概就会变成这样。尽管周不讨厌也恨不起来,但就算是亲生母亲,性格上也有些应付不来的。\\

『你爷爷寄水果来了,也给你分一点。礼拜六给你寄过去,那天下午你就呆在家里啊!要是拒收或者不在家的话饶不了你啊?』

「自说自话就把我日程给安排了……」\\

虽然这周六没什么特别的打算,倒是没什么问题,不过这种事情不该早点联系的吗。\\

「怎么了吗?」\\

自言自语似乎给听到了,真昼用平常的表情看着周这边。\\

「嗯,老妈说礼拜六下午要把爷爷给的水果寄来。大概是苹果之类的吧」

「你会削皮吗」

「……削皮器能削吗」

「削是能削……不过会削掉厚厚一层,有点浪费营养呢」\\

「这话像是咱老妈说出来的」这个感想还是咽到心里吧。\\

「大不了连皮啃就是了」

「真粗野啊」

「毕竟麻烦么」

「真懒啊」\\

真昼意见一如既往的直白,周只能露出苦笑,耸耸肩不管了。

真昼虽然一副无语的样子,不过还是认可地说了「嘛反正到了胃里都差不多了」。\\

「对了,不知道烂掉之前吃不吃得完,椎名你要一点么?」

「那就要吧。毕竟水果那么贵\footnote{日本的水果是真的贵。}」\\

真昼说的事情有些成家之后为生活奔波的感觉,不过要说的话她一直就是这副样子吧。\\

「周六是吧,我那天就先做点午饭顺便当做回礼了」

「明明是我一直受照顾啊」

「没事,反正给你做饭我也不讨厌」\\

真昼轻轻地微笑了。

她的微笑让周想起了昨天的事情,周有些尴尬地错开眼睛,简单回了句「……那就拜托了」。

\subsection{安息之地敌人来袭}

接过礼物之后立刻亲手送上的打算或许是个错误。\\

听到门铃声和「\ruby{周——}{\jpfont あーまね}」这充满俏皮的高声时,周就掌握了所有情况并抱住了头。\\

\vspace{2\baselineskip}

真昼周末来做午饭的提议原本让周求之不得感激涕零。周原本还以为这是上天的恩惠。

事实上,她做的培根意面也很好吃。浓酱和黑胡椒的刺激相得益彰,美味得不得了。\\

并不是真昼有什么过错。是的,不是真昼有什么过错。\\

有错的是被千叮咛万嘱咐要呆在家,结果还没注意到这事的自己——以及这位超爱惊喜,会做出奇葩行为的,和自己有血缘关系的女性。\\

「……那个,藤宫?不是快递……」

「不是。这是老妈拿着钥匙穿过大门直达了……」\\

回想起来,错误还是在于把这千方百计想来视察的母亲说的话当真了。

那母亲,不搞点事那是不可能的。\\

「……诶,你母亲?」

「咱老妈估计是来看我日子有没有好好过吧……不事先说好是因为怕我装模作样」

「哦……」

「你这副赞同的样子让我心情很复杂啊不过现在这不重要」\\

问题是现在在这里的真昼该怎么办。

要是老妈在大门外,叫真昼立刻回家就好。然而,既然已经到家门口了,便没法让真昼回家了。

话虽如此,啥都不做就把母亲领进门的话,碰上真昼肯定会发生莫须有的误会吧。真昼肯定也不期望这样。\\

就在自己烦恼着如何是好的时候,门铃声的间隔越来越短了。\\

(——啊真是的)

「……抱歉啊椎名,先到我房间去一下吧。拜托了」

「诶,嗯,嗯?」

「这个你拿着,我想办法把老妈支到外面去,之后你就回家吧。真的抱歉不过拜托了」\\

真的是迫不得已,周选择了隐蔽的方针。\\

虽然她做了午饭,不过已经收拾干净了,这一点没有问题。

鞋子藏鞋柜里就发现不了,她带到家里来的毯子之类的私物让她拿进房间里就好。\\

真昼在房间里这段时间,只要周在母亲大致粗查一遍之后求着做饭吃,母亲应该是会答应的。房间的视察周是准备全力拒绝来应付过去的。

故意要求冰箱里的东西做不出来的东西,然后一起去买菜,在这期间让真昼逃离——这是预定的计划。\\

周告诉真昼没有其他办法了,递给她多出来的的钥匙并认真恳求后,真昼虽然困惑着还是嗯地点头同意了。\\

另外,不用储藏室是因为,现在这个季节没空调还是很冷的。

周的房间有空调还有软软的座垫,这样就不会搞得她坐在空空荡荡的地板上腰疼身子冷了吧。\\

「……那就拜托你了。我现在去应付老妈……」\\

面还没见,周已经很扫兴了。周到玄关的时候,真昼也静静走进了周的房间。\\

确认真昼进去之后,周不情不愿地开了门。\\

「哎呀周可真是慢。精神就好,还以为你睡着呢」\\

很快映入眼帘的是暑假以来就没见过的母亲。

明明是自己母亲,容貌上却没体现出年龄,还挂着家里一直见到的那副笑嘻嘻的表情。体现不出年龄的不只是外表,还有行为也是一样。\\

「行了行了精神着呢您就回去吧?」

「你就这么对你妈的吗……花了几小时才过来的,连点慰劳都没吗?」

「远道而来诚为感激,请回吧」

「还说这种话啊。这么不可爱,和修斗一点都不像呢」

「男人要什么可爱啊」\\

虽然周发出了呕吐的声音,母亲——志保子并没有心情不好的样子,呵呵地笑着就当成是叛逆期接受了。\\

「那我进来了?」

「等等,没让人进吧」

「这边可是拿我和修斗的钱租的?」\\

「我说啊妈,要来就先说一声。都这么大了」

「哎呀,要是不突然袭击的话,不就看不到儿子有没有好好过日子了么?」

「唔……你看没问题吧,都收拾好了」

「还真是,吓着了。周你在家啥都不会,其实意外地挺能干嘛。没想到啊」\\

志保子到了客厅环视了一圈,好像赞赏着一样感慨地点着头。\\

当然,收拾好是多亏了和真昼的共同作业,能保持清洁时多亏了真昼的建议和提醒。以上基本都是真昼的功劳,然而现在这没法和志保子说。\\

「皮肤也挺好的,营养也有好好摄取呢」

「……嗯」\\

周稍微移开了点视线,因为这也是托了真昼的福。\\

「菜也有好好做呢……咦,看上去是两人份的?」\\

涂着指甲油的手指指着餐具的部分。

午饭是两个人吃的,盘子当然也是两人份的。这里是周粗心没注意到,不过志保子眼神也真是好。\\

「因为朋友来了」\\

周并没有说谎。

虽然周不那么确定,不过两人已经建立了接近于朋友关系的交情,应该不算有错吧。虽然说性别没讲出来。\\

周尽量憋着动摇淡淡地回答着说。志保子回了一声「哦~」好像没接受这套说法一样,又把目光放回了客厅。

总算是勉强糊弄过去了,然而差点都要出冷汗了。\\

「……嘛,合格了……简直好得不像是一个男生自己住嘛」\\

志保子观察了一阵,重复了几次质疑回复之后,阐述了总评。

某种意义上是理所当然的吧,大部分事情都有真昼掺和进来。\\

「没啥妈要担心的事情吧」

「是啊,真是吓了一跳。家里你还啥都不会的,看来是成长了啊」

「……我也是会成长的」\\

周心里自嘲着「哪来的脸这么说」回答之后,志保子也笑嘻嘻地称赞了「你也努力了呢」。

因为不是自己的功劳,心里默默地难受。\\

然而实情是不可能说出来的,周只能忍着求她回去。\\

生活检查姑且算是做完了吧。

或许不用求做饭也能回去了——周甚至产生了这样的想法。\\

「最后就是检查房间了呢」\\

然而志保子最后投下的炸弹让周不禁瞪大了眼睛。\\

检查房间,也就是私室……寝室的检查。

里面当然有真昼。要是被发现的话,容易预测得到下场比当初设想的会面还要惨。\\

「喂搞啥啊,自己房间就算是妈也不能进啊」

「哎哟,是有什么见不得人的东西么」

「正常来说男高中生的房间总有一两个见不得人的东西吧」

「还承认了啊」

「是啊承认了所以别进来」\\

这里必须尽全力阻止。就算面子丢了,真昼的存在也必须隐藏到底。

现在,要是周房间里的真昼给看到了,志保子毫无疑问会以自己方便的形式朝着愉快的方向想入非非。无论如何都得避免。\\

就好像固执地不让通行一样,周挡在门和志保子中间说着NO拒绝着。志保子很快就看出这是藏着什么东西,说着「有秘密不告诉妈了,你还挺像个样子嘛——」笑嘻嘻地逼近过来。\\

要是到了关键时刻,哪怕有些内疚,周也是抱着不惜来硬的也要拒绝的打算在和志保子对峙的。\\

然而,房间里发出了哐镗的一声。\\

「周」

「嗯」

「藏着什么啊」

「……和妈没关系」

「这样啊,懂了懂了」\\

嘿嘿的笑容变得更浓了。

这种笑带着不准拒绝的压力。每次看到这种笑容,周都非常不适并且反抗的精力会减掉不少。

这已经成了习惯,改不掉了。\\

趁着周支吾时的破绽,志保子把手摆到了门把手上。\\

现在再后悔已经为时已晚了。\\

为了确认这声声响,志保子穿过周的旁边打开了房间门。\\

门前看到的是——背靠床边,膝盖上抱着垫子的美少女。

而且还眼睛闭着,重复着规律的呼吸……简单来说,看到的就是正在打盹的真昼。

\subsection{天使大人被气势压倒}

打盹这事本身是常有的。

身处于开着空调的暖和房间,又是刚吃饱了午饭的时候,光这两条对打盹来说已经是充足的环境了。

虽然周涌现出「正常来说会在男人房间睡着吗」的疑问,不过真昼姑且是把周认定成了无害生物,说不定是不小心睡着了吧。\\

这也怪不得真昼。不出声傻傻待着也挺无聊,而且总有些事是没办法的。\\

周抱头烦恼的原因,是真昼在母亲志保子过来的时候遭到目击,而且还是在这个状态下。\\

百分之百会被误会的。

要是站在别人的角度,周自己也肯定会误会,觉得两人关系已经好到能进房间还大意到打盹的程度。\\

周脸上抽着筋瞄了母亲一眼,发现她看着真昼的眼神十分灿烂。周还听到了「哎呀哎呀可以可以」这样的心理活动,大概是错觉吧。\\

「哎呀真是的,找了个这么可爱的女朋友!周还真是不能小瞧啊!」\\

志保子「呀」地发出了一道不符合这年龄的高亢声音,让周的头开始疼了起来。\\

她不但完全误会了,而且还进入了兴奋状态。

就算是儿子带了女朋友来,一般也不会那么高兴的吧。

然而现在志保子就是这么高兴,理由肯定是因为她喜欢可爱的东西没错了。\\

确实,真昼有一副任何人都会承认是美少女的外表。

她睡着时没有防备,平时的假面也取了下来,更重要的是,表情和动作无法遮掩的容颜清晰可见。

无比端正的那副容颜,现在正处于安详而放松的状态。\\

虽然周已经见得习惯了,然而每次看见真昼时都还是觉得她美貌极品、非常迷人。\\

真昼她那天真无邪的睡脸没有防备,可爱到让人不禁想去摸摸。

抱着周的垫子睡得香甜的那个样子,强烈勾引起周不太想大大方方说出来的那类欲望。\\

像那样的,连已经看惯的周都赏识的美少女,在志保子眼里是儿子的女朋友(暂定)。

恐怕是这点让她情不自禁兴奋起来了吧。\\

「莫非不让妈进来是因为女朋友在里面?不知不觉你也成了个男的啊」

「才不是咧!从头到尾都不是!既不是女朋友也不是什么别的!」

「哎,不用找借口的哦?只要你挑的,妈都不反对」

「哎所以说不是这个问题!不是交往关系啦!压根就不是!」

「说啥不是的,房间都进来了哇」

「还不是您老人家来的这么突然啊!就算只在客厅您不也得误会嘛!」

「最根本的问题是,周要是没这意思的话,根本不会把女孩子领进家里来,女孩子没这意思的话,也不会跑到对方家里去的哦?」\\

被志保子这么一说,周使劲思考着反驳的论据却难以觅得。\\

正如她所说,周基本上把家当成自己的领域,不怎么愿意让别人进来。

虽然一开始让真昼进来是因为没拗过她那架势,不过在那之后,即使不考虑做菜这事,说到底也是因为周中意真昼的性格才会像这样把她放进家里来。\\

(要说喜欢的话,确实是喜欢没错啦)\\

对周而言,就算不考虑外表的因素,也是挺喜欢真昼这个少女的。\\

她有着学校里不表现出的,辛辣、耿直,同时却不坦率的矛盾性格;她看似冷淡无情实则爱照顾人;她说话一针见血;她被出其不意的时候会慌乱地露出与年龄相符的样子;她极为偶尔地会露出天真无邪的笑容。如今,周已经觉得以上这些全部都是真昼的魅力了。\\

尽管这样的感情算不上是恋爱,但至少周认为她是很有魅力的少女。\\

「虽然我作为朋友很喜欢她,但可别把对异性的喜欢全当成恋爱了。再说,这家伙也没这意思」\\

他们的关系并没有甜蜜到能让周老实赞同志保子的说法。说到底,要是真昼被误会成对周有意思,她也会不乐意吧。\\

「这可说不准哦?你才是,该不会觉得自己能理解女孩子复杂的心情,就有些自以为是了?」

「要怎么说才明白我们不是这种关系啊……椎名,求求你起来吧……」\\

就算千言万语都已说尽,志保子也是不停把话题引向恋爱的方向。周只能烦恼地按住额头。

周希望真昼能快点起来,认真的。\\

「嗯……」\\

或许周的许愿起了效,又或者是她被吵醒了。\\

真昼缓缓抬起合着的眼皮子,发出甜甜的声音抬起脸来。\\

亚麻色的头发顺着肩膀滑落下去。

焦糖色的眼睛朦朦胧胧水汪汪的,那副样子毫无防备,甚至让周不好意思直视。\\

可能是意识还微妙地没有完全觉醒,真昼睡眼惺忪地仰视着周,让周微微错开了视线。\\

「椎名,睡着这事先不说,现在我被误会了,来帮忙解释一下」

「误会……?」

「我说咱家女朋友啊,你名字叫什么?」\\

真昼思考着那句话的意思,显得还是软绵绵的。志保子则毫不客气地靠近过去,露出了老好人一样笑嘻嘻的表情。

面对这无忧无虑的笑容和亲善的眼神,真昼好像还在刚醒来的混乱中,肉眼可见地惊慌失措着。\\

「呃,那,那个」

「初次见面的时候,互相报出名字是很重要的呢!」

「呃,椎,椎名真昼……」

「哎呀小真昼,名字好可爱呀!我叫志保子,别客气直接用名字喊我就行」\\

真昼在气势所迫之下不禁报上名字,然后往周那边看了过去,好像在说着「藤宫,救救我」似的。然而周自己还巴不得别人来救他,因为实在是帮不上忙所以摇头拒绝了。\\

因为是自家的老妈,所以周很清楚,她一旦失控起来就停不下来的。\\

看她对真昼兴致勃勃的样子,大概是想和真昼做一次彻底的交流吧。

也不知道她有没有注意到真昼这个最要紧的人正在困惑着。\\

「那,那个,母亲\footnote{母亲:日语原文为 {\jpfont お母様},是一个可以称呼对方母亲的敬称。}」

「噢!已经认我当妈了啊!」

「藤宫!」

「我和周都姓藤宫,对吧周」

「妈啊椎名为难着呢」

「周,女朋友不用名字喊可不行哦?」\\

因为志保子实在不听人说话,周皱起了眉头,但志保子却没有介意的样子。看她这嘿嘿笑着的样子,该说她是胆子大呢还是脸皮厚呢。\\

「那,那个,志保子」

「什~么?」

「我,我和藤宫——」

「你说哪个藤宫?」

「……周、周君不是那种关系」\\

听到志保子这做作的话,真昼明显一副狼狈样但还是努力试图否定着。

因为志保子的催逼,真昼犹犹豫豫地喊了周的名字,并窥视了周那边几眼。志保子则由于成功让真昼喊出名字而露出了满脸的笑容。\\

「噢,那就是今后会变成这种关系吗」

「呃,那、那个,不是」

「哎呀我真是的,是不是当电灯泡了」

「那,那个,让我好好说清楚!我和周、君,不是那种关系,只是在一起吃饭,那个,就是因为周君不会做饭」

「你能当个好老婆呢,小真昼。咱家这周啊家务活不会干还得一个人过日子。如果是那样的话务必要支持他啊」

「啊,那个」\\

周觉得,真昼已经尽力了。

然而,要顶住志保子这势头把事情说清楚,这种事情大概是做不到的吧。\\

定期来家里,亲手下厨做菜,一起在桌前吃饭,听到这些的时候,志保子的眼神变得更灿烂有气势了。

事已至此,周是阻止不了志保子了。能阻止的大概只有父亲修斗了吧。\\

「……椎名,放弃吧。我妈一兴奋就不听人说话的」

「怎么这样……」\\

周已经到达大彻大悟的领域。他早早放弃了解释,选择默默看着失控的母亲。

\subsection{天使大人感到羡慕}

「话说周还真找得到这么漂亮一女朋友啊,妈都吓到了」\\

两人同时闭口不言。周是因为疲于否定,而真昼则是由于不知所措。

志保子将这一沉默视为肯定——不如说是不管两人怎么说都会被她视为掩饰难为情的肯定——以毫不掩饰好奇的眼神盯着真昼。\\

「怎么样,小真昼你看周有在好好过日子吗?」

「嗯……那个……至少死不了人吧……」

「你倒是说点好话啊」

「可是一开始那时候房间那么脏」

「要不要那么严格,现在有保持干净吧」

「那不是因为我有帮忙打扫吗」

「那个,嗯,我是真的很感谢你啦,饭啊打扫啊什么的」\\

在这些方面周对真昼是抬不起头的。

正因为有她在,周才得到了现在舒适的生活,即使要下跪谢恩,周也会毫不犹豫地去做。但由于真昼不喜欢,所以周并不会真的这么做,不过周是有打算在平日里尽可能努力以慰劳真昼的。\\

只是,志保子把这段发言往不太好的方向理解了。\\

「我说周,不只是这次,平时也一直让真昼帮忙啊,真是个让人头疼的孩子……听那个说法,难道你们是在同居吗?」

「不是!怎么样才能想成那样啊!只是住在隔壁啦!」

「哎呀,那就是命运的邂逅呢!真好啊周,能有个这么漂亮能干的姑娘照顾你」

「漂亮能干是没错啦但是命运的邂逅什么的我有意见」

「多浪漫啊不挺好么」

「我说的不是这意思!是说我们根本就没在交往!」

「哎哟哎哟」\\

志保子肯定以为周是在难为情,而周的脸则是真的快要抽筋了。\\

母亲总是将事情一厢情愿地解释成能作为自己美好妄想的食粮的那种东西,而不知被这样的母亲烦恼了多少次的儿子,发出了这几个月以来最沉重的叹息。\\

至于被这惊人的气势压倒的真昼,则是交替看着周和志保子,明显一副不知所措的样子。\\

「小真昼小真昼,虽说这可能是咱做家长的偏袒自家孩子,周啊虽然嘴巴毒而且不坦率,不过他其实很诚实很绅士的,你可以当作买到了件好货哦。但因为他还没有女性经验,这一方面就需要小真昼好好把握了」

「说啥呢赶紧闭嘴吧妈」\\

后半部分的补充相当多此一举。\\

「我说的不都是事实嘛。倒是周之前为什么不找个女朋友啊,明明长得和修斗一样外表也还不错的。还是因为太土了吗」

「要你多管」

「给小真昼看看你帅的一面如何?」

「不给而且这家伙也没想看」

「又来了又来了。啊,要不小真昼把他打扮成你喜欢的样子?只要打扮起来的话周也挺好看的」\\

真昼看着志保子笑嘻嘻地推销着周,或许是因为束手无策,所以露出了个含糊的笑容。

能让那个沉着冷静的天使大人畏缩到如此地步,在某种意义上,志保子或许非常厉害。\\

「妈,椎名她真的很困扰啦。话说你赶快回去吧」

「让母亲回去,你还真是长大了啊。不过打扰了你和女朋友甜蜜时间也是事实,我差不多也该告辞了吧」

「真的赶紧回去吧」\\

周已经疲于坚决否定,而真昼想必也被志保子这兴致搞得很心累吧。\\

往真昼那边一看,她似乎也稍微有些疲倦。

这也是当然的。她基本上是个文静的姑娘,却被迫参与到了这情绪高涨得连亲儿子都会感觉到累的对话当中来了。\\

周在心里做下稍后慰劳她的决定,同时对着志保子往门外甩着手。志保子则回了一个微微不满意的表情。

即便如此,志保子也没说要留下,应该是姑且顾虑到了这边的意思。虽说这顾虑明显是在错误的方向上。\\

「啊,小真昼也交换下联系方式吧。咱家周的生活方式之类的各种事情,之后都跟我说说」

「哎,好、好的……?」\\

在最后,志保子还建立起了这让周想要求饶的联系,使周捂住了自己的额头。\\

真昼无可奈何地顺着这势头用手机交换了联系方式。

毫无疑问,这样一来志保子也会开始多管起真昼的闲事吧。\\

(真的抱歉了)\\

看着志保子满面笑容地握住真昼的手嘱咐着「周就拜托你了」,周便决定稍后给父亲发条消息说「求求你把妈控制一下吧」。\\

\vspace{2\baselineskip}

「好累啊……」

「抱歉来了阵台风」\\

尽管志保子的滞留时间不长,但两个人已经精疲力尽,正并排坐在沙发上。

周坐得很沉,捂着脸,长叹了一口气。真昼虽然坐得有些拘谨,但平时挺得直直的背也比往常要弯曲。\\

「真的抱歉,误会没解开就让妈回去了」

「没事,毕竟没有什么损失……」

「啊损失还是有的……看我妈那样子是中意起椎名了……估计之后会管你不少闲事……」\\

在这一点上,周由于给真昼添了麻烦,所以真的很对不起她。

首先是儿子的女朋友(误会),再加上志保子喜欢可爱的东西,恐怕志保子对真昼中意得不得了,会想好好地关照她吧——甚至会到多管闲事的等级。\\

「志保子真的很重视藤宫呢」

「说好听点是这样,说难听点就是缠人……」\\

虽然这和溺爱还是不一样的,然而志保子对周的疼爱也并非周的所愿。

因为也有周太邋遢的过错,所以周并不太好提太多意见,但即使如此周也觉得她管得太多了。\\

周对母亲很感恩也很重视,但坦率来讲,他同时也觉得很麻烦并且希望保持距离。\\

「……真好啊」\\

周看着低声细语的真昼。\\

「哪里好了」

「你母亲,虽然这么热闹但是挺温柔的」

「那该叫又吵又过度干涉吧」

「……就算那样也好啊」\\

真昼不是在客套,而是真的露出了羡慕的表情。她用几乎要消失的轻轻淡淡的声音嘟哝了一句,然后垂下了眼帘。

一望而知,她的表情忧郁而昏暗。仿佛一碰就会崩坏的那副样子,无论谁看到都会觉得柔弱。\\

真昼表露出的柔弱与虚幻,看上去绝不仅仅只有疲劳。这样的她似乎感受到了周的视线,忽然抬起头轻轻地微笑了。

真昼恢复了往常的表情,仿佛在说什么事都没有一样,然后罕见地把身子靠在了沙发的靠背上。\\

「小真昼,啊」

「……怎么了啊,突然来这么一句」

「不是……只是感觉好久没有人叫我名字了。一般都是叫姓的」\\

没人用名字称呼那个人气爆棚的天使大人,令周感到有些意外,不过这应该是周围人都觉得用名字称呼真昼太过惶恐而不好意思吧。

因为她在学校是天衣无缝的天使大人,周围人不敢那么随便叫她。

还有,用外号称呼她的人倒是不少。虽然她本人讨厌得要死。\\

「要是没有好朋友的话,也就爸妈会叫了吧」

「爸妈才不会呢,绝对的」\\

真昼以冷淡的声音做出了秒答。\\

周不由得往真昼的脸看过去,却发现她的表情上没有任何的颜色。

她面无表情,仿佛一切都脱落下来,甚至可以视作无机物一样。或许缘于其端正的美貌,周甚至误以为眼前是个人偶。\\

然而这也只是一瞬。真昼注意到周的视线后,收起了这张无表情的脸,眉毛低了几分,像是有什么困扰一样。\\

「……总之,就是很少见了」\\

小声嘟哝了一句后,真昼轻轻吐了一口气。\\

周早就已经看出来,真昼和父母相处得不好。\\

触及父母的话题时,真昼偶尔会露出冰冷的表情。「没和父母出去吃过饭」、「讨厌生日」,从这些发言就很容易想象她的家庭环境有问题——然而,周哪里能想象出,父母连她的名字都不会喊呢。\\

『……真好啊』\\

方才的那句细语,究竟是以何种心情编织而出的呢。\\

「真昼」\\

自然地,周说出了他未曾叫过的名字。\\

真昼焦糖色的眼睛啪地眨了一下。

或许是因为出其不意,真昼像是在发呆一样,显露出了隐藏于平时的态度和表情里的某种稚嫩。用茫然自失这个说法来形容她,大概很贴切吧。\\

「叫个名字谁都可以吧」

「……说的也是」\\

周生硬地补了一句之后,隔了一会儿,真昼露出了淡淡的笑容。

那微微安心的笑容,在周的心里泛起了涟漪。\\

「……周君」\\

自己的名字被小声叫了出来,让周心里的涟漪更大了。\\

到刚才为止,或许是因为真昼只有面对志保子才这么叫,所以周没怎么挂在心上……然而,像现在这样,被真昼面对面叫出名字之后,周胸中就翻滚起了一些痒痒的、让人心焦的东西。\\

「在外面请不要这么叫」

「……这种事情知道的啦。倒是你别在外面说漏嘴了啊」

「知道的。这是秘密嘛」\\

周无法直视一脸微笑的真昼。\\

于是,周简单地回了一句「噢」,假装要改变姿势而往旁边看去,逃开了她的笑脸。

\subsection{钥匙的去向}

周六母亲的突然来袭之后,周和真昼互相的称呼发生了改变,但除此以外并没有什么大的变化。

两人并没有关系突然变好。不过,随着称呼变得亲密了一些,真昼的态度也多少软化了一点。\\

「……那个,周君」\\

周日傍晚,真昼比平常来得更早,而她的脸上的表情带着微妙的尴尬——或者说是困扰。\\

虽然让她进来了,但真昼这不明不白的态度还是让周一脸困惑。

周想过真昼是不是对直呼名字有所抗拒,不过她叫周名字的时候并没有犹豫,所以原因大概是另有所在吧。\\

总之两人先坐在了沙发上。周看向真昼,发现她从裙子的口袋里掏出一块手帕。\\

正当周想着突然怎么了的时候,真昼将叠得仔细整齐的手帕展开,把在其中包着的亚光色的钥匙拿给周看。

周对这把钥匙有印象,因为这就是昨天交给她的那一把。\\

「……钥匙,还给你。昨晚结果还是没有还成。那个,不小心忘记了错过了还你的机会……很抱歉,这样」

「这样啊」\\

看起来她是由于就这么把钥匙拿回了家,所以心里感到过意不去了。\\

周搞明白真昼这奇怪的样子是怎么回事之后,看向放在手帕上的钥匙。\\

仔细想想,真昼差不多是每天都会来这边做晚饭。虽然周会去给开门,不过有时周绕了路所以不在家,有时手上忙着停不下来,就需要让真昼在门外等一小会儿。\\

现在这个天气让人站在门口等着,对女性来说是不是太苛刻了呢。

听说身体受凉是女性的大敌,而且设身处地想想,周自己杵在门外也不怎么舒服。\\

反正真昼差不多每天都会来,那让她拿着钥匙,对周来说也会更轻松些吧。\\

「虽说你就这么拿着也行呐」

「哎?」

「等什么时候我们没关系了再还我就行」\\

直说的话,周要是把钥匙给了真昼,那就表示要受她一段时间的照顾了,然而真昼却不安地看着没有接下钥匙的周。\\

「但、但是」

「不如说每次都跑去开门好麻烦」

「真话说漏嘴了哦」

「反正你也不会乱用咯」

「话是这么说……」\\

再怎么说,从她那得到晚饭和来自己家做饭也已经有一个多月了,周自认为也算是理解了真昼的人品。\\

真昼她首先拥有常识和健全的思想,性格上就压根做不来坏事。

就算她拿了钥匙,想必也不会把钥匙给别人或是趁着周不在的时候跑来偷偷做些什么吧。她是可以相信的人。\\

「每次都按完门铃等着,你也麻烦吧」

「就算那么说,感觉你太没有警戒心了」

「我是因为信任你才给你钥匙的来着」\\

听到周这句话,真昼瞪大了眼睛,又一时语塞般皱起了眉。

真昼的表情上浮现出困惑,以及另一种不知是什么的感情。\\

不过周这边把钥匙给真昼只是为了省点事,要是真昼不愿意的话周是准备老老实实让步的。\\

而真昼则是来回看了钥匙和周好一会儿——然后轻轻叹了口气。\\

「……那好吧。我暂时借下了」

「嗯」

「……周君真是的,都搞不明白你是心大还是粗心了」\\

「真是的」,真昼无奈地以带刺的声音往周扎了过去,周便只得苦笑了。\\

「这才是我的风格嘛」

「这种话才不是自己说自己的时候用的」\\

哼~地,真昼以冷淡的声音指正,结果却反而让周笑得更灿烂了。\\

看起来真昼已经熟悉起了周,以至于能进行这种没营养的对话了。

不过,真昼都允许了周称呼她名字,要说还没熟悉倒是反而有些不可思议。\\

虽然真昼看着周的眼神饱含无奈,似是在说着「拿这人真是没办法啊」,但那眼神与其说是冷淡,不如说是带着一丝丝温暖。

真昼也明白周那是在插科打诨吧。\\

「那我就不客气地用了啊。你家要是出了什么事情我可不管哦」

「比如说?」

「……趁你不注意打扫屋子吓你一跳?」

「那还真是感激不尽啊」

「做好一堆吃的把冰箱撑满?」

「然后早饭也有得享受,晚饭菜色也会增加咯」\\

真昼的恶作剧实在是和平——不如说是喜闻乐见、求之不得。不过真昼却由于轻描淡写地被带过去而有些微妙的不满。\\

威胁没有威胁的样子,这如实地反映出了真昼的善良,实在是令人欣慰。\\

「总觉得被当成笨蛋了」

「我可没这么干」\\

看来再笑下去真昼就真要闹别扭了。虽然周也想看看真昼闹别扭的样子,不过他还是收起了笑容默默地看着真昼。

\subsection{天使大人与奖励}

望着走廊里贴着的写着许多学生名字的纸,周轻声叹息「嘛也就这么回事吧」。\\

上周考试的名次已经出来,于是周便和同年级的同学们一样过来看了。

要论结果的话,第21名,和往常差不多,看着还行但不怎么显眼。写题的时候手感上跟以往没有什么变化,现在看到排名一如既往,周也稍稍安心了下来。\\

顺带一提,真昼依旧雄踞年级第一之位。\\

尽管她真的是个才女,但周也深知她也没有欠缺努力,只得叹服不愧是她。

周也经常能看见她晚饭后学习的身影。\\

虽然也有原本脑子好使的原因,但要论将她送上第一宝座的,果然还是坚持不懈的努力吧。\\

「椎名同学又是第一啊……」

「不愧是天使大人,脑子实在好用啊」\\

听见混在喧嚣中的这般论调,周不禁撇起了嘴。\\

「咋啦周,一脸不爽的。排名不妙?」\\

和周一起的树看见周的样子,有点惊讶。

顺带一提名单只有前50名,所以树不是来看自己的排名,只是陪着周过来看。\\

「没啥。21名」

「哦,这不是比上次还好了点嘛」

「差不多吧。这点只是误差」

「哎呀哎呀聪明人说的话感觉就是不一样」\\

树做作地边笑边挖苦着周,而周「好好好」随意地应付过去,再次看向排名表。\\

周觉得,真昼真的有好好努力过。

虽然她不怎么愿意把努力给人看见,但对在暗处默默努力的她来说,即便别人看上去理所当然,但这也是她付出了莫大的努力才得到的成果吧。\\

即便周围的人会夸奖她「真厉害」,但这些人却对她付出的努力一无所知,因而也不会犒劳她的努力。

对真昼来说,这应该让她非常苦闷吧。\\

「……至少,我来补足下吧」

「嗯?你刚说了啥」

「没啥。喂,我回教室咯」

「好嘞~」\\

\vspace{2\baselineskip}

「咦,周君,这是什么?」\\

真昼看上去是从超市回来直接就进了周家里。她正打算把食材放进冰箱,结果注意到了这多出来的白盒子。\\

「嗯?啊,是蛋糕」\\

白盒子里面放着的是蛋糕。估计看见盒子的形状真昼多少也料到了,只是姑且问问周确认一下吧。

顺带一提,千岁经常在社交网站上发自己喜欢的糕点店,周就是去那儿买来的。\\

「……你喜欢吃蛋糕吗?」

「倒也不算。这是给你买的」

「怎么又来」

「你不是考了年级第一嘛,稍微庆祝下咯。年级第一,祝贺祝贺」\\

听见是给自己买的的时候,真昼眨了眨眼。

看来是真的很出乎意料吧。\\

「其、其实每次都考第一,并没有什么好庆祝的」

「就算是那样,你也一直在努力,偶尔来些奖励不也挺好的嘛。草莓奶油蛋糕不喜欢吗?」

「哎?倒、倒也不是不喜欢……」

「嗯,那就好。吃完饭来吃咯」\\

即便察觉到真昼吃了一惊说不出话的气氛,但周还是就这样结束了会话。\\

要是太过顾虑真昼反而会让她陷入困惑,所以态度还是干脆一点为好。\\

在周看来,真昼这个人在对待他人上算是很尽心尽力的类型,但对待自己来则是十分地严格,没有什么大事就不会让自己放松。

要是没有谁来表扬、犒劳一下她,那她就会一头扎进要做的事情里而不知休息。周甚至觉得,她是不是基本上不知道撒娇这一行为。\\

虽然周和她的相处不算太久,但多少也搞清楚了她的性格。一直受她照顾的恩情,周希望这样算是报答了一点。\\

周看着仍呆呆僵在厨房里的真昼,苦笑着轻轻地叹了一口气,在她重启之前一直都看着她。\\

\vspace{2\baselineskip}

饭后,看着真昼以微微紧张的神色把蛋糕放在盘子上端过来,周不禁笑漏了嘴。\\

「为、为什么要笑啊」

「没、没什么」

「感觉就不像是什么也没有」

「别在意」\\

周不过是看着真昼动作发硬的怪样子感觉有些有趣,仅此而已。

但要是笑得太过了会坏了真昼的心情,那原本犒劳她的目的就达不成了,于是周笑得差不多就停了下来。\\

真昼顺带把咖啡也拿了过来,和蛋糕一起放在桌上,坐在了周的旁边。

这些动作也微妙地显得不自然,令周想要发笑,但毕竟真昼本人就在旁边,周还是忍住了。\\

真昼畏畏缩缩地朝上瞄了一眼周。\\

「嗯,祝贺祝贺」

「……谢谢你。不过……」

「好啦好啦你就乖乖收下吧。毕竟你也确实努力过了咯」

「虽然,是这么回事」

「那就快吃吧。偶尔也让自己放松下嘛」\\

「反正已经买了给你了」,周这么补了一句,真昼才以略带抱歉的神情微微点头,拿起了叉子和盛着蛋糕的盘子。\\

「感激不尽」

「请吧」\\

周轻轻地摆摆手,真昼则拿起叉子,以慎重的动作将蛋糕切成一口的大小送进嘴里。\\

虽然女孩子会给人挑剔甜食的印象,但既然是千岁都常吃的店那应该就没问题吧。\\

其佐证就是,真昼尝了一口,稍稍睁大了眼,然后微微放松了嘴角。

虽然真昼很少有表情变化,不过最近她也开始变得慢慢会流露出易懂的喜怒哀乐了。\\

真昼慢慢吃着蛋糕,同时脸上浮现出了柔和的表情,让这不过是吃东西的场景变得好似一幅画作。\\

「……?怎么了吗」

「不,没什么」\\

真昼突然发现周凝视着自己,不解地歪起了头。

那与平常相比稍显幼稚的表情,令周方才还在盯着看的视线不自觉地迷离起来。\\

取而代之,真昼则是开始凝视着周。她仿佛是突然想起了什么一般,用叉子叉起一口蛋糕,朝着周这边伸了过来。\\

她变成了所谓的喂食的姿势。\\

「咦,我、我不是想要吃,是说」

「不是吗?」

「……呃,啊,那个……要是送过来了,我还是会要的」\\

这样的场景周实在是没有想过,周显而易见地狼狈着,最后一个不小心同意了下来。\\

毕竟已经是这个年纪,更何况对方还是异性,再加之要被不得了的美少女喂食,某种意义上说不定算是幸运——然而周还没有把自己的羞耻心舍弃到了能老老实实为之高兴的地步。\\

「本来就是周君买来的东西,周君你也有吃的权力」\\

而提案的真昼似乎完全没有意识到这些,仍以平常的表情把蛋糕放在周的嘴边。\\

周就算看向真昼,也只看得到她不解的样子,周便下定决心一口咬下了蛋糕。\\

在嘴里泛开的,是无比甘甜的味道。\\

「……好甜」

「毕竟是蛋糕嘛」\\

显然并不只是蛋糕甜的原因,但真昼估计没有注意到吧。

周即使小口咀嚼,也是觉得甜得不得了,应该是精神状态产生了相当大的影响。\\

「……看来是什么都没感觉到啊」\\

周这边可是甜味害羞味心痒味全部尝了个遍,真昼那边却一脸没事人样。\\

这实在是令人不甘,于是周说了一句「稍微给我下」,从真昼手上抢过叉子,以同样的方式叉起蛋糕伸向真昼。

有借有还,被搞了怎能不还手。\\

「嗯」

「……那个」

「吃了」\\

或许是因为周语气有些强硬的原因,真昼怯生生地,像是被以同样的方式喂食的小鸟一般,一口吃下了蛋糕。

周死死盯着真昼的脸,看见了那脸上微微泛起的红晕。\\

「于是乎,感想如何」

「很、很好吃……」

「不是这个,是问的被喂的感受。如何」

「……感觉非常害羞」

「是吧。所以说,对别人做这种事可是要被误解的啊。要做的话女生之间做做算了」\\

周说着「这下明白了我的感受了吧」,呼地别过头去,而真昼以几乎要消失的声音回答了一声「嗯……」。\\

应该是把周当作无害的人看待,真昼才会做出那种事情吧。

真昼这样无意识地做出这种事情让周很困扰,不过周感觉也不算坏,因而也没有什么好怪罪的。

只不过,那甘甜的滋味依旧在口中回荡。\\

(太过不设防我也很难办啊)\\

被真昼信任本身是挺令人高兴,但那样不自觉而毫无防备地做出这种事实在是让人够呛。\\

周得出这样的结论之后,看着旁边的真昼微微害羞地缩着身子,轻轻地叹了口气。

\subsection{圣诞的过法}

「我说周,在你家开个圣诞party可以吗?」

「不可以」\\

周一口回绝掉唐突的提议之后,千岁的脸蛋显而易见地鼓了起来。\\

平安夜就快要来临了……对与家人分开并且孤单一个人的周来说,这个节日跟他并没有多少关系。不过千岁和树似乎是想和周一起过,才这样来发出邀请的吧。\\

千岁午休时间特意赶来周和树的教室做出了如上提议,而她现在正因为周的秒回而鼓着脸蛋。\\

「有啥不好的啦反正周你也是一个人……啊,莫非是女朋友」

「没有没有不存在的」

「那有什么问题啦。还是你不愿意?」

「周要是不愿意的话那就算了吧」\\

他们这样做,也是以自己的方式为朋友着想着吧。

虽说,估计也有个原因是他们想要一个能自由自在秀恩爱的地方。\\

他们露出这样抱歉的表情周自己也过意不去,而且周也并不是讨厌开圣诞party。

周之所以不情不愿的,是因为在私人场所看到他们那激烈得不同寻常的身体接触很羞耻,另外还有要跟真昼说明得花上一些功夫。\\

要是说得极端点的话,只要事先跟真昼说好在他们离开之前不要到家里来,再把平时真昼存在的痕迹消灭就好了。\\

「也不是不愿意啦……行了行了,24号对吧?估计天黑之前就会解散的,那之后你们再去两个人你侬我侬亲亲热热。千万别在我家撒太多狗粮啊」\\

周也不至于非要拒绝,于是就答应了,然后千岁的脸上就变得笑嘻嘻的。\\

「没法子,就这样凑合吧」

「你以为你谁啊」\\

因为千岁嘴上说得有些嚣张,周就没什么顾忌地捏了她的脸。然后千岁有些口齿不清地说着「疼疼——阿树~周欺负我~」开始求救了。\\

「喂,周你别欺负小千啊?她的脸只有我能捏」

「行行你就帮我好好捏捏她吧」

「交给我吧」

「别啊~!」\\

周想着这应该也能给他们个借口亲热,便把捏脸的机会让给了树。不出所料两个人果然开始捏脸嬉闹起来了。

被捏的千岁实在是一脸高兴的被捏样儿,周看着这场面耸了耸肩。\\

「……我能回去吗?」\\

当然说是回去这也是自己的教室,不过在被秀一脸之前周还是想要和他们保持点距离。\\

「不可以~。不好好做安排可不行。蛋糕和午饭都得准备!」

「我可不会做啊」\\

再怎么说周也做不来圣诞用的午饭。

真昼的话估计普普通通就能做出来,然而没办法拜托她帮忙吧。\\

周连连摆手表达自己做不来,不知为何千岁则是凝视着周。\\

「咋了啊」

「就是感觉你不会做饭怎么变得这么健康啊」

「这种事情怎么都无所谓吧」

「算了算了小千,周也有他的情况吧」

「哎~,阿树你不也挺想知道的」

「他说之后会告诉我的」

「我没说过」\\

周使劲瞪了一眼树告诉他别自说自话做下约定,树则好像是故意似的放声大笑了起来。

不会缠着不放是他的优点,不过偶尔会像灵光一闪一样开人玩笑也是他不好的地方。\\

「真是的……不过,饭叫个外卖就好了吧。蛋糕倒是得先预约」\\

先不管树对自己的窥探,周提出了一个现实的建议。

要说当然也是当然,周自己又做不出蛋糕,而且饭菜也做不来,那么自然就该准备现成的东西吧。\\

「啊,那我要吃披萨~!蛋糕我去老地方预约了啊,现在应该还能预约」

「圣诞不该吃烤鸡吗」

「阿树不也更喜欢披萨嘛」

「那倒是。还是小千了解我」

「哎嘿嘿~」\\

虽然他们自说自话就决定吃披萨了,不过周自己也并不讨厌披萨,而且披萨也挺有party气氛的,所以周觉得也没什么不好。

照这节奏,午饭估计就定下来是周和树经常点的那家披萨宅急送了吧。\\

听到披萨,周忽然想起了真昼。

真昼像小动物一样啊呜啊呜嚼着吃披萨的样子,让周觉得有种神奇的可爱,这应该是因为周平时看到的一直是她优雅吃饭的样子吧。\\

想起前日还喂蛋糕给真昼吃,周感觉脸上自然就带了点热意。\\

(以后再也不做那种事了)\\

互相喂食这么羞耻的事是再也做不出来了吧。他们又不是树和千岁那样的亲热情侣,应该也不会再有机会了。\\

「……周,怎么了?」

「啊,没事没事。那预约蛋糕就交给你了」\\

一瞬间,周想起一些事情而有点恍神。千岁对此感到诧异,担心地探出身子看向周那边。于是,周连忙把那件事情赶出脑子,恢复到了平日里的面孔。\\

「好嘞~!披萨我也去约啦~!」\\

听着千岁兴高采烈的声音,周决定回到家之后向真昼询问圣诞节的安排。

\subsection{天使大人的圣诞约定}

「圣诞节的安排吗?应该没有吧」\\

周洗完碗,向坐在沙发上的真昼询问之后,得到了十分干脆的回答。\\

本以为八成会有个女生聚会什么的,但好像真昼并没有这种预定。

或许周的意外显露在了脸上,真昼看到周之后露出了一丝傻眼的表情。\\

「基本上,那些和我有交情的女同学大多是有男朋友的。男性的话就算来邀请我也会拒绝,所以不管怎么说都是没有安排的」

「男的听了都要哭了」\\

外出时的真昼防御极为坚固,那些怀着淡淡的期待邀请真昼的男生们在这坚固的防御面前也只得默默忍着眼泪吧。

周是觉得,真亏那帮人干得出邀请真昼这事啊。要是对自己没有足够的自信,邀请天使大人简直是想都不敢想。周甚至佩服起那些性格阳光的家伙们来了。\\

「……他们就这么想和我一起过吗」

「幸运的话可以拉近距离嘛」

「为了什么?」

「要说的话,想要交往?」

「为什么想要交往呢」

「……还不是想交往之后做些这个那个的嘛」

「动机不纯呢」\\

周在心里对被干脆地抛弃了的诸位男生合掌致意,接着进行了补充说明。\\

「嗯,不过应该也不全是那样的家伙啦,别太疑神疑鬼啊。你的话应该能明白男生看向你的眼神的类型吧」

「也是呢,倒也并不全都是非礼的目光。比如周君你的就不是对吧?」

「那我倒真没干过呐」\\

可爱啊想摸摸头啊这种程度的事倒是经常浮现在周的脑子里,不过并不至于这个那个想入非非的程度。

再说要是想了那种事情,真昼就会早早察觉到然后疏远自己吧。\\

因为周是无害的男人,所以才得以坐在她身边。要是哪怕露出了一点凶相,她就会立刻离开自己。\\

周没有特别想要女朋友的愿望,而且食欲这边对他还更加重要,因而周也没有打算让现在的关系崩溃掉。\\

「也是呢。周君的话看起来一开始就对我没什么兴趣」

「算是啦」

「所以说,可以信任」

「那可是感激不尽」\\

虽然这信赖的方式感觉对男人来说有些不太好接受,不过对「安全的男性」这一立场周暂时没有什么不满。\\

「……于是乎,问完了我安排的周君的安排如何?」

「嗯?啊啊,我的话二十四号白天树他们会过来我这,虽然预定不变,但晚饭可能会晚一些,所以想着先跟你提前说下」\\

总算是又绕回了正题,周再次进行了解释,而真昼则是明白了一般点了点头。\\

「我知道了。那就等圣诞party结束之后叫我,我再过来做饭。在那之前我会做好准备的」

「哦,抱歉了」

「没事,祝你过得开心」

「……不觉得寂寞吗?」

「一个人,已经习惯了」\\

听到真昼一副没事的样子做出的回答,周有一点心里不好受。

或许真昼的脑内突然闪过了父母的事情,使她的脸上浮现出略带几分自嘲的苦笑。\\

「……啊,那个」

「怎么了?」

「……虽然是个十分冒昧的请求,就算圣诞前夜不行,圣诞节那天可不可以一起,这样」\\

不知为何,提出这样一个提议让周十分害羞。

虽然周并没有什么特别的意思,但一般来说,一起过圣诞节的请求,往往是有着特别的意义的。\\

周真的,没有别的意思。

只不过是,不愿看见真昼那一副寂寞的表情低着头的样子罢了。\\

听见这一提议,真昼眨了眨眼。\\

「在一起,是要做什么吗?」

「嗯?啊,也没什么事好做啊。抱歉」\\

被指出那一点,周也不好再坚持邀请了。\\

要是考虑被他人看见所造成的麻烦,一起出门就压根没有可能。

那就只剩在待在家里了,但这个家里也几乎没有可以引起真昼兴趣的东西吧。\\

这么一来只剩下两人在一起什么都不做这一个选项,但那样一来气氛怕是会变得十分尴尬。\\

要是这样的话,还不如两人各过各的还更好——周如此考虑着,正准备撤回自己刚刚的提议,真昼却静静地注视着周。\\

「……那,我想试试,那个」\\

真昼挺起劲的样子,令周颇感意外。\\

真昼纤细的手指,指向了电视的方向。

准确来说,是电视柜里放着的游戏机吧。

最近晚上的时候都有真昼在,因而周也没怎么开过机,可真昼却似是对它感兴趣,小声说着「那个东西,我还没有玩过……」表达出自己的期望。\\

一对并没有在交往的男女玩着游戏过圣诞节,这事听上去甚至有种超现实的感觉。\\

「啊,嗯,也不是不行……可以吗?玩游戏什么的」

「不行吗?」

「也不是说不行啦」

「那,这样就好」

「哦、哦」\\

周虽然想着「就这样就好么……」,但既然这是真昼的愿望,周还是决定尽自己所能将其实现。

周想要给真昼一些小小的乐趣。不过反正周圣诞节也没有什么特别的安排,就算只是能和真昼一起吃饭也算是赚到了吧。\\

「嗯,也不管什么圣诞节了,悠闲地过就好了咯」

「是呢」\\

轻轻一笑的真昼不知为何,让周感到难以直视。周点了点头之后,若无其事地别开了脸。

\subsection{天使大人与并非本意的相遇}

「Merry Christmas!」\\

然后,到了圣诞当日。

学校已经放了寒假,在这想必大家都按着自己的过法度过的日子,树和千岁抱着行李聚在了周的家里。\\

时间大约是下午一点。

外卖点的披萨和果汁已经摆在桌上了。拖到这个点,论原因还是圣诞节订单太多,就算预约了也无济于事,送达的点还是迟了。\\

但是这个点吃午饭也不算太晚,况且两人也都是过了晌午才过来,没等太久,所以都不怎么在意。\\

「好好好merrychristmas」

「周说的好应付!再说一遍!」

「Merry Christmas」

「虽然发音很标准但还是觉得好应付哦?」\\

周希望千岁不要把自己和原本就活泼喧闹的她相提并论。\\

树注意到周即使这样也已经算是提着兴致了,于是一边哄着千岁,一边露出了平常那爽朗而略带轻浮的笑容。\\

「算啦算啦。这种事随便了啦,总之吃饱了玩,玩够了睡咯」

「别睡我家,蠢货」

「开玩笑啦。要睡也是去小千家了啦」

「记得要趁父母不在啊」

「哎~,周你是不是在想什么色色的事情~?」\\

周无视掉一脸坏笑的千岁,走向厨房去取餐具和茶杯。\\

千岁虽然撇着个嘴一脸不悦,但还是说着「我来帮忙~」跟在了周的后面。\\

不必说,厨房自然是被整理的干干净净。毕竟这已经算是真昼的领土了,自然各种道具和调料也都按她的方便排得整整齐齐。\\

「干净得简直让人意外呢」

「谢夸」\\

周随意应付着千岁,从餐具柜上取下分披萨用的小盘子和杯子,递给千岁其中的一半左右,却发现千岁正盯着餐具柜看。\\

「……咋了啊」

「没啥~?」\\

看见千岁咧开嘴的笑脸,周有种仿佛是要纠缠上来的感觉,所以背上一阵发抖,不过还是决定坚持无视千岁。

虽然周注意到了千岁心里似乎产生了一个不得了的误解般的什么,但因为她没有说出口,周也没法确定具体是什么。\\

看到千岁心情变得稍微愉快了一些,周面部抽搐着返回了树待着的客厅里。\\

\vspace{2\baselineskip}

「不过嘛~这屋子可真整洁啊。那么大真奢侈」\\

听着屋里音响传出的圣诞风的音乐,千岁差不多吃完了午饭,然后歇息了一阵子,环视了一圈这只有三个人的客厅小声说道。\\

客厅那么宽敞是多亏了租下这里的父母,而整洁则是仰赖真昼过来帮忙打扫,因而周也没什么可以评论的,只好答了一句「谢了」。\\

「嘛~有一段时间可是超乱的~。还真亏你弄得干净啊」

「啰嗦」

「嗯嗯,有股女人的味道呢~」

「你是怎么才会这么想」\\

屋子变得干净,怎么就跟有女性扯上关系了,周一点点都没法理解。\\

「嗯~?大概是直觉吧。总感觉打扫屋子的方法跟周的性格对不上。还有就是书摆放的方法、电线的走法、不让东西损伤摆在一起这些~。另外还有一些餐具不符合周的喜好吧~」

「……那是我妈的」

「呼~嗯?」\\

周姑且是把那些东西都放在了最里面,不过拿餐具的时候似乎还是被千岁看到了。

因为光是周这有的餐具还是不够,真昼便从自己家里带来了几样,但周没有料到平常往好里说坏里说都那么粗神经的千岁会注意到这样的细节。\\

「也无所谓就是了~?你说是吧~阿树~」\\

周微妙地迟疑了一下,而千岁意味深长地看着他,然后笑嘻嘻地靠在了树身上。

似乎这已经是常有的事,树并不怎么惊讶,而是把手伸向千岁让她坐在自己膝间,接着顺势就抱住了她。这样的姿势,对周来说非常难以直视。\\

「在别人家不要这么秀」

「羡慕啦~?」

「那倒没」\\

与其说是羡慕,不如说是感觉待不下去,所以周希望他们能收敛下。然而,考虑到对他们来说这才是正常模式,恐怕那句劝告也无济于事吧。\\

跟树黏在一起,看上去心情大好的千岁把身体靠在树胸前,仰头望着天花板和树的脸。\\

「……现在大家是不是都在这样亲亲热热呢」

「别忘了还有那些正留着血泪的家伙们啊」\\

大家都是这样,这种事完全是天方夜谭吧。肯定也有人会和家里人,或是和朋友一起度过今天。一个人过的人想必也同样存在。

视单身为屈辱的人可是要多少有多少,千岁这发言要是公开出来恐怕有些危险。\\

「男生都那么想要恋人的吗?」

「应该是吧。我倒没有特别想要就是」

「我倒觉得那是因为周本身就是个怪人呐」

「闭嘴」

「圣诞节前大家都是一副坐立不安的样子呢。特别是单身男生。话说最近一堆人涌去天使大人那里想要邀请她一起过圣诞,结果被全部干脆拒绝了,尸体都快堆成座山了耶。据说是已经跟人约好了,所以不行」

「咦,是嘛」\\

周觉得,那个约好的人可能就是自己。

虽然周有种被当成了表面上好听的拒绝理由这样的感觉,不过考虑到真昼拒绝他人给内心带来的罪恶感,真昼再怎么用这借口周也不会介意。反正也不会蹦出自己的名字,应该没有问题。\\

「那个时候男生露出的绝望表情可真是。虽然很失礼,但我还是笑出来了」

「你别笑啊」

「毕竟啊,平时明明没什么关系,一到圣诞节突然就想要耍帅一起过那当然是不可能的咯?没有提前建立好关系就已经迟了一步了,而且『虽然交往不深但让我们一起过圣诞节加深关系吧』这么好的事情哪可能有啊。还有,就是那种人啊,才会一边说着大家一起开party一边找机会两人独处,在女孩子看来很吓人的」\\

「真昼哪是连那种人的邀请都会答应的便宜女人啊~」,千岁吐着舌头说完,然后或许是她想起了这事不太开心,便和树蹭到了一起。

虽然和真昼方向不同,但千岁也是个美女,所以也有各种各样的烦恼吧。想到受欢迎的女人会为人际关系所烦恼,周就有点同情起了千岁。\\

「有那么多邀请,椎名她也挺不好过啊」

「……周你真是对天使大人没兴趣啊」

「嗯啊」

「对周来说邻居才是天使大人呢」

「赶你出去嘞」

「别嘛」\\

「真是烦人」,周稍微狠瞪了千岁一眼,千岁就以滑稽的姿势说着「好可怕~」并抱住了树。\\

「不过倒是没有否认被邻居关照了呢」\\

周一时语塞,而千岁则心满意足地笑了出来。\\

「别瞪人家啦~抱歉抱歉」\\

由于千岁的道歉没什么反省的语气,周就又瞪了她一眼。接着,千岁发出可爱的「咿呀~」声抱紧了树……然后看向了树身后的窗子。\\

发现千岁看着那边愣住了,周心想着「有什么事」也跟着把视线移了过去。映入眼帘的,是在那蓝天下轻轻飘落的一抹抹白色。\\

「……啊,阿树快看!下雪了!」

「哦哦,看来是个白色圣诞节啊」\\

现在已是十二月下旬,就算下雪也并不稀奇。

有太阳时下雪是有些少见,不过对恋人们来说反而是值得高兴的吧。

虽然还没到晚上,但从气温来看雪恐怕是要一直飘到夜里,这么一来无疑会是个银装素裹的平安夜。\\

这下情侣们可要兴奋啦——周边想着,边默默看着身边的这对情侣打开窗户跑出阳台。「反正他俩要在外面亲热一阵子,我去准备点暖和东西吧」,周抱着这样的想法刚一起身——就听见了阳台上传来了不同寻常的声音。\\

「哎?为、为什么在这」

「呃、哎?」

「啊」\\

最后听见的那一声,是最近周已逐渐耳熟的,令人感到甘甜而清爽的声音。\\

一股不好的预感猛烈地涌了上来。\\

周察觉到阳台上的两人愣住了的气息,连忙跑了过去,恰好见到——似乎是正好走到阳台上看雪的真昼,隔着栏杆与那两人相遇了。

\subsection{天使大人与困惑}

太糟糕了。周看着旁边姿势端正地坐着的真昼,叹了一口气。\\

迎来阳台遭遇这一惨剧之后,他没有办法,只能把真昼请来家里了。

反正就算试图糊弄,这两个人也毫无疑问会胡思乱想。所以干脆老实说出来,还更能防止一些多余的臆测和误会吧。\\

并且,不好好封上他们的嘴的话,之后的事情会很恐怖。\\

「……那个,真的很对不起……」

「……并不是你的错……」\\

虽然真昼以饱含歉意的细小声音道歉了,但唯独这件事上真昼并没有错。

因为今天是白色圣诞节,而且这还是今年的初雪,所以真昼就不由自主地去阳台上看雪景了吧。\\

如果周听见了打开窗户的声音恐怕会阻止两人,不过由于房间里放着音乐,所以实际上他并没有听见。

而且,真昼也有尽量注意不发出声音吧。周是完全没有注意到。\\

看着相互反省的两个人,千岁两眼放光,把脸猛凑了上来。\\

「原来,周的邻居是天使大人吗!?」

「那个,天使这个称呼可不可以……」\\

真昼似乎再怎么说也不愿意被当面称为天使大人,于是委婉地拒绝着,然而千岁却一副笑嘻嘻的表情,根本不知道有没有在听。\\

至于树,他一边挠着脸交替看着周和真昼,一边微皱着眉头。\\

「嗯嗯。那么……根据目前为止的信息来推断,椎名住在周的隔壁,经常给周做饭,我说的对吗?」

「……嗯」

「算、算是吧……那个,因为藤宫有恩于我,而且看藤宫那个样子就知道吃得不健康,所以很在意……」\\

真昼淡淡地说明了两人开始交流的契机,并且解释了为什么交情会持续下去后,树虽然嘴上说着「原来如此」,但表情上似乎还是微妙地无法接受。\\

如果站在树的立场,周估计也没法接受吧——周这样的普通男生,竟然能得到真昼这样优秀的女性来照顾他。\\

「嗯,我算是知道怎么回事了。不过不管怎么说,这个状况下椎名你对周没有其他意思才是很不可思议啊。这都快算是走婚妻\footnote{走婚妻:原文为 {\jpfont 通い妻},意思是平时不同居,有需要时才到丈夫家里的妻子。}了吧」

「噗」\\

听到了这平时完全没听过的单词,周不禁笑了出来。\\

走婚妻——被这么一说,只看状况的话或许是挺像的。周每天的晚饭真昼都有在做,假期里也时而有她的午饭吃,并且偶尔还来帮忙打扫。听上去说不定确实挺像这么一回事的。

区别就在于,两边互相都没有带着爱情吧。\\

听到树这么一说,真昼虽然稍稍睁大了眼睛,不过很快就转变成了对外用的笑容坚决否定道「没有这个打算也不可能」。

想着真昼对树和千岁是用和在学校同样的方式相处,周心里便觉得有些痒痒的。\\

「我也没有什么不好的想法,所以椎名才会来给我帮忙的吧」

「周这么说倒是没问题啦。不过,还真是奇怪的组合……那个才女给周做饭啊……布偶也是送给椎名的?」

「……算是吧」

「哦~」

「好烦」

「我还什么都没说呢?」

「光是脸就好烦」

「好过分!」\\

千岁笑嘻嘻的……不如说是坏笑的表情,让周本就烦躁的心非常不舒服。\\

目前还只是确认事实,所以千岁没怎么捉弄周,但是他可不愿意受她捉弄。因为对真昼也会有影响,可以的话他希望能无视千岁。\\

「我说,两个人都冷静一下」\\

树一开始就注意到了周的样子有变,所以没有千岁那样调戏周的样子。

树会在周真的不高兴之前停手,是个会察言观色、能为别人顾虑的男人。可以的话真希望他能在窥探之前就停下,不过这事还是没办法的吧。\\

周微妙地瞪着眼,而树劝诫了因为谜团解开而满心欢喜的千岁之后,不知为何端正了姿势,连同整个身体朝向真昼低下了头。\\

「……那个,椎名,咱家的周受你照顾了」

「我啥时候成你家孩子了」

「彼此彼此,谢谢你能为了藤宫和他这么要好」

「别顺着说下去啊,搞得我好像废人一样」

「确实挺废的」

「你这家伙」\\

确实,树也一直说着周,周自己也有所认知……不过被指出来还是让周十分心情复杂的。\\

真昼似乎懂得这种玩笑,抓住了机会故意装了个傻之后,看着周和树的对话嘻嘻地微笑着。\\

虽然这笑容不至于到只给周看到的真实面孔那种程度,不过也不完全是对外的装模作样,这让树也露出了有些愣住的表情。

周捅了捅树表示「有女朋友的人别看呆了」,接着不开心的千岁也同样……不,是更用力一些地捅了捅树,让周觉得莫名有趣。\\

只不过,看到真昼有些疑惑地把脑袋稍微歪了过去,周便当作无事发生般恢复了原本的姿势。\\

「……所以说,虽然我们并不是你们那么甜蜜的关系,不过要是给别的家伙知道了肯定会引来麻烦事的,这你总明白吧」

「知道知道,不会跟别人说的」\\

周这是暗中威胁说「要是告诉别人了会怎么样你懂的吧」,不过树很轻易就答应了,这让周感到意外。\\

「千岁你也是」

「我也没那么多嘴啦~。而且,这么可爱的家伙给周做饭什么的,说出去估计也没人相信」

「配不上还真是抱歉了」

「我没说到这个地步啦~」\\

千岁说的并没有错,而且周也对自己有认识。

普通的男生,正在得到学园偶像级别的天使大人照料,这种话谁都不会相信的吧。

就算有人相信,肯定也会骂周不配。\\

这样的事情周也不是料想不到,所以他才不想让周围人知道这个事实。麻烦事可还是算了吧。\\

「真是有够低声下气的」千岁笑看着周,不过她的视线忽然像是被吸过去一样转移到了真昼身上。\\

千岁先是用热情的眼神注视着,然后紧接着叹了口气,又注视了上去。

真昼也一副不太舒服的样子,看上去有点不知所措。\\

「那个,怎么了?」

「……我又一次觉得,椎名怎么这么可爱的啊」

「咦?谢谢……?」\\

千岁正面夸奖了真昼,然后目不转睛地盯着真昼的容颜。\\

「这么近看还是第一次,果然漂亮得能说是天使大人呢。相貌端正皮肤白皙长得漂亮睫毛又长头发顺滑身体苗条还凹凸有致」

「那、那个……?」\\

千岁的老毛病似乎又犯了,周大大地叹了口气。\\

周不擅长与千岁相处。

周并不讨厌千岁,他还挺欣赏千岁的人格……然而无论如何都有应付不来的地方。比如容易高涨的情绪,比如偶尔会太过关心别人的事情,这些地方都让周疲于应付。因为周的家人中也有类似的人在,所以这种意识就更加强烈了。\\

也就是说,千岁与母亲的相似之处让周难以应付。\\

不只是性格,千岁的嗜好也和母亲很相似……特别喜欢漂亮和可爱的东西。\\

不阻止的话,总感觉真昼实在有些可怜,于是周骂了一声「真是的」,同时轻轻拍了拍快要把手伸出去的千岁的脑袋。\\

因为目的是制止和吐槽,周用的力气真的很轻,不过受到冲击的千岁还是小声喊了句痛,然后收起了伸向真昼的手。\\

「这点事不至于拍脑袋吧」

「这家伙很怕生的,没熟悉之前不要身体接触」

「熟悉之后就没问题了吗?」

「这个你问椎名。注意阶段啊阶段」\\

真昼明显摆出了要逃跑的姿势,看来阻止千岁应该是正确的选择。\\

看到了真昼略微……不如说是相当困扰的样子,千岁似乎也理解了周阻止她的理由。\\

「对不起,我太兴奋就想摸摸看了」

「哈,哈啊……」\\

就算突然听到想摸的心声,真昼也依旧是一副困扰的样子,仿佛不知如何是好一般,用眼神向着周求救。

「啊,椎名,千岁虽然是个精力旺盛的怪人不过并不是坏人……吧」

「我说你这算是袒护我吗?不是吧其实是在损我吧?」

「你看看你现在这样子能否定吗?」

「不能!」\\

千岁光明正大否定之后又注视了一会儿真昼,接着以认真的表情再次把手朝真昼伸了出去。

这次是把手心伸出去的姿势。\\

「那么就从朋友开始吧,请多多指教」

「哎?好、好的,请多多指教……?」\\

被请求握手后,真昼也畏畏缩缩地握住了伸出来的手。\\

千岁一旦中意上谁就会想要变得要好,从她的性格来看感觉真昼要有的被折腾了。不过既然是普通的朋友关系,周倒是也没什么插嘴的余地。

只希望千岁能在相处中有所节制。\\

「嗯嗯,又诞生了新的友情啊」

「你倒是好好管一下你的女朋友啊」

「我尽量」\\

周对着每次都让千岁差点暴走的树吐槽一句之后,看着握着真昼的手笑嘻嘻的千岁再一次叹了口气。\\

\subsection{暴风雨过后}

「真的对不起啊」\\

到了傍晚,树与千岁回去之后,周向略显疲惫的真昼做出了道歉。\\

一下子被不认识的人缠上还被知道了秘密,真昼估计也感到困惑、疲劳了吧。

周感觉这样的对话在志保子那时候也有发生过一遍。\\

「不,毕竟原因是我自己的粗心」

「她很吵吧」

「……是个很活泼的人」

「老实说她吵也没问题的哦」

「虽然精力有点旺盛,不过感觉还挺有趣的」

「这哪只是有点……算了,你不在意的话那倒没事啦」\\

周是觉得那绝对能说是吵闹啰嗦了,不过客气的真昼对她的评价实在是非常委婉。\\

庆幸的是真昼没有多么讨厌千岁,然而周并不晓得她和那人成为朋友到底会不会是件好事。

千岁和真昼算是迥异的类型……在新鲜感的意义上或许是件好事……吧。\\

当然,要是千岁让真昼太过困扰的话,周是打算提醒一下她的。不过周现在打算先留着心眼默默守望这两人。\\

「我周围没有那样的人,所以还是有些开心的」

「千岁那种类型的家伙确实是不怎么见得到……要是她太缠人就打她脑袋啊?」

「暴、暴力是不好的,我会努力用语言阻止」\\

尽管有种两个人都默认千岁会暴走的感觉,不过千岁确实是经常把满腔的热情往奇怪的方向上使,所以这种提醒还是有必要的。\\

周在心中立下了稍后直接去劝告千岁的誓言,同时转身朝向窗户的方向眺望起飘落的雪花。\\

如果不是这天气,也就不会暴露给那对情侣了……不过,下雪或许表示着对恋人们的祝福,所以周也不好抱怨太多。

真昼似乎也是喜欢看雪的,她注意到周在看雪,于是同样欣赏起了雪景。\\

因为是冬天,所以太阳早早落下,周围已经暗了下来。

现在看天黑程度已经算得上是晚上了,雪下得也很薄,所以在家里的灯光下才勉勉强强能看得出雪。\\

「是白色圣诞节呢」

「是啊。不过,和我们没什么关系就是」

「那么漂亮不也挺好吗」\\

因为两人完全没有交往关系,所以说实话,白色圣诞节之类的其实和他们没有什么关系……不过,既然真昼喜欢,那下雪也不坏吧。

飘舞的小小雪花渐渐为夜幕低垂的世界饰上一层银白的淡妆。按照这个样子,就算雪下个不停,最后似乎也不会有多少积雪。\\

「不过,要是雪下得太大了会让公共交通瘫痪的,还是适度最好」

「这时倒现实起来了啊」

「因为人只靠浪漫是活不下去的」

「说的是」\\

能有这样的对话,或许也是多亏了雪天所赐。\\

两人相互轻轻笑了笑,接着真昼站了起来。\\

「那我去把晚饭拿来咯」

「咦,拿来?」

「我先在那边做好了炖牛肉了。再怎么说,烤一整只火鸡两个人也吃不下吧……」

「我根本不会想到拿整只火鸡来烤啊」

「只是周君不会做菜而已……明天午饭就在蛋包饭上盖浇炖牛肉吧」

「听上去好好吃……」\\

那种东西在吃之前就知道肯定很美味,所以今天晚饭还没吃,周就开始期待起明天的午饭了。\\

「我鸡蛋喜欢煎熟一点」

「巧了,我也更喜欢这种传统的方式。那我去把锅拿来」\\

真昼啪嗒啪嗒往门外走去,打算暂时回一趟家。周呆呆地望着她的背影,回忆起了吵吵闹闹的白天。\\

被发现,实在是预料之外。

周自己原本就被怀疑着,所以如果只是让疑念加深倒还算预料之中……然而他根本没想过,那个时刻真昼竟然会露面。\\

结果上来说,情况都讲清楚了,周也得到了理解自己的人……然而,周的心情还是有一点点的复杂。\\

——两人之间的秘密,若是能再保持一会儿该有多好。\\

(想什么呢我)\\

不用再对树和千岁到处藏着掖着了,生活明明应该会轻松许多才对。然而,周却感觉有些郁闷。他自己也困惑着,不知道是怎么回事。\\

从结果上看明明并不算差,可是周总有哪里觉得不舒服。\\

「怎么了吗?」

「……没事」\\

真昼抱着锅回来之后,看到周的样子疑惑地歪着头,不过再怎么说周也不能把这说不清道不明的情绪吐露给她吧。\\

看着周如同掩饰一样的平常表情,真昼至始至终都是一副摸不着头脑的茫然模样。

\subsection{天使大人的幸福的味道}

「……呼,真好吃」\\

真昼的料理一如既往地好吃。\\

因为是圣诞,端出来的是比平时更加精致的料理。

真昼用小火炖透的炖牛肉做成了肉馅饼,现在两人正边切边吃着。\\

享受完切饼的快乐之后,那酥脆的口感配合炖牛肉浓郁的酱料,吃起来正可谓是幸福一刻。\\

真昼似乎是特意将其和进了面料里。周佩服着真昼谜一般的高技术,在吃完了今天第二个蛋糕的时候歇了下来。\\

顺带一提,连蛋糕都是真昼亲手做的。\\

她和进肉馅饼面料的同时似乎也顺便和进了另一份甜点用的面料,最后做出了千层酥。这已经是店里师傅的级别了。\\

「你喜欢吃就好。……吃得还真是多啊」

「嗯。因为好吃」

「谢谢夸奖」\\

周也逐渐开始习惯真昼的微笑了。\\

每次夸奖说好吃的时候,她都会露出安心一般的笑容,所以看到这副笑容已经像是周的日常一样了。

看到这比平时柔和得多的表情就好像是周的特权,这让周心里莫名痒痒的。\\

「……明天是蛋包饭吗……超期待的」

「喜欢蛋包饭吗」

「有蛋的都喜欢」

「啊原来如此……记得蛋卷之类的都吃得狼吞虎咽的」

「好吃嘛没办法」\\

就算喜欢吃蛋,要是不好吃的话周也是不吃的。他能那么有食欲,还是因为真昼的料理美味吧。\\

尽管周心里觉得自己独占非常奢侈,不过周并没有让给别人的想法,而是准备一直享受到真昼不做为止。\\

「……周君啊,吃饭的时候一脸幸福呢」

「事实嘛,因为真昼做得好吃啊」

「夸奖是谢谢啦,不过你这幸福还真廉价呢」

「不不还挺贵的……你搞清楚自己的价值吧……」\\

毕竟是天使大人亲手制作,食用这些料理的权利想必是一部分男生梦寐以求的吧。\\

「对我来所就只是每天在做的东西呢」

「我也真是幸福啊」

「……至于这么夸张吗?」

「那是,毕竟每天都能吃到好吃的料理」\\

周基本上没有什么物质欲望,还是食欲更加强烈。对他来说,能每天吃到新鲜出炉的美味料理就是最大的幸福。\\

「怎么样才能这么会做菜啊」

「以前照顾我的人告诉我的,『如果有人能给你带来幸福,一定要抓住他的胃』」

「抱歉啊让你抓住了我这种人的胃」

「就当是预先练习了」\\

看到真昼那小小的笑容,周不禁心里一跳。\\

「……话说回来,那个照顾你的人也很厉害啊」

「是啊,那个人的料理相当好吃,我还远远比不上。那个人的料理有幸福的味道」\\

看着真昼微微露出的柔和笑容和望着远处的眼神,周默默地感到了安心。\\

听这个说法,真昼应该是很受那个照顾她的人疼爱,而且也看得出,真昼同样敬爱着那个人。

父母对她冷眼相待,而那个人则代替父母教会了真昼很多重要的事情。\\

那样的人能在真昼身边,真的是万分侥幸吧。

听语气那个人应该是女性。周认为,正因为有她在,真昼才能像现在这样过着正经的日子。\\

「听上去好吃得不得了吧。不过对我来说你的才是幸福的味道」

母亲先暂且不提,虽然父亲的料理也很好吃,但是真昼的更符合周的口味。\\

真昼的料理是那种每天吃都不会腻的安心味道,让人心情平静的同时又让人满怀期待,不仅完全吃不腻,甚至越吃越想吃。

不过真昼的负担已经太大了,所以周是不会把这些话说出口的。\\

周嗯嗯地点头之后,就看到真昼愣住了。\\

或许该说是猝不及防吗——真昼以呆呆的,毫不掩饰稚气的表情看着周。\\

「……真昼?」

「啊……没事」\\

真昼听到声音后回过神来,慌忙摇头,然后低下了头。

她紧紧抱着喜爱的坐垫,轻轻吐了口气。与刚才的样子截然不同,现在的真昼身上能感受到有奇妙的魅力。\\

「怎么了吗」

「……只是想着,我这种人也能做出幸福的味道吗」

「虽然不懂你自卑的理由,不过你做得好吃到我想天天吃啊」

「……谢谢夸奖」\\

真昼朝着周往上瞄了一眼,露出了有些害羞而又满足的表情轻轻地微笑着。倒是周想要低下头埋起自己的脸了。\\

这极少给人看到的表情,即使心里没有作为异性的喜欢,也让周的心脏不由分说地跳动起来。\\

真昼取下平时的假面,露出了这算得上毫无防备的笑容,而周现在满脑子都想着要让脸冷却下来。

这慢慢涌上来的热量要是暴露出来就不好了。要是两人互相害羞的话,一定会发生尴尬的。\\

「啊,那个……对了真昼」

「嗯」

「明天是中午开始对吧?」\\

周受不了这样的气氛,强行转换了话题,但真昼并没有太介意,而是思考着周说的话。\\

「嗯,就是这么约好的吧?中午做好饭,然后就是约好的游戏……对吧」

「好」

「不乐意……吗?」

「不是,只是确认一下……虽说平安夜已经过去了,不过圣诞这种过法真的可以吗?」

「不可以的话我就不会提出来了……我很期待」\\

她脸上再次露出了小小的柔和的微笑,让周没法正视她,只能敷衍了句「噢」,然后靠在真昼反方向的扶手上,掩盖住自己的羞耻。

\subsection{天使大人与圣诞}

第二天,真昼来到了周的家里,她的样子有点坐立难安。\\

这是出于节假日去异性家玩时常会有的紧张……当然不是这么回事。因为真昼之前的迫切希望,今天两人约好了要一起打游戏,她现在这副模样大概就是兴奋流露出来时的表现。

据说她是第一次玩电子游戏。从这个角度来看,说她是深闺大小姐也毫不过分吧。\\

「那我先去把午饭做上」

「嗯。麻烦蛋煎熟点」

「知道了啦」\\

就算客人要求很多,也没有损到她的心情。只见真昼转身甩起围裙,快步走向厨房开始做起了午饭的准备。想必她现在心情一定很不错。

「真昼居然这么期待啊」周这么想着,莫名觉得有点尴尬,或者说是心痒。\\

(不过,她期待的只是玩游戏罢了)\\

她绝不是因为能两人一起玩才感到期待。

望着那一束轻轻摆动的马尾,周轻轻地露出了苦笑。\\

\vspace{2\baselineskip}

「……这应该怎么操作呢?」\\

吃完午饭,两人坐在电视机前的沙发上,盯着电视的画面。\\

周试着问了下真昼想要玩什么类型的游戏,却得知她甚至对游戏的种类都不太了解,周没办法,只好打开某有名的国民级2D游戏,把手柄递给了真昼……不出意料,真昼手忙脚乱,不知如何是好。\\

「呃,这个嘛,首先要移动的话拨动这里的摇杆,跳跃的话按这个键……」\\

真昼平常一直都是一副冷静沉着的样子,现在却一脸不解地反复看着电视和手柄进行操作。看着那样的她,周感觉莫名的治愈。\\

虽说是没有习惯,但如此无拘无束的玩法周还是第一次见到。\\

看着真昼不知多少次躲也不躲直接向敌人突击然后挂掉,周切实地感受到,就算是天使大人,也有她不擅长的事。\\

「……打不赢」

「别说通关,你这连第一个敌人都没干掉啊」

「不要吵啦」

「习惯就好习惯就好。这种事要靠身体记住」\\

听了周说的「什么事情都重在挑战」后,真昼老实地继续起了游戏。

看着一脸认真地挑战着本是娱乐性质的游戏的真昼,周甚至感到一阵欣慰,不自觉地流露出了笑容。\\

只是真昼老卡在第一个敌人那里,周看着一直没有前进的画面,比起笑意,他的心中渐渐涌上了更多的不安。\\

真昼看向了周这边。

看见那表情,周仿佛听到了嘟起嘴的音效。大概是错觉吧。\\

「啊,你看着,这里得这样」\\

由于再在这里卡着可能就会影响到干劲,周便把手放到真昼正握着的手柄上,给她做起了示范。\\

周全通这游戏已经不是一回两回,就算是真昼卡住的地方周也能轻松突破。

事实上,这只是真昼的操作实在太菜,正常人是不会卡在这里的……不过周并没有讲出来这点。\\

「喏,这个敌人虽然会以一定速度不规则移动,但只要它一看见你就会加速朝你冲过来。这时候掐准时间跳起来……」\\

周如同要盖住真昼的小手一般握住手柄进行操作,一边做着浅显的讲解一边做着示范。\\

画面上,玩家角色如周所说的一般地移动着,避开了敌人。

虽然这并不是什么大不了的动作,但对一直失败的真昼来说似乎很是新鲜,她不禁「喔」地发出了一声感叹。\\

真昼装饰着长长睫毛的双眼顿时睁大,表情也明朗起来。

因为凑得近,周获得了真昼的下睫毛也很长这项新发现。同时,他看着高兴起来的真昼,也微微笑了出来。\\

而当周这样望着她端正的侧脸时,真昼也许是察觉到了视线,也向周这边反望回来。\\

因为周之前为了能够到她手上拿着的手柄而靠了过来,两人现在的距离比想象的还要近。

不如说,两人的胳膊和手已经碰上了,距离已经近到周能够感受到真昼呼出的空气轻轻拂过皮肤的地步。多亏于此,真昼的体温和甘甜的香味直接传了过来。\\

「抱歉」\\

周注意到自己的手已经几乎要包住真昼的手,慌忙连同身子拉开距离,真昼则像是现在才注意到这事一般,吧嗒吧嗒眨了眨眼,视线开始迷离起来。\\

「没……没关系。我才是,十分抱歉」\\

看着真昼开始泛起红晕的脸颊,周后悔自己搞砸了。

真昼并不是很喜欢身体接触。就算两人差不多已经熟络,握住她的手也许还是会让她不快。

虽然真昼看起来有些害羞,但这并不代表没有厌恶感。\\

「真的抱歉」

「那个,我其实并没有那么在意的哦?」

「不讨厌吗」

「……虽然是被吓了一跳,但到不了讨厌的程度。毕竟也不是不认识的人」\\

看来心胸宽广的天使大人原谅了这边的无礼举动。

真昼轻轻带过不予追究,周也安心下来,继续开始了游戏。\\

周准备这次让真昼自己打,于是看向了画面……可看着依然挂掉的真昼,周开始认真地烦恼起如何才能提高她的游戏技术了。\\

\vspace{2\baselineskip}

结果上来说,真昼跌跌撞撞地好歹算是通了第一关,这时周决定先不玩这个游戏了。\\

对一个完全的新手来说,总是挂掉是十分伤干劲的。因此周打算让真昼试一试别的游戏以减轻压力。\\

「真昼,你身子歪了」\\

于是乎,下一个游戏周选择了在现实中也很常见的赛车游戏给真昼……结果真昼身子跟着歪了起来。

这个游戏没有必要用重力操控,再说手柄里也没有装陀螺仪。\\

歪身子是完全没有必要的……但也许是无意识的举动,真昼正以拿着手柄的状态左摇右摆。\\

而真昼本人似乎是正集中于游戏,没有回答。\\

与刚才的游戏不同,因为现代人平常乘车的机会不少,这种开车的游戏似乎很容易上手。或许也是她学习过的体现,尽管技术很烂,但好歹能玩得起来。

真昼一脸认真地摇晃着身体,努力地操控着车。\\

(超可爱的啊这)\\

真昼像不倒翁一样晃悠着,莫名地显得很可爱。而她一脸认真,又拼命地玩着游戏的样子更增添了可爱感。\\

出现大弯道时,真昼的身体也自然地跟着大大地倾斜。

嘭的一声,她倒在周的大腿上的时候,周为了憋住不笑也是拼上了命。\\

「……其实,身子没必要跟着一起歪哦?」

「我、我不是故意这样的」

「嗯,我知道。但你刚才就一直歪着」\\

周拼命抑制住嘴唇的抖动,把真昼扶了起来。

该说是果然吧,真昼的身子又软又轻。个头小当然也是原因,但是那纤细到看似一折就会断的身体,也让人犹豫是否该触碰她。\\

真昼被周扶起来之后,也许是出于羞耻,正红着脸而发着抖。

这一举止也像小动物一般可爱,让周终于忍不住笑出了声。\\

「是、是把我当笨蛋了吧」

「没有没有。只是有点想笑」

「这就叫把人当笨蛋啦」

「你难道觉得我会把认真的人当成笨蛋?」

「倒没有这么想……」

「就是嘛。我只是觉得你很可爱而已」

「……那个可爱绝对是指像小孩一样惹人发笑的意思」\\

这回答似乎略带几分闹别扭的情绪。要是开玩笑开得太过分真昼怕会不高兴,所以周的感想就到此为止。

不管内心再怎么想,只要别在脸上漏出来就没问题,所以还是只在心里默默想着吧。\\

真昼摆出有些不满的表情。周朝她轻轻一笑之后,她哼地别过了头。

\subsection{天使大人与圣诞礼物}

尽管中途出现了让天使大人闹别扭的事态,但天使大人一回到游戏中,就把这些事情抛诸脑后,再次回到了一脸拼命的表情。\\

虽然跌跌撞撞,但游戏本身真昼应该是大致熟悉了,还算能玩起来并且勉强跟得上。\\

原因大概是在于,现在玩的跟一开始玩的游戏不同,是以控制车辆为主题的游戏。

尽管总是脱离赛道冲进泥地、撞上墙什么的,但真昼还是能够往前开的。\\

本来周还在担心不擅长游戏的真昼会不会一直往反方向开,不过真昼的表现比周预想的要顺利,这让周松了一口气。\\

顺便周让画面分成两半也玩了起来,不过真昼无意识的干扰让周有些不好受。\\

真昼似乎还是有不自觉地歪身子的习惯,不时地把头靠到上胳膊附近又移开去,如此反复着。

此时空气就会泛起轻柔的香味,让周无法冷静下来。\\

虽说如此,但毕竟对手是最弱的电脑玩家,周还是一马当先。\\

「……为什么你开得这么快啊」

「玩得久习惯了」\\

多玩几遍的话,赛道就能记下来,转弯也自然会变得熟练。习惯之后,对手的妨碍也能利用视野和障碍物在一定程度上阻止。\\

看到真昼那无法接受的表情,周苦笑着回答了她,然后静静切回了单人模式。

考虑到她经验不足,周觉得应该先让她在更大的画面上进行练习。比起让真昼看到周的技术对自己失去信心,还是先让她习惯下与电脑对战为好。\\

还好真昼倒是有干劲,回到单人模式也依然认真地注视着屏幕。

保持这样的话,再怎么说也应该可以跟上电脑玩家吧。\\

看到勤奋这一特质在玩游戏上也体现了出来,周觉得很想笑便默默笑了出来。而真昼似乎从气氛上发现了周在偷笑,不满地砰砰拍着周的膝盖。

这副样子有趣得让周笑得更欢了,而真昼皱着眉头小声嘟哝了一句「周君大笨蛋」。\\

\vspace{2\baselineskip}

「赢了」\\

苦战两小时有余。

真昼得到并保持住了闪耀在画面边缘的第一的文字,冲过了终点,稍显骄傲地看向周。\\

面对电视苦斗良久,她总算获得了光荣的第一。

她不知重复了多少次的倒数第一,却依旧没有放弃,一名一名慢慢地提升名次,最后好不容易终于获得了第一,因此带来的感动也特别强烈吧。\\

看着这如同说着「总算做到了」的洋溢着成就感的表情,周也直率地送上了称赞的掌声。\\

「太棒了。很努力了啊」

「嗯」\\

大概是被夸了很高兴吧,真昼平常的那副表情稍稍害羞地柔和了一些。

这并不是「嘻嘻」这样一目了然的笑容,而是略微泛起的夹带着喜悦的柔和的害羞神情,与她平时的那种冷淡感相比简直甜到无法想象。\\

虽然,最近真昼那副平常的冷淡脸上也时不时会露出与年纪相称的少女表情,但今天真昼这副比起以往更加与年纪相称的表情,实在是可爱到过分。

这天真无邪的微笑,甚至令周的理性开始松弛,脑子里开始冒出想要抚摸她的头的欲望了。\\

顺着这股,想要抚摸猫的欲望般的,想要疼爱她的冲动,周的大脑不自觉地向手臂发出了指令……然后慌忙地收回了无意识间抬起了的手。\\

「怎么了吗?」

「啊啊没,没什么。就是觉得你玩得好多了」

「有进步吗?」

「有啊有啊。跟一开始相比简直是天差地别」

「谢谢夸奖。因为很开心,所以自然地就努力了」\\

周实在无法继续看着「哼哼」地笑着的真昼,于是带着掩饰的目的,从放在屋内柜子里的篮子中取出了一个小盒子。\\

「给,得了第一的奖励」

「咦,那个,其实没必要」

「不要奖励的话就当作是从某个蓄着白胡子的胖老爷爷那拿到的吧」\\

是的,这其实是昨天周不小心忘了送出去的圣诞礼物。\\

虽然生日和圣诞隔得不久,要再选个礼物送出去有点难办,不过这回正好有个看中的东西,因而并没有像生日那次一样那么折腾。\\

真昼似乎是听到圣诞礼物这一词汇,才重新想起来今天是圣诞节这。她啪哒啪哒眨了眨眼,然后小心翼翼地接过了礼物。

告诉她「现在打开也可以」后,真昼再次小心地解开了包装丝带。\\

(虽然,也不是什么贵重的东西)\\

真昼打开纸盒,慢慢地从里面取出的,是皮制的钥匙包。\\

送太贵的东西反而会让对方很难办,因此周并没有选择大牌子的东西,而是单纯地挑了一件感觉和真昼挺配的东西。

这是件印着花朵和常春藤图案的,设计上适合日常使用的朴素的钥匙包。虽然周对花不怎么熟悉,并不清楚上面刻着的是什么花,但看着那纤细的形状觉得和真昼很配,于是便选了这一件。\\

「不是给了你把多的钥匙吗。不过不用这包也没问题就是啦」

「不,我会感激地使用的……周君的眼光比想象中更好呢」

「比想象中是什么意思啦」

「嗯,毕竟平常总是运动衫配运动裤……如果只说服装的话那甚至是比眼光还基本的问题……」

「这么好用的衣服可没别的了啊」\\

毕竟周根本没有机会让真昼看见自己认真搭理的外表,而且周也觉得这种事情麻烦所以尽可能不想去做,因而只让真昼见过穿校服和宽松的室内装这两种样子。

因此在谈及眼光之前,周大概已经给了真昼一种邋遢的印象吧。不过周确实是邋遢,所以这个印象也不像能消掉的样子。\\

「……要是好好打理打理说不定会变得挺帅的哦?初中那时候的周君不就有好好打理嘛」

「那是我妈强行把……等等你怎么知道的」

「志保子阿姨说着『明明好好打理的话能有这个样子呢』把照片给……」

「可恶啊那家伙」\\

周实在没有料到因为母亲工作需要才被迫换上外出用打扮那时的照片会流出来,在内心里向着不在场的母亲送去了大量的抱怨。\\

「……那样的打扮不适合我啊」

「真的吗。周君,只是为了不和别人过多地对上眼神让前发遮住了眼睛,但其实五官挺端正的啊……」\\

一只小小的手,伸向了周的脸庞。

洁白的手掌触碰到周的额头,将垂下的前发捋起,让周的视野变得比平常更加宽广了。\\

周以除泡澡以外许久没有过的宽广的视野看向真昼,映入眼帘的真昼脸上则浮现出略带惊讶的神情。

周想着这并不是什么值得吃惊的事情,自己的脸不过是个不算丑也不算帅的普通脸罢了,所以真昼一直盯着周这事让他不禁感到不可思议。\\

「……怎么了啊」

「没什么,就是感觉眼睛比以前有生气啊,这样」\\

真昼嘴里说出「几个月前还是一副死鱼眼来着呢」这般虽然非常过分却令周无法反驳的话之后,仰着头直直盯着周。

明明再怎么看也没什么好看的,真昼却一直静静地看着周。\\

这样子被异性,还是这般不得了的美少女凝视,周总觉得感到很羞耻。\\

不过,周也不是个甘愿一直被搞的人,作为报复没有多想便撩起了真昼的一束落在脸上的侧发,让她姣好的脸庞露了出来。

虽然对触碰真昼有点犹豫,但周转念一想反正真昼也没多想就碰了自己的头发,自己这样做应该也没事吧。反正也不是摸头,周希望这是安全的。\\

(不过啊,她这确实是个美人啊)\\

重新这么一看,周再次体会到真昼那美貌的迷人。

就算跟那曾经落在周屋子里的杂志上印着的美女之流相比,恐怕也还是真昼更加美丽迷人吧。\\

再说,照片本来就不太可信。

因为,照片仅仅记录了一瞬,是可以加工的,不论是保持原样、美化,或是伪造都可以做到。\\

与之相对,眼前的真昼即使没有加工处理,却还是那么可爱而美丽。\\

周一直盯着这这令人总看不够的端正的面容,而真昼的目光则渐渐开始迷离了起来。\\

正当周想着怎么了的时候,真昼突然把手从周的头发上移开,然后垂下了眼帘。

真昼扭扭捏捏,看上去很不自在。她把手柄完全放开,然后抱起了旁边的靠垫。\\

「那个。呃……对了。我也有给周君的圣诞礼物」

「哦,哦哦,谢了」\\

周正想开口问是什么,真昼却如同不让周问出口一般从放在一边的包里拿出了一个包好的袋子塞给了周。\\

「那,我就先去准备晚饭了」

「嗯?哦,哦……?」\\

说完这句话,真昼便迅速从沙发上站了起来。面对这过于快速的发展,周感到不知如何是好。

\psline

天使大人现在是 40\% 左右吧大概。

\subsection{天使大人与新年预定}

圣诞节过去之后,世上就是一片年末的气氛。\\

夜景用的照明灯还留着,然而装饰那么多的圣诞树已经撤下,各种鲜艳的装饰变化成了和风。

贩卖的东西也全面变成了新年装饰和食材,平安夜的样子已经完全不见了踪影。\\

「变化还真是快啊」周想着,一边望起已开始进入过年准备的周遭,一边把脸埋到围巾里取暖。\\

黑白的千鸟格围巾,是真昼送来的圣诞节礼物。

听真昼说,脖颈处的装饰很重要,周便从她那儿收到了这个手感非常舒适、挡风保暖性能好的,兼具了实用性和装饰性的好东西。\\

因为平时不带围巾,于是周便心怀感激地用上了,同时确认起手上提着的购物袋里的东西。\\

尽管采购基本是说好分担进行,但是为了减轻要做菜的真昼的负担,基本上是周带着备忘录把东西买齐的。

今天很冷,似乎是要做火锅,袋子里装着蔬菜啊蘑菇啊肉之类的。蔬菜这么多,是真昼默默地主张着要好好摄入营养吧。\\

周再次确认没有遗漏,在越来越冷的天气下发着抖快步回家了。\\

「你回来啦」\\

回到家已经是傍晚了,所以真昼前来迎接了周。

无缘的他人迎接房子的主人这个场景尽管有些奇怪,不过周最近已经开始习惯了。\\

「嗯,刚回来……买了点年糕片回来没问题吧?」

「是想涮火锅啊」

「嗯。还有,买了拉面最后吃」

「……我吃不了这么多哦?」

「我会吃掉一大半的所以没关系」\\

周以前吃得并不多,不过多亏了真昼的料理,现在晚饭还是吃得挺多的。\\

她可能也是留意着卡路里,吃的量也就是不会发胖的程度。而周吃得比她多,有些微妙的担心,所以目前有开始做肌肉锻炼。\\

真昼的感想似乎是说周太瘦了应该多长点肉,所以周希望尽可能多长点肌肉而不是脂肪。\\

「周君吃的话倒是行啦。那个,给我一下,我放冰箱里。周君去漱口洗手吧」

「好嘞」\\

周把装着货的塑料袋递给真昼之后,老老实实走去了洗手间。\\

\vspace{2\baselineskip}

「说起来真昼年怎么过」\\

今天周也是吃完了一如既往非常美味的晚饭。正在收拾打理的时候,他忽然觉得有些在意便向真昼询问了。\\

「过年……回去也没意义,就呆在这边」\\

听到这语气实在平淡的回答,周醒悟了自己的失误,而真昼却似乎没有怎么介意的样子。

因为和家人相处不好,所以一提起家庭关系她无论如何都会摆出冷淡的态度。\\

只是,这样的话,真昼岂不是得一个人过新年吗。

周基本上来说有半年一次给家人露个脸的约定,所以在遇到真昼之前是准备在长假时回老家的。\\

「周君是要回老家吗」

「这个啊,姑且家里人有叫我露脸来着」\\

周瞄了一眼真昼,不知是不是错觉,感觉她的眼神比平时的表情要更冷一点。\\

似乎真昼理所当然地以为会自己一个人过,没有怀疑地以为周会回老家。\\

「……如果回去的话,感觉关于你的事情会被问这问那的啊」

「真辛苦啊」

「老爸大概听老妈说完也就那样了,不过妈大概会老想打听吧」

「明明我们经常说话的,真是不可思议呢」

「说真的你不知不觉就跟老妈熟络起来了啊……」\\

不知为何真昼不知不觉和老妈打好了关系,结果周不知道的时候流出了照片和秘密……想着这些,周感到有些虚脱,不过真昼看这样子应该是自愿和老妈相处的,于是周心情上姑且觉得这样也可以接受吧。

稍后得叮嘱志保子不要说出多余的事情。这个先不说,周看向了真昼,不知该如何是好。\\

一想到真昼有时露出的空虚的表情和寂寞的眼神,周就无论如何……都不想放着她一个人。\\

「我觉得吧,最近也跟妈见过了,虽然对爸有些抱歉不过这次不回老家应该也可以吧。反正春假会回去的」\\

所以,如果不会给她添麻烦的话,周还是希望能和往常一样和她一起吃晚饭。\\

「……是这样啊」

「嗯。还想吃你的荞麦面\footnote{日本风俗,在除夕夜(12月31日)会吃荞麦面。}来着」

「还真是嘴馋啊」

「因为是真昼做的」

「……明明基本是买来的?」

「就算这样也好啦」\\

就算只是买来的荞麦面煮一煮也好。

因为,两个人慢慢吃面共度时光,这才是更重要的。\\

「……真是个怪人」

「吵死了」\\

对着发表出失礼感想的真昼,周刻意做出了不高兴的回答,而真昼回了个小小的微笑。\\

「……谢谢」

「谢什么啊」

「什么都有」\\

真昼没有再说更多。或许是心情好上了几分,她露出了明朗的表情抱紧了喜爱的坐垫。

\subsection{天使大人与新年准备}

12月31日,除夕。

今天是一年的最后一天,是为一整年画上句号的日子。

虽然一般来说这一天都是在为明年做准备和大扫除中匆忙度过的——\\

「那个,真昼啊」

「怎么了?」

「……我在这干闲着真的行么?」\\

周悠闲地坐在客厅的沙发上,望着从早上就开始穿着围裙在厨房里忙活的真昼的背影。\\

真昼从早上就过来,是为了准备年菜。

既然决定了两人一起跨年,那当然也要准备两人份的年菜。\\

周本想着真昼会去外面买一些对付过去,但看起来她是打算自己做的样子。连家庭主妇都感到头疼的家务,这位花季女高中生居然一个人搞定,实在是令人惊讶。\\

周真心叹服真昼的能干,但本人却说「就算要买这种东西也要事先预订,现在已经没办法了」。

听完真昼的解释,周虽然理解了真昼亲自动手的原因,但还是对不辞辛劳亲自制作年菜的真昼脱帽致敬。\\

当然了,能省事的地方真昼也会省事,像黑豆这种东西煮起来花时间不说,还要占掉一个炉灶,所以她就直接买来了成品。\\

「周君,就算你因为在那闲着感觉良心不安,你觉得你能帮上忙吗?」

「不能」

「那不就是嘛。比起过来碍事,还不如在那老实待着更好」\\

周乖乖遵从观点实在辛辣的真昼的旨意,老实地坐在沙发上,但什么都不干还是让周冷静不下来。\\

就算是周,也并不是什么活也没有干过。

毕竟大扫除昨天就已经完事了,家里也已经屯好了不出门也可以支撑一段时间的,包含了年菜材料的大量食材。

虽然周不是什么也没做,但要是跟现在的真昼比起来那就是没出多少力了。\\

「昨天你把家具家电这些都搬开来全部打扫一通也累了吧,今天你就休息着吧」\\

真昼以言语关心着负责力气活的周,不过并没有回头看周这边而是继续打理着料理。\\

顺带一提,真昼自己家的大扫除似乎是已经完成了。再说真昼她似乎有在认真地进行着定期打扫,大扫除并没有费她太多时间。\\

「哎呀,就算是这样但还是感觉……有点抱歉啊」

「我也是喜欢做饭才做的,并不觉得累哦」

「就算是这样啦」

「没事,我很享受的」\\

真昼以「小意思」的语气说完,然后就把精神集中到了料理上去。周不知如何是好,抱住了头。\\

\vspace{2\baselineskip}

「真昼,午饭买来了」\\

考虑着让已经在年菜上忙得不可开交的真昼再准备午饭实在有些过分,周就去便利店里随便买了些午饭。反正真昼本来食量就不大,一袋三明治应该就没有问题了吧。

真昼正好也暂时脱下了围裙,打算稍事休息的样子,从时间上来说是正正好。\\

「特意去准备午饭,谢谢你了。抱歉哦,我实在是抽不开空准备午饭」

「哎呀要说的话拜托你准备年菜这边就已经很对不起你了啦……来,开饭吧」\\

到了兼作休息的午饭饭点,真昼老实地回到了客厅。\\

「三明治和咖啡没问题吧?」

「嗯,谢谢」\\

真昼轻轻点头,从周那接过午饭,然后坐到了周的身边。\\

「话说做得怎么样了?」

「有一部分是买来的已经做好的,品种数目也有控制,所以现在差不多已经做完了。另外还剩下很多东西等着放凉了装起来。另外周君你好像很喜欢鱼肉末鸡蛋卷的样子,那部分是我自己做的哦」

「为什么你会知道啊」

「你说过自己喜欢鸡蛋料理的吧」\\

尽管当时只是随意一说,但真昼却认真地记下来,还特意用烤箱去烤制了。听见烤箱的工作声,周还想着是在做什么,看来做的是鱼肉末鸡蛋卷。\\

「喜欢微带甜味的口感对吧?」

「你很了解我嘛」

「再怎么说也已经好几个月了,喜好什么的还是能记住的」\\

真昼说着很让人高兴的事,开始吃起了火腿生菜三明治。\\

周也边啃着买来的饭团,边看着厨房那边,眼睛望着的地方放着真昼带来的小尺寸的重箱\footnote{\ruby{重}{chóng}箱:装食品用的多层方木盒,常用于新年的年菜({\jpfont おせち})。}。

年菜应该会装在那重箱里面吧。\\

明明是一个人住居然连重箱都有就已经让周始料未及了,当真昼拿出那个不但涂漆还贴了金箔的,看上去就很高级的重箱的时候,周都有点被吓到了。\\

「实在是,让人只能说感激不尽了啊……该怎么说呢,刚刚开始独居生活的时候还真是想象不到,今年后半年伙食上能这么充实啊」

「我倒是想感叹亏你还能活到今天呢」

「好过分。其实靠便利店这些地方卖的东西意外地能过下去哦?」

「但是不健康啊,真是的」\\

虽然真昼无奈地叹了口气,但她那表情混着苦笑,仿佛在说着「拿你没办法」一样,让周的心里稍微跳了一下。\\

「有我在的话,可是不会允许不健康的饮食生活的哦?」

「你是我妈么」

「都怪周君过得太不健康了。明年的饮食我可是要安排得更健康的哦」\\

看着真昼微妙地提起了干劲,周想到「真昼明年也打算在一起啊」,便莫名地害羞起来,偏开了眼睛。\\

不过,真昼把周这样的态度理解成想要过邋遢日子,便以略带不悦的表情看向了他。结果,周花了一小会儿功夫才解释清楚了自己不是那个意思。

\subsection{天使大人与跨年}

太阳快落下的时候,真昼已经做完了所有东西并且装进了重箱,现在开始准备起了晚饭。\\

话是这么说,但因为晚饭是过年的荞麦面,所以只需要买来已经加工到煮之前那步的荞麦面,然后煮面并且准备好放到面里的佐料就好了。\\

鱼板是年菜的素材有剩,正好可以加进面里。菠菜只要烫一烫就行,葱也是只需要剁碎就好。

最费工夫的是鲜虾天妇罗,不过真昼对于麻烦的油炸食品也是在毫无怨言地油炸着。\\

「还有,南瓜有多出来,就顺便做成天妇罗了」

「哇……真是豪华的除夕荞麦面啊」

「偶尔来一次也挺好的吧」\\

那样说着的真昼最终完成的除夕荞麦面,比起老家吃的版本果然更为奢侈。\\

大鲜虾天妇罗准备了一人两只虾。而旁边附带的南瓜天妇罗里放了大量的菠菜和葱花,点缀着切成了扇形的鱼板,也是清爽的样子。\\

另外,真昼似乎喜欢先放面再放天妇罗。周的那一份也并没有直接把天妇罗浇在面上,而是分盛在另一个盘子里,这细节上的体贴让周非常感激。\\

「哇」

「来吃吧」\\

或许是觉得周只吃这些不够,真昼把年菜多出来的东西也盛在小盘里端了出来。

周看到真昼坐下,各自合掌对食物表达感谢之后,开始吃起了面。\\

虽然说是店里卖的东西,或许因为买来的是贵一些的荞麦面,周一咬下去,荞麦面的香味就扩散开来。

面汤也是浓淡适中,让人放松下来。这从胃里温暖到全身的味道,正适合于寒冷的冬日。\\

「哈……这才是快过年的感觉……」\\

周喝下面汤,长吐了一口气,发出了小声的感慨。\\

看着电视悠闲地吃着荞麦面等待新年的来临,这种感觉果然是不错。

在老家,每年吃着荞麦面看着年末特别节目和一年一次的歌曲节目已经是周的惯例了,今年能以同样的方式过年也是难得。虽然说,在他旁边的不是家人而是勤劳的外人少女。\\

「吃着荞麦面,一下子就涌出了一年过去的感觉呢」

「说的是……今年发生了很多事情啊」\\

虽然这么说,但其实与真昼的交流占据了这「很多事情」的绝大部分。

刚开始独居生活的时候,周一丝一毫都没有想过会有这样的美少女来给他做饭。\\

「这是周君第一年一个人过日子呢,肯定会很辛苦的吧」

「你倒是挺习惯的啊」

「嗯,我大部分的事情都能自己完成呢。周君什么都不会还想一个人过,这样实在不行哦?」

「唔……话是这么说啦」

「多让人头疼啊,真是的」\\

真昼责备周的时候,与其说是看不下去,更像是觉得愉快的样子,表情也很柔和。

她似乎并不以照顾周为苦,始终都是一副温和的表情。\\

「……今年真的受你照顾了」

「就是说」\\

真昼伴随微笑的全方位肯定虽然稍微有些扎心,不过幸好她并没有不愿意的样子。\\

「……明年也请多多照顾」

「知道啦。周君要是没了我,就会笔直奔向自甘堕落的颓废生活了」

「无法反驳」

「……知道的话自己注意点啊?」

「就当作明年的抱负吧」\\

就算有所留心,让真昼勤勤恳恳照料一阵子之后,决心恐怕还是会融化掉。不过周把这样的想法留在了心里,没有对本人说出来。\\

当然,身边物品的收拾整理之类的周会去做——不过会拜托她做饭这事是不会有差了吧。

尽管周也发现自己离不开真昼的饭菜,不过他对此已经无可奈何了。\\

就算对真昼宣言说要改善,周也只会遭到笑话。于是他摆出一副僵硬的表情,而真昼只是愉快地露出了小小的笑容。\\

\vspace{2\baselineskip}

「快要到新的一年了呢」

「是啊」\\

周吃完荞麦面,在沙发上看着歌曲节目,不知不觉中就快到0点,要进入新的一天了。

或许真昼只在必要的时候才会看电视,看样子她对现在的歌并不熟悉。周看着真昼静静地,同时又开心地观赏着歌曲节目,时间就过去得比想象更快了。\\

画面变成了转播含有除夜钟\footnote{除夜钟:在午夜之前,日本全国的佛教寺庙会敲响108下新年钟声。}的风景,让周再次感受到新年即将来到。\\

真昼坐在旁边,垂下眼帘,静静聆听着除夕钟声。\\

片刻之后,听到第107次钟声——\\

「新年快乐」\\

到达0点的瞬间,真昼看着周挺直了身子后弯下腰来,周也附和着端正姿势同样做出了新年的祝福。\\

「新年快乐……感觉有些不可思议啊,两个人跨年」

「呼呼,是啊……今年也请多多照顾」

「我才是……不如说应该是我拜托你来着」

「这个没法否定呢」\\

周朝着哧哧笑着的真昼苦笑出来,同时注意到了膝盖上的手机在振动。\\

似乎是树和千岁他们发来了新年的祝福,App的图标上多出了几个数字。

而真昼也是一样……不如说或许真昼那边更多吧。她的手机也在振动着。\\

最近发条消息就能完成新年祝福,相较以前变得方便了。\\

「我稍微回几条」

「我也是」\\

恐怕真昼那发来了大量新年祝福吧。不过周隐约觉得,真昼应该没有告诉过男生联系方式。\\

看着真昼熟练地连按着屏幕打字回消息,周佩服地感叹着「这方面倒是很像女高中生啊」,同时自己也开始给树和千岁回信。

他们的消息既有普通的『新年快乐』,也有『和椎名要好地跨年了吗?』这种多余的窥探。虽然被说中了,但周还是做出了否定的回复。

很快,树就回了一句像是在开周玩笑的『少来了』,于是周也重复着被开玩笑和否认的过程,享受着这段对话。\\

突然,周的上胳膊压来一阵重量。接着,周感受到了空气中飘着的甘甜香气。\\

周战战兢兢地往旁边看去,便看到了真昼合上眼睛靠在了周的胳膊上。\\

(——等下等下等下)\\

虽然没有发出声音,但周现在其实相当惊慌失措。\\

尽管以前也有打瞌睡这事发生过,然而谁能想到,真昼会在自己旁边,而且还靠在自己肩膀上睡着呢。\\

真昼为什么会睡着,这不需要想也能明白。\\

现在的时刻是已过零点半的深夜。\\

作息规律的真昼应该不怎么会熬夜,而且今天她一直在忙活做年菜,尽管没有表现出来,不过真昼想必是相当劳累了吧。

想必她已经没有体力去抵抗睡魔了。\\

理由是可以理解的。\\

尽管如此,周却没想到会偏偏在这个时候睡着。\\

靠在周身上睡着的真昼,仿佛无视周的混乱和狼狈一样,露出了安详的睡脸。长长的睫毛、端正的鼻梁、粉红的嘴唇,都处于无防备的状态之中。

尽管周并非第一次见到真昼的睡脸,但距离这么近却还未有过,这让周的身体开始僵硬起来。\\

「真昼,醒醒」\\

周客气地喊了一声,不过没有得到反应。

真昼或许是因为相当疲倦,陷入了沉沉的梦乡。无论是跟她讲话,还是轻摇她的肩膀,她都没有醒来的样子。\\

周轻轻拍她的腿,摇晃触碰到的身体,也都没能让她起来。\\

做出这样的事情之后,真昼靠着的部分偏移了一点,开始往前倾斜,周便慌忙接住真昼拉了过来……结果却意外地形成了抱到怀里的姿势,让周变得更加慌张了。\\

(……味道真香啊)\\

吃完饭后,真昼回了一次家,洗了洗澡或者做了些什么其他的事情。或许是洗发露的花香加上本人的体香,现在的真昼正微微散发出甘甜的香气,让周非常不自在。

另外,周还感觉自己接触到了什么柔软的东西,就更加冷静不下来了。\\

由于真昼睡得太熟,周不忍心叫醒她。再说,周甚至觉得,如果不用强硬一点的方式,根本无法让她起来。\\

(该怎么办啊)\\

新年刚开始就遇到这样的意外事件,让周抱住了自己的头。

\subsection{无防备的天使大人}

新年伊始,便被不得了的事态袭击了的周,以僵硬的表情看向怀里的真昼。\\

她真的睡得很熟。\\

真昼大概是想着「周是可以放心的人」,毫无警戒地熟睡了,而周在着急感与害羞感的侵袭下,理性已是摇摇欲坠,真的想要一头撞在墙上。\\

明明不想,可周还是忍不住把意识集中在真昼的触感上。\\

这副纤细的身体,虽然紧致却不失柔软,无处不体现着那女性特有的娇柔。

特别是在身体互相接触的部分,比起画面,那富有质量感的感触,更加无情地削磨着周的理性。\\

(——这可怎么搞啊)\\

过于出乎意料的事态,与这从未感受过的柔软一起向周袭来,令周陷入了极大的混乱之中。\\

女孩子居然这么又软又香啊……对这第一次知道的事实,周产生了微妙的感慨,但接着理性便突然刹车阻止周产生非分之想。\\

周越觉得自己不该去想,怀中传来的柔软感就越鲜明,让周的大脑陷入了一片混乱。\\

虽然周还是试着思考该如何解决这一事态,但他感觉完美平安地解决是不可能的。\\

总之,周还是总结出了三个解决方案。\\

一、强行让真昼醒来

二、搬回真昼的家

三、让真昼睡周的床自己睡沙发\\

第一个的话,主要问题是周不想把现在正熟睡着的真昼弄醒。毕竟是自己的原因才让她这么累的,可以的话还是想让她能安稳地睡着。\\

第二个的话,初看上去应该是风险最小的,但是,这样面临着要掏真昼的衣服找到钥匙然后擅自进入女性的屋子这样的大难题。做到这个地步的话,就算是真昼,事后知道了也可能会对自己产生厌恶吧。\\

那么第三个,让她睡自己的床这一选项应该是最为安全而且容易实施的了……但要是这么做,周有自信自己精神上会死掉的。\\

就算平常两人就一直在一起,让露出了谁见了都会迷上的天真可爱的睡脸的真昼睡在自己床上,周有种自己的理性啊什么的会坏掉的预感。\\

让女孩子睡在自己的床上这种场景,已经是让男生欲罢不能了,好巧不巧对方还是个勤劳努力的美少女。

尽管这么做可能让真昼有些不满,不过这也是没办法的事。\\

但是,这就是最安全,也是周能做出的最好的体贴与妥协了。\\

下定决心的周,各用一只手放在了靠在自己身上的真昼的背上和膝内,缓缓地把她抱了起来。

也有睡着的缘故吧,真昼的身体轻得如同羽毛——那倒也不至于,但真昼的身体抱起来感觉还是很轻。\\

虽然感觉应该没那么容易弄醒,但周还是尽量平稳地把她抱到了自己的屋子里。横抱的状态下门把手开得很是勉强,但过了这一道坎之后,就只要让她躺在床上就好了。\\

纤细的躯体沉入床中。

周把毯子和被子给真昼盖好,便完成了晚安的准备。\\

真昼没有要起来的样子,传进周耳中的只有规律的呼吸声。

仍带着几分幼气的端正美貌,在平日里的美丽之上又添上天真的睡脸,令周不禁心跳加速。\\

让真昼好好地躺在床上之后,周在床边蹲了下来。\\

(……难受啊)\\

要论的话,真昼睡在自己床上的这一场景、柔软的感触、这毫无防备的可爱睡脸、在男性家里能睡着的信赖、由此产生的不设防,这一切的一切都是原因吧。\\

当然被如此信任周也很高兴,但却也让周不由得感觉自己完全没被当成男的看。

估计在真昼眼里,周只是个『实在没用的必须要人照顾的安全放心的无害的男孩子』吧。\\

周偷偷瞄了一眼真昼,但真昼则对周内心的纠结毫无察觉,依旧是一副安宁的睡脸。\\

(睡得那么香,都不知道我的烦恼)\\

既然这么没有防备,那要不然我也躺进去好了……周一瞬间闪过了这样的念头,但转念一想,两人没有交往的情况下,一起睡那就实在是太过分了,便否决了冒出的想法。\\

要真这么干了,感觉真昼起来的瞬间可能就不肯对自己说话了,而且还感觉她会以冷淡的眼神说出一句「你到底在想些什么啊」。所以,为了自己好还是不要付诸行动吧。\\

不过,只是稍微摸一摸应该情有可原吧。这么想着,周把手伸向了真昼的头。\\

丝滑如绢,舒润如绸,光亮如玉——周用指尖轻轻梳过这正如此般词汇所言的光泽长发,没有一丝阻碍便顺滑地直达发梢。\\

连这地方的保养也是十分上心啊——周一边对女性的努力感到赞叹和畏惧,一边轻轻地把指尖滑向了真昼的脸蛋。\\

也许因为真昼体温不算太高,那水润光泽的雪白肌肤,比起周的手来还略凉一些。

周轻轻抚过真昼的脸颊,而后,看着那无比安心的睡脸,静静地露出了苦笑。\\

「晚安」\\

明天……准确来说,今早醒来之后,她肯定会很吃惊吧。周这么想着,但又觉得,她都已经让自己这么抓狂了,这点小事应该算是容许范围之内吧。\\

真是拿你这个家伙没办法啊——周这般苦笑着,再次轻轻地抚过真昼那软软的脸蛋。

\subsection{天使大人的醒来与羞耻}

早上,周起床之后也没有听到日常生活的声音。\\

家里安静得能听到窗外的鸟鸣,睡在周房间的真昼也没有已经起床的迹象。

时间上已经过了日出时分,不过真昼可能是因为昨天太累所以睡得很熟吧。\\

另外,要说周的话,尽管睡是睡着了,不过想着自己床上有真昼就不怎么睡得下,结果到最后也没能睡深,现在这个点就起床了。

周身体上倒是并没有多难受,然而他的难受却是在另一个意义上的。\\

周做了做拉伸以放松因为睡沙发而僵硬的身体,同时缓缓站了起来。

总之周打算先去观察一下真昼的情况。虽然说去拿衣服换才是主要目的,不过周也是顺便准备去观察真昼情况的。\\

周静静地打开自己房间的门。\\

里面一片安静,熟睡在床的真昼果然还是那副样子。\\

不过,要说不同之处的话,真昼或许是翻了几回身子而横了过来,头发也像河流一样摊在床上。\\

「呼、呼」真昼发出着可爱的呼吸声,而周蹲下来望着她。\\

真昼睡觉时非常天真可爱。

也许真昼平时都绷紧了神经,很多时候都是挂着高冷的表情……然而睡觉时,她的表情就松弛了下来,可爱到周想去摸的地步。\\

(……睡觉时真是可爱啊)\\

当然,她即使起床也无疑是美少女而且很可爱,不过现在的周则是更接近于观赏小动物时的感情。

周既想要抚摸那柔顺的秀发,也想要戳戳那柔软的脸颊。正因为平时无机可乘,到了现在这样无防备的状态,周才更加想要调戏她。\\

周情不自禁地把手伸向那看上去很软的脸蛋摸了摸。\\

光滑的脸蛋传到指尖的是和昨天一样的柔软。这软乎乎的样子让人想要一直摸下去,周便不由得用手指肚戳了上去。\\

因为软软的很舒服,周就像是疼爱真昼一样的感觉摸了摸。尽管周有留意少使力气,然而还是让静静睡着的真昼发出了「嗯……」的沙哑而甘甜的声音。

接着,周还没来得及把手拿开,真昼合上的眼睛就缓缓睁开了。\\

一双焦点没有重合的,湿润的焦糖色眼睛看着周……准确来说,是看着周的方向。

真昼松垮的表情上还留有稚嫩睡脸的余韵,十分天真可爱。不如说,那惺忪的眼神明明有意识同时又在恍惚中,使得现在这个样子显得比刚才更加稚嫩。\\

真昼露出了疏忽大意显露无疑的表情,接着又落下眉梢再次闭上了眼睛。

周正想收回手指,真昼却把脸往周的手指上磨蹭起来,同时喉咙里发出了撒娇一般的呼噜呼噜的细声。她把脸蹭上来的样子,就好像是在说不要离开一样。\\

周明白这显然是睡迷糊了。

真昼没道理对周那么撒娇,而且平时的真昼也不会做出这么松垮的表情和动作。\\

即使如此——真昼做出的撒娇的小猫一样的动作,使得周的心脏和理性一大早就受到了考验。\\

是该收回手呢,还是顺着感情摸脸疼爱呢。\\

心情上,周相当倾向于后者。

这么软萌的真昼可是不怎么见得到,而且周也对真昼会撒娇到什么地步很有兴趣。\\

不过,周感觉,如果真的付诸行动的话,真昼清醒的瞬间就会闹别扭不说话了。由于周非常清楚真昼会羞耻得不能自拔,所以他不知道现在应该如何是好。\\

总之,由于真昼很可爱,周停留在了观察睡迷糊的真昼这一步。\\

尽管真昼的意识已经苏醒了相当一部分,不过不知道是因为脑袋不清醒还是因为没注意到这是周的手,现在真昼正把脸蹭到周的手指上打着盹儿。\\

周原本只是打算来观察情况和拿衣服来换,不知为何却变成了这样的身体接触。周感受到了说不清的心痒,以及自己脸上因此而集中起的热量。\\

「嗯、嗯……」\\

过了一阵子,也许终于是清醒了,真昼再度睁开了眼睛……\\

「……咦」\\

四眼相对之后,真昼把目光移动到旁边的周和碰到脸的手指,僵住了。\\

再接着,真昼一跃而起。\\

「早上好」

「……早、早上好……」

「你在我家睡着了所以把你搬这儿了。我没有其他意思,都想让你谢谢我什么都没干了」\\

周抢先解释了真昼躺在周床上的理由,这样一来真昼也没有吵闹而是老实下来了。

不过,因为睡在了男性的床上这个事实,真昼的脸渐渐发红,捏着被子提起来掩在嘴角上。\\

这套动作也微妙地很可爱,让周不禁别开了眼睛。\\

(这什么状况)\\

周姑且是借出床铺的人,现在他却觉得像是自己不好一样。

确实擅自摸脸是很对不起,不过周只是稍微摸了一下下,也并没有打算要做什么事情。\\

周因为真昼的可爱心跳不止,又因为罪恶感而觉得痛心,心里五味杂陈。而看真昼那边,她的脸依旧一片朱红,有些略微的不开心……这倒不至于,但朝着周露出了有话要说的眼神。\\

「……周君喜欢摸脸吗」

「嗯?」

「圣诞节那会儿,还有昨天睡觉之前你不都摸了吗」

「……原来你醒着啊」\\

昨天应该是真昼熟睡的时候摸的,本人应该意识不到才对。

然而真昼却知道有这事儿,说明那时真昼是醒着的。\\

「……那、那个是,嗯……被放到床上的时候醒过来了……那个,除了装睡还能怎么办嘛」

「就不觉得我会做什么事情么?」

「……周君,应该是不会做那种事的……而且,也有为了确认这点,才装睡的,嘛」\\

周似乎是被确认是否真的可以信任了。\\

结果上来看,好在周最终得到了信任,不过他真是希望真昼以后不要做出在男人面前睡着这种没有戒备的事情了。

就算是周,他也不觉得自己下次再见到的话能只戳个脸就完事了。\\

「……嗯,能得到信任倒是好啦,不过以后别这么干了。我也是男的」

「那、那还是知道的,嗯」

「还是说想要我做出什么事来?」

「怎么可能啦」\\

真昼满脸通红强烈否定,然后又钻进了被窝里。周把「这是我的床啊」这句吐槽咽进了肚子。\\

在真昼的害羞消退之前,周只能把窝成一团浑身发抖的真昼静静放在一边了。

\subsection{天使大人的害羞与不悦}

真昼从羞耻中回过神来,先回了趟家,然后换好了衣服回来了。\\

不过,她似乎还在害羞着,每每与周对上眼便会微妙地偏开视线,搞得周也开始尴尬了起来。

虽说万幸真昼还愿意一起坐在沙发上,可周却觉得如坐针毡。\\

「……原谅我吧」\\

周总觉得不好受,下意识地向真昼做出道歉。真昼则瞄了一眼周,然后轻轻叹了一口气。

大概是脸上的害羞已经褪去了吧,真昼姑且算是恢复了一如往常的表情。\\

「我没有生你的气。周君并不需要跟我道歉」

「不过啊」

「我只是因为露出了见不得人的样子,后悔自己的大意而已」

「见不得人什么的……其实很可爱啊」\\

那不负天使这一外号的,正如天使般的睡脸、醒来之后的惺忪睡眼、还有那毫无戒备放松下来的天真表情,全都十分可爱。\\

与平时那冷静而沉稳的表情截然不同,在睡迷糊的时候真昼会露出十分幼气的表情,这是周的新发现。

这个表情可爱到了让周想要更多看看的地步,不过真昼应该是不想自己疏忽大意的表情被看到吧。\\

周并不觉得那表情很不像样子或者见不得人,所以想要否定那一部分,结果不知为何却让真昼咬着嘴唇用抱在怀里的靠枕嘭嘭地拍起了周。\\

周并不痛,真昼大概也不是认真的,但周还是搞不明白真昼怎么突然就拍起自己来了。\\

「干嘛啊」

「……周君这种地方真的是不行」

「什么啊……那你要我怎样」

「这种话是不能轻易说的」

「我又不是跟别的人这么说……」\\

数起周身边的女性,除了真昼和千岁就没了。

虽说千岁确实算得上是可爱,但一提到她周下意识就觉得是个麻烦,而且也没必要当面称赞她,因此除了真昼周也没有谁能夸了。\\

看到僵住的真昼,周感到有些疑惑,耸了耸肩。\\

「我说,你早就习惯被这么说了吧?也不差我一个」\\

再说周向真昼表达自己觉得她可爱早就不是一次两次了,事到如今真昼还在介意这里,让周匪夷所思。\\

真昼的话应该对自己长得有多漂亮心知肚明,被夸奖也应该早就习惯了。

照理只是被周一个人说了几句,不至于让真昼害羞成这样吧。\\

周这么想着,真昼却不知为何变得一脸不快。\\

「所以说你到底怎么了从刚才开始」

「……什么都没有」\\

最后,真昼又嘭地用靠枕补了一发物理攻击,然后哼地扭过了头,丢下一句「我做年糕汤去了」,穿起围裙去了厨房。\\

周手上拿着被塞过来的靠枕,面对微微不悦的真昼,除了望着她的背影以外别无他法。\\

\vspace{2\baselineskip}

吃完年糕汤之后,真昼恢复了一如往常的表情。

刚开始吃年糕汤的时候真昼还板着个脸让周微微有些不适应,但这年糕汤和年菜都很美味,让周吃着吃着就入了迷,等周回过神来,就发现不知什么时候真昼的心情已经恢复了。\\

两人一起离开餐厅坐回沙发上的时候,一切都恢复了往常。\\

「说起来啊真昼,新年参拜你去吗?」

「新年参拜吗?去倒是不大想去……毕竟我不喜欢人挤人的地方,总有种被盯着的感觉」

「那还不是因为你……」\\

是个不得了的美人啊——周正想这么说,却突然想起自己刚刚还坏了真昼心情,便把这话咽了回去,回答道「这也是没办法啊」。\\

「周君打算去新年参拜吗?」

「在老家那时我都是跟着爸妈一起去的,不过现在还拿不定主意。至少我是想着没必要挤着元旦去」

「同意」

「千岁那俩好像是在千岁家里培养感情,而且说起来现在的孩子们也不怎么会去新年参拜的样子。那就过一阵子再去吧」\\

要跟过去比的话……特别是十几二十多岁的这些年轻人们去做新年参拜的比例似乎少了不少,并不是说周这些人有什么特殊的。

虽然也不是不想去,但周明白人多得动都动不了只会让人筋疲力尽,因而他想着等人少下来了再去也不迟。\\

「再说了,新年前三天还是想悠闲点过啊。我的话福袋什么的倒也不在意」

「我的话倒是对福袋有点兴趣呢」

「你是要去购物中心吗?」

「……我是没有朝着那人堆突击的勇气呢」

「同意」\\

周作出了跟刚才的真昼很像的回复,然后把身子靠在了沙发上。\\

反正,也不是说新年就非得去哪里不可。

周大致上希望避开麻烦事,只要能这样悠悠闲闲地过着日子便十分满足了。而且考虑到方便做饭,整个新年期间真昼似乎都打算在周家里过的样子,这下不管是聊天对象还是伙食都不用愁了。\\

「这可真是个豪华的新年啊」,周这么想着,偷偷瞄了一眼坐在一旁的真昼,轻轻地笑了起来。

\subsection{天使大人与初次见面}

『明天可以去周家里吗』\\

三号,真昼回去之后,周收到了这条父亲发来的短信。\\

『周你不回老家那就算了,但我还是想看看儿子的脸啊。顺便听志保子说了,感觉也得和邻居打个招呼』\\

似乎是母亲她好好地把真昼这号人物和自家儿子受了人家多少照顾告诉了父亲,于是作为家长父亲便觉得有必要向她打一个招呼了。\\

要是志保子她不知道这回事的话周肯定是要全力拒绝的,但现在她不但知道了,真昼自己和志保子之间还有不少来往,周便觉得就算拒绝也是无济于事了。

反正事到如今也没啥要藏着掖着的了,周对父母来视察不回家的儿子这件事本身也没有什么抗拒感。\\

父亲——修斗如果和志保子一起来的话,反而是可以让容易暴走的志保子冷静下来。\\

周料到就算自己现在拒绝了,过后志保子也会强行跑来见真昼,便向先来约时间的父亲回了一个肯定的答复,然后给真昼发了条消息。\\

\vspace{2\baselineskip}

「嗯,那个,我待在你们家庭团聚的地方没问题吗。不会打扰你们?」\\

第二天一早真昼便来到周的家里,显得稍有些紧张。

某种意义上也是当然。毕竟正受着自己照顾的男生的父母突然说要想要见自己。\\

真昼似乎是暗地里和志保子有过联络……不如说是经常从志保子那里发来联系的样子,大概已经是熟悉了吧。光是志保子那还好,偏偏这回父亲也要过来,真昼会感到紧张这也是情有可原。\\

「我说啊,我爸是要跟你打个招呼才来的,我妈的话似乎很中意你的样子不如说你在她还高兴哩。这么一来反而是你不在不行」

「就,就算你这么说……」

「好啦别那么畏畏缩缩的啦。稍微忍一忍的话我会很高兴的」\\

让真昼和自己的父母打招呼这事虽然听起来有些超现实,但既然父母已经有了见面的意思,那就没办法了。\\

虽然占用了真昼的时间这点上有点对不起她,但从父亲的性格来看要是不跟真昼见个面怕是不会善罢甘休的,所以希望真昼能忍一小会儿。\\

「……志保子阿姨她是怎么介绍我的呢」

「放心吧。我跟父亲强调了好多次是恩人的关系,讲清楚了不是我妈那自我幻想时间里的那种关系」\\

好像在志保子她脑子里,真昼已经是儿媳,或者说是可爱的女儿了,所以周全力否定了这一点。

修斗那时也苦笑了会,回答说『是志保子她平日里的坏毛病呢』并接受了,这么一来应该就不会有误解了吧。

\vspace{2\baselineskip}

看着真昼放心下来抚过胸口的样子,周一边苦笑着说着「抱歉啦」一边等待着。正好这时门铃响了起来。\\

公寓大门的话他们手上有钥匙可以直接打开,因而周已经料想到他们会直接进到家门前。

看见真昼的身体悚地一抖,周一边微笑着安慰着她一边起身走向玄关,解开防盗链拧动了门把。\\

打开门之后,门前站着的是周已经见惯了的父母的身影。\\

「半年不见了呢周」

「好久不见,爸」\\

看见露出平和笑容的父亲——修斗,周也同样露出了略带安心的微笑。

身边萦绕着安稳的氛围的修斗,是那种在一起就能让人平静下来的性格,周也一见面就不禁放松了下来。\\

「对妈的时候就是一副鬼态度呢……」

「还不是妈你突然就不请自来。事先说一声的话我就会好好接待啦」\\

主要那时候有真昼在,所以周才是那样的一副态度,要是只有周一个人的话他的态度也会缓和些吧。\\

「总之,进来吧。……这提的都啥啊」

「带了各种各样的东西啦。嘛这个先放一边,小真昼呢?」

「里面」\\

周简短地回答之后,便陪着脱掉鞋子的父母回到客厅,正赶上稍稍有些坐立不安的真昼看向这边——她的眼睛瞪得大大的。\\

真昼的惊讶也不无道理。\\

修斗他那年轻的外貌实在难以想象已经是个三十大几的人了。就算除开从儿子眼里来看的加分,那容貌也还是三十左右的水平。

看着那几乎可以称得上是娃娃脸的年轻而端正的容貌,周已经不知几次想着要是能多继承点那基因该多好了。\\

由于这个男人有着与自己不同的柔和长相,看上去实在是个友善的好青年(虽然年纪上已经算是中年了),所以两人常常被怀疑血缘关系。虽说两人一起走的话看上去倒像是年龄差大的兄弟来着。\\

「小真昼,好久不见呢~」

「啥好久不见啊,还没一个月吧」

「在我心里已经好久不见了哦」\\

看到志保子跑向自己露出满脸的笑容,真昼也摆正了坐姿微微露出外出用的笑容回答说「很久没有见过您了」。

不过,真昼还是以略带困惑的眼神望向修斗,而注意到这视线的修斗则保持着一脸平和的笑容站在了志保子身边。\\

「初次见面。我是周的父亲,藤宫修斗。椎名的事情我已经听志保子说过了。儿子一直都受你照顾了」

「初次见面。我是椎名真昼。我才是,一直都在受周的照顾」\\

配合行了漂亮的鞠躬礼的修斗,真昼也做出了礼貌的问候。\\

真昼担心的,大概是修斗他会不会是跟志保子一样的性格。不过修斗是个温厚而有常识的人,所以真是希望真昼能尽快安下心来。

能控制住志保子的只有修斗一个,志保子也对修斗强硬不起来。虽说喜欢得一塌糊涂也是一个理由吧。\\

「哎呀,没必要那么谦虚哦?反正周是个邋遢仔啦」

「身为邋遢仔真是对不起了」

「好啦志保子,这种话不能说啦。……周,平常一直受人家照顾,有好好地感谢过人家没?」

「有尽我所能」

「那就好」\\

以「应好好对待女性」为教育方针的修斗,似乎是在担心周有没有好好感谢过真昼。

再怎么说,把事全部丢给真昼,自己在一边享受这种事情,周自己心里也过意不去。因而周觉得自己是最大程度上关照着真昼的。\\

修斗听了周的回复放下心来,再次与真昼合上了视线。\\

「……实在是,该怎么谢谢你才好呢。好像不但平常做饭都是靠你,连年菜都麻烦你来做了吧……」

「我一直都很感谢人家,也尽我所能慰劳她啦」

「嗯。……周君也意外地挺关心我的」

「意外是什么啦意外」

「毕竟嘛……」\\

「看上去一副大大咧咧的样子却很能注意到细节呢」,周听到真昼这么说,无法反驳大大咧咧这个事实,结果一时语塞,而看着的修斗则露出了柔和的笑容。\\

「看你们关系这么好就再好不过了。周你也不要给人家椎名添太多麻烦啊」

「……知道的」

「椎名也是,要是周有什么地方不好的话希望你能好好指出来。虽然看上去不像,但这孩子其实还是很坦率的,要是有你讨厌的地方应该会很快改正过来的」

「……周君很温柔,所以,讨厌的地方什么的……那个,只有一点点」

「有呢」

「……与其说讨厌……说是缺点更准确吧」\\

真昼略微害羞,好像有些难以启齿一样,搞得周都想问问自己到底是什么缺点会让真昼说起来这么害羞了……\\

志保子则是不知为何,似是有了头绪般「哈哈」地咧着嘴笑着看向这边。周能做到的就只有瞪着她说「搞什么啊」了。

\subsection{天使大人的憧憬}

「请用」\\

就算是亲生父母也一样是客人,招待也是当然的,不过真昼坚持说要自己端茶上来,于是周便拜托她了。

他还真没想到,真昼为了自己喝而拿来的茶具和红茶,竟能在这种地方派上用场。\\

周的父母坐在这平时周和真昼两个人坐的沙发上,露出了满脸的温和笑容。\\

「哎呀小真昼真是谢谢啦,你已经完全适应了呢」

「是、是的」

「这事原本应该得让周来做的哦?」\\

让周泡茶的话,恐怕只能泡出红茶的涩味,所以真昼才会亲自动手的。然而,志保子却露出了略显无奈的表情。\\

「没有,只是我自己愿意的……」

「也是,要是周来泡的话,热水温度太随便了,也没办法」\\

虽然说得并没有错,但是被指出这些还是有点让人不爽的。

话虽如此,周也无法反驳,只能老老实实地闭上嘴,结果却被志保子笑嘻嘻地看着了。\\

「说起来啊周,开始好好用名字称呼小真昼了啊」\\

听到这突然的一言,周和真昼都僵住了身子。\\

因为叫她名字已经很自然,所以周就把这事忘了。上次见母亲时,周还是用椎名这个姓氏称呼真昼的,而真昼叫周叫得也很别扭。

而现在,看到两人相互之间称呼得那么顺畅自然,就志保子那个性,肯定会胡思乱想的吧。\\

「……有什么关系」

「嗯,挺好挺好,关系亲密是件好事」\\

志保子故意没有进一步追问下去,只是眉开眼笑地观察着周这边。周则感觉到自己的脸上一阵抽搐。

说不定被开玩笑反而还更好一点。这种时候的志保子,脑袋里绝对是在快乐脑补着两个人这样那样的关系。\\

「志保子,别再逗周了」\\

不过,修斗这时踩下了刹车。\\

「志保子这习惯不好啦。别开周太多玩笑了」

「行咯,虽然很可惜但就算了吧」\\

只要修斗说的话志保子都会乖乖听,周作为被折腾的儿子真是对此感激不尽。\\

「话说回来,看到儿子和可爱的女孩子关系那么要好,果然还是很棒的吧」

「我倒是一直担心着志保子的坏习惯会不会失控哎」

「哎呀,修斗会阻止我的吧?」

「虽然我觉得既然有自知之明最好还是改掉,不过志保子这种地方我也喜欢所以没办法呢」

「哎哟……我说修斗你啦」\\

虽然说修斗是阻止了志保子,不过这次父母又开始微妙地形成了二人世界,周也不掩饰自己的叹息了。\\

修斗大体来说是个有常识的人,不过却会无意识间疼爱自己的老婆,有时会产生让其他人难以接近的氛围。

幸好这个样子在家人前才会表现出来,在外是不会产生这么露骨的氛围的。然而,或许因为这里是周的家里,所以修斗就放松下来了吧。\\

长年不减的恩爱在儿子眼里算是表示夫妻和睦的好事,不过周还真是希望他们设身处地,为旁边看到这些场景的自己着想着想。\\

两人一旦变成那副模样,周便不想进去打断,于是就死了心坐到餐厅拿来的椅子上,再次深深叹了一口气。

真昼也坐到准备在旁边的椅子上,静静地看着周。\\

「……你爸妈关系真好啊」

「是啊。虽然在外面不是那个样子,不过在家里就是那种感觉了」

「是吗」\\

周苦笑着回答之后,真昼眯起眼睛看向志保子和修斗。\\

她的表情并没有表示出不快,相反地,是如同看到耀眼的东西时那样的感觉。

真昼的眼神中渗透出憧憬和艳羡,就好像见着什么珍贵的东西一样。\\

看到她以虚幻渺茫的微笑望着两人,周情不自禁差点把手伸了过去——\\

「啊,周,怎么了嘛?」\\

接着,志保子似乎回到了现实世界,周听到了她的声音立刻把手收了回来。\\

「怎么了个什么啦。还不是你们俩沉浸在二人世界让我们待不下去了嘛」

「哎哟羡慕了?」

「没有没有,不存在的。我是觉得这种事情给我在家里做啦」\\

似乎两人并没有注意到周差点去握住了真昼的手。真昼好像也同样没注意到,正因周说的话而露出苦笑。\\

周不知道自己为什么会把手伸出去。

只是,周总觉得……不希望让那样的真昼孤单一人。\\

看到她现在已经回到了平时的样子,周稍微放心了一些,同时为了不被察觉而回到了平时板着面孔的模样。\\

「所以,爸妈看到儿子的脸满意了么」

「看到真昼倒是挺满意了……」

「喂」

「有一半是开玩笑的啦。目的还没有完成呢」

「目的?」\\

周还以为志保子的目的是新年的走访和给真昼打个招呼,然而志保子似乎还有其他的目的。\\

「你们还没去新年参拜吧?」

「我准备等人少一点之后再去」

「对吧?小真昼也还没去吧。发的消息里是这么说的」

「是的」

「就猜到是这样,所以咱把和服拿来了哟~」\\

看来志保子是想和真昼去新年参拜的样子。

事到如今,周终于明白了志保子为什么会满脸笑容地提着一大包行李过来。不知是今天第几次,周又叹了口气。\\

志保子喜欢可爱的东西,也喜欢给人穿衣打扮,肯定是不想放过这次机会的吧。

和服的话,光是周知道的范围里家里就有几件。他们似乎是把这些给带过来了。\\

「我的梦想就是给女儿穿上和服去新年参拜……小真昼的话我觉得肯定适合」

「妈你就是想要个换衣服的洋娃娃吧」

「没有的事哦?不过很大的原因是想让真昼穿上呢」\\

志保子「毕竟肯定很适合」的自信满满的见解是正确的。

不如说,感觉没什么衣服会不适合真昼。\\

在周所见的范围里,男性化的服装、大小姐那样高雅的打扮、平时带着饰边和蕾丝的很少女的服装,真昼都穿过几次,每一种都很适合真昼。所谓美少女,大概是不择衣装的。

和服恐怕也会非常适合真昼吧。\\

藤宫家只有一个儿子,所以想给女儿打扮的志保子似乎不愿放过这个机会。\\

「……要是真昼愿意的话,就让她穿上过去呗」

「为什么说得好像周不去一样?」

「要是让学校里那帮子人知道我和真昼一起出门就不好了吧」\\

如果只是父母和真昼一起去新年参拜,他们看起来就像是一家人,不会有问题。

而如果带上了周就有问题了。\\

外表不显眼的周和真昼一起参拜,如果给同年级的同学见到了,可以想象寒假过去之后将是哀声一片的地狱场景。

再怎么说,周也不会想在承担这种风险的情况下还去新年参拜。\\

「不被发现就可以了吗?」

「可以是可以啦不过正常来说肯定会……我说妈啊,不会是」

「哼哼,就是为了这种时候才拿来了这么多东西的哦?」

「哪种时候啊!?」\\

和服、衬衣、小饰品,周就觉得如果只是这些和服相关的东西的话行李不会那么多,结果看来是为了欺负周而带来了更多的行李。\\

「修斗也很来劲的」

「爸……」

「难得的机会,不是挺好的吗。我是觉得,既然是年度活动,可以的话最好还是一起去吧」\\

被这么一说,周就难以拒绝了。

志保子的提议也包含了修斗重视家庭的意向,周要是拒绝的话,会感觉有些不好意思。\\

「可是啊」

「没问题,相信妈吧。肯定会把周打扮成判若两人的帅哥的!」

「这是在说现在的我很挫吧」

「和修斗长得那么像当然是不挫的,不过发型和给人的感觉都是土里土气的啦。这种的是叫不阳光吧」

「吵死了」\\

周也知道自己土里土气,但周是自愿打扮成这样的,不希望别人一一指出。\\

「要是打扮好的话明明还挺能看的,就是你嫌麻烦……」

「多管闲事」

「真是可惜……我说小真昼啊,你也想看周整理好的打扮吧?」

「咦?」\\

志保子突然将话题抛给真昼,让真昼肉眼可见地惊慌失措着。

尽管周希望志保子不要对真昼那么步步紧逼,然而志保子却是毫不客气。\\

「周要是打扮好的话,我觉得真昼也应该会对周刮目相看的。别看周这样,其实长得还挺不错的哦?他虽然性格不坦率,但是遗传了修斗的绅士风度,只要好好打理就真的是个好男人啦」

「呃,那个……是、是啊……?」

「不想一起去新年参拜吗?」

「想、想去是想去啦,可是」

「喂别出卖我啊」\\

不怕一万就怕万一,周是希望尽可能地拒绝的,而真昼却瞄了一眼吐槽的周。\\

「……周君不愿意的话,那就算了」\\

真昼发出了有些沮丧的声音微皱着眉头,让周突然感到一阵呼吸困难。\\

真昼本人似乎没打算表现出来,然而她明显是一副遗憾的样子。这副样子似乎并不是故意彰显出的,而是自然流露出来的。

她静静摇着长睫毛朝下看着,让周产生了强烈的罪恶感。\\

志保子丢来了好像在说「让小真昼伤心」的指责般的目光,而修斗的视线则好像在说「放弃才更快一点」。在两道视线下,周发出了唔唔的低吟。

这岂不就像是在欺负真昼一样了吗。\\

「……行吧」\\

由于那样的一副表情,周不得不败下阵来。

\subsection{天使大人与新年参拜}

「好了,已经可以了」\\

周被志保子这也不是那也不是地摆弄头发、折腾脸部、搭配服装,在总算得到解放的时候感到了少许的疲劳。\\

周对衣着没有太大兴趣,所以这段时间很是痛苦。不过周照了照镜子,看到辛苦确实有了成效,镜子里映出的是平时的周无法比拟的端正样貌。\\

志保子选择的是深灰色的切斯特大衣、白色的高领衫、黑色的运动裤,这是简洁而又不那么休闲的搭配。

因为是要去新年时期可喜可贺的活动,志保子有注意让衣服不要显得太轻便,目前的搭配给人一点微微的正式感。\\

周并不喜欢花花绿绿的衣服,这黑白而稳重的打扮也是符合周的喜好的。\\

周还确认了一下发型,偏长的前发经由剪子、打蜡和志保子的手艺巧妙地往旁边梳开,露出了平时常常藏在前发后面的眼睛。

将眼睛露出来,使得周给人的印象明朗了许多。不仅如此,周的头发也做成了更加厚实的造型,酝酿出优雅的气质。\\

被母亲和树嘲笑不阳光的周已经不在这里,站在镜子前面的是一个让人刮目相看的清爽男儿。\\

「明明稍微弄弄就是个好青年了,为什么就是不做呢」

「没有兴趣」

「周你老是这样。不过,因为板着个脸,不笑的话也清爽不起来就是」\\

板着脸这句话是多管闲事,然而这是事实所以周也无法否定。\\

「那我给小真昼调整去了,你在客厅等着啊」\\

周是在自己房间搞了这些,所以并不知道回自家换了趟衣服的真昼是什么样子。

由于真昼会自己穿和服,所以她先回了趟家,穿上再过来。从会自己穿和服这一点,就能看出真昼的能干。\\

周目送志保子离开房间之后,再一次看了看镜子里的自己。\\

由于很久没有打扮成这样过了,周觉得自己简直就像另外一个人。\\

「……嗯,应该不差吧」\\

尽管站在真昼旁边可能还显得有些寒酸,不过现在的周比起平时应该是好上了几倍。

稍稍摆弄着不再遮挡视线的前发,周小声地自言自语说,「偶尔这样或许也并不坏」。\\

\vspace{2\baselineskip}

周在客厅和修斗一起等了几十分钟之后,听到了家门打开的声音。

周有听说女人出门的准备需要花费大量的劳力和时间,所以对等待这事本身是没有不满的。然而,他很担心真昼有没有被志保子性骚扰。\\

周迫不及待从沙发上站了起来往门口看了过去,只见真昼静静地走进了客厅。\\

看到真昼的第一眼,周就不禁出了神。\\

平时,真昼不会穿和服,周也没有看到的机会。周尽管觉得应该会很合适——但却没有想到竟然会这么合适。

由于穿着振袖和服在人流中难以行动,所以真昼选择了小纹和服。淡粉色基调的梅花纹小纹和服非常合身,以至于让人怀疑这身衣服是不是原本就是真昼的。

她平时不怎么穿粉色衣服,而目前的打扮在典雅的感觉中还带着女人味。

淡色的长发只留一束在旁,其他部分都用发簪固定在上方。雪白的脖子和摇摆的装饰更加凸显出女性的感觉,非常美丽动人。\\

配合上衬托出原本的美丽的化妆,这一切将清秀美女的氛围体现到了极致。\\

「怎么样?我觉得还满可爱的。小真昼底子好,我这么花心思打扮真的是值了」

「嗯,相当好看」\\

听到修斗直率地笑着这么夸奖道,真昼也有些难为情地垂下了眼帘。连这个动作都那么迷人,所谓美人还真是可怕。\\

「喂,周,不好好说出感想可不行啊」

「我觉得挺好的吧」\\

再怎么说周也没法在父母面前对真昼赞不绝口,于是就送上了不痛不痒的称赞,不过志保子好像非常不满意的样子。\\

「……周就是这种地方不好哦?」

「吵死了」\\

尽管周被志保子批评了,但他并不打算在父母面前说出更多的夸奖,所以把脸朝向了别的地方。

志保子尽管对周感到有些无奈,然而她似乎是很了解周的性格,所以叹了口气就把周放过了。\\

「真是的……话说,小真昼,你觉得怎么样?周这样简直就是变了个人吧?」

「是、是的。和平时完全……」

「平时要是打扮成这样肯定能受欢迎的,可他就是不干,真是亏啊」\\

对周来说这是多管闲事,但志保子正叹着气,好像是真的觉得可惜一样。\\

「明明长得和修斗这么像了,结果还不好好利用,真是让人失望啊。太可惜了~」

「志保子吗,行了行了,周也是这个年纪了嘛」

「那不是应该更想要受欢迎吗?」

「要说的话周是那种只要有一个人就好的性格,觉得其他人都很烦吧」

「哎呀」\\

修斗原本打着圆场,却反而给志保子的妄想点上了一把火。

确实,周比起被众人喜欢,更希望只有一个人在自己旁边……修斗是这么对志保子说的,实际上周也很赞同,然而在现在的情况下,说得不就好像真昼就是那个人一样吗。\\

在志保子光辉灿烂的笑容下,周把抽着筋的脸转了过去。\\

尽管周心想着「为什么就非得胡思乱想不可呢」,但他也知道事实上在其他人眼里就是这样的。

至少,周可以断言说真昼对他而言是特别的。\\

虽然这是事实——\\

周不让真昼发现地偷偷看了真昼一眼之后,轻轻叹了口气。\\

(要说喜欢的话,那肯定是喜欢的)\\

周确实对真昼有好感。

不过,要断言这是恋爱感情的话,周觉得还是有区别的。\\

「妈你想的事情都是不存在的。别说这么多废话了,赶紧去准备开车吧」

「真是没劲……算了算了,修斗,去准备开车吧」

「是啊」\\

周似乎成功地转移了话题,两个人都开始做起了出门的准备。

周将去哪个神社的选择交给爸妈,目送着两人出门前往停车场的背影。\\

「……要带的东西我都在包里了,没太多要准备的。真昼你呢?」

「嗯,都在这个包里」

「这样」\\

突然就只剩下了两个人,周感到有些坐立不安的同时,检查了家里窗户有没有锁好,并且拔下了多余的电器的插座。\\

周关掉客厅的灯,再次看向真昼。

果然,就算不仔细看,周也依然觉得真昼很漂亮。周在父母跟前没有能尽情称赞,然而无论让谁来评判都无疑是和服美人的真昼实在是非常养眼。\\

「怎么了吗,周君」

「嗯,就是觉得这和服真适合你啊。就是那种清秀的和服美人的感觉,挺可爱的」\\

周从女性打扮的时候应该称赞,这件事周本来就从修斗那儿学到了。周本应在看到之后立刻称赞的,不过在父母眼前称赞实在是太难为情了。\\

周说出坦率的感想之后,真昼连着眨了几次眼,接着微微染红了脸抿紧着嘴唇。

想起之前真昼也是这个反应,周露出了小小的苦笑。\\

「啊,你是不喜欢被夸是吧?抱歉」

「不、不是的,不过……周君,还挺」

「还挺?」

「……没什么」\\

看着真昼扭开了脸,周虽然不明不白的,但是看真昼也没有说出口的打算,便只好老实放弃,和真昼一起走向门口。

考虑到走路,真昼穿的不是木屐而是长筒靴,是和洋折衷的风格。即便如此,周大概也能看到她可爱的姿态吧。\\

真昼叮铃铃地摇着发簪的装饰,同时穿上了长筒靴,然后静静走向了提前一步到外面去顶住门的周。

两人的距离,比想象中要更近。很少见的,这次是真昼往周那儿接近之后,轻轻地踮着脚尖。\\

「这意思是让我听她说话吗」,周心想着把门锁上后弯下了腰。接着,真昼将手形成环状捂在嘴前,靠近了周的耳边。\\

「周君」

「嗯?」

「那个……很帅,哦?」\\

小声耳语了短短一句之后,真昼穿过了周的身侧,快步走向了电梯间里面。而周就这么把头嗵地压到了门上。\\

「……太狡猾了」\\

刚才那句宛若回击一样的耳语,让周的心好像揣了只兔子一样怦怦直跳。\\

因为真昼的事情,周花了好一会儿才让一下子变得火辣辣的脸颊冷却下来,结果被提前等在停车场的父母投以了怀疑的视线。

\subsection{牵起天使大人的手}

从住的地方出发,开了将近一个小时车,周一行人便到达了坐落在这一片地区的有名的神社。虽然如他们所料,比起在电视上看到的时候人已经少了不少,但看来还是不至于到没人的地步。\\

「虽然人已经少了挺多了,不过还是剩下不少啊」

「是呢」

「小真昼,不要走丢了哦。虽然我们会留意,你也有手机,要汇合也不算麻烦,但即使如此还是一起去参拜比较好吧」

「好的」\\

身着和服的真昼在一行里行动最不方便,步速也是最慢的。虽说她脚上穿的是长靴,但穿着和服还是会限制步幅,因而走得比较慢也是自然。

虽然到不了得挤着人才能前进的程度,但人还是多到经常撞到肩,所以周这边仍必须留个心眼。\\

「那就出发吧」\\

志保子领着众人扎进人群里,打算首先去趟洗手处\footnote{洗手处:原文为 {\jpfont 手水舎},是指给参拜者洗手漱口以清净身体的地方。}洗手漱口。果不其然,真昼吸引了很多人的眼球。\\

穿着和服来的人也不算少,按理说穿着和服的真昼不会太过显眼,然而事实并非如此。

说到底,真昼就算没有装饰只是穿着校服的样子就已经够吸引人了。清纯系正统美少女的和服姿态,要说不显眼那是不可能的。\\

就连她漱口的动作都是那么美丽,吸引着旁人的视线。\\

「……怎么了吗?」

「没什么」\\

尽管周觉得旁人看着真昼让他不是滋味,但周并没有说出口,而是跟着父母一样洗完手漱完口,然后跟了上去。\\

虽然周也在放慢步子等着真昼,但毕竟真昼不是日常穿着和服,下摆的处理似乎对她来说还挺难,加之人也多,导致真昼的步伐变得比平常要慢。\\

「真昼,还好吗」

「嗯,就这点程度……哎呀」\\

被其他的参拜者撞到了肩,真昼身体失去平衡,眼看就要摔倒,于是周赶紧用手扶住。\\

「看上去不太好啊」

「……抱歉」

「好啦,手伸过来」\\

毕竟现在是让真昼穿着不习惯的衣服走着,很有必要照顾下她。

周把手伸向衣袖中露出的小小的手掌,而真昼则仰起头看向了周。\\

看到真昼这副模样,周觉得她或许不乐意,正打算把手收回来,真昼却慌忙把自己的手放在周的手中,再次仰头凝视着他。这么一搞,周也迷糊了起来,凝视了回去。

两人这么盯了一会之后,真昼先移开了视线,紧紧握住了周的手。\\

连让周表示疑惑的空隙都没有,两人眼看便要顺势走到赛钱箱面前了,于是周一边清楚地感受着牵着的那只手传来的触感,一边把这小小的疑问埋在了心里。\\

\vspace{2\baselineskip}

「花了挺长的时间啊,许了个什么愿?」\\

趁参拜完稍稍离开队伍的时候,周向刚刚静静地许着愿的真昼问道。

真昼以称得上是示范的美丽动作进行了参拜,闭眼合掌的时间有周的两倍长。看到她结束合掌后那优雅的行礼,周差点入了迷,现在回过神才想起来问真昼许了什么愿。\\

「只是无病无灾啦」

「真是个平淡的愿望」\\

要说的话,这倒也是真昼的风格。

周想着真昼这人既没物欲又没钱欲还没名欲还能许什么愿,结果一如所料,甚至让人感觉有点扫兴。\\

「还有」

「什么?」

「……想一直过着这样,平稳的日子」\\

这同样是很有真昼风格的愿望。

这个愿望像是不大喜欢刺激和变化的真昼会许下的,而且也只有喜欢平稳和安宁的真昼才会有这种愿望吧。\\

「要我妈在那可就不平稳咯」

「那样也有那样的乐趣啦」\\

是这么回事么……周虽然怀疑,但看着真昼本人很高兴的样子,便闭上了嘴,以一副温柔的表情牵起了她的手。

毕竟现在还没有完全穿过人多的区域,而且父母已经先参拜完在稍远处等着了,要是走去那里的这段路上摔着了也麻烦。\\

虽然周是抱着这样的想法牵起了手,但真昼却微微眨了眨眼,略带害羞地垂下眼帘,回握住了周的手。\\

「你们两个,这边这边~」\\

志保子的声音十分明亮而富有活力,很容易分辨出来。

像是被催促着一样,两人走向父母所在的地方,这时志保子则瞪大了眼,然后用手捂着嘴,似是在微笑般地望着这边。\\

「哎呀哎呀」

「咋了啊」

「想着你俩怎么就自然地牵起手来了呢」\\

听到志保子这么讲,周才反应过来自己牵着真昼的手走到她面前这一失策。

这岂不是在说,真昼对周来说是特别的存在了么。志保子胡思乱想后整天这么坏笑,对周来讲可是一点也不好笑。\\

「……是为了不让她走丢啦。而且穿着和服还很容易摔着」

「说的也是。穿着和服很难走路,确实需要一个护花使者吧。我可是保护着志保子呢」\\

修斗是个明白人,没有对周牵着真昼的手一事感到奇怪。他也和周一样,轻轻地牵着志保子的手。

要是能像父亲那样灵巧地伸出手牵起对方的话,那就没这么多累人事了,但周从性格上便做不到,因而他很感激真昼坦率地把手牵了过来。\\

看着志保子的注意力转向了修斗,周松了口气,正想悄悄松开手,可真昼却没有放松手上的力气。

虽然动作很轻,但周还是理解了真昼不愿松开手的意思,轻声问她「怎么了」,却也没有得到回答。她仅仅是用细细的手指抓着周。\\

「小真昼小真昼,我打算去买些热饮,汁粉\footnote{汁粉:原文为 {\jpfont おしるこ},一种日本的红豆沙甜品,一般放入麻糬等食用。}和甘酒\footnote{甘酒:又称醴,是一种甘甜的日本传统浊酒,以白米发酵酿成。}你要哪种?」

「那我就要汁粉吧」\\

由于志保子的打断,周错过了提问和放开手的时机,只好继续握着那娇嫩的手。\\

「你呢?」

「……那就甘酒」

「好好」\\

不过,要是真昼不讨厌的话那这样也不错——周抑制并忍住那心中泛起的微微瘙痒感,告诉了志保子自己要什么,然后重新握紧了真昼的手。\\

\vspace{2\baselineskip}

没多久,志保子就从店里回来了,并把买来的各种东西分了下来。再怎么说这时候不放手也没法喝,于是两人便暂时松开了手稍做休息。\\

父母则一起喝着甘酒放松地笑着。

虽然不至于进入二人世界,但两人还是亲亲热热了起来,所以周也没什么兴致搭话,喝起了刚刚到手的甘酒。\\

虽然甘酒很有营养,被誉为能喝的点滴水,但令周享受的还是米中沁人心脾的甘甜与回韵。一口下去,周不禁叹出一口夹带感叹和安心的吐息。

虽然汁粉也难以舍弃,但既然是新年,考虑到气氛,周便选择了甘酒。从个人喜好上来看是选对了。\\

周瞄了一眼真昼,发现她神情安稳,一点点地喝着纸杯里的汁粉。\\

「汁粉好喝吗?」

「很好喝哦」

「让我尝一口」

「给。我也能尝一口吗」

「嗯」\\

难得有这个机会,两人便决定交换热饮各尝一口。周换过了杯子,将那微粘的红豆色的汁粉送到了嘴边。\\

嗅着空气中飘着的红豆独特的香味,周将汁粉含在嘴里,一如所料有种甘甜而浓厚的风味扩散开来。觉得略微有些过甜,大概是因为周不那么喜欢吃甜的。

真昼似乎是挺喜欢甜味的东西,这个甜度对她来说或许正好。\\

「好喝」\\

真昼似乎也挺中意甘酒,微微弯起眼角露出了笑容。\\

「……还真是自然呢」\\

守望着两人的志保子小声地感叹道。\\

「咋了啊」

「不用在意哦……今天是个冷天挺幸运呢」

「明显是天气暖和更好吧」

「你们俩说不定是那样,我们的话……是吧?」\\

志保子向同样守望着两人的修斗寻求同意,修斗则以平和……而微妙地混有苦笑的笑容,微笑着回答道「确实呢」。\\

在那微妙的温暖视线中,周略感不适地抖了抖肩膀,而真昼则以不可思议的眼神望着那样的周。

\subsection{天使大人与幸福的氛围}

「小真昼,料理真好吃啊」\\

从新年参拜回来后稍微休息了一会儿便已是傍晚,真昼换上了衣服如常开始了晚饭的准备……不过志保子要在周家里住一晚,正为了观察真昼的手艺而待在厨房。\\

周的老家在几个小时的车程之外,所以志保子和修斗都很累了,而且他们好像原本就打算住一晚。虽然周希望他们取得家主的许可,但原本的家主是修斗所以周也没法抱怨。\\

所幸为了以防万一,被褥多准备了一套来客用的,大概他们两个人会一起用吧。反正在老家他们也是一起睡的,并不会有多大变化。\\

「谢谢阿姨」

「明明是女高中生居然这么能干。我读高中的时候可做不成这样」

「老妈你现在也没真昼能干吧」

「你刚说了什么吗」

「没」\\

听到厨房传来了猛降一调的声音,周装作什么都不知道,靠在了沙发上。\\

在一旁休息的修斗责备周说「别老欺负志保子」,但是平时被欺负,啊不,被调戏的都是周这边,所以这点报复应该是在可以接受的范围吧。\\

「真是失礼」的声音从厨房传向了装傻的周,但志保子马上又恢复成了明亮的声音向真昼搭话。

而真昼面对志保子的搭话也没有迟疑地应答着。似乎她已经基本习惯了志保子的气势和性格,脸上的表情很安稳。

周在远处看着两人关系良好地做着饭菜,放下心来轻轻地叹了一口气。\\

「志保子她,真是中意椎名啊」\\

修斗同样望着两人的背影,微微地笑着。\\

「嗯,又能干又可爱性格还好,老妈中意她也是自然的嘛」

「那周怎么想」

「……没什么特别的,只是觉得她是个好人,还挺可爱的」

「这样啊」\\

周一度以为这是若无其事的确认,但修斗的性格不是会深究的类型,所以大概只是他单纯地对周的想法感兴趣吧。

而修斗也没有再过多地追问周的回答。\\

「能让周想要每天吃的料理,真是期待啊」

「味道绝对可以保证。只要老妈不干多余的事」

「不用担心,志保子也想吃椎名的料理,顶多也就是帮帮忙啦」

「那就好」\\

并不是说志保子做的饭不好吃,只是她和真昼细腻的调味不同,大多是粗略的调味。

细腻的调味这活一般是修斗来干,而志保子会优先考虑分量和开心。\\

当然,志保子是有着正值大胃口时期的儿子的主妇,这样做几乎是理所当然的,但周的喜好是真昼精雕细琢的味道,要是真昼的料理的魅力受到影响就不好了。\\

好在志保子似乎也只是在给真昼帮帮忙,并没有在此之上的举动。于是周安心地叹了口气,望着两人料理的情景。\\

\vspace{2\baselineskip}

「嗯,真美味」

「谢谢叔叔」\\

不管怎么说平常那刚好只够两个人用的餐桌,不可能让四个人围着吃饭,所以最后晚饭是拿出了储藏室里收着的大一号的折叠桌来吃的。\\

真昼听到修斗坦率的感想而放下心来,身体变得不那么紧绷了。

除了料理实习课,真昼似乎从来没有给周以外的人做饭吃过,所以显得有些紧张……但因为修斗那温和的笑容,真昼终于不再僵硬了。\\

「真的很好吃啊。这样的话不管是一个人住还是结婚都不用担心了」\\

志保子望着这边感慨地念叨着。周尽管因此而差点脸上抽筋,但仍然面无表情地啜饮着味噌汤。\\

周已经相当熟悉了高汤带来的浓郁的味道。

周完全习惯了真昼的调味,已经不怎么想吃真昼的料理以外的东西了,这就是每天都吃真昼的菜的缺点吧。\\

「周,感想呢?」

「当然很好吃。一直以来都谢谢了」\\

虽然就算志保子不要求,周也打算这么说,但听上去就像是被催着才说的。

两人独处的时候,周每天都没有忘记赞美,但这次父母在场,所以周克制了一下。虽说结果还是失败了。\\

这次周也一如既往地做出称赞,但真昼好像有些心神不宁,倒不如说好像不舒服一样扭动着身子,小声地回了一声「……嗯」。\\

她脸上有一抹淡淡的红晕,大概是因为周的父母在这里吧。

接连不断地从三个人收到赞赏,即使真昼习惯了周的感想,肯定也多少会有些害羞。\\

「小真昼真可爱啊」

「志保子,别太捉弄她了」

「我没打算捉弄她啊。真的,我只是觉得她是现在难得一见的纯洁少女哦?」

「没、没有这样的事……」

「有的有的。该说是纯洁还是纯情呢」

「周君!?」\\

纯洁肯定是没错的。真昼面对不怎么帅气的男人把衬衫前面打开也会满脸通红,可以说是纯情又天真了吧。\\

「哎呀哎呀,在我们不知道的时候发生了什么吗?」

「没什么」

「什么也没有!」\\

从真昼那也传来了否定的声音。

说纯情、纯洁这些并不是在贬低真昼,但真昼似乎是不怎么喜欢被这么说,正强烈地否定着,所以周也就没有再继续说下去。\\

「对我来说,只要周不做出会让椎名受伤的事剩下的就随便了。但捉弄她还是要有个度哦,周」

「我知道的啦」

「……看吧,这不就是在捉弄我吗」

「纯情可是真心话」\\

真昼坐在旁边,啪嗒地打了一下周放在桌子底下的腿。

真昼脸颊稍红地微微瞪着周,在周回答「抱歉抱歉」之后,她端整的容貌上浮现出赌气的表情。这番举止微妙地有些可爱,但周为了不惹真昼生气,忍住没有笑出来。\\

「……怎么说呢,这个,我们秀的东西又在眼前被秀了回来啊」

「不也挺好吗。周的表情也比平时柔和多了」\\

\vspace{2\baselineskip}

「唔,抱歉啊妈妈他们的份也让你来做」\\

晚饭结束之后,众人谈笑了大约两个小时,便解散了。话虽如此,由于父母会在客厅睡觉,要回家的只有真昼一人。\\

因为父母先去洗澡了,所以只有周一个人出去送她。虽然没什么送行的必要,但姑且是以防万一,同时也有为今天志保子他们脱线行为道歉的考量。\\

「啊,没事的。我还挺开心的」

「这样啊」\\

所幸她好像没有心情不好的样子。

倒不如说,可能还挺开心的。\\

「……而且……」

「而且?」

「……稍微,明白了一点,幸福的感觉」\\

随着如同微弱的叹息似的呢喃,真昼脸上浮现出伴着寂寥的笑容。

那副笑容,仿佛风一吹便会消散一样。周能够发现真昼眼睛中混着的微弱的憧憬,是因为察觉到了她的家庭环境吧。\\

周感觉不能放着她不管,便忍不住把手放在她的头上,故意用有些粗鲁的动作揉着。\\

真昼并没有露出不乐意的表情,只是吃惊地抬头看向周。\\

「干、干嘛啊」

「没啥」

「没啥是什么啊……头发都乱了」

「反正回去要洗澡吧」

「这倒是没错啦」

「……不行吗?」

「也、也不是,不行……至少,先跟我说一声嘛」

「摸了」

「那是先斩后奏」

「抱歉」\\

「只要事先说明就会给我摸吗」周产生了这样的想法但没有说出来。在他坦率地道歉之后,真昼轻轻叹了一口气。\\

「真是的……我的话还好,随便乱摸女孩子的头真的不好哦」

「不是,又不会摸别人……」\\

可以触碰异性的身体的,基本都是关系很亲近的人,这一点周还是清楚的。像现充一样随便进行身体接触之类的,周怎么样也做不到。\\

姑且,周把真昼算作比较亲密的人,所以会一边确认着真昼并不会讨厌一边摸着。但是,周并不会想对真昼以外的人这么做。

说到底,周对其他人连身体接触的想法都根本不会有。这种希望,最多就是惩罚恶作剧的千岁的时候才会产生。\\

周补充说他不可能摸其他人后,真昼就变得老实起来,没有甩开头上的手。\\

「……看到之后就在想,周君和修斗叔叔实在太像了。就算我才刚认识他,也看得出来」

「哪里啊。性格和长相都不太像吧」

「……很像啊,真的」\\

真昼大大叹了口气,而这次周稍微有些生气地来回揉了揉真昼的脑袋,但她似乎并不讨厌。\\

(……有那么像吗?)\\

确实,站在一起的话,会有人误以为两人是年龄相差大的兄弟,但是,周和修斗的氛围其实完全相反。

性格也是,虽说不算完全相反但也肯定不相似。\\

然而她却说很像,这是怎么一回事呢。\\

虽然周脑中浮出了好几个疑问,但真昼好像没有说下去的打算,眯上了眼睛任由周摆布。\\

摸够了之后周便拿开了手,紧接着真昼就突然回过神来,看向周的眼神带上了微妙的慌乱。\\

「怎么,还想我再摸一会儿吗?」\\

周抱着捉弄的心态问了问,而真昼微红着脸回道「请别捉弄我了」,周便决定就此作罢。\\

真昼似乎是稍微有些不高兴,露出一副不满的表情打开了自己家的门钻了进去。\\

在周正后悔自己有些做过头了的瞬间,真昼从门缝窥向了这边。\\

「……周君」

「怎么了」

「……笨蛋」\\

真昼脸颊上挂着淡淡的红色,编织出好像在闹别扭,同时又稍稍带着一点撒娇一样的话语,然后关上了门。\\

(……谁才是笨蛋啊)\\

都怪真昼,让周的心脏突然猛跳起来。\\

周轻轻叹了一口气,接着为了让稍稍发热的身体冷静下来,他靠在走廊的墙壁上,吐出了仿佛比平时更白的气息。

\subsection{天使大人与新学期}

新的学期开始了,但并没有发生什么太大的变化。

尽管大家都度过了一个各随所愿的寒假,但并没有什么像暑假回来那样的变化。没有谁下狠心来一个形象变化,班里也还是原来这群人。\\

周静静地坐在座位上,观察着比平时更加喧闹了几分的教室。这时,一个人从旁边走了过来。\\

「哟,周,挺有精神的啊」

「托您的福」\\

在周之后来到教室的树也没有什么变化。

虽然从圣诞以来周就再也没见过他,不过他还是带着一如既往的轻浮的笑容。\\

「怎么样,过了个好年吗?」

「……嗯,算是吧」

「怎么欲言又止的。是有了什么进展吗?」

「进展什么啊我说……不是那回事,什么也没发生」\\

其实并不是什么都没发生。虽说并不是出自双方的本意但真昼还是在周的家住了一晚,然而这种事情周不可能说得出口。

不难想象,跟树说了他就会跟千岁说,然后两个人就会过来嘲笑捉弄。

除此之外就只是父母来这边去了新年参拜,应该可以算在什么都没发生的范围之内吧。\\

「……哦~?」

「什么都没发生哦」

「你这么说的话就当是这样吧」\\

虽然周有些不爽他的坏笑,但由于吐槽起来太麻烦,周就随他去了。\\

「找点东西来转移话题吧……」周抱着这样的心态环视着教室,但并没有什么特别的事情。

女生们一如既往地围在王子也就是门胁周围。无论是他在包围中略带困扰的表情,还是周围的男生们的嫉妒,都没有任何改变。\\

「还是一如既往啊,那个」

「那也是平日的光景了」\\

完全旁观的周,以及有着女朋友因而对其他女生没兴趣的树,看到门胁的人气,只能苦笑着环视周围,找找有没有什么别的新鲜事。\\

「说起来,听说椎名好像有男朋友了」\\

听到几个女生围在一起讨论的内容,周僵住了身子。\\

「啊,丽萨是这么说的吧。去新年参拜的时候看到她和一个男的牵着手」

「是啊是啊。椎名完全没有和别人交往的意思,是不是因为有男朋友了啊」

「听丽萨说,那个人长得还挺帅的,不过没在学校里见过。她怀疑是别的学校的人」\\

不知是不是错觉,似乎班上的视线都朝向了那几个谈话的女生。就连那个门胁,似乎也朝着她们竖起了耳朵。

只有树的视线倒是朝着周这边。\\

「周啊」

「不知道」

「我还没说呢」

「跟我没关系」

「行吧」\\

面对小声而断然否定的周,树一阵苦笑,然后突然掀起了周的刘海。\\

「我说你,长得还可以嘛」

「你这么说我只感觉是在搞我」\\

树虽然性格轻率,长相上给人一种轻薄的感觉,但算得上是帅的那一类人。

这样的帅哥说别人长得好也只感觉是在挖苦。\\

对于自己的外貌,周自认为尽管还行但并不算什么帅哥,所以并不想听到这种对外貌的评价。\\

周甩开了树碰着前发的手,皱起眉头之后,看到了树苦笑的表情。\\

「你就是这样的家伙啊」

「就你多嘴」

「也是,挺有你的风格的」\\

面对一如既往冷淡的周,树没有生气反而笑了起来。\\

\vspace{2\baselineskip}

「你的事情在学校传开了啊」\\

晚饭后,周在餐桌前面对着真昼感叹了一句,真昼也理解其话中的含义,表情变得僵硬起来。\\

最困扰的肯定还是真昼吧。

就听到的传闻而言,那个人是周这件事似乎没有暴露,但真昼突然被追问是不是有男朋友了,肯定还是很累的。今天真昼来到周家里一直有些微妙的生硬,步子也显得沉重,应该就是因为这个吧。\\

「……没暴露是周君还算好,但是发生了好多误会,解开这些误会花了很大功夫」

「只要牵个手就是男朋友了吗」

「不知道。总之先干脆地否定说只是熟人,之后就只能等流言自己消散了」

「嗯,也没别的办法」\\

不管怎么说自己被当成真昼的男朋友还是很对不住她,所以周希望流言能尽早消散。老被别人询问说那个人是不是男朋友,感觉会给她带来压力。

从周的角度来说也是一样。每次听到流言时,周的心里既有歉意又有难为情,从而冷静不下来。所以,周也希望他们赶快忘掉。\\

周长叹了一口气,而真昼只是轻轻垂下了眼睛。\\

「……就这么,看起来像恋人吗」

「不知道啊。在我看来,我这样的人肯定当不上真昼的男朋友。你怎么都会选个更能干的帅哥,而且就算和我站在一起,比起男友来说也更像是普通的熟人吧」

「才不是什么『你这样的人』」

「嗯?」\\

听到比预想更大声的回应,周不由得重新看向真昼,而真昼一改之前略带惆怅的表情,不知为何脸上浮出了有些……生气的、果决的表情。\\

「周君对自己的评价很低,但其实不是那样的。我觉得周君是一个很优秀的人。又温柔、又善解人意、又绅士,而且,人品非常好……用心打扮的时候,也非常帅气,嗯」\\

一直听着不像是在说自己的赞扬之辞,周的脸颊自然地红了起来。

周没想到真昼居然如此看好自己,再加上她说得太过认真,所以被夸奖的周不由得害羞了起来。\\

真昼也慢慢意识到了自己说的话有多羞耻,说到一半便开始吞吞吐吐。

即便如此,真昼依然看着周的眼睛,表示出刚才说的都是真心话,而这让周更加害羞了。\\

「这、这样啊……谢谢」

「……所以说,那个,这个……请不要,那么贬低自己」

「哦、哦……」\\

被如此直白地夸奖,周也无法否定,因为现在的氛围连谦虚都无法容忍。

脸颊染上淡淡红晕的真昼低下头,因为羞涩而微微颤抖着,周也对心中扬起的羞涩和焦躁无可奈何,小声地念叨起来。\\

「……那个……我去洗碗」

「好、好的」

总之周现在能做的,就是打个马虎眼赶快逃开。

这也可以说是战术性的撤退。一直看着她因羞涩而颤抖的身姿,对心脏相当不好。\\

周一个深呼吸之后,站起来把碗筷端到水池那边,而真昼陷进了客厅的沙发,脸埋在坐垫后面。她似乎也因为刚才没习惯的称赞而羞耻着。\\

面对着如此身姿,周小声地念道「这么害羞的话就别说啊」,却同时又觉得,因为真昼的称赞,自己的心里轻松了几分。

周大概是因为被肯定而安心了不少吧。\\

尽管周心里这么想,但害羞的事情依旧是害羞的,所以,明明是寒冷的冬天,周却用冷水心无杂念地洗起了盘子。

\subsection{天使大人与朋友}

『我说~周,把天使大人借给我好么?』\\

吃完晚饭以后,千岁打来了电话。\\

周平时都是用App和千岁发消息交流的,不知为何这次她却打来了电话,而且还是从周这里询问真昼的事情,不清楚她到底什么意思。

就算说要借,真昼也不是周的所有物,想约的话还是应该去问本人吧。\\

「不要问我啊,问椎名去」

『她现在在你那里么?』

「……在是在啦」

『那你帮我问问明天放学后要不要一起玩吧』

「自己问」\\

「这家伙没有问联系方式么」在这样一想之后,周忽然想起,圣诞节那会千岁一直在拼命开真昼的玩笑,没顾得上要真昼的联系方式来着。

由于周毫无疑问拥有真昼的联系方式,而且还频繁待在真昼身边,所以千岁就来联系他了吧。\\

虽然千岁的想法可以理解,但周还是想对她说自己并不是她的信鸽。\\

总之还是让她和本人商量比较好吧。周这么想着,把手机递给了旁边云里雾里的真昼,告诉她「千岁找你」就自己靠在沙发上了。

真昼看起来有点为难,但还是老实地接过手机放在耳边。\\

「你好……咦,明天吗?是、是的,并没有什么安排……」\\

真昼大概被千岁的连珠炮缠住了吧。周看着真昼为难的样子苦笑起来。

她似乎没有不耐烦,而只是一味地被唐突的提议惊讶到,对怎么应付有些不知所措吧。\\

真昼往周这儿瞟了一眼,所以周回了一句「要不要去还是你自己决定。她是想和你一起玩而不是我」。\\

真昼偶尔也会和朋友出去玩,可是不过几个小时她就会优先回来准备晚饭。

周觉得她偶尔也该歇一歇。千岁那样的蛮缠能不能算是「歇」先暂且不提。\\

「嗯、嗯……那个,那么就这样说定了……」\\

也不知道真昼是不是因为周的那句话而下的决定。真昼回复千岁时,周甚至听得到电话另一端传来一句「太好了!」的轰隆,真昼更是下意识地把手机拿开了耳朵。\\

周对千岁实在太高的兴致有些受不了地笑了笑,然后和真昼对上了视线。

她看起来也有些为难,尽管如此,她的嘴角还是浮现出一丝带着安心和喜悦的微笑。

等到千岁安静下来以后,真昼又拿起手机和她聊了起来。\\

看到那个让人欣慰的样子,周微微笑着注视着她。\\

「谢谢,手机还给你」\\

一挂掉电话,真昼便把手机客气地送回到周的手里。

看来事情已经谈妥,明天她就要被千岁带到哪里去玩了。\\

「很突然吧,千岁基本上都这样」

「嗯,吓了一跳呢」

「那家伙也不是什么坏人啦,就是有点强势」\\

虽然周觉得并不是「有点」的等级,但周还是做出了比较温和的评价。她肯定算不上坏孩子,只是有点咄咄逼人。\\

真昼大概也是早就理解了千岁的性格,她虽然苦笑着,但幸好并没有觉得厌烦。周可以说是树的好朋友,但却和他的女朋友合不来,这样的事虽然也不稀罕但还是让周有点悲伤。\\

「明天也不用把我放在心上,好好玩吧」

「好」

「……啊对了」

「怎么了」\\

虽然周很想让她好好享受,但还是不得不提醒她一件事。\\

「要是被性骚扰了不用顾虑直接揍她吧。那家伙像我老妈一样,很喜欢可爱和漂亮的东西,遇到像你这样的美女估计会想要摸个遍的」\\

尽管上次姑且是阻止住了,但千岁实在是喜欢可爱的东西。

虽然真昼生日那会儿周依靠了千岁的慧眼,但周还是有点不放心真昼单独和她待在一起。\\

真昼是有着「这才是算得上美少女」的外貌的少女,那种可爱和漂亮仅仅走在街上就可以引人注目。

虽然想让她提防别人的搭讪,不过千岁的魔手也是得留意的。\\

「我觉得你要是不喜欢的话她也不会做这种事,不过如果不拒绝得干脆点的话,她可能会得意忘形然后就一直缠着你了,要注意一下……怎么了吗」

「……没什么」\\

看到真昼紧紧闭着嘴唇的样子,周觉得有些奇怪,但是真昼没有说出她在想什么,而是静静移开了视线。\\

\vspace{2\baselineskip}

等到和千岁一起去玩的那天,周也久违地早早回家度过了一段安静的时光。\\

最近真昼总是待在周的身边,周像这样独自一人的时间也就只有假日了。

即使是假日,真昼提出做午饭的时候,周总是会接受她的好意,所以他就更少一个人度过了。\\

当然,周也不是讨厌那样……甚至还觉得蛮惬意,但偶尔这样一个人的空间也挺好的。

虽然说,周倒是觉得旁边有点冷清。\\

(到最后,真昼也完全习惯了我家啊)\\

周已经产生了一种真昼理所当然会在自己身边的感觉,但实际上自他们相识以来也才经过了几个月而已。\\

尽管那样,感觉两人的距离好像一起过了好几年一样,估计是因为他们间的相性很好吧。

不过度干涉,呼吸同一片空气——这样的距离,对于周来说感觉很舒服。\\

头疼的是,周已经舒服到了不想放手的程度。\\

(我还真是单纯)\\

要断言有明确的好感,两人之间又没有那种热切。但仅仅作为邻居和朋友来说,周的独占欲又太过强烈了。

周对真昼有着朋友以上的好感,尽管如此,把她视作恋爱对象这种想法仅仅是有一点点火星而已。意识到这一点,周感觉到了心里有一阵说不出来的瘙痒。\\

再这样下去继续将好感的天平倾向真昼的话,恐怕就没法再回头了。

所以,周把心中的微热收进心底,掩饰起来。\\

如果示好的话,真昼只会觉得困扰吧。

虽说真昼对周也表示出了一定的好意,但是周觉得那不可能是出自恋爱的感情。不如说,怎么会有人喜欢上这么添麻烦的废人呢。\\

虽然真昼曾给予过周肯定,但周觉得她根本不可能喜欢上自己。如果对真昼表达出方向错误的好感,只会让双方的关系变得不干不脆的。\\

周把心里焦躁不安蠢蠢欲动的感情压制住,接着静静地往窗外看去。\\

冬天的晚上总是来得很快,周围已经罩上了一帘昏暗的帷幕了。

虽然现在才刚过六点,但主观上也可以说已经到夜晚了。\\

毕竟是千岁,不可能带着真昼逛到很晚。但天这么黑,让两个容貌端正的两个女高中生在外面晃,周还是有点不放心。\\

『什么时候结束?』\\

周给手机从不离身的千岁发去消息,立刻就收到了『马上就散咯』的回复。\\

看来千岁也没打算放学后玩多久。周松了口气,又问了句什么时候会到车站,然后从沙发上站起来,往洗手间走去。\\

(前些日子的发蜡还有一些吧)\\

尽管周不太愿意,但既然要在外面见真昼,也没办法了。\\

虽说周基本上自己不怎么想弄这些,不过父母把让自己变帅的方法全部都教给他了。只是那时的发型的话,还是可以重现的吧。\\

周看向镜子,里面映出了平时那样沉闷阴暗的自己。\\

周拿起了发蜡,亲手改变起了那个土里土气、一点也不时髦的自己。

\subsection{迎接天使大人}

现在是隆冬时节,没有太阳的晚上气温很低。\\

考虑到防寒和装饰性,周穿的是浅灰色的毛衣、藏青色的水手外衣,还有内层绒毛的黑色卡其裤。然而即使这样周依然觉得冷,那穿着校服加外套的真昼该有多冷啊。\\

尽管真昼在冬天会穿厚实的紧身裤,然而那显出女高中生范儿的,高度勉强不至于违反校规或者不雅的裙子,这么冷的样子,使得周都想让她穿上运动裤了。

偶尔擦肩而过的女高中生也摇晃着短得没意义的裙子。周痛切体会到,女高中生对于美的努力有多么可怕。\\

想着这些事情,周用真昼送的围巾捂住嘴角,快步前往最近的车站。\\

似乎真昼是去了大型商业设施,路上用到了电车。距离公寓最近的车站是能步行走到的距离,而且听千岁说电车马上就要到了,所以现在差不多时候正好吧。\\

周在行走的时候,风轻轻吹动了周弄好发型的头发,但并不至于把头发吹乱。

如果头发乱成一团就不得不修了,很是麻烦。周心里想,平日里就打扮的人类真是值得尊敬啊。\\

周想着这些事情默默走着,接着就看到了车站。

考虑到公寓的方向,真昼应该会从这个口子出来。在车站口附近等着的话,应该就能确保遇上真昼了吧。\\

周倚靠在车站口的墙上,一边看着时间一边等着真昼。没过多久,车站里就走出了眼熟的亚麻色直发少女。\\

「真昼」\\

周一搭话,真昼听到熟悉的声音就没有警戒地回过头——接着,视野里看到周的瞬间就愣住了。\\

「诶……嗯?为,为什么」\\

为什么,指的应该是这身打扮的事情吧。

周会来迎接这件事恐怕千岁已经告知她了,但她似乎没想到周会以新年参拜那时的样子过来。\\

再怎么说,周也不会想用平时随意的样子和发型就直接过来。

如果被周围人看到,将谜之男人和周等同起来的话,周会很困扰。而且,要在真昼旁边走路,如果一点样子都没有的话,连真昼都会被轻视。

尽管周的目的是乔装,但至少也应该把仪容整理到能站在真昼旁边的程度吧。\\

「是想着我自己办不到么。再怎么说也不可能用平常那副样子来接你啊」

「……这倒是啦」

「不合适吗?我对着镜子确认过来着,还是很奇怪吗」\\

周一身平凡而朴素的穿搭,发型则是和前几天新年参拜的时候用的一样,因而自认为应该不算奇怪,不过从美感优秀的人看来说不定还是不行。

周偶尔感到有视线往这边看,或许也有可能是因为自己的样子很奇怪吧。\\

周尽管做了挺多打扮,看来却还是显得土,这让周稍稍有点受打击。但真昼却连忙摇着头给予肯定说很合适,让周总算松了一口气。\\

「那就好。行吧,现在大冬天的一会就要天黑了。而且一个人回家还很危险」

「……这,这种事情我还是知道的啦」

「还是说,不想我来接你?不想一起走的话那在后面跟着也行。我走前面」

「我,我没有,讨厌的意思。那个……谢谢你了」

「嗯」\\

看起来没有被讨厌,安心下来的周从口袋里掏出了手伸向真昼,真昼则慢慢地把手放了上去。

或许是天气太冷,手上传来的感触比预想的还要冰凉。\\

「手这么冷啊。手套呢」

「今天拿去洗掉了。倒是周君你的手怎么这么热啊」

「我是把手抄在口袋里了啦」\\

周是以两手抄着口袋这种不希望好孩子照做的姿势过来的,因而也没什么好称道的。\\

除此以外周没有说更多的话,只是像包住般握紧了那只纤细的手。

真昼的手,真的是纤细而弱不禁风,轻易地就被周的手裹了起来。\\

「……好暖和」\\

真昼轻轻感叹道,如同在微笑般眯细了双眼。\\

那纯真的表情令周不禁心脏加速,但周还是把意识集中在握着的手上,没有把这悸动表现出来。

周把手握了上去,顺带把装着她和千岁购物成果的包包袋袋接了过来,然后迈开了步子。\\

真昼突然抬头望向周,周回道「咋了啊」。

真昼盯了周一会,最后总算是移开了视线。

那微微泛红的耳朵和脸颊,不知是寒冷,还是看太久了的羞耻感所致。\\

「好啦,回去了。要顺便去趟便利店么?这个季节肉包子很好吃哦」

「……我喜欢豆沙馅的」

「你还真喜欢甜食啊……晚饭吃啥?」

「备好了溏心蛋、叉烧和笋干那就吃拉面吧」

「大冷天吃拉面听着就很有胃口呢」

「确实呢」\\

虽然周没在意过冰箱不大清楚,不过听上去这些都是以前就买好了的。\\

虽说汤料和面确实只能去买,但配菜都是精心手工制作的。仅仅是想象那厚厚的叉烧和入了味的溏心蛋,周便不住地咽起了口水。

那味道,一定能沁入这寒冷的身体吧。\\

「……吃完豆沙包不知道还吃不吃得下呢」

「那豆沙包就一人一半好了。这样就吃得下了吧」

「……嗯」\\

真昼略带害羞地接受了周的提议,令周微微笑了起来,牵着真昼的手握得更紧了一点。\\

\vspace{2\baselineskip}

「椎名啊,好像又被看见和那个男人走在一起咯」\\

第二天,周被树以「流言还没散呢你咋又添了一把火啊」这样的眼神看着,说了一句「你问我我问谁」别开了头。

\subsection{天使大人与朋友的烦恼}

进入二月,真昼的「疑似男友的谜之男子」的传言终于平息了。

关于这一点,在那之后周没有再和真昼在外见面也是一个很大的原因吧。尽管周去接她的时候引燃了谣言,但在这之后并没有其他的音讯,火焰最终姑且算是熄灭了。\\

即便如此,「不是恋人但和真昼相当亲密的男人」这个认知似乎已经定格了下来,并且也有散布着真昼对那个男人有意思这样的无根无据的传言……不过真昼本人以不容追究的笑容否定,那方面的传言也总算是渐渐平息了。\\

千岁似乎是在走廊目击了那个情况,听千岁说,真昼「有一种不由分说的威压感」。由此可见,真昼应该也是相当讨厌那传言吧。\\

虽然那也是理所当然的,但被全力否定到那种程度,周感情上还是稍稍有点悲伤。不过,他同时也认为这是没有办法的。\\

明明真昼那边没有恋爱感情,只是有亲近的感觉就被别人胡乱猜疑,那肯定会有生气的想法吧。\\

周自己也只能苦笑了。\\

\vspace{2\baselineskip}

「说起二月?」

「期末考试」

「喂,为什么正值花季的男高中生会有那种土气的想法?」\\

放学后,千岁跑来——或者说是不请自来——到了周的家里。她听到周的回答之后,毫不掩饰她无语的心情。\\

听千岁说是有事要商量还是什么的,但是也许是心理作用,周感觉她过来就是为了找真昼玩的。

顺带一提真昼在厨房泡茶,客厅里只有周和千岁。\\

「虽然不知道男高中生有没有花季,但我认为这是学生理所当然的想法……」

「正在享受青春的男高中生都应该说情人节的吧?」

「因为没有享受青春所以我不懂」

「少来~」\\

千岁明明知道谣言不是真实的,却还是笑嘻嘻地朝这边看。因此,周瞪了她一眼。

即使这样,千岁还是没有停下笑容,周就只好死心了。\\

「那么,要商量什么?」\\

听千岁说,她特地来到周的家里,是因为有事情要撇开树跟周、真昼商量。\\

「嗯。我在想给阿树的巧克力要怎么办。初中的时候呢,倒是普通地送了熔化再凝固的巧克力,不过我觉得啊,作为高中生还是想要做得更时髦一点」

「那样的话有椎名的意见就足够了吧」\\

周不会做料理,就算被问及巧克力要怎么办,他也只能回答不清楚,顶多只能告诉她树的喜好。然而,千岁跟树的交往时间更长,那种事情她早就了解透了吧。\\

「虽然我也会去问\ruby{昼儿}{\jpfont まひるん},不过周姑且也是男生嘛~我也想要听听男生的意见」

「姑且个头啊我是正宗的男生」

「我认为男生和女孩子独处的时候是会出手的哦」

「我说啊。那种事是要在交往过程中经过对方同意才能做的事,而且说到底我们连那种关系都不是」

「周那方面教养真好啊,还是该说想法健全呢」\\

虽然得到了想法健全的评价,不过周觉得自己的想法是很普通的。\\

确实,男人能够对不喜欢的女性做出那种行为,但是「能够做」和「付诸行动」还是不一样的。再说,周绝不可能不顾对方的想法。\\

要说周对真昼不会产生那种欲望是假的。周觉得,如果身边有一个外表和内在都很迷人的女性,那么男人特有的欲求多少涌现一些也是没办法的事。\\

即使如此,周也不会产生要做什么的想法。

因为,周对真昼最先产生的是这样的感情——不想让她哭泣,不想被讨厌,想要珍视她。\\

还有,明明真昼都宣言过,要是周做了什么就会在社会方面和要害方面给周造成巨大打击。周还没有愚蠢到因为一时的欲望就对真昼出手。要是那样的话,恐怕真昼会毫不客气地实现她的宣言。\\

「嘛,这也可以说是周的优点,或者说就是因为这个才得到昼儿的信赖吧」\\

千岁给真昼取了昼儿这样可爱的绰号,而真昼在厨房里听到了却没有否定。由此可见,虽然不知道她是否情愿,她对这个绰号应该还是认可了吧。

对于真昼来说,比起当面被喊作天使大人,似乎还是现在这样比较好。\\

「我偶尔会怀疑你是不是男的」

「不是说了我是男的吗。哪有这样骨头架子不凹不凸的女人」

「也就是所谓的草食系吗……我觉得周也可以更贪一点哦」

「就我这外表还贪得无厌不恶心吗」

「用之前那个造型不就好了嘛。话说,想看」\\

树和千岁很早以前就明白,真昼的传言中的那个人就是周。再加上前几天周也承认了,所以事到如今周也没有再作隐瞒。

只是,周并不想特地给他们看那个造型,而且也嫌麻烦。\\

「不要用那种说法。不如说我也不喜欢打扮成那样」

「又不会少块肉~」

「我的精神力和发蜡会少」

「要不要这么省!」\\

周无视了喊着「小气!」鼓起脸颊的千岁,这时真昼苦笑着从厨房里回来了。

她手上拿着托盘,上面载着装有千岁要的奶茶的杯子。\\

沙发前的折叠桌子上放置了三人份的奶茶之后,周从沙发上站起来,坐到了附近地板上的坐垫上面。

周用「快坐下」的眼神催促真昼之后,真昼尽管有些抱歉,但仍然拘谨地坐到了周之前一直坐着的地方。\\

「都能有那样的传言了,在学校也用那个造型明明就能受欢迎了嘛」

「我才不要。肯定很麻烦,而且我本来就没想要受欢迎」

「诶,好不容易有个情人节这么重要的活动啊。周就不想要情人节的巧克力吗?你看,比如说很受欢迎的门胁君什么的,应该会收到很多吧?不羡慕吗?」

「诶,才不要,会得糖尿病的」\\

恐怕门胁王子殿下会收到大量的巧克力吧,但如果全部吃完的话,身体肯定会长赘肉。\\

「再说,如果考虑要回礼的话会很郁闷的。就算是推测,算上义理巧克力和本命巧克力,他估计会收几十份吧。白色情人节的三倍回礼,对高中生的钱包压力不会太大吗」

「你都默认三倍回礼了吗,真行啊。不用在意回礼什么的,我也会送你巧克力的哦。要什么样的?」

「甜的东西我既不喜欢也不讨厌啊……不太甜的就行吧」

「知道了,我会塞进去各种玩意儿的」

「不要混进什么奇怪的东西啊」

「没问题,能吃的」

「我说啊」\\

虽然不知道她想放什么,但是千岁好像并没有打算送正常而美味的巧克力。\\

「昼儿要送谁巧克力?」

「在班上有来往的女孩子呢」

「不给男生吗?」

「……如果给的话,就算是义理巧克力也会闹出大事……」

「啊」\\

很容易想象到男生们会沸腾。也不难想象,沸腾之后就会发生没有意义的争斗。

在普通的男生看来,收到天使大人送的巧克力已经算是上天的礼物了。因此,如果真昼赠送男生巧克力的话,恐怕会引起不得了的骚动吧。是真昼的人气可畏,还是男生的脑洞可怕呢。\\

周苦笑着理解了,不送男生巧克力才是真昼的上策。\\

「我也会给千岁哦」

「哇昼儿爱你。我也会给你哦,和给周的不一样,好好做的巧克力」

「喂」\\

千岁笑着紧紧抱住真昼。

千岁的动作并不是性骚扰,让周松了口气,同时也瞪着千岁表示刚刚那句话可不能听过就算了。接着,千岁露出了好像泄了气一样的笑容。\\

「开玩笑的。给周的也是能吃的哦?」

「总觉得能吃的和好吃的是两种……」\\

看到千岁明显想要往巧克力里使坏,周烦恼地按住了额头。而千岁毫不掩饰自己的愉快,笑着对周说道「你就期待着吧」。

\subsection{情人节的喧嚣}

情人节当天,学校里和想象的一样吵闹,大家都是一片冷静不下来的氛围。\\

男生中有很多人一边坐立不安地期待着什么,一边又装作没兴趣的样子。

很多男生都认为今天能不能收到巧克力会决定一个男性的等级,所以表现出这样态度的男生才会那么多吧。\\

「大家都心浮气躁呢」\\

认为评级根本无所谓的周一边事不关己地想着「真辛苦啊」,一边把视线投向了因为别的而不感兴趣的树。

树悠闲地看着班里的喧嚣之景,满不在乎地回答了周「是啊」。\\

「拥有女朋友而显得从容的的树先生,请发表一下对于今年情人节的见解」

「果然从男生的心情来讲今天能不能收到巧克力事关自己的未来,所以都非常拼命啊。而且,坐立不安地想着能不能从椎名同学那里收到巧克力的男生大概有六成吧」

「男生方面她好像连义理巧克力都不会给,因为会闹得一发不可收拾」

「我想也是……话说回来,周君呀,有从那个人那里拿到巧克力的预定吗?」

「不清楚。至少我看不到要送的样子」\\

因为真昼好像会送巧克力给女生但是不给男生,所以周无法期待会得到她的巧克力。不过就算收不到,周也不会有什么不满。

不用说,如果收到了,周当然会很感激,但是有或者没有其实都无所谓。\\

说实话周认为情人节就像是零食公司的促销一样,其实并不是那么重要的节日。\\

树朝着一眼就看得出没什么兴趣的周,苦笑着感叹道「真是淡泊啊」,并把视线从周身上别开看向班里最为热闹的方向。\\

「……不过嘛,那个是真的厉害」\\

树指的「那个」,自然是吸引了班里几乎所有女生的大红人。\\

王子身处集体中央,脸上浮现出天真无邪、讨人喜欢的笑容。接连不断地有女生走来,递给他装有巧克力的袋子。

明明还没开始上课,但他本人准备好的手提包里已经鼓鼓地塞满了礼物,其人气之高可见一斑。\\

「不愧是王子啊」

「旁人咬牙切齿的样子也很厉害」\\

大概没从任何人那里收到巧克力的男生用嫉妒的目光看着门胁。

在被评级之前就等级差异就摆在了眼前,所以他们已经是无可奈何了吧。\\

虽然说在周看来,他也只是在意着门胁收到那么多的巧克力带回去好像很辛苦,以及门胁会怎么处理这些巧克力。\\

「受欢迎的男生真辛苦啊。把那些拿回去吃肯定要费一番力气吧」

「是呢。不过这样都不胖还真是厉害」

「与我无缘的话题呢」

「小千给你的巧克力已经准备好了哦。做好觉悟吧」

「什么觉悟啊」

「俄式的」

「不要啊放了什么进去啊」\\

在前几天的对话中,周就察觉到了她没打算做普通的巧克力,现在看来她好像把什么多余的东西混进去了。\\

「我想想,红辣椒、芥末、哈瓦那辣椒三位一体巧克力一块,梅干浓缩精华巧克力一块,剩下的是普通的巧克力」

「那家伙都做了些什么啊」

「好像是想要周大吃一惊」\\

从某种意义上来说也许会惊愕,但基本是在痛苦的意义上吧。\\

「……吓得不敢吃啊」

「放弃吧。这是我试吃时也走过的路」

「你是吃着玩的吧」

「算是吧。小千做的东西不管是什么我都吃」

「可恶的笨蛋情侣」\\

树这样的人,不管千岁端出什么来他都会吃吧。

说起来并不是千岁不擅长料理,问题在于她过于富有挑战精神了。虽然她正常做料理时确实可以做得正常,但好像一旦她想到什么就会做一些让人头疼的加工。

平时牺牲的基本都是树,而周没想到这次会轮到自己。\\

从树的反应来看,巧克力应该还在能吃的范围,虽然不需要过度恐惧但令人忧虑的东西还是会忧虑。\\

朝着稍稍有些消沉的周,树送去了闯过难关之人所特有的名为「你放弃吧」的温暖目光。\\

\vspace{2\baselineskip}

「来吧周,请收下!」

「多谢」\\

千岁下课后到班里来迎接树,顺便给周送来了巧克力,而周则不那么感兴趣似的回答了她。\\

能够收到巧克力,他当然很感激。\\

虽然很感激,但是,因为里面放了剧毒物质,所以周心里不可能坦率地高兴起来。\\

因为周打算一个不剩全部吃掉,所以肯定会在哪里碰上之前说过的超辣巧克力和超酸巧克力,接下来几天里,周恐怕会一边感觉战战兢兢,一边吃下去吧。\\

「我想着你肯定已经从阿树那里听过了,但还是好好期待里面的东西吧!」

「我可不太喜欢辣的东西啊……」

「这还在可食用的范围里哦?我也好好试吃过了,其实也挺好吃的!」

「那是因为你喜欢吃辣吧……真是的」\\

因为周并不那么嗜好辛辣食物,所以果然还是提不起兴趣。而且周也不太喜欢酸的东西,因此这巧克力可以说是准确地聚集了他不喜欢的味道。

其他的普通巧克力应该还是做得挺好吃的,只有这点算是好消息。\\

「啊,里面还有超甜和超苦的」

「感谢提前通知」\\

听到千岁干脆地又丢出一颗炸弹,周简直烦恼得想要抱住自己的头。

超甜应该是放了大量砂糖,而超苦恐怕是可可含量99\%的巧克力吧。\\

只是那样的话周还能熬过去。他并不讨厌苦的东西。\\

树好像也是第一次听说,「小千……你这家伙……」微妙地抽动着脸,但千岁依然保持着笑容。\\

「都说了没事啦,肯定还有清口的」

「清口?」

「那我们走啦~拜拜~」\\

千岁没有回答周的疑问,牵起树的手走了出去。今天似乎是他们的情人节约会。\\

收到了来自树的「祝你好运」的安慰和激励的话语,周好像累了一样叹着气挥手目送两人。\\

看到他们的身影消失后,周也想着差不多该回家了,于是披上外套,从桌子旁边的挂钩提起了包。

虽然周并不会对自己孤零零一个人有什么不满,但待太久也只会被现充男女们晒到,所以他还是打算早点离开。\\

周正准备背起包回家的时候,忽然看向全年级最充实的男人那边。

礼物攻击似乎终于有所收敛,门胁眺望着桌子上积攒着的令男生们垂涎的东西,眼神里好像望着远方一样。他的桌子旁边挂着的袋子也塞满了宝贝。\\

周立刻就明白了他在想些什么,于是就怀着同情走到了他那里。\\

「门胁」

「嗯?啊啊,藤宫吗。有什么事吗?」\\

做了将近一年的同班同学,就算是没什么存在感的周也被记住了名字。

因为周跟自己主动搭话的情况只有事务联络,面对意外的对象门胁也觉得不可思议。\\

对于这样的态度,周微微地苦笑着,打开了背包前面的小口袋的拉链。\\

「倒不是有什么事找你,这个给你」\\

周从中取出几个紧凑的三角形一样叠在一起的超市塑料袋,扔给门胁。\\

这些袋子是真昼说「以防万一预先准备几个放进去的话,之后用到的时候会很方便哟」而买好放进去的,应该就是她预计到了这样的情况吧。虽说这次要用上的不是周。\\

门胁一边困惑着「这是什么」一边把三角形展开,结果出现了一个比想象中要大的超市塑料袋。\\

毕竟塑料袋不是很厚,所以可能会破,但是周也不至于帮忙帮到这个地步,便决定这一点还是请本人想办法。\\

「我想错了吗?」

「不……错是没错啦……」

「是吗。应该会挺辛苦,加油吧」\\

恐怕之后会有人在校园里目击到门胁抱着鼓鼓的包吧。\\

周一边怀着「受欢迎的男生真辛苦啊」的感想,一边挥着手离开了教室。


\subsection{天使大人与情人节}

虽说是情人节,但家里并没有什么活动的气氛,周正如往常一样回到家休息着。\\

现在还没到做晚餐的时间,所以真昼就在身边。但是她一点也没有浮躁的气氛,也没有要对周采取什么行动的迹象。

周本来就没有期待能够收到巧克力,所以没什么所谓。他感到有些微妙的悲伤只是由于男人的矜持所致。\\

「今天学校里弥漫着甜蜜的味道呢」

「因为是情人节呀」\\

真昼貌似是送巧克力给了有交情的女生,对于男生却连义理巧克力都没给,所以从倾慕天使大人的男生们那里一直能听到非常沮丧的声音。\\

周自己虽然抱有「为什么明明没什么大不了的瓜葛却以为能受到巧克力呢……」这样的疑问,但男生们果然还是会期待一下的吧。\\

「不过,情人节什么的只是与一部分帅哥有关的活动,和我们这种不引人注目的男生没什么关系就是了」

「好像悟透了一样呢」

「不是我自满但我从来没收到过本命巧克力哦。也只有从千岁那里收到了俄式义理巧克力」

「俄式义理巧克力」

「似乎是在普通的巧克力里面混了几个含有刺激物的巧克力」\\

千岁送的巧克力里面好像混入了超辣超酸超甜超苦等等各种各样的,仿佛会破坏味觉的内馅的巧克力,所以周吓得不太敢吃。\\

「又是不得了的东西……」

「我之后会吃,如果我看起来很痛苦的话还请体谅一下」

「会好好吃下去呢」

「那当然。毕竟也算是为我准备的,自然会吃咯。又不是毒」\\

虽说巧克力里面是刺激物,但是对身体没有害处,所以周是打算怀着对她做巧克力给自己吃的感谢而吃下这些的。

既然是特意抽出时间做的,收到的人当然应该吃掉。虽然说这刺激物让人非常提不起兴趣。\\

「这样吗……」

「嗯,除此之外也没有收到别的巧克力,对我这样的非现充来说情人节什么的都是无关的话题」\\

即使只收到一个义理巧克力也已经足够了吧。\\

考虑到一个月后还礼的日子,周不知道回礼要怎么办而困扰似的垂下眉头。真昼则是静静地看着他。\\

\vspace{2\baselineskip}

晚餐后,周吃下千岁的巧克力倒在了桌子上。\\

从千岁那里收到的巧克力盒有着等间隔的划分,里面放着12块松露。

里面有四种算是中奖,也就是说有三分之一的几率抽中。\\

其中大奖只有超辣的那一种,所以周心想,除此以外的11块都能正常地吃下去,于是拿起一块——结果就是这副惨样。\\

「抽中了呢」

「……想用几天吃完结果就是这样……」\\

在厨房里做着饮料的真昼注意到周的情况,以带着些许怜悯的声音搭话过来。\\

即便周勉强咽下去了,可他嘴里的感觉不是辣而已经算是痛了。虽然周很清楚辣味不是味觉,但现在问题根本不在那里。\\

该说是幸运吗,这东西并不是真的吃不了,而是停留在了能忍受却很难受的程度。\\

对于冲出鼻腔的芥末特有的尖锐刺激,周一边佩服千岁真亏能把这样的挥发成分包裹起来,一边忍着产生于本能的泪水骂道「也不用那么下功夫吧」。\\

攻击鼻子和眼睛的是芥末,灼烧舌头的是辣椒和哈瓦那辣椒粉。由于强烈的味道——不如说是痛感,仅仅一颗就让周遍体鳞伤。\\

「节哀。不过换种想法,先见到地狱,剩下的就是天堂了」\\

话虽如此,但现在的痛苦却是怎么都无能为力的。\\

周正从心底渴望这疼痛快点消失,便听见了轻轻的一道叹息,接着从侧面传来了哐当的声音。\\

「来,清清口」\\

周抬起头,只见旁边有一杯冒着热气、释放出香甜气味的马克杯。

杯子里是浓茶色的液体。\\

「可可?」

「很像呢。Chocolat chaud……说简单点就是热可可。虽然不那么甜,但对于换口味来说应该是足够了」

「帮了大忙……」\\

总之现在周想先冲掉这疼痛。\\

周拿起马克杯将热可可倒入口中,温和而浓郁的味道就在嘴里扩散开来。

这杯热可可虽有巧克力的甜甜香气,但并没有过分的甜。那甜中带苦的感觉,是一种很容易入口的让人安心的味道。\\

「好喝」

「那就好」\\

虽然真昼的回答很平淡,但周没有在意,而是慢慢地品尝着热可可,试图掩盖住嘴里的痛感。

巧克力里面并不是放入了大量的刺激物,说到底也只是把那些东西混到生奶油里面凝固起来,用巧克力厚厚涂抹一层还敷上了糖粉。因为这些东西,尽管一开始的冲击很强烈,但过了一会儿就渐渐缓和了。\\

周喝完的时候,舌头总算变成了平常的样子,不过他还是感觉麻麻的。\\

「哈……那家伙还真的全部混在一起了啊……」

「有那么辣吗?」

「那当然,毕竟放了红辣椒、芥末、哈瓦那辣椒啊。真是的……幸亏有清口的,这要是在外面吃的话我怕是已经死了」

「所谓的不幸中的万幸呢」

「太对了」\\

周轻轻地骂了一句「可恶的千岁」。不过,她应该也是想给一个惊喜才这么做的,所以也不能太怪她。

中奖版本以外的巧克力应该是正常的味道,而且她也并没有恶意。她不是仅仅只让别人吃,而是自己也有试尝,所以周也只能报以苦笑。\\

「话说回来,还真是少见啊,热可可什么的。平时不都是热牛奶吗?」

「……嗯,是吧」

「这,难道说是情人节才做的吗?」\\

基本上真昼比起热可可更多是喝热牛奶和奶茶,不过既然她这么少见地做了这样的饮料,周还是带着些许的期待问了一句。\\

「……算是吧」

「嗯,thank you。帮大忙了」\\

看到真昼稍稍地点头,周安心地叹了口气。

如果这时被否定了,就显得周自我意识过剩一样,这会让他感到非常羞耻。但好在周似乎猜对了。

真昼应该只是因为难得的情人节而借了借活动的兴头,但周还是觉得非常感激。\\

周再一次告诉真昼「很美味」之后,真昼好像身体不舒服一样颤抖了一下。\\

「怎么了吗?」

「……那、那个」

「嗯?」\\

周想着自己坐在身边催促的话她应该不好说出来,所以留意着以温柔的语气再次做出了询问。

周轻轻催促之后,真昼就把半边脸埋在紧紧抱着的抱枕里,仰望着他这边。真昼微蜷着身子,那仿佛含着些不安的仰视,可爱得让周不由得想抚摸她的头。\\

她的举动犹如小动物一样,微妙地可爱而招人微笑。周静静地等待着,但真昼只是颤抖,却一点都不往下说。\\

「……我、我回去了」\\

不仅如此,她还突然站起来提起了行李。\\

当周发出「哎?」的声音时,真昼已经踏出啪嗒啪嗒的脚步声离开了客厅。

周还没回过神来,家里就响起了门打开之后又关上的声音,接着又响起上锁声。转眼间,真昼就没了踪影。

一切发生得太快,以至于周不禁发出了「哎哎……?」的声音。\\

(我做了什么吗……?)\\

因为实在是没想过会让她逃走,所以周一半是困惑,另一半则是「难道自己做了什么坏了她心情的事情吗……」这样的不安占据心头。\\

周一边担心着「明天见面的时候如果她心情还是不好的话怎么办」一边打算去看一眼她走出的家门。这时,他突然发现自己房间的门把手上挂着一个纸袋。

这是她离去前拿着的浅粉色纸袋,纸袋外侧有用贴纸固定的留言卡。\\

『一直以来承蒙你的关照,这些是我平日的感谢』\\

留言卡上写着她特有的稍微圆润的,却又一丝不苟的礼貌的字迹。周往袋子里一看,里面装着用巧克力色丝带包装的粉彩色盒子。\\

周虽想着为什么纸袋会在这里,但却马上意识到是在那个时候挂上的。\\

看起来,真昼好像是觉得直接给太羞耻了。再加上她说过不给男生,这使得她产生了相当大的犹豫。\\

(普通地送给我不就好了嘛)\\

想到真昼在这种时候相当拘谨,周就苦笑起来,同时坐到沙发上取出里面的东西。\\

盒子的包装很可爱,表现出真昼风格的女人味。

周一边对能不能收下而感到微妙的不安,一边慢慢地拆开包装打开了盖子。\\

盒子里面放着的,是一个个用塑料包装着的,浸在巧克力里的圆片蜜饯橘子,也就是所谓「橙香四溢\footnote{橙香四溢:原文为 {\jpfont オランジェット},是一种用巧克力包住砂糖腌制的柑橘类的原产法国的点心。}」的点心。

鲜艳的橙色与有光泽的深巧克力色的对比非常绚烂,使得这个点心看起来很好吃。\\

涂在外面的巧克力还有白巧克力的版本,水果方面也有单独包装的柠檬版本在里面,因此周感觉根本不会吃腻。\\

在橙香四溢的旁边,还有另一张留言。\\

『因为你好像不太喜欢吃甜的,所以我做成了容易入口的东西。要是能合你胃口就好了』\\

留言纸像这样写着,让周想起大概十天前的事情。\\

『要什么样的?』

『甜的东西我既不喜欢也不讨厌啊……不太甜的就行吧』\\

她好好地记住了和千岁的对话,并且还照顾到了周的喜好。\\

真昼风格的细致关怀,自己的喜好被放在了心上,再加上这本来就是她的赠礼,三件事情加在一起使得周不由得害羞起来,脸颊微微有些发烫。\\

周注视着为了便于食用而一个个分开包装的普通版橙香四溢,用手拿了起来。

放出艳丽光泽的巧克力和鲜艳的橘子的对比非常漂亮。周就慢慢地吃下了一口。\\

在嘴里扩散开来的,是蜜饯橘子的酸甜和黑巧克力的不会太甜而恰到好处的微苦。

两种味道巧妙地相互衬托,实现了完美的协调。\\

(好吃……)\\

周感觉这些比买来的巧克力更美味,恐怕是因为这是真昼亲手做的吧。

这样想着,周又咬了一口。\\

真昼的橙香四溢,又酸又甜,又带点微苦——不知为何却让周感觉到了无比的甘甜。

\subsection{情人节的次日}

「藤宫,昨天多谢了」\\

第二天周来到学校之后,因为门胁过于自然的搭话而不由得僵住了。\\

虽说昨天稍微有一点来往,但周没想到他会特意为了这点小事而跑来道谢。\\

门胁的表情和他被女生围住的时候并不一样,像是老好人一样明朗。被他笑着搭话的周也在旁边若有若无的视线下感到十分难受。

周本来就不喜欢受人瞩目,面对这种充满好奇的视线还是会感到有些心生怯意。\\

「啊,那点事情不用在意啦。看你也挺不容易的」

「算是吧……」\\

门胁露出了仿佛在看远方的眼神,周也对他报以了「果然受欢迎的男人很不好受啊」的同情。\\

门胁本人自知受到欢迎,却并不为此而骄傲。正因如此,他才会受到周围人们的喜欢,并且嫉妒他的男生们也不会真的讨厌他。

或许,为这点小事特意道谢的守规矩的品格,也是他得到其他人喜欢的原因。\\

「总之还是帮大忙了。还是来道个谢」

「没事的啦,有困难的时候互相帮助嘛」\\

周也不是为了卖他人情而帮他,他做的也并不是那么值得被感谢的事。

周轻轻笑着说不要在意之后,门胁也稍微露出了安心的笑容。\\

面对他发自内心的笑容,周围的女生们顿时喧闹了起来。周只得苦笑着感叹,这个笑容应该在面对女生的时候用啊。\\

\vspace{2\baselineskip}

「你和优太发生了什么吗」\\

门胁离开之后,看到了他们谈话的树过来搭话了。\\

优太是门胁的名字。树和班上所有人关系都比较好,自己也是待人和蔼、能炒热气氛的感觉,当然和门胁也有着一定的来往。

周有时也会感叹并且困惑,这样的男人居然愿意当自己的朋友。\\

「诶,因为他收了太多巧克力,有些走投无路了,我就把自己存着的购物袋给了他而已……」

「啊。看来比他预计的还要多,到最后出了岔子啊」\\

树当时也在旁边看着那一大堆的巧克力和女生的好意。听到周的解释之后,他理解地露出了带有同情的苦笑。\\

那时,两个人的感想就是,有那么多的话带回去肯定很辛苦,所以周给他帮忙也不是什么不可思议的事。

周自己倒是觉得,只是帮了个小忙,并不需要什么道谢。\\

「就只是这样了,也没做什么了不起的事」

「该说是像你的风格吗。……不过,常备塑料袋啥的……怎么感觉你像辛苦的家庭主妇一样,特别是看你拿手机看超市广告的时候」

「我是男的啊。不过,应该是受了某人的影响吧……」\\

某人当然指的是真昼。该说都是她害的,还是该说托她的福呢。\\

由于伙食费两个人各出一半,所以周为了尽可能节省,有时会浏览网上的广告,有时会向真昼提议去做广告里的便宜商品做得出的东西。在树的眼里,这样做就显得更加像个为家庭而奔波的人一样了吧。

或许,周所做的事情,比起一般的一家之主,反倒远远更像是主妇干的事。虽然说料理全都是交给真昼的。\\

「有个顾家的搭档真好啊」

「才不是什么搭档啊。……千岁呢」

「小千?嘛,嗯。只要不把奇思妙想付诸行动的话,应该……也不是做不到吧」

「但是那家伙肯定会乱搞哦?」

「……这一点也很可爱对吧?」

「喂别闪开视线啊」\\

往好里说往坏里说,千岁都是个喜欢寻求刺激的随性的人。

普通做的话,她似乎也能做到一般女高中生等级的家务,但她要是起了玩心或者心情有变就会搞出很多事情。\\

「不过,她说结婚之后应该会老实点」

「要让你爸答应得花多久啊……」\\

树的父亲是当下很少见的,对交往管得很严的人。因为不待见千岁,所以树的父亲心里对当前两人以结婚为前提的交往感到不满。

千岁的父母倒是一直很欢迎树。周还觉得有点惊讶,因为一般都是反过来的……\\

「长大之后会慢慢说服的啦,就问他不想看孙子吗」\\

树做作地耸了耸肩膀,但眼中写满了认真,表示不惜发生争吵,这件事上也绝不会听父亲的话。

从平时的表现也看得出,他对千岁的爱很深。周觉得树从高中就开始考虑结婚很了不起,同时决定给他加油。\\

「……反正你爸放弃之前估计你也不会让步的,加油吧」

「嗯。你也加油」

「加油什么」

「和那个人……对吧?」

「……我和她又不是那种关系」\\

「别随便瞎猜」,周说着把脸别开之后,旁边就传来了树明朗而愉快的笑声。

\subsection{与天使大人的约定}

周从超市买完真昼要求的食材回来以后,发现真昼已经坐在他家的沙发上等着了。\\

虽然这是和平常一样经常看到的场景,但要说不同的话,这次的真昼正抱着靠枕、手绕膝盖坐在沙发上。

虽说真昼的姿势看起来像是小孩子闹别扭一样,但她的表情比起闹别扭更应该说是害羞。因为真昼的可爱,周也有点不知道该往哪看才好。\\

周想着「幸好真昼穿着长裙」,稍微移开目光,走到冰箱那边把食材都放了进去。回到客厅,周发现真昼正偷偷看着他这边。

周坐到真昼的旁边,发现她的视线正微妙地移开着。\\

「真昼,昨天谢谢了。味道很不错」

「……那就好」\\

周虽然知道真昼恐怕还在意着昨天的事,但出于感谢的本分还是老实地传达了自己的意思。真昼听到之后,把自己的脸半埋进靠枕里看着周。\\

「想要什么回礼?」

「我也不是想要回礼才送的」

「我懂我懂,但对于你的真心实意还是应该用真心实意回馈对吧?光收礼不回礼的男生多丢脸」\\

周以礼尚往来为信条,在他看来,真昼特意为自己做了那么美味的东西,自己有所回馈是理所应当的事。在这件事情上他并没有让步的打算。

毕竟,真昼以前似乎没有送过男生礼物,这次特意迎合周的口味做了巧克力,肯定是费了不少功夫。\\

「……我已经从周那里得到很多了」

「不如说反倒是我总是在从你那获取吧。你一直在为我做饭,我也总是给你添麻烦」

「我做这些是因为我乐在其中……周君大概没有意识到吧,你也给了我很多东西,有那些就足够了」\\

周感觉自己就没有给予过真昼什么东西。不如说周因为一直都在单方面地获取,甚至都想做出回礼了。但真昼似乎并不这么觉得。\\

「这两件事不一样啊……算了,我就想想有什么东西你会喜欢吧」\\

即便周在不知不觉中给了真昼什么东西,那和白色情人节的回礼想必也是不同的。

在情人节收到了巧克力,就要在白色情人节回礼,这是一种不可缺少的礼仪。\\

周以「不打算让步」的眼神盯着真昼,接着她就眼神飘忽着点头答应了了一句「……好」。\\

「总之决定要送什么还有一个月左右的时间考虑,能找到什么你会喜欢的东西就好了」

「……有那么空闲么?下周开始就是期末考试,再没多久又是休业式了」\\

真昼有些惊讶地指出了这一点。确实,下周期末考试就要开始了。

虽然今天的学校里还残存着情人节的余味,但应该马上就会过渡到考试前的紧张气氛。

不过对于周来说,这并不是什么需要焦虑的事情。\\

「考试只要和平时一样发挥,肯定不会留级,也用不着到现在才焦虑。真昼你也是吧」

「嗯,平时好好学习的话,考试前就能更轻松了」\\

周平日里就做好了预习和复习,对待学习也很认真,所以几乎不会因为考试而感到为难。

即使不临时抱佛脚,周自认为也能维持住和平时一样的成绩,而且事实上他也是一路这么走过来的。考试前,周顶多也就是坐在书桌前的时间比平时稍微久一点而已。\\

而真昼甚至提前掌握了上课内容,并且和周一样也是不会落下预习复习的类型,所以在她身上连着急的影子都见不着。不如说,能提早结束一天日程的考试对她来说反而更加轻松吧。\\

「嗯,你就等着吧,也别太期待了」

「……好的。周送给我的东西,我全都会珍惜的」

「那么夸张啊」

「熊熊我也有好好对待着」\\

看来生日送的小熊布偶有被好好珍视着。

周因为有看见真昼用着钥匙包,所以知道对方有在好好使用,但对于小熊布偶却有不安……现在看真昼的样子,她似乎还挺中意的。\\

听到真昼叫出「熊熊」这么可爱的名字,周差点没笑出来,但要是真笑出来恐怕会被瞪,所以还是忍住了。\\

今年要还是像这样和真昼一起的话,下个生日要送什么好呢……周感到望眼欲穿。\\

周笑着对真昼回复道「那真是太好了」,却发现她忽然死死地盯住自己。\\

「……说起来,我还不知道周君的生日在什么时候」

「啊,我的吗?11月8号」\\

周这时才发现自己还没有告诉过真昼自己的生日。周把生日告诉真昼后,她的瞳孔细细地眯缝起来。

经过几个月的相处周已经开始明白了,这是真昼略微有些生气的表情。\\

「……那个,周」

「嗯?」

「去年那时候我们早就认识了吧?」

「是啊」

「为什么当时不说呢」

「因为你没问嘛。你也没说过自己的生日吧,那也是我看了学生证才知道的」

「唔」

「说到底那时我们关系也没好到现在这种程度吧。说了生日你也会想着『这家伙在说什么呀』」\\

就算周对真昼说「其实今天是我生日」,恐怕那时的真昼也只会回复一句「是吗」之后就没有下文了吧。\\

就算是从周的角度看,这样做也显得是在要求礼物一样。周讨厌这样,而且他的脸皮也没那么厚。

没有必要说、关系也没好到可以说,因此那时也就没有说出来,仅此而已。\\

「……但是」

「不用那么在意也没关系哦?」

「……那,今年你的生日的时候我会好好祝贺你的」\\

真昼似乎还没有满意,她转向周这边,紧紧地握住衣服袖子,对周宣言道。\\

看来她是因为只有自己被庆祝生日而感到不满吧。真昼的眼神仿佛在说,等到周的生日时,她会比自己生日那天更加认真地庆祝。望着这个眼神,周露出了略带无奈的笑容。

听到这番话,周情不自禁地感到开心……不知不觉里,笑容中的无奈就转为了单纯的喜悦。\\

结果,真昼也与周有同样的想法……今后,她依然愿意待在周的身边,这件事比什么都要令他感到开心。\\

「真昼同我约好了要一起相处到那时候了啊」\\

听到周不经意间说出的话,真昼睁圆了那清澈的焦糖色眼睛——然后一瞬间就红着脸,把手中的靠枕嘭的一声拍在先前一直握着的衣袖那边。\\

被当面这么说好像让她感到了羞耻。

急于掩饰的真昼把气撒到周的身上。看到这让人发笑的场景,周又一次差点笑了出来。\\

「……也没有,讨厌周君……在一起,感觉还挺安心的,所以,可以」

「是吗,谢谢了」

「……我、我没有什么别的意思」

「这我还是知道的啦」\\

因为真昼补充的那一句,周向她点头表示理解,不知为何却迎来真昼一副稍有不满的神情。

\subsection{选择礼物的方法}

周在学习上本就很勤奋,上课态度也很认真,所以并没有太费劲便通过了期末考试。\\

他和真昼一起复查试卷,算出来的分数跟平时差不多,况且平常他在学校里的态度也挺好,留级的事情是可以不用担心了。

树也考了个不错的分数,千岁似乎也避免了挂科,这么一来周熟悉的人就都不用担心留级的问题了。\\

考试之后是毕业典礼,那时会欢送和周没什么关系的高三学生。在毕业典礼后就是休业式了……不过在这之间还有一个节日很成问题。\\

「……该回什么礼呢」\\

这正是情人节的胜者们会迎来的回礼之日。

且先不论周这到底算不算是胜者,但既然从真昼和千岁那收了巧克力,那么他自然打算回礼。\\

只不过,周还在头疼该送什么好。

要送千岁的话,他打算去那家买了圣诞蛋糕的店准备一个白色情人节款的点心套装,再来点她正在收集的角色周边,那就应该问题不大了。\\

问题在于真昼。

周总觉得,无论送什么她都会欣然收下。

周送的东西她都平常地收下了,而且她看起来比较在意心意,对送的东西并不是很关心。说实话,这样反而让周最头疼。\\

就算想要从喜好的角度来挑选,周也只知道她喜欢甜的东西和可爱的东西,而这种东西女生差不多都喜欢。因而,周一直在头疼到底该选什么好。\\

再怎么说,上回说过的磨刀石肯定是免谈,因为这样不但一点意思也没有,而且预算方面还很紧张。但即使不考虑磨刀石,周还是很烦恼要选什么。如果可以的话,比起实用品而言,周这次更想要送享受用的东西。\\

周暂且先去了杂货店。他望着里面的白色情人节特卖区,但看着这些,周却不太想象得出她真正高兴的样子。

要是这次挑的礼物,能让真昼做出跟上次收到熊布偶时一样的反应就好了。\\

(送两次熊布偶毕竟也没意思啊)\\

虽然可爱的熊布偶货架上倒是摆了一堆,但送两次一样的东西还是欠缺新鲜感吧。

但话说回来,以周那贫乏的想象力,能想到的女生喜欢的东西,除了小饰品以外也没别的了。\\

可周也不敢完全确定,两人的关系到底可不可以送小饰品。\\

如果送的话,估计真昼还是会好好收下,但问题是她到底会不会高兴。

虽然,周是觉得两人按男女说来算是关系不错了……但是,送小饰品到底能不能让她高兴呢。\\

这要是树给千岁,那肯定没有问题,但周给真昼送就得画一个问号了。\\

周这样一脸烦恼地在特卖区旁边晃来晃去,恐怕看上去就像一个可疑人物吧。

虽然周也换上了外出用装束,但还是改变不了男性在可爱的杂货前晃来晃去很可疑这一事实。\\

周正念叨着这也不行那也不行,突然有人从身后搭了一句「在找什么吗?」。\\

一回头,周便看见一位身着店里围裙的妙龄女性微笑着站在身后。

她大概是看不下去周这实在苦恼的样子,所以才会过来帮忙。要不然的话,她也不会特意向这跟个可疑人物一样晃来晃去的周搭话。\\

「啊,那个……白色情人节的回礼,我拿不定主意」

「在这边没有看上的吗?别的地方也有些东西常被选来回礼,我带你去看看吧」

「啊,不是这个意思……只是说和她的关系有些不好形容,不知道该送什么不会被讨厌」

「怎么说?」

「她不算是女朋友但挺亲近的……打个比方说,小饰品这种东西,从称不上是喜欢的人那儿收到,会不会感到高兴呢」

因为解释起来很害羞,周的说明便有些含糊,但女店员听完后却笑了起来,恐怕是觉得周的烦恼比较逗人吧。\\

「男性烦恼这种东西是很常见的哦」

「那他们是怎么决定的?」

「大部分人虽然犹豫,但最后还是会决定购买。如果关系亲近的话,就算送了,一般也不会被讨厌的哦」\\

听了「不会讨厌」这话,周稍稍安心了一点。但即使如此,要送真昼饰品还是让他心里有些慌慌的。

真昼虽然身上打理的很整洁,但并不怎么会佩戴饰品。她偶尔倒是会戴一戴,不过每次戴的都是高级的东西。\\

真昼审美的品味很好,因而周并没有自信自己选出的东西能得到她的认可。\\

「有需要的话,到那边去给你介绍几件吧,在女性间很火的」

「……麻烦你了」\\

听见这求之不得的提议,周不自觉地摆正了姿势点了点头。\\

\vspace{2\baselineskip}

「然后我就买了」\\

跟树讲了事情的经过后,周就遭到了笑话。树笑话周的眼神和前几天那个店员一样。\\

两人正在食堂一角吃着每日套餐。一讲到白色情人节的话题,周不小心就把这事讲出来了。\\

「……别多嘴。不过啊,明明没有交往却送对方饰品,还是显得有些恶心是吧」

「你怎么这么矫情啊,男子汉做事得靠勇敢和气势懂不。那个人啊,反正只要是周送的,收到什么都会开心的哦?」

「……虽然是这么回事啦」\\

以真昼的性格,不论送什么她都会高兴地收下吧。

但周希望送一件能让她真的高兴而且会用的东西,因而还在担心这到底达不达得到要求。\\

「结果你买了个啥?」

「……手链,粉金色的,主题是花」\\

感觉对真昼来说,比起给人以冷淡感的银色和华贵感的金色,还是这华丽中泛着柔和与可爱感的粉金色比较适合。\\

身为学生,高价的贵金属是肯定买不起的,所以这里只是谈的外观——周自己觉得,在这种颜色的饰品中,他挑选出的这一款设计精致优美,很适合真昼。\\

「咋了,听上去不是挺能让她高兴的嘛」

「……不会被觉得恶心?」

「我说你多虑了吧。为什么你这种地方这么消极……」

「给女性送礼什么的我可是只给她好好送过啊」\\

母亲肯定是成不了这样的对象,千岁也不算数。不如说给千岁的东西是她自己闹着要的甜品,周甚至不怎么觉得那算是礼物。\\

「你在这种事情上面还真是缺乏自信啊……」

「不如说怎么可能会有自信……可是那家伙哦?」

「熊布偶那时候她挺高兴吧」

「虽然是这么回事啦」

「我说啊周,重要的是心意啊心意。既然你钱也花了东西也买了,那剩下的就只有加入你的心意了啊」\\

由于树说得轻巧,周嘟哝着「要是真能那么干脆就好了啊」用手扶住了额。\\

看来,直到白色情人节当天,周都得要在这个决定到底好不好的纠结中度过了。

\subsection{天使大人与白色情人节}

白色情人节当天,周等待着真昼的到来,神色有些微妙的紧张。\\

学校里的气氛虽然并不如情人节时那么喧嚣,但还是感觉得出来,胜者们坐立不安地打算回礼,而女生们则期待着那些回礼。

顺带一提,门胁是规规矩矩的,回礼一律用了点心,但是周想到仅仅买那些东西恐怕就要花掉几万日元,所以光是看着都觉得头疼。\\

周并没有在学校交给真昼回礼,因而他像现在这样在家里等着她。

虽然今天周提前回了家,一直在做着心理准备,但他不擅长给女孩子送礼物,还是会对此感到紧张。\\

周姑且是没穿平时的汗衫和运动衫,而是穿了件白衬衫,外面再套了一件灰色V领针织衫,并用休闲裤搭配上了多层的衣服。

周觉得这样应该看不出平常的邋遢感,但他不清楚真昼看到这身打扮会怎么想。\\

正当周坐立不安地等待真昼到来时,他听到了从门口传来的开锁声。

或许是由于紧张,周条件反射地端正了自己的坐姿。\\

真昼一如既往地用备用钥匙进了门,走到客厅露出脸,看到了周,然后呆住了。\\

「咦,为、为什么是那个发型?」

「姑且是白色情人节,我觉得穿得端正一点比较好……你要是觉得不自然,我就换回来」\\

虽然看起来成功让真昼吃了一惊,但这身打扮评判似乎不佳——周这么想着,正准备直起身站起来,却发现真昼拼命地摆着手,好像在否定似的。\\

「才、才没有那种事,我只是有点,被吓到了」

「这样啊」\\

看真昼冷静不下来的样子,或许比起这个还是平常的造型要更好吧。

真昼尽管在旁边坐着,但她一直显得心神不定的。\\

「……你冷静不下来的话,我还是换回去吧?」

「不、不用,这样就好……只是,那个,一身没什么意义的帅气打扮」

「什么叫没什么意义啊」

「明、明明平常那种沉静的气氛更让人安心……这个样子,我冷静不下来」

「那我换回来」

「……保持这样就好」\\

真昼捏住周的袖子,抬头朝他看过来。

不知是不是因为羞耻,真昼脸上微微泛红,眼睛湿润地仰头望着周,这让他的心脏怦怦跳动起来。\\

真昼本人应该不是故意的,但她抓着衣服用仰望的视线实在是让人精神上承受不住。而且因为距离很近,周能闻到她身上甜甜的香味,在各种意义上都很辛苦。\\

这些东西在周的脑海里挥之不去,但真昼似乎也意识着周的打扮,尽管扭扭捏捏的却又不想让周换回去。两个人都红着脸,无比尴尬。\\

周一边游离着视线一边笨拙地回答「哦、哦哦」,接着就像是要蒙混过关似的把放在旁边的纸袋随手塞给真昼。\\

「给,这是回礼。不用太期待」

「……谢谢。我可以现在就打开吗?」

「嗯」\\

虽然真昼在周面前打开礼物让他觉得羞耻,但周也没有阻止她。\\

为了凑点装饰,周姑且买了个丝绒材质的小盒子,把礼物装了进去。不过他感觉盒子里的东西与盒子并不相称,这一行为说不定有些画蛇添足了。\\

真昼用洁白的手指打开藏青色的箱子,里面立着的是前几天买到的粉金色手链,并附赠了折好的纸片。\\

真昼不太喜欢引人注目的饰品,于是周就挑选了这条重视简约、重视品质、以花为主题的手链。

手链各处都点缀着迎着光线闪闪发亮的水晶玻璃,设计上可爱与优美兼备。\\

真昼焦糖色的眼睛一直凝视着盒子里的手链上闪耀的粉金色光芒。\\

「这个,不合你的喜好吗?」

「没有啊,这个很可爱」

「那太好了。我觉得很适合你,所以才买来的」

「……谢谢」\\

真昼听到周的「觉得很适合你」这句话,害羞地垂下了目光。

这副模样很惹人怜爱,令周不禁忘记了呼吸。\\

「……还有,这个是?」\\

周想着移开目光,但视线却又凝视着真昼,好像被钉住一般。看到真昼发现了放进去的赠品,周挠了挠自己的脸颊。\\

「啊,那个啊。嗯,呃,我觉得光手链可能还不够,嗯……毕竟一直都受你照顾,就有点想要为你实现愿望」\\

放在里面的赠品,是周亲手制作的券。券能用3次,名字是『不管什么事情都会答应的券』,好像骗小孩的东西一样。券上附有周画的小熊插画,周甚至觉得自己画得还挺不错的。\\

由于平时一直受真昼的照顾,周就希望在力所能及的范围内实现她的小愿望,于是把这张卡券放了进去当作赠品,但真昼似乎盯着画上的小熊而肩膀颤抖不已。\\

「呼、呼呼。这插画是周君亲手画的吗」

「吵死了画得差不行啊」

「不是啊,还挺有趣的」\\

虽然周感觉真昼隐含了画得差劲的意思而皱起眉头,但看到真昼脸上天真无邪的笑容,他也就不想再抱怨什么了。\\

「……那,我可以马上用一次吗?」

「想让我做什么?」\\

周没想到真昼会冷不丁地说要用一次,不过要是真昼有什么愿望,周也本就打算在力所能及的范围内帮她实现。

这样想着,周凝视着她。接着,真昼轻轻地将装有手链的盒子转向了周。\\

「……请周君帮我戴上」

「这种事不用券我也会做的啦……悉听尊便」\\

真昼说出来的愿望真的非常微小,因而周苦笑着说道「就算不用愿望券,只要拜托一下我也会做啊」。

「明明用在更重要的事上就好」面对说出可爱愿望的真昼的谦恭与可爱,周的表情也自然而然地柔和了起来。\\

由于真昼伸出了手,周于是就接过盒子,放在自己腿上,接着从中取出了手链。

周一边听着细链子摩擦的声音,一边为了不弄坏而小心翼翼地打开手链的卡扣,将手链轻轻缠在真昼的手腕上。\\

他小心翼翼地把卡扣扣上后,色彩柔和的手链微微闪耀着,仿佛给真昼的纤细手腕增添了光彩。\\

果然,这种颜色很适合真昼雪白的肌肤。

因为真昼的美貌可谓清秀美人,所以周估计比起华丽的饰品,还是低调、有品位的更加合适。这下他可以挺起胸膛说,自己没有选错了吧。\\

「嗯,很适合你」

「……谢谢」\\

周心想「一直碰着也不好」,就轻轻地松开手。接着,真昼就好像要把戴有手链的手腕温柔地抱住一般将其贴在胸前,脸上浮现出柔和的笑容。

她的脸颊泛起红霞,微微翘起的嘴角露出了笑容。周想撇开视线,却被那份笑容迷住而挪不开眼。\\

与满脸的笑容不同,那既端庄又天真烂漫的甜美笑容,深深地印在周的脑海里。\\

与平时展现出的无语笑容、纯粹的喜悦又不同。现在这美丽笑容尽管留有些许稚嫩,但很有女性魅力;既端庄又具有诱惑性,牢牢地吸引着周的视线。\\

(……真难熬)\\

让周看到这样的笑容,只让周一个人看到这样的笑容——这种事让周心里痒痒的。\\

周为了控制住疯狂跳动的心脏而试图转移视线,但结果还是没能做到。最后,周一直盯着她,直到真昼注意到自己被注视着,羞耻地用靠垫遮住脸为止。

\subsection{白色情人节的次日}

「白色情人节怎么样了?」\\

第二天,被树问到感想的周紧皱起眉头。\\

树姑且是照顾到周,没在学校问起这件事,但回家的路上二人顺便来了趟快餐店,周一坐到座位上他就笑着问了过来。\\

周只是因为想要偶尔尝尝咸的东西才来的,要是早知道会被问这种事情说不定不来更好。\\

「你问怎么样……只是正常地送给她了而已」

「她高兴了吗?」

「……算是吧」\\

要说真昼高不高兴,那肯定是高兴的。

虽然真昼没有笑得像孩子那样欢,但她那含有腼腆、甜美和难以形容的妩媚的笑容,让周觉得她应该是相当高兴。\\

那美丽的笑容,光是回想起来,周就会觉得浑身不自在。

周一边压制着从内部往脸颊处窜起的热量,一边用尽可能平缓的声音回答。树似乎是理解了一样抱着胳膊,嗯嗯地点着头。\\

「从你那反应看来感觉挺顺利的。她想必很开心还露出了很可爱的笑容吧」

「!?」

「你看吧。你们关系确实变好了嘛」\\

听到这比起戏弄更像是感慨的语调和声音,周咬住了自己的嘴唇。

树不会触及周不想被让他人知道的事,但除此之外会以好朋友的身份准确地指出周的心中所想,因此非常难对付。就算周想要还击,可是树本来就和千岁关系不错,事到如今再说起这个也毫无意义,结果周并没有可以还击的手段。\\

树温和地笑着看着支支吾吾说不出话来的周,那微妙的温暖眼神很令人生气。\\

周无可奈何,只能边吃炸薯条边转过脸去。这时,树朝他露出了苦笑。\\

「就我来说还是很高兴的哦?周终于要有春天了呢」

「不是那样的」

「对她来说是怎么样的还不清楚吧」

「……那种事情,不可能的」\\

确实,周亲身体会到了真昼对自己非常信任。既然这样,周就打算和她亲近到自己成为她最能信赖的人为止。至少,从真昼现在的交友范围来看,周是最能让她敞开心扉的人。\\

只不过,要说那是恋爱感情,那也不对吧。

周偶尔将真昼作为异性来对待,从而会感到害羞,但那是异性间常有的事。虽然周承蒙了真昼的好意,但他认为那不是异性恋爱意义上的好意。\\

虽说最近周多少整理了一下自己的外形,但自己是废人这一点毫无改变。周不太能想象真昼会喜欢自己这样类型的男生。\\

「你就是在这种地方自卑啊。真的是,觉得自己不可能会被喜欢上的那类人」

「不如说那被上天赐予了一切的……啊不,她那些也算是用努力换来的。你难道觉得那样努力、可爱又厉害的女孩子会喜欢我这种没有什么可取之处的人?」

「如果所有的美少女都跟有能力的美男子在一起的话,那些没轮上的家伙们可能会发起恐怖袭击哦」\\

周感觉那不是身为美男子的树可以说的话。\\

「算了,你要这么说的话那现在就当是这样。……那么我就作为朋友来预言一下吧」

「什么啊」

「你迟早会改变。不如说,已经出现了变化的征兆。接下来只差你迈出那一步」

「……你又懂我什么了」

「哈、哈、哈,你以为我跟你当了几年朋友了」

「一年都没到」\\

听到周冷静的吐槽,树「好像是这样啊」地哈哈大笑起来。

虽然两人轻松交谈的内容如此,但是,比起在老家一起度过小学和初中的男性友人们而言,树尽管从高中开始才成为周的朋友,他却相当理解周,并且能够为周着想。\\

「话说回来啊」

「嗯?」

「你虽然说自己配不上她,但那个说法和态度简直就是在承认你对她有好感一样啊」

「我把炸薯条插你鼻子里哦」

「对不起」\\

明明有点感动,但树最后还是说了多余的话,因此周拿起了炸薯条吓唬他。不过周觉得,马上就轻易道歉也挺有树的风格。\\

\vspace{2\baselineskip}

「今天挺晚的呢」\\

周的回家时间比平常晚了一个小时,迎接他的是穿着围裙的真昼。

周不由得想到她是哪家的新娘,这该是因为之前和树的对话吧。真昼明明没那个感情但自己却擅自妄想起来对她本人也不好,所以周慌张地把这念头从脑海里驱逐出去。\\

「嗯,和树去吃炸薯条了」

「……明明是晚餐前」

「没问题晚餐我会全部吃完」\\

周就算吃饱了,真昼的料理也还能吃得下。况且炸薯条也是保守地选了小份,所以现在周的肚子并没有很饱。

就算真昼端出一直以来的份量周也有信心能吃完。\\

「虽然我想是不是会胖……但因为周君太瘦了,说不定多长点肉更好呢」

「你才是要不要多长点肉啊。好像身子骨会折断一样挺可怕的」

「还没到会折断的程度哦」

「是吗?你看,明明这么纤细」\\

真昼有着非常符合少女的纤细腰肢。运动她也样样都会,所以说是纤细也不仅仅是纤细,而是既紧绷又柔软的感觉。\\

真昼乍一看好像真的会折断一样。周试着抓了抓真昼的手腕,发现她的手腕细得可以轻松地用手指围过一圈,仿佛用点力就会折断一样。这样一来,父亲「女孩子就应该温柔礼貌地对待哦」的教诲也可以理解了。\\

周在和真昼牵手的时候也曾有过这样的想法——她的手实在是太纤细,周很担心她会不会在自己不知道的时候受伤。

真昼纤细的手指也好像有点小事就会折断一样,以至于让周怀疑这么纤细会不会有问题。\\

周像是描摹一样地确认着手指的触感和结实程度,接着就看到真昼缓缓扭动着身体。

她稍稍低下头,但视线是向着周握着自己手指的手。\\

看到真昼浮现淡淡红晕的脸颊,周才迟迟发觉自己是没经过许可就不客气地触碰着她,便慌忙松开了手。\\

「……那个,抱歉。我记得你讨厌被别人擅自触碰的吧」

「没、没有……我不讨厌,被周君触碰」\\

真昼说出的话让周一瞬间怀疑了自己的耳朵.。在周的凝视下,她似乎也注意到自己说了什么,突然把头抬了起来。

真昼的脸比先前更红,眼眸也因为羞耻而微微湿润了。在她的目光下,周变得非常坐立不安。\\

「不、不是说要你碰我哦。只是说不想让其他男人碰我而已」

「哦、哦哦」\\

就算真昼这么说,周心中的鼓动也平息不下来。

虽然周明白真昼把自己当做亲近的人而特别对待,但听到那种说法的话,周恐怕会按有利于自己的方式去理解,所以还是希望她不要这样为好。\\

「……对、对了。昨天那个,你没戴着呢。啊啊不对,并不是在催你」\\

周为了掩饰心脏的跳动而问了问,接着,真昼看向自己的手腕,轻轻地用手指描着刚刚周握过的地方。\\

「……做家务的时候戴着的话会碍事,而且这样也会更快坏掉。……我想好好珍惜它,所以只会在休息日戴着」

「……这样啊」\\

听到真昼说出这么惹人怜爱的理由,周差点就一屁股坐了下去。\\

不存在听到这么可爱的话语还不产生意识的男生吧。

真昼珍惜着自己送的礼物,并且打算好好戴上,这两件事都好好地传达给了周。周涌起的百感仿佛要溢满心中一样,让他很是苦闷。\\

周用眩晕的脑袋感受着自己怦怦的、甚至有些吵闹的心跳,做了一个长长的深呼吸以让自己冷静下来。\\

「……你要是喜欢的话,我就很开心了」

「我很喜欢,也很珍惜。不管是熊熊,钥匙包,还是手链」\\

「护手霜倒是没有客气地用着」,真昼有些害羞,嘴角又有些笑意地说道。看到真昼这副样子,周实在无法忍耐,从没脱鞋呆呆站着的状态中急忙脱下鞋进入了家里的走廊。\\

「……我去换衣服」

「好、好的。快去快回,周君」\\

明明已经回到家却感觉好像被新娘送行一般。周体会着这样的心情,心脏又剧烈地跳动起来,于是便快步走进房间蹲在了地板上。


\subsection{休业式与树的相求}

意外地没意思啊。周远远看着校长在台上神色严肃地致辞,强行忍住了哈欠。\\

虽说到了休业式的日子,但周也没什么特别的感慨就迎来了这一天,现在正在听着台上的校长讲话。老实说,校长的讲话无聊到简直要睡着的地步。

周的周围几乎所有学生都是一样的心情。认真听讲的学生屈指可数,大半都是随便听听,昏昏欲睡地看着台上。\\

再怎么说也不能露骨地挂着一副没劲脸,所以周还是摆出了一副认真的表情。然而,他的心里还是盼望着早点结束,校长的致辞就左耳朵进右耳朵出了。\\

要是这是自己的毕业典礼,周说不定还会有些感慨,但只是休业式的话周根本涌现不出激动还是什么的情绪。\\

虽然直说出来不太好,但周确实觉得休业式完全无所谓。所以,周一边装出优等生的样子,一边消磨着无聊的时间。\\

\vspace{2\baselineskip}

「……啊肩膀好酸」

「都怪校长的话太长了啊」\\

典礼结束后一回到教室里,众人纷纷说着这样的话。

不过,他们的声音中却稍稍显出了活力,大概是因为只要等接下来的班会结束,接着就是为期约两周的自由时间了。\\

班里的同学们,都因为终于要从无聊的课堂中解放而嘴角露出着笑意。周在座位上望着这些人,轻轻地叹了一口气。\\

明天开始就是春假了,要怎么度过这段时间呢。\\

这段时间周姑且也在父母面前露过脸了,即使考虑到路费也还是不回家比较好。但这样的话,春假就变得空闲很多了,就算把这段时间拿来预习高二的课程,也还颇有闲暇。

因为事先没有找到合适的工作,要做短期兼职的话恐怕凑不满天数。另外,能在放假时一起玩的朋友也只有树和千岁而已。\\

「我说周君呀」\\

说曹操曹操就到,周身后的树向他搭话过来。

周一回头就看见树非常爽朗的笑容……对周来说,这可疑的笑容让他有种不好的预感。每当树露出这样的笑容,都是想找周帮忙或者会给周带来什么麻烦事的时候。\\

「什么啊」

「你明天以后有空吗?」

「算是有空吧」

「嗯嗯,我就觉得会是这样。太好了太好了」

「……什么啊」\\

树笑容满面,拍了拍挂在自己座位上的书包。

他应该在昨天就把橱柜和书桌的大部分行李带回家了才是,现在那书包里却鼓鼓地塞满了物品。况且今天也没有课要上了,按道理剩下没带的东西至多也就笔袋、文件或者钱包什么的。这样一来,他书包塞满东西的样子就显得很不自然。\\

「……那个是?」

「要换的衣服」

「为什么要带衣服过来」

「借我住一阵」\\

树起劲到了简直可以在话语句尾看到心形符号的程度,用着巴结的语气缠着请求周。所以,周彻底皱起眉头也是没办法的事情吧。\\

「那个,你知道什么是\ruby{报告|联络|相谈}{{\jpfont ほう}|{\jpfont れん}|{\jpfont そう}}吗」

「当然知道了,就是\ruby{访}{\jpfont ほう}问\ruby{连}{\jpfont れん}夜\ruby{噪}{\jpfont そう}音吧」

「那只是在晚上扰邻吧混蛋。你打算吵死人吗」

「开玩笑的啦。虽然想住在你家是真的」\\

树做事基本上不怎么会缺少事先告知。

这么说来他是遇到了什么情况才不得不仓促住在外面,但周想不明白会是什么事。\\

「早上我和老爸吵架了」\\

像是回答周的疑问一样,树轻易地就把缘由说了出来。\\

「……因为千岁的事情?」

「嗯。我老爸一生气,好几天都听不进去别人话的。然后我也不能住在小千家里吧。就算小千的父母愿意收留我,不过再怎么说……」

「住我这就能了吗」

「感觉你会收留我的」\\

恐怕树的想法是,房间还没整理那会他也过来住过几回了,所以觉得没问题吧。\\

周也并不是不愿意让树住在自己家里。

只是,过来这边做饭的真昼会不会觉得不情愿,这是一个问题。\\

如果真昼要被迫在一个休息的场所开启天使大人模式,会不会太过辛苦了呢。

毕竟她只对周展现着自己的本性,在树面前的话还是会掩藏起来的吧。\\

另一个问题是,最近真昼有时会做出微妙地可爱的举止,有时会露出害羞的样子,让周情不自禁产生作为异性的意识。要是让树看到这样,肯定会产生子虚乌有的误会,这很令人害怕。\\

「……我跟那家伙联系一下」\\

毕竟不能不问一下真昼的意见,所以周给她发了条消息。回家前,她会发来购物的清单,在那时她应该会看到这条消息才对。\\

看着周熟练地把消息发了出去,树不知为何好像有些傻眼似的叹了口气。\\

「怎么了,你们在同居吗?」

「要不要让你不开空调不盖被子睡地板啊」

「我该赞扬你收留我的慈悲,还是该哀叹你冻死我的冷漠呢」

「我就想哀叹你那子虚乌有的妄想」\\

周朝着树露出了「你这家伙在说什么」的目光,接着树耸了耸肩。

想耸肩的是自己好吧——周不想因为奇怪的误会而让真昼烦恼。\\

树还算是个识趣的人,应该不会捉弄真昼。但周感觉树会趁着真昼不在的时候戏弄自己,这让他有些郁闷。\\

周看着树的笑容叹了口气,这时真昼似乎碰巧打开了手机,回了一条『你去买三人份的食材的话,我会照常做三人份的』答应了树的留宿。\\

「她说可以」

「太棒了,可以吃到椎名亲手做的东西了」

「你目的不会是这个吧」

「也有一点这个目的。我也想尝试一下周赞不绝口的料理嘛」

「……别给她添麻烦啊」

「就算给你添麻烦也不会给那个人添麻烦的啦」

「也不要给我添麻烦啊」\\

周对着嘿嘿笑着的树的额头就是一发弹指。树一边喊痛一边愉快地笑着,周也故意深深叹了口气给树看。

\subsection{树的情况}

「那,你打算待到什么时候?」\\

周放学后买点东西回到家,歇了一会儿后看向如自己家一样随意的树。

考虑到真昼在家,周最近没怎么放他进来过。不过他已经来过这个家里好几次了,所以才有这样了如指掌的感觉吧。\\

树盘起腿喝着咖啡,因其美型所以还挺像样的。他仿佛在思考似的,把视线在空中徘徊着。\\

「嗯……总之想先住三天。真是麻烦啊」

「你老爸也不是什么坏人,只是欠缺接受他人主张的灵活性而已」

「你可以直接说成是既顽固又不懂变通的生错时代的混蛋老爹哦」

「我说啊」

「我怎么忍得了自己的交往对象被父母说这说那」\\

虽然树说着「反正成年了也要离开家」,但他其实并不是真的讨厌他父亲。

树的父亲是一位通情达理的男性,一旦令自己满意后就会亲切地对待对方。他现在这样只是因为千岁不太令他满意,但在周看来,他还是一个挺好的人。\\

他不认同树和千岁的交往,很大程度上是因为树的家世还不错,所以希望能选择和长子相配的女性吧。

而且,恐怕树的父亲不擅长应对千岁也是一个单纯的原因。\\

只不过,对于不分青红皂白就被否定的树来说,似乎就是因为这样才会选择离家出走。\\

「在这方面周可真舒服啊。能想怎样就怎样」

「因为我父母超恩爱的嘛,而且也希望儿子选择喜欢的对象」

「真的羡慕你父母」\\

树能成长到现在这样也是严格教育的结果,所以也不能对此太过否定。

听他本人说,把头发染成明亮的颜色,打扮成看起来很轻浮的外表也是一种反抗。\\

「就算你这样说,但其实还是挺尊敬你父亲的吧」

「虽然在人格方面我很尊敬他,但作为家长还是不行吧。又不是只要压迫就行的……明明适度地给点甜头就行了,但他却只用鞭子教育所以自然会被反咬一口啊」

「被给甜头的一方认识到这一点真的好吗」

「明明把我放养我都能接受,他却打算把我关进笼子套上项圈,所以我才会反抗,仅仅是这样而已」\\

树耸着肩说道「活了几十年他却好像连这都不懂」,接着把剩下的咖啡一口气喝光了。\\

「我也不再是小孩子了,各方面都有考虑过」

「考虑?」

「虽然到大学毕业为止不得不靠父母养活这让我很气愤,但是从这之后离家出走两个人一起生活要花多少钱、手续怎么办之类的,我都有考虑」

「以离家出走为前提啊」

「如果他不认可我们的话」\\

周觉得树作为高中生能有这种觉悟,在某种意义上来说很厉害。然而,周和树的父亲保持着还算良好的关系,终究还是希望他们二人能够和解。\\

总之,在树父亲的怒气缓和之前他应该都会待在这里,但周还是希望他们尽早和好。\\

「嘛,这几天你就好好放松一下吧。幸好这几天放假,有的是时间」

「有个朋友真好……!」

「别黏过来,难受」

「我受伤了!作为精神赔偿我要求椎名的料理!」

「就算没受伤你不也会吃吗」

「诶嘿」

「装什么可爱啊真恶心」

「好过分这骂得更直接了……哦哟哟」\\

树虽然装作在哭,但他脸上还是在笑的。看着他这副样子,周一边觉得无语,一边稍微松了口气。\\

虽然树和他父亲争斗是常有的事,但今天早上的这次似乎更加严重一点。也许是心理作用,周感觉树在学校里是强打着精神,但现在多少有些恢复了。\\

不过,这种想法无论如何也没法对本人说出口,所以周一边假装冷漠地对待树,一边轻轻地叹了口气。\\

\vspace{2\baselineskip}

太阳下山之后,真昼来到了周家里。

她空着手是因为周已经准备好了真昼要求的食材吧。\\

因为周事先告诉了真昼今天树也在家,所以她就算看到树毫不客气地休息着,也没有露出动摇的样子。不如说倒是树微妙地有些慌张。\\

「赤泽,好久不见了」

「嗯,好久不见。突然到访你们的爱巢……疼、疼疼疼,我知道啦只是开个玩笑啦。突然打扰你们真抱歉,有我这样不习惯的家伙在很困扰吧」\\

因为周默默地踩着树的脚所以他喊着疼,尽管如此,树还是笑嘻嘻地露出了讨人喜欢的笑容。\\

「才没有那种事。热闹一点才更开心嘛」

「有这家伙在也只是吵闹啊」

「不应该说这种话哦」\\

周因为受到责备所以闭上了嘴巴,却看见树笑嘻嘻的,于是在真昼看不见的地方拧了一下树的侧腹。

另外,由于树有着男生的理想体型,所以基本没有可以捏的部位。\\

「那么我就去准备晚餐了,请随意」\\

两人展开一场小小的攻防战的时候,真昼脸上浮现出天使的微笑,穿上围裙前往了厨房。

真昼大概是觉得,因为实在不知道要和树说些什么好,所以就把他交给周了吧。\\

树眺望一阵真昼的背影之后,收起了脸上的窃笑。\\

「……关系这么好,钥匙都给了啊」

「你烦不烦」\\

真昼会用钥匙进来,是因为完全养成了习惯吧。由于她没按门铃就进来了,所以让树注意到了这一点。\\

「『请随意』是因为椎名同学把这里认作自己的安身之处才会这样说的吧?那个态度看起来已经像是太太一样了哦」

「我可以把你赶出去吗」

「我虽然也想说是玩笑,但你认清楚了,客观来看就是这样的哦?」\\

周刚想抓住树的脖子就给他逃开了。因为他已经坐在地毯上打开了游戏,所以周也只能从沙发上下来,边用膝盖轻轻地撞他一下,边坐在他旁边打算用游戏消磨时间。\\

过了一会儿周就开始听见把盘子拿出来的声音,到底还是不能让真昼一个人做所有的家务活,所以周站起来往厨房走去。\\

「我来帮你,把装好菜的盘子端走就好了吧」

「谢谢」\\

周一如既往地把料理摆在桌子上后,只见树一脸无语的样子。\\

「……怎么说呢……」

「什么啊」

「算了,我不说了」\\

看着话不说完就去整理游戏机的树,周发出了「啥跟啥啊」的含有些许困惑的声音。


\end{document}
