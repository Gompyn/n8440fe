% \subsection{75 天使様との昼食}
\subsection{75 天使様との昼食}

% 「周~、今日から私達も一緒にご飯食べるね!」\\


% にこやかな笑みを浮かべながら真昼を伴ってきた千歳に、周は先日の真昼の『考えがあります』という発言を思い出して頬をひきつらせた。\\


% 確かに、千歳が樹と一緒に食べるという名目なら、真昼を伴っても単に友人をつれてきたと誤魔化せる。周りに多少妬まれはしても、怪しまれる事はないだろう。\\


% 千歳に手を引かれた真昼はにこにこと笑みを浮かべており、いつもの天使様のように振る舞っている。


% ただ、心なしかしてやったという表情にも見えるのだから、周は頭を抱えたくなった。\\


% 「あー、俺席外した方がいいんじゃないのか?」


% 「そんな事ないですよ。こちらからご一緒させていただくんですから」\\


% 逃がさない、という意思をひしひしと感じる。


% 千歳は根回しされて……というより千歳発案な気がしてならない。にこにこ、いやにやにやしている千歳に眼光を鋭くしても、彼女はどこ吹く風だった。\\


% 樹も根回しされているのか、それとも千歳と共に食事をするのが嬉しいのか、いつもの笑顔で「いいんじゃね、一緒で」とのたまう始末。\\


% 周としては、先日よいと言ったものの、やはり周囲の羨ましそうな視線に気圧されてたじろいでしまう。\\


% 「あれ、白河さんと椎名さんも一緒に食べるの?」\\


% 今日も一緒に食べるつもりだったらしい門脇までひょっこりと顔を覗かせたので、周の胃は微妙に痛みを覚えていた。\\


% 「はい、ご一緒させていただくつもりです」


% 「そっかー、賑やかになるなあ」\\


% 門脇はのほほんと笑っているが、賑やかどころではなくなる気がする。


% 反対するような反応は門脇にはなく、精々真昼がくる事に驚いているといったところだ。\\


% もう詰んでいた。\\


% 「……諦めろ周、包囲網が出来てる」\\


% 門脇に聞こえないように小さく呟いた樹に、周は疲れたように大きく一つ吐息を吐き出したのだった。\\


% \\


% 「椎名さんお弁当なんだ」\\


% 周や樹は食堂でいつも食べているので、日頃教室で食べている真昼達もそれに合わせる形となっていた。


% それぞれ男性陣が注文した昼食を持って席に着くと、門脇が真昼が広げたお弁当の存在に気付く。\\


% ちなみに真昼は周の真正面に座っている。千歳がそこに促したので逃げる隙がなかった。\\


% 「ええ、夕食の余り物を詰めてる事が多いですけど」\\


% 夕食の残りは周の朝ごはん分別に残しつつたまにお弁当分で取り分けているので、今日はそれを詰めてきたのだろう。というか昨日の夕食に出たとりつくねの照り焼きが詰め込まれていた。\\


% 「へえ、もしかして手作り?」


% 「ええ。といっても大したものは作れませんが」


% 「まひるん嘘はよくないよー、すごく料理上手なのに」


% 「ちぃは椎名さんに弟子入りしたらどうだ」


% 「いっくんひどい」


% 「ちぃは料理の味付けだけ教えてもらえばいい。料理自体は出来るのになあ……味付けを奇抜にするから」\\


% 千歳は決して料理が出来ない訳ではないのだが、悪戯心がうずくと新たなる味を求めて普通から逸脱していく。その悪癖さえなければなあ、と樹がよくぼやいている。\\


% 「じゃあまひるんに今度マンツーマンでお料理教室開いてもらおー。毒味役に周呼んで」


% 「おい毒味言うな。あと椎名に迷惑かかるから急にそういう事言うのやめろ」


% 「いえ、私は迷惑とか思ってませんよ。千歳さんと一緒にお料理するの楽しそうですから」


% 「わーいまひるん好きっ。楽しみー! 周も予定空けといてね!」\\


% 千歳は真昼の隣に座っていて、満面の笑顔で真昼にべったりとくっついている。


% 真昼もそれを微笑みながら受け入れているので、すっかり仲良くなったなあと感慨深さを覚えたところで――気付く。\\


% (今みんなの前でナチュラルに遊ぶ約束をさせられたよな)\\


% 千歳を見ても、本人は真昼と仲良く笑い合っている。故意なのか、偶然なのか、分からない。


% ただ、周囲に居て微妙に聞き耳を立てている同級生の人間達と目があって「ウラヤマシイ」と声にならない妬みが飛んできたので、頬がひきつった。\\


% 「……なあ樹」


% 「ん?」


% 「これ俺殺されない? 大丈夫か?」


% 「大丈夫だ多分」\\


% 真昼のファン、というか真昼に思いを寄せている男子達から結構な視線を浴びているので、気が気ではない。\\


% まだ千歳主導だから殺気を向けられたりはしていないのだが、これが表で仲良くなってから真昼が何か言い出した時が怖い。まず彼らの中で「何であんなやつが……」と言われるだろう。\\


% 「よかったじゃないか藤宮」


% 「……俺がお前だったらそこまで妬まれずに済んだんだろうなあ」\\


% 門脇ほどの美形で多才な人間なら、真昼とも釣り合いがとれて妬まれはしても仕方ないなと諦められただろう。\\


% 「俺は藤宮が羨ましいけどなあ」


% 「どこがだよ」


% 「色々」\\


% 含みのある言い方をした門脇が苦笑したので、首をかしげるしかない。\\


% 「まあ、優太の気持ちも分からないでもないなあ」


% 「マジか」


% 「人間持ってるものに気付きにくいもんだ。その上ない物ねだりってな。ちぃもよくない物ねだりする」


% 「というと?」


% 「椎名さんにあってちぃにないものだな……」


% 「いっくん今絶対変な事考えたよね?」\\


% 話を聞いていたらしい千歳がにぃっこりと、満面の笑みを浮かべた。しかし、目が笑っていない。\\


% あっこれ地雷踏んだな、と察したので二人が仲良くお話し合いし出すのを眺めて、ちらりと真昼を見る。\\


% 真昼は千歳が樹とじゃれあい始めたのに困惑していたが、周と目が合うと微笑みに表情を変える。


% それが、天使様の笑顔ではなくていつも家で見せるようなはにかみに近いもので、周は気恥ずかしくなって視線を逸らす事になった。

