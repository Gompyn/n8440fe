% \subsection{75 天使様との昼食}
\subsection{75 天使様との昼食}

% 「周~、今日から私達も一緒にご飯食べるね!」\\
「周~,今天开始我们也一起吃饭了哦!」\\

% にこやかな笑みを浮かべながら真昼を伴ってきた千歳に、周は先日の真昼の『考えがあります』という発言を思い出して頬をひきつらせた。\\
周看着一脸坏笑地把真昼带了过来的千岁,想起了真昼前几天说过『我有一些打算』的事情,脸角不禁开始抽搐。\\

% 確かに、千歳が樹と一緒に食べるという名目なら、真昼を伴っても単に友人をつれてきたと誤魔化せる。周りに多少妬まれはしても、怪しまれる事はないだろう。\\
不过确实,如果是以千岁和树一起吃饭这一名义,那么就算带上真昼,那也可以装作只是千岁把朋友也一起拉来了。尽管多少会让周围心生嫉妒,但就事情本身应该不会过于奇怪。\\

% 千歳に手を引かれた真昼はにこにこと笑みを浮かべており、いつもの天使様のように振る舞っている。
被千岁拉着手牵来的真昼,脸上泛着笑容,保持着一如往常的天使举止。

% ただ、心なしかしてやったという表情にも見えるのだから、周は頭を抱えたくなった。\\
不过,不知为何,那表情看上去又似乎在说着「搞定啦」一般,另周不禁想要扶额 。\\

% 「あー、俺席外した方がいいんじゃないのか?」
「额——,我是该识趣地让个位嘛?」

% 「そんな事ないですよ。こちらからご一緒させていただくんですから」\\
「没关系的。我才是,可以在这里打扰你们吗?」

% 逃がさない、という意思をひしひしと感じる。
真昼的话里满满的都是「别想跑哦」的意思。

% 千歳は根回しされて……というより千歳発案な気がしてならない。にこにこ、いやにやにやしている千歳に眼光を鋭くしても、彼女はどこ吹く風だった。\\
肯定又是千岁在煽风点火……不如说估计是千岁出的馊主意。周狠狠的瞪了一眼嘻嘻地笑着——或者说是坏笑着的千岁,被瞪了的千岁则摆出一副事不关己的表情。\\

% 樹も根回しされているのか、それとも千歳と共に食事をするのが嬉しいのか、いつもの笑顔で「いいんじゃね、一緒で」とのたまう始末。\\
不知是因为也添了把柴还是只是因为能和千岁一起吃饭感到高兴,树一脸平日里那种笑容地「一起吃饭不挺不错嘛」这么劝道,让这闹剧告一段落。\\

% 周としては、先日よいと言ったものの、やはり周囲の羨ましそうな視線に気圧されてたじろいでしまう。\\
虽说前几天周也给出了同意,但周围这满是羡慕的视线还是令周如坐针毡。\\

% 「あれ、白河さんと椎名さんも一緒に食べるの?」\\
「哎呀,白河和椎名你俩也一起吃吗?」\\

% 今日も一緒に食べるつもりだったらしい門脇までひょっこりと顔を覗かせたので、周の胃は微妙に痛みを覚えていた。\\
门胁似乎今天也打算一起吃饭,突然横插一脚进来,令周莫名地有点胃疼。\\

% 「はい、ご一緒させていただくつもりです」
「嗯,欢迎你加入进来」

% 「そっかー、賑やかになるなあ」\\
「哈哈,这下可就热闹起来了啊」\\

% 門脇はのほほんと笑っているが、賑やかどころではなくなる気がする。
虽然门胁一脸笑容说得漫不经心,但周觉得这场面实在不该热闹起来。

% 反対するような反応は門脇にはなく、精々真昼がくる事に驚いているといったところだ。\\
周表现出反对的态度,并非是因为门胁,而纯粹是真昼跑来一起吃饭让他吃了一惊 。\\

% もう詰んでいた。\\
周已经被将死了。\\

% 「……諦めろ周、包囲網が出来てる」\\
「……放弃吧周,你已经被包围了」\\

% 門脇に聞こえないように小さく呟いた樹に、周は疲れたように大きく一つ吐息を吐き出したのだった。\\
树以门胁听不到的音量小声跟周说道,周则只得疲累地长吁一口气。

% \\


% 「椎名さんお弁当なんだ」\\
「椎名你是带便当的啊」\\

% 周や樹は食堂でいつも食べているので、日頃教室で食べている真昼達もそれに合わせる形となっていた。
周和树平常都是在食堂吃饭的,于是配合他俩平常在教室里吃的真昼他们也去了食堂。

% それぞれ男性陣が注文した昼食を持って席に着くと、門脇が真昼が広げたお弁当の存在に気付く。\\
周和树他俩买好了午饭,在座位上坐定后,门胁在意起了真昼拿出来的便当。\\

% ちなみに真昼は周の真正面に座っている。千歳がそこに促したので逃げる隙がなかった。\\
顺带一提真昼坐在周的正对面。千岁设了个局让周无处可逃。\\

% 「ええ、夕食の余り物を詰めてる事が多いですけど」\\
「嗯,虽说基本上是没吃完的晚饭」\\

% 夕食の残りは周の朝ごはん分別に残しつつたまにお弁当分で取り分けているので、今日はそれを詰めてきたのだろう。というか昨日の夕食に出たとりつくねの照り焼きが詰め込まれていた。\\
剩下的晚饭除了分出一部分作为周的早饭外,有时真昼也会取一部分作为便当,今天大概就这么做了吧。不如说便当盒里装满了昨天晚饭的照烧鸡肉丸。\\

% 「へえ、もしかして手作り?」
「哎,难道是自己做的?」

% 「ええ。といっても大したものは作れませんが」
「嗯。虽然这么说但也只是些小打小闹呢」

% 「まひるん嘘はよくないよー、すごく料理上手なのに」
「昼儿哟撒谎可不对哦~,明明超擅长做饭的不是嘛」

% 「ちぃは椎名さんに弟子入りしたらどうだ」
「小千你要不拜椎名为师吧」

% 「いっくんひどい」
「阿树好过分」

% 「ちぃは料理の味付けだけ教えてもらえばいい。料理自体は出来るのになあ……味付けを奇抜にするから」\\
「椎名你只要叫小千她调味就好了。单论做饭她还是能搞定的……就是调味太过诡异了」\\

% 千歳は決して料理が出来ない訳ではないのだが、悪戯心がうずくと新たなる味を求めて普通から逸脱していく。その悪癖さえなければなあ、と樹がよくぼやいている。\\
虽然千岁肯定是会做饭的,但她总是想要皮一下,去探索新奇的味道,结果就让她的调味脱离常轨了。因此树也常常「要是小千她没这个怪癖就好了啊」这么叨念着。\\

% 「じゃあまひるんに今度マンツーマンでお料理教室開いてもらおー。毒味役に周呼んで」
「那昼儿你下次就给我开个一对一的烹饪课吧~。试毒就拜托周啦」

% 「おい毒味言うな。あと椎名に迷惑かかるから急にそういう事言うのやめろ」
「喂试毒是什么鬼啊。另外突然这么说会给椎名她添麻烦的吧」

% 「いえ、私は迷惑とか思ってませんよ。千歳さんと一緒にお料理するの楽しそうですから」
「嗯,我不觉得麻烦哦。跟千岁一起做饭听起来很有意思呢」

% 「わーいまひるん好きっ。楽しみー! 周も予定空けといてね!」\\
「呜哇昼儿我爱你~!好高兴~!周你也记得空出时间来哦!」\\

% 千歳は真昼の隣に座っていて、満面の笑顔で真昼にべったりとくっついている。
坐在真昼一旁的千岁,以满脸笑容扑了上去紧紧抱住了真昼。

% 真昼もそれを微笑みながら受け入れているので、すっかり仲良くなったなあと感慨深さを覚えたところで――気付く。\\
真昼一脸微笑,放纵千岁的撒娇,而周则感慨着这俩人的关系可真是好啊——然后,突然意识到。\\

% (今みんなの前でナチュラルに遊ぶ約束をさせられたよな)\\
(刚才在大庭广众之下我好像自然地被约好了一起玩来着)\\

% 千歳を見ても、本人は真昼と仲良く笑い合っている。故意なのか、偶然なのか、分からない。
周看向千岁,可她却依旧和真昼互相说笑着。周搞不明白,这到底是千岁的圈套,还是偶然的结果。

% ただ、周囲に居て微妙に聞き耳を立てている同級生の人間達と目があって「ウラヤマシイ」と声にならない妬みが飛んできたので、頬がひきつった。\\
只不过,沐浴在周围那些微妙地竖起耳朵听着这边的同学们眼中射出的好似说着「羡慕啊」的嫉妒目光之下,周不觉地皱起了脸。\\

% 「……なあ樹」
「我说啊树」

% 「ん?」
「咋了?」

% 「これ俺殺されない? 大丈夫か?」
「我感觉我可能要被做掉了耶?」

% 「大丈夫だ多分」\\
「没事啦大概」\\

% 真昼のファン、というか真昼に思いを寄せている男子達から結構な視線を浴びているので、気が気ではない。\\
被这些真昼的粉丝——嘛,其实就是对真昼有想法的男生们这么狠狠地盯着,周实在是顶不住。\\

% まだ千歳主導だから殺気を向けられたりはしていないのだが、これが表で仲良くなってから真昼が何か言い出した時が怖い。まず彼らの中で「何であんなやつが……」と言われるだろう。\\
虽说到底主导的还是千岁,所以小命应该还是能保住的,但如此推算万一真昼说了些啥,让两人的关系摆上了台面的话那就很吓人了。首先肯定会被他们「为啥是你个小子啊……」这么抱怨吧。

% 「よかったじゃないか藤宮」
「藤宫你这不是挺幸运的么」

% 「……俺がお前だったらそこまで妬まれずに済んだんだろうなあ」\\
「……要我有你那样那也不会被嫉妒到这个程度啊」\\

% 門脇ほどの美形で多才な人間なら、真昼とも釣り合いがとれて妬まれはしても仕方ないなと諦められただろう。\\
如果邀请的是门胁这种配得上真昼的多才多艺的帅哥的话,那么就算他们再怎么嫉妒,也会知难而退的吧。\\

% 「俺は藤宮が羨ましいけどなあ」
「我倒是挺羡慕藤宫的呐」

% 「どこがだよ」
「我有哪好羡慕的」

% 「色々」\\
「有很多啦」\\

% 含みのある言い方をした門脇が苦笑したので、首をかしげるしかない。\\
门胁面露苦笑,含蓄地发出感叹,令周不大能够理解。\\

% 「まあ、優太の気持ちも分からないでもないなあ」
「嘛,优太的那感觉意思我大概也能明白」

% 「マジか」
「真的?」

% 「人間持ってるものに気付きにくいもんだ。その上ない物ねだりってな。ちぃもよくない物ねだりする」
「人总是不知已有之物的珍贵,而垂涎于不可得之物啊。小千她也是,经常冒出些不讲道理的要求」

% 「というと?」
「所以说?」

% 「椎名さんにあってちぃにないものだな……」
「在说椎名她有而小千没有的东西啊……」

% 「いっくん今絶対変な事考えたよね?」\\
「阿树你刚才是想了些乱七八糟的事情吧?」\\

% 話を聞いていたらしい千歳がにぃっこりと、満面の笑みを浮かべた。しかし、目が笑っていない。\\
千岁似乎是听见了树的话,脸上露出了灿烂的笑容——然而,眼中却毫无一丝笑意。\\

% あっこれ地雷踏んだな、と察したので二人が仲良くお話し合いし出すのを眺めて、ちらりと真昼を見る。\\
周察觉到树这是踩到地雷了,便在一旁观看着两人的亲密交流,顺带朝着真昼瞄了一眼。\\

% 真昼は千歳が樹とじゃれあい始めたのに困惑していたが、周と目が合うと微笑みに表情を変える。
树和千岁突然开始打情骂俏,令真昼有些搞不清状况,但和周对上了眼之后,她的脸上便露出了笑容。

% それが、天使様の笑顔ではなくていつも家で見せるようなはにかみに近いもので、周は気恥ずかしくなって視線を逸らす事になった。
映入眼中的,并非那天使的笑容,而是近似于平日里在家中的羞涩的微笑,这让周也害羞了起来,不自觉地偏开了视线。
