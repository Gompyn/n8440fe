\subsection{围裙与手制料理是男人的浪漫}

真昼同意在周家里做饭的同时,提出了如下条件。\\

\begin{itemize}
    \item 周出材料费的半数加上若干人工费。
    \item 如果有事不能一起吃饭至少提前一天通知对方。
    \item 食材的采购和饭后的处理由两人分担。
\end{itemize}

关于第一条中的人工费,是周不好意思占用真昼的时间所以才提出的。在这一点上真昼做出了让步,而其它部分则没有发生什么争执,顺畅地决定了下来。\\

关于由真昼来做饭这一点,由于是必然的结果,所以并没有什么可烦恼的。\\

于是乎这么决定后,次日,真昼便早早地拎着——准确来说是两只手抱着购物袋过来做起了下厨的准备。\\

「……还真是新得几乎没有好好使用过的痕迹啊……」

「啰嗦」\\

现在家中有一位穿着围裙的女性。周明明身处于这一男人的浪漫具现一样的状况,可却不知为何感到如坐针毡。\\

将头发扎成一束的真昼带来的新鲜感也是原因之一,但主要理由还是在于厨房基本就没使用过这点被真昼再次指出所造成的尴尬吧。\\

「明明有这么多好东西却被放着吃灰」

「你能用上的话那不就不吃灰了么」

「那是结果论吧。多好的厨具啊,怀才不遇得都要哭了」

「那就用你擅长的料理让它们破涕为笑吧」\\

周干脆地表达自己不行,真昼则一脸无语地看着他,但也许是料到如此,周只是叹了口气而并没有抱怨什么。\\

「那么,做饭用的调味料有么」

「有啊,你当我傻吗。保存方法和保质期也都没问题」

「哎呀真是意外」

「因为都没开封呐」\\

大部分调味品都是没有开封,放在阴凉避光的地方。应该不必担心吧。

尽管买来了,这些东西却得不到展现自己的机会。周基本没下过厨房,其实到最后都没使用过这些。对调味料来说,能被真昼这位厨师使用,也算是物尽其用吧。\\

「这可不是什么好嘚瑟的事情。不过,要是不够的话我回家拿一趟来用就好」

「那就好」

「总之既然有基本的调味料的话多少是能做出点东西的。啊,今天的菜单就我一个人定下来可以吧」

「我也不太清楚这些东西,能吃的话什么都可以的。我不怎么挑食」

「这样啊。那我就动手了……调味料放在哪了告诉我一下」

「都放在这个篮子里」

「……还真是都没开封呢……」\\

真昼瞄了一眼塞满了调味料的篮子,无语地皱了皱眉,但因为周事先说过,马上便恢复了表情,在水龙头旁边洗起了手。\\

「那我就开始做饭了。你就在客厅或者房间里等着就好」

「行。反正我也帮不上忙」

「你这还真是干脆呢……嘛算了。要是你不会料理还晃悠来晃悠去的我也很难办」

「你也很直接啊」

「反正是事实。跟你说也没必要拐弯抹角吧」\\

正如真昼所说,自己显然是个累赘,周便老实地走回客厅观察真昼的背影。\\

真昼洗完了手,迅速投入到调理工作中。\\

虽然不知道她要做什么,但从准备好的材料看应该是日式的。

如此美味的料理竟是在自己家里做出来的,不禁让人感到不可思议,怀疑是不是在做梦。然而,真昼正摇晃着背后扎成一束的秀发处理着食材却是不争的事实。\\

(……怎么说呢,感觉跟有了老婆一样)\\

尽管两个人互相并没有这样的感情,然而目前的状态看上去实在像是成了家一样,让周不由得心生想象。\\

周自然是对真昼没有一丝一毫的非分之想,不过有个美少女在自家厨房这个状况本身就足够让人浮想联翩了。\\

果然,就算不论好感,仅仅是可爱的少女正在做饭这一场景,便足够让周有种莫名的感动。\\

「……你不会在想些乱七八糟的事吧?」

「别瞎猜啊」\\

真昼没有回头的突然发问让周脸差点抽筋,但也万幸真昼没有回头让周不致败露。\\

这家伙可真是敏锐啊——周既佩服又感到发凉,收起了微微涌起而不至于邪念的男心,继续观察着真昼的背影。
