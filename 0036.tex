\subsection{天使大人与新年准备}

12月31日,除夕。

今天是一年的最后一天,是为一整年画上句号的日子。

虽然一般来说这一天都是在为明年做准备和大扫除中匆忙度过的——\\

「那个,真昼啊」

「怎么了?」

「……我在这干闲着真的行么?」\\

周悠闲地坐在客厅的沙发上,望着从早上就开始穿着围裙在厨房里忙活的真昼的背影。\\

真昼从早上就过来,是为了准备年菜。

既然决定了两人一起跨年,那当然也要准备两人份的年菜。\\

周本想着真昼会去外面买一些对付过去,但看起来她是打算自己做的样子。连家庭主妇都感到头疼的家务,这位花季女高中生居然一个人搞定,实在是令人惊讶。\\

周真心叹服真昼的能干,但本人却说「就算要买这种东西也要事先预订,现在已经没办法了」。

听完真昼的解释,周虽然理解了真昼亲自动手的原因,但还是对不辞辛劳亲自制作年菜的真昼脱帽致敬。\\

当然了,能省事的地方真昼也会省事,像黑豆这种东西煮起来花时间不说,还要占掉一个炉灶,所以她就直接买来了成品。\\

「周君,就算你因为在那闲着感觉良心不安,你觉得你能帮上忙吗?」

「不能」

「那不就是嘛。比起过来碍事,还不如在那老实待着更好」\\

周乖乖遵从观点实在辛辣的真昼的旨意,老实地坐在沙发上,但什么都不干还是让周冷静不下来。\\

就算是周,也并不是什么活也没有干过。

毕竟大扫除昨天就已经完事了,家里也已经屯好了不出门也可以支撑一段时间的,包含了年菜材料的大量食材。

虽然周不是什么也没做,但要是跟现在的真昼比起来那就是没出多少力了。\\

「昨天你把家具家电这些都搬开来全部打扫一通也累了吧,今天你就休息着吧」\\

真昼以言语关心着负责力气活的周,不过并没有回头看周这边而是继续打理着料理。\\

顺带一提,真昼自己家的大扫除似乎是已经完成了。再说真昼她似乎有在认真地进行着定期打扫,大扫除并没有费她太多时间。\\

「哎呀,就算是这样但还是感觉……有点抱歉啊」

「我也是喜欢做饭才做的,并不觉得累哦」

「就算是这样啦」

「没事,我很享受的」\\

真昼以「小意思」的语气说完,然后就把精神集中到了料理上去。周不知如何是好,抱住了头。\\

\vspace{2\baselineskip}

「真昼,午饭买来了」\\

考虑着让已经在年菜上忙得不可开交的真昼再准备午饭实在有些过分,周就去便利店里随便买了些午饭。反正真昼本来食量就不大,一袋三明治应该就没有问题了吧。

真昼正好也暂时脱下了围裙,打算稍事休息的样子,从时间上来说是正正好。\\

「特意去准备午饭,谢谢你了。抱歉哦,我实在是抽不开空准备午饭」

「哎呀要说的话拜托你准备年菜这边就已经很对不起你了啦……来,开饭吧」\\

到了兼作休息的午饭饭点,真昼老实地回到了客厅。\\

「三明治和咖啡没问题吧?」

「嗯,谢谢」\\

真昼轻轻点头,从周那接过午饭,然后坐到了周的身边。\\

「话说做得怎么样了?」

「有一部分是买来的已经做好的,品种数目也有控制,所以现在差不多已经做完了。另外还剩下很多东西等着放凉了装起来。另外周君你好像很喜欢鱼肉末鸡蛋卷的样子,那部分是我自己做的哦」

「为什么你会知道啊」

「你说过自己喜欢鸡蛋料理的吧」\\

尽管当时只是随意一说,但真昼却认真地记下来,还特意用烤箱去烤制了。听见烤箱的工作声,周还想着是在做什么,看来做的是鱼肉末鸡蛋卷。\\

「喜欢微带甜味的口感对吧?」

「你很了解我嘛」

「再怎么说也已经好几个月了,喜好什么的还是能记住的」\\

真昼说着很让人高兴的事,开始吃起了火腿生菜三明治。\\

周也边啃着买来的饭团,边看着厨房那边,眼睛望着的地方放着真昼带来的小尺寸的重箱\footnote{\ruby{重}{chóng}箱:装食品用的多层方木盒,常用于新年的年菜({\jpfont おせち})。}。

年菜应该会装在那重箱里面吧。\\

明明是一个人住居然连重箱都有就已经让周始料未及了,当真昼拿出那个不但涂漆还贴了金箔的,看上去就很高级的重箱的时候,周都有点被吓到了。\\

「实在是,让人只能说感激不尽了啊。……该怎么说呢,刚刚开始独居生活的时候还真是想象不到,今年后半年伙食上能这么充实啊」

「我倒是想感叹亏你还能活到今天呢」

「好过分。其实靠便利店这些地方卖的东西意外地能过下去哦?」

「但是不健康啊,真是的」\\

虽然真昼无奈地叹了口气,但她那表情混着苦笑,仿佛在说着「拿你没办法」一样,让周的心里稍微跳了一下。\\

「有我在的话,可是不会允许不健康的饮食生活的哦?」

「你是我妈么」

「都怪周君过得太不健康了。明年的饮食我可是要安排得更健康的哦」\\

看着真昼微妙地提起了干劲,周想到「真昼明年也打算在一起啊」,便莫名地害羞起来,偏开了眼睛。\\

不过,真昼把周这样的态度理解成想要过邋遢日子,便以略带不悦的表情看向了他。结果,周花了一小会儿功夫才解释清楚了自己不是那个意思。
