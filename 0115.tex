\subsection{上学风景}

「感觉到有视线」\\

随着与学校的距离变短,众多视线扎向了周,让他不禁发出了疲劳般的嘟哝。\\

视线的内容多种多样,有表示「和真昼牵手一起走的是谁」的,有混着嫉妒和好奇的,还有带着羡慕的。\\

虽说是预料之中,可一旦亲身体会,他就觉得这种视线比想象的更加难受。\\

幸好,并非所有视线中的感情都是负面的,但不习惯的东西到底是没法习惯。

周一直以来欣然度过的是平凡不起眼的生活。因而,现在他感到难以平静。\\

「没办法。周君乍看上去就像换了个人一样」\\

带着表现出两人是恋人的目的,两人牵手紧靠着前行,当然会有同样去上学的男生看过来。\\

只不过,体育节展示出来的周和现在走在真昼旁边的周似乎有很大差距。虽然没人讲出来,但这些人的视线中都切实地带着询问的意思。\\

「区别有那么大吗?」

「嗯。怎么说呢,当然发型有变化,外表是不同了,不过更重要的是周君挺直了背,表情也很自信,所以印象会有很大区别」

「抱歉了啊平时这么不争气」

「请不要自虐……说到底周君已经改变了。虽然两种周君我都喜欢,但我讨厌自卑的周君」

「我不想被说讨厌,所以会注意的」

「那就好」\\

真昼微笑着把身体紧靠了上来,便又有视线看过来了。\\

这次的视线混着杀气,让周脸上差点微微抽筋。然而,真昼朝着周围露出了极品的天使微笑后,这些杀气就彻底消失了。\\

能让周围人全部闭嘴的天使大人在某种意义上是最强的。\\

在相对改善了一点的视线中,周感觉刺刺的难受,但还是重新握住真昼的手看向前方。马上就到学校了。在学校里,周会沐浴在更多的视线当中,所以他现在就开始有点胃疼了。\\

「现在视线都成这样了,待会儿怎么进教室啊」

「请放弃吧……还是说,你讨厌这样?」

「不讨厌。我已经决定要好好改变了」\\

从真昼表白那时起,周就明白他不能再和过去一样了。

周已经决定,就算为了在她旁边也不能表现得丢脸。与其疏于努力,不如抱着多多少少胃痛的心理准备,让自己能够配得上真昼。\\

真昼闻言,回答说「……是吗」,相扣的手指上更加使劲了。\\

「咦,昼儿?」\\

周注意到旁边的真昼耳朵微红,正打算跟她讲话,就听到后面传来了声音。

听到熟悉的声音和亲昵的爱称,周回过头,看见千岁眨着睁大的眼睛站在那里。\\

她一副目瞪口呆的表情,看了眼真昼,然后看向在她旁边的周。

看到牵着的手,千岁「哦呵~」地笑着,小跑到两人跟前,猛地拍了拍周的后背。\\

「早上好~总算到这一步了吗兄弟」

「你好吵」

「昼儿也早上好~看来很顺利」\\

千岁砰砰地使劲拍着他,满脸笑容,看上去心情很好。\\

今天,周一直都感受着好奇和嫉妒的眼神。而千岁的眼神是纯粹的好意,让周心里微微发热。\\

「恭喜昼儿啦,不枉我一直关注着」

「是啊,有很多事情都是找你商量的」

「嗯嗯。比如周这么迟钝怎么办之类的」

「……真昼」

「因、因为,周君事实上就是很迟钝嘛」\\

听她这么一说,周不太能反驳。

明明真昼一直都在表现着这一点,周却没能好好应对,确实责任都在他身上。真昼会去和千岁商量大概也是没办法的。\\

曾帮真昼商量的千岁说着「毕竟是周嘛」这种让人不太高兴的评价,然后再次抬头看向了周。\\

她会露出这种观察般的眼神,恐怕是因为她第一次看到了好好打理的周吧。\\

「哎呀~话说回来,第一次见到周的那种男人形态啊~」

「这说法是怎么回事」

「阿树和小优都是那么说的。嗯嗯,虽然不如阿树,但是弄得挺好啊」\\

千岁再次笑着啪啪地拍起了周的背。这是她以自己的方式在为周着想吧。

她的话语,听上去就像是「即使外表变了也还是和往常一样」的激励,让周的嘴角稍稍放松下来。\\

「对你来说第一肯定是树咯」

「那当然。对昼儿来说第一是你,所以也没什么好抱怨的吧?」

「说的是。我是真昼的第一就好」\\

周并没有想成为千岁心目中的第一。只要真昼说周是第一,那就足够了。\\

他瞟了真昼一眼,发现她牵着手,把脸靠在周的胳膊上,轻声细语道「……周君是第一」。

或许是因为在千岁面前宣言而感到微妙的害羞,她脸上挂着淡淡的红晕。\\

「好少女啊~昼儿太可爱了。要是周不在就能抱起来疼爱了」

「好好好,上学路上就别干这事了,到教室之后随你吧」

「哇,太棒了,男朋友同意了哦昼儿。过会儿给我抱抱~!」

「呃,请、请手下留情……?」\\

由于莫名其妙要被抱,真昼一边困惑着,一边又点了点头,而千岁带着满面的笑容走在真昼的旁边,大概是迫切地想要祝贺真昼吧。\\

见证两人的亲密后,周把目光从真昼身上移开,看向了周围。

或许是因为已经快到学校,视线的量变得更多了。\\

(……进了教室,大概会提问不断吧)\\

在大量的视线中,周想象着几分钟后的未来,露出了不让她们发现的、小小的苦笑。
