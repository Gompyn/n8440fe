% \subsection{175 登校}
\subsection{175 登校}

% 「忘れ物はないですか?」\\


% 朝食を食べて身支度を改めてした後に家を出ようとすると、真昼が声をかける。\\


% 忘れ物、と言われても一応事前準備はしてあるので、恐らくない。


% 今日は午前中のみの日程なので教科書は必要ない、上履きと筆記用具と課題、提出書類くらいなものだろう。全て先んじて鞄に詰めて確認までしているので、問題ない筈だ。\\


% 「特にないと思うけど」


% 「本当に?」


% 「むしろ何でそんな疑うんだ」


% 「……こちらをお忘れでは?」\\


% 若干あきれたように告げつつ真昼が周に見せたものは、暑苦しいから後で締めようと思っていた学校指定のネクタイだ。


% ああ、と思わず声を漏らせばため息をつかれる。\\


% 「一応休暇明けの集会があるのですから、身なりはしっかりするべきです」\\


% 全くもう、と言いながらネクタイを首に巻こうとしている真昼に、なんだか面映ゆい気持ちになりつつ軽く屈んでおく。\\


% もちろん夏休み前までは基本的に毎日やっていた事なので自分でも出来るのだが、真昼がしてくれるというのなら止めるつもりはなかった。\\


% 大真面目にネクタイを結んでいる真昼の姿に、小さく笑う。\\


% (……後で気付いたら照れるんだろうなあ)\\


% 自分で新婚夫婦のような真似をしているのだから、微笑ましい。\\


% 周としてはしてくれる事がありがたいし気遣いもありがたい、当たり前のようにしてくれるのもありがたいし、何より後で気付いた真昼が照れる姿を見られるのが嬉しいので、良いこと尽くしだ。\\


% せっせとネクタイを結び整えてくれている真昼を眺めていると、視線の質がいつもと違う事に気付いたのか、不審げな眼差しが返ってくる。\\


% 「……どうかしましたか?」


% 「いや、何でもないよ。可愛いなって思っただけ」


% 「最近の周くんは可愛いと言ったら誤魔化せると思ってませんか」


% 「思ってないけど、可愛いのは事実だからな」


% 「……そういう所がたらしだとあれほど……いえ、私限定ならいいですけどっ」\\


% 微妙に強気を窺わせる声で言い切って周のネクタイをきっちり締めた真昼に、これ以上はからかうまいと笑うのは控える事にする。\\


% 代わりに、真昼の頭を一度撫でて、真昼の手を取る。\\


% 「じゃ、行こうか」


% 「……誤魔化された感があります」


% 「気のせいだ」\\


% 素知らぬ振りをして、自分の鞄を背負い真昼の鞄を持ちつつ玄関に向かう。


% 真昼は自分で持とうとしていたが、彼女に何でもしてもらう訳にはいかないからこれくらいしないと甘えっぱなしになってしまうだろう。\\


% 譲る気はない周に真昼がほんのり嬉しそうに頬を緩めて、周の二の腕に軽く頭突きをする。\\


% 「どうした?」


% 「……何でもないです」


% 「何でもなくないと思うけどなあ」


% 「周くんはさっき何でもなかったので、私も何でもないのです」


% 「それを言われると追求できないなあ。はいはい、なんでもないんだな」\\


% 笑って玄関で靴を履き、真昼と一緒に家を出る。\\


% 「行ってきます」\\


% 誰に向けるでもなく呟くと、真昼はじっとこちらを見た後に小さく追従するように「行ってきます」と口にした。\\


% 真昼の返ってくる場所はここなんだな、と思うと嬉しくて頬が緩んでしまうのだが、真昼からの追求はなかった。


% なにせ、真昼の方もほんのりと赤らんだ顔で嬉しそうに笑っていたのだ。周の事を言える訳がない。\\


% 幸せそうにしている真昼の手を握れば、真昼も同じように握り返した。

