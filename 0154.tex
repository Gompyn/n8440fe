% \subsection{154 一瞬見えた色}
\subsection{154 一瞬見えた色}

% 自宅に帰って次の日、真っ先にした事は掃除だった。


% 流石に帰宅当日は疲れていたのでしなかったが、二週間も家を空けていれば部屋も埃が溜まっている。僅かなものではあるが、真昼も一緒に家で過ごすためなるべく清潔にしておきたいところだ。\\


% そんな訳で真昼仕込みの掃除術を駆使して、周は掃除をしていた。ちなみに真昼は真昼で自宅の掃除をしているらしいので、周一人である。\\


% 真昼のお陰で掃除は得意ではないものの維持する事には問題がない。真昼いわく『ちゃんとこまめに掃除していれば大きな労力は要りません。後回しにするから不必要に労力と時間が奪われるのです』との事。


% 真昼の教えの通り定期的に軽い清掃をするだけで綺麗な状態を保てていた。\\


% 今回も、埃が多少家具に降っているだけなので、掃除に時間はかからなかった。\\


% さっと家具をほんのりと化粧する埃を拭いて掃除機をかけてついでに窓も拭き終えたところで、周は時計を見上げる。\\


% 既に時刻は十五時過ぎ。


% いつも通っているスーパーのセールは十六時から始まる事が多いので、そろそろ向かった方がいいだろう。\\


% (我ながら思うけど、所帯染みてきたなあ)\\


% スーパーへ行くのは帰省前に冷蔵庫を空にしたせいで、本日の夕食の材料がないのだ。朝昼はカップラーメンや冷食で済ませたが、夕飯はそうはいかない。


% 買い物担当は周であるが、材料費は折半だ。なるべく安く済ませようという考えはおかしくないのだが……高校生男子が食費を気にするのは些か所帯染みているだろう。\\


% 自分でも自分の変化にふっと笑って、とりあえず軽く汚れた服を着替えるべく自室に着替えを取りに行った。\\


% \\


% 「……ん?」\\


% スーパーに行く最中、考え事をしながら歩いていると、見覚えのある色素の薄い色の髪が見えた。


% つい振り返ってしまうが、当然後ろ姿しか見えない。真昼のような髪の長さでもなければ、そもそも性別からして違う。染めたような色ではなく天然のあの色の薄さは、珍しい。\\


% 珍しい事もあるもんだな、と思いつつ到着したスーパーに入って本日の夕食の材料をかごに放り込んでいると「あれ」と聞き覚えのある声が背後から聞こえた。\\


% 「こんな所で会うなんて珍しい」


% 「九重か」\\


% 門脇を通じて騎馬戦で親しくなった青年が、周と同じようにかごを腕に提げている。


% ちなみにかごの中に入っているのはお菓子やジュースなので、彼の方が余程男子高校生らしい買い物をしていた。\\


% 「藤宮って家こっちなの?」


% 「おう。九重はこっちらへんじゃないと思ってたんだが……」


% 「僕はただ友達の家に泊まりだから買い出しにきただけ。藤宮は……ご飯?」


% 「ん。夕食の買い出しだよ」\\


% 見れば分かる通り、周の手にしたかごの中には生の鶏肉や大根、牛乳や豆腐といったおやつとはどう間違っても認識出来ないようなものが入っている。\\


% 「そういえば藤宮は一人暮らしなんだっけ。えらいね」


% 「まあ真昼がご飯作るんだけどな……」


% 「……そういえば言ってたっけ……すごい生活してるよね」


% 「だな。真昼には感謝してるよ」\\


% 彼女が居なければ周の食生活はズタボロだろう。掃除は多少出来るようになっても、未だに料理は不得意のままだ。


% 仮に居なくなってしまえば、周の今の生活は成り立たなくなる。\\


% 小さく苦笑しながら「真昼様様だ」と呟けば、九重はそっとため息をつく。\\


% 「なんというか、ほんと……あれだね、首ったけ?」


% 「そうだな。真昼もだけど」


% 「自信満々に言えるんだね」


% 「愛されてるって自覚は持ってるよ」\\


% 付き合う前は好意に自信が持てなかったが、今は違う。真昼に大切にされて好かれているのは自覚しているし、彼女が周の側に居る事を望んでいる事も分かっている。


% 自意識過剰とかではなく純粋に事実だと認識していた。そう出来るようになったというのが、自信のついた証拠かもしれない。\\


% あっさりと、淀みなく答えた周に、九重は先程まで苦笑していた周に代わって苦笑する。\\


% 「まあ、自信がついたならいい事だと思うよ。相思相愛なのにうじうじしてたあの時よりいいんじゃないの」


% 「厳しいなあ」


% 「だってどう考えても好かれてるって見えてたのに。ま、僕には関係ないけど、君らが幸せならそれでいいんじゃないの」\\


% 肩を竦めた九重に彼なりの賛辞を感じて、頬を緩める。\\


% 「……ま、優太も納得してたし、僕はこれで丸く収まったと思ってるからね」


% 「え?」


% 「ううん、なんでもない。じゃ、僕はレジに行くから」\\


% 何故そこで門脇、と思ったものの、追求をする前に九重はさっさとこちらに背を向けて去っていったので、周は困惑しつつもスマホにメモした夕食の材料をかごに放り込むべく彼に背を向けたのだった。

