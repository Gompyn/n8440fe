\subsection{天使大人与生日}

周向树和千岁寻求完建议后,总算选好了礼物,在真昼的生日当天以一副紧张的表情看着她的背影。\\

以车站前的可丽饼屋卖的特制可丽饼(冬日限定非常莓果特辑)为报酬,周说动了千岁帮自己忙买了个东西,并把这东西也加进了礼物……可现在周却苦恼着该在什么时候把这礼物送出去。\\

而那过生日的本人,正和往常一样做着晚饭。\\

虽然周不清楚菜单,不过真昼似乎是在做和食的样子,但怎么看她都没有什么特别的感觉,表现得跟平时一样自然。

从当事人身上完全感受不到生日的氛围。不如说那淡定程度,简直让人觉得她是不是根本就不记得这回事。\\

甚至到了晚饭端出来后也没有发生变化。两人在餐桌上虽有对话,但进餐还是一如往常。\\

周真的拿不定主意该什么时候把东西交给真昼,于是看向藏在沙发后面那放着礼物的纸袋,皱起了眉。\\

总之周先收拾好了餐桌。等他回到客厅的时候,发现真昼正坐在那刚好两人位的沙发上,看着似乎是自己带来的书。

就连看书的模样也美如画作,到底是不虚天使之名。\\

虽说周对要不要坐在真昼旁边有些微妙的犹豫……但一直退缩也不是个办法,于是周提起放在那里的纸袋,坐到真昼身旁。\\

真昼突然抬起了头。

大概是注意到了周的气息和纸袋摩擦的声音,真昼那焦糖色的双眼看向了周,然后又移向了周拿着的纸袋。

真昼的表情似乎有些不解。看来,都到了这个地步,她还没有注意到自己生日的事情。\\

「嗯,给你的」\\

周把纸袋推出去放在了真昼膝上,使真昼脸上更加茫然。\\

「这是什么」

「今天不是你的生日吗」

「是倒是……话说为什么你会知道。我可不记得自己有跟谁说过这回事」\\

真昼的眼里微微露出警戒的态度,但听到周说「你上次把学生证落屋子里了吧」之后,她或许是接受了这个解释,便恢复了平时的表情。\\

「其实,没必要在意的。反正我也不过生日」\\

那冷淡而透出排斥感的声音,应该不是周听错了吧。

真昼那眼神,如同对生日这词汇本身抱着忌讳感一样。\\

周明白了,原来如此。\\

明明是生日,她的态度却毫无变化,其原因,并非是不记得生日的事情。

因为生日很烦人所以故意忘掉的——应该说是这么回事。

若非如此,她也不会用那种语调吧。\\

「啊这样啊。那就当作是平常受你照顾的回礼吧。权当我一厢情愿想要报恩」\\

周以「你不过生日也没关系,但作为感谢平日照顾的回礼是另一回事。这就当作我表达感激的心意而不是生日礼物」这个说法把礼物塞了过去。

每天都吃着这么好吃的饭,偶尔还来帮忙打扫屋子,虽然都是小事,但也实在是受照顾了。即便只是一点一点地,周也想要回报真昼。\\

周虽然很轻易地就退让了,但却执意要把礼物送过来,这让真昼有些混乱。她有些困扰地皱着眉,接过了礼物。

真昼的视线,集中在纸袋里面用另一层袋子包装的东西上面。\\

「我可以现在打开吗?」

「嗯」\\

看见周点头,真昼紧张地从纸袋里把盒子拿出来,小心地打开包装纸,解开缎带。

看着别人在自己面前慢慢打开礼物,周感到格外紧张。\\

里面放着的是树推荐的护手霜。因为是套装一起卖的,所以这个大盒子里还附带着一点小点心。\\

顺带一提这并不是那种带有香味的时尚品,而是以没有香味、适合家务、亲和肌肤、滋润保湿为卖点的东西。

周也确认过网上的评价,效果应该是不用担心的。\\

「抱歉,不是什么值钱东西。看你干家务手应该会干吧。虽然也有带香味的,不过那种你估计有了所以没买。听说这东西对皮肤好而且挺有效的」

「是实用品啊」

「要说的话,你更看重实用性吧」

「也是。谢谢你了」

看着真昼微微露出笑容,似是在说「你还挺了解我的嘛」,周也稍稍放松了嘴角。

看来印象不坏。\\

之后虽然还有一件东西……但要当面打开周还是觉得有些害羞,如果可以的话周还是想真昼回到家再发现那个东西。\\

可事不如愿,在把护手霜放回纸袋里的时候,真昼眨了眨眼,似乎是注意到了纸袋里还有一件东西。\\

「……是还有一件东西吗?」

「啊。呃,那个,怎么说。就是个来自于独断和偏见的附赠」

「附赠?」

「……附赠」\\

周撇开视线,只回答了这么一句。真昼歪着头搞不明白周的意思,但她觉得不如直接打开来得快,便从纸袋里把那东西拿了出来。\\

为了让那东西尽可能不起眼,周用了跟纸袋一个颜色的包装,还将其塞在了最底下,但果然这个大小还是很显眼。不如说真亏能在打开护手霜的盒子之前都没让她发现。\\

那东西的包装并非盒子,而是聚酯塑料袋。其大小,正好够真昼双手抱住。

看着她把那深蓝色的丝带小心地解开,周想着「我要不要先离开一下」的时候——真昼正好把里面的东西取了出来。\\

她用两只手小心地把里面的东西提了起来,相当意外地眨巴着她那两颗大眼珠子。\\

「……熊?」\\

真昼说着的,便是那东西的原型。\\

那是一个不算太大,大概小学生抱着大小正适合的布偶。

布偶身上的软毛颜色很淡,与真昼的发色很相近。它的脸上透出天真的感觉,上面缝着一双乌黑、光亮、圆润的眼睛,眼睛里正映着真昼的身姿。\\

「都高中生了还玩偶啊」她说不定会这么想。\\

尽管如此,听了千岁「不管到了什么时候,女生总会喜欢可爱的东西」的建议后,周就选择了这个。\\

再怎么说男的一个人跑去买这东西实在是非常害羞,周便以车站前的可丽饼为报酬让千岁陪着自己去买了。

结果从挑选到打包周一直在被千岁笑嘻嘻地看着,说不定其实一个人去买羞耻感还会少一点。\\

「……我觉得女孩子会喜欢这个」\\

周挠着头,不知是在跟谁解释般嘟哝了一句。\\

这种事周实在是不擅长。

说到底,给异性送礼这件事,除了小时候送给母亲以外,周就没有干过。他甚至没有想到自己会去做这种事情。\\

从男的那里收到这么可爱的玩偶会不会让她受不了啊……周偷偷瞄了眼真昼,看到她正紧紧盯着熊的脸不放。

真昼看不出是高兴还是不高兴,只是呆呆地望着布偶熊。\\

「嗯,不喜欢的话扔了也行」\\

「如果不喜欢的话那也没办法」周想着,玩笑般地说了一句,结果真昼却皱着眉刷地把头扭了过来。\\

「我不会做那种事情!」

「嗯、嗯。看椎名的性格我想应该也不会的」\\

真昼的否定比预想要强烈,令周一边退缩一边点了点头。而真昼则再次看向手中的熊布偶。\\

「……我不会做,那么过分的事情。会好好珍惜」\\

真昼纤细的手腕,像是要将其拥入怀中般,紧紧抱着熊布偶。

那姿态看上去,既像是孩子不愿喜欢的玩具被拿走,又像是母亲慈爱的拥抱。

可以说的,便是她极为珍重地抱着布偶这件事。\\

仿佛能配以啾的音效一样,真昼紧紧把布偶抱在怀中,并稍稍垂下眼帘看着它。\\

那脸上的表情,既不是平常的那种冷淡的表情,也不是被周的脱线惊呆时的表情,而是心安的、柔和的、泛着慈爱的、爱惜的表情。

还有她那天真无邪的纯洁微笑,美丽又惹人怜爱,让周不禁屏息。\\

(——不该看的)

望见这样的表情,周不由自主地会对此产生意识。\\

让顶级的美少女露出了这样的表情,还被自己看见了这一事实,就算没有恋爱上的喜欢,也足以让周心跳加速了。\\

真昼那珍惜地抱着布偶,露出淡淡微笑的姿态,已经可爱到不论谁看见了大概都会着迷的程度。就算是知道自己无欲无求的周也差点入了迷。\\

为了确认自己脸上积蓄了多少热度,周伸手捂了下自己的脸。手上传来了比平时更加明显的热感。

由于自己害羞得太过明显,所以周用真昼听不到的声音骂了一句「……靠」。\\

幸好,真昼正紧紧抱着熊布偶,把半张脸埋进里面,因而并未注意到周。

那副模样也是一样地可爱,让周好不容易才忍住了发出怪声的冲动。\\

「……这么喜欢的话,我也就心满意足了」\\

周想着说些什么,于是挤出了这么一句,真昼则稍稍把眼睛露了出来。\\

「……我是第一次,收到这种东西」

「咦,以你的人气这算是日常贡品吧……」

「你把我当成什么了……」\\

这带有稍许无奈的声音与表情,反而让周安心了下来。这大概是因为不再需要直视那样的表情了吧。\\

「……我没有告诉过别人过我的生日。因为不喜欢生日,所以我从来都不说」\\

真昼在断言「不喜欢」之后将视线移向了布偶熊。

她看着布偶的眼神很安详,与嘴中的话语截然相反,让周莫名觉得不太自在。\\

「一般,不认识的,或者没什么关系的人送我礼物我也觉得可怕,所以不会收」

「我送的倒是收了啊」

「……藤宫又不是不认识的人」\\

真昼小声地回答着,然后把脸埋进布偶里仰头看向了周。周则开始后悔自己直视了她这件事。\\

她那无意中向上看的眼神,还有放松下来的、与年龄相应的天真感流露出来的表情,实话说,相当令人怜爱。

那可爱令人不自觉地产生了想要摸摸头的冲动,于是周在不经意间把手伸向了真昼的头,然后慌忙用力收了回来。\\

「……怎么了吗」

「没、没什么」\\

不知是注意到了周一瞬间动了的手,还是察觉到了周那几近爆发的急躁感,真昼咚地歪了歪头。

仅是这样,周的视线便差点被夺走了。美少女这种生物还真是可怕。\\

但是再怎么说,直接回答因为可爱所以看呆了,周还是会感觉羞耻,而且就算说了周也确信真昼只可能回答「啊?」。\\

而且,如果那样说的话周在各种意义上都会死亡,所以还是决定把这个冲动深藏于心。\\

「……谢谢你,藤宫」\\

周撇开了脸,而真昼纤细的声音再一次传进了他的耳中。

\psline

※娇化输出 20\%
