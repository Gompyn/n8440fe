% 171 届いた手紙


%  それを見付けたのは、千歳と真昼のためのケーキを買って帰った際、エントランスにある郵便受けを覗いた時だった。\\


%  いつもの広告に紛れて、見覚えのない封筒が一通入っていた。


%  丁寧な字で『藤宮周様』と書かれていて、一体誰がこんなものをと何気なく捲って、目を疑った。\\


%  裏には、送り主の名が書かれている。\\


%  ――椎名朝陽、と。\\


% (……真昼の父親、だよな)





%  母の名は小夜だと聞いているので、母の名ではない。


%  そして、周の事を知っているのは恐らく彼だけだろう。


%  恐らくあの時真昼が迎えに来たところを見られていたのだ。軽く調べれば、周が真昼と親しくしているのも見えてくる。\\


%  ただ、周にわざわざ手紙を送る理由が分からない。実の娘相手ならまだしも、娘の彼氏に送る必要が見えなかった。\\


%  真昼いわく自分に関心を持っていない、という事ではあるが、関心がないなら様子を見に来たりはしない。


%  真昼の父親の意図が全く見えなかった。\\


%  困り果てて、とりあえず一度家に戻って千歳が帰宅してから開く事に決めて、バッグの中に手紙をしまい込んだ。\\


% \vspace{2\baselineskip}


% 「帰ってきてから様子が変なのですけど、何かありましたか?」\\


%  課題をひいこら言いながら七割程終わらせて千歳が帰宅したところで、真昼は周の顔を覗き込んできた。


%  真昼が帰ったら手紙を開封しようと思っていたのだが、隠し事があるのに気付いたらしい。\\


%  隠したい、というよりは手紙に記された用件が分からないので迂闊に真昼に知らせない方がいい、という判断だったのだが、真昼に怪しまれるくらいなら最初から隠さなかった方がよかったかもしれない。\\


% 「あー、いや、なんつーか」


% 「はい。……あ、私に言いたくない事なら無理には聞きません」\\


%  あくまで周の意思を尊重する、といった姿勢の真昼に、周は組んだ足を戻しつつ、彼女を見やる。\\


% 「言いたくないというか、真昼が聞きたくないかもしれないというか」


% 「私が聞きたくない……ああ、そういう事ですか」\\


%  両親の関係だと気付いたのだろう、次の瞬間には淡く苦笑している。\\


% 「まさか、あの人がまたこの辺りに居たのですか?」


% 「いや、そうじゃないけど……俺宛に、手紙が」


% 「周くん宛に? 送り主は?」


% 「……椎名朝陽って書いてた」


% 「ならうちの父親ですね」\\


%  あっさりと頷いた真昼の表情は、思ったよりもショックの色が見受けられない。実に淡々とした様子で、ショックというよりはただ少し驚いているといった感じだ。


%  ただ、若干眼差しが冷たくなっているのは、彼女が両親から受けた仕打ちのせいだろう。\\


% 「まあ、何故周くんに手紙を送ったかとかどうやって私と周くんの仲を知ったのかとかその辺りは気になりますが、私が関与する事ではないでしょう」


% 「中身、気にならないのか?」


% 「他人に宛てた手紙を覗く趣味はないです。私の父からであろうと、宛先は周くんですので」\\


%  きっぱりと言い切った真昼に、自分が気を使いすぎて逆に真昼に気遣わせているな、と感じた。


%  といっても、真昼も受け入れているというよりはむしろ関わりたくないといった風に見える。\\


%  いつもより少しだけ落ち着きなく視線を揺らした彼女は「読むならどうぞ。席でも外しましょうか?」とひんやりした声で尋ねられ、周は小さく苦笑して首を振る。\\


% 「ん……なんつーか、側に居て欲しいというか。真昼が嫌なら一人でもいいけど、彼女の親からの手紙って緊張するからさ」


% 「それならここに居ます。……手紙の内容を私に知らせるかどうかは、周くんに任せます」\\


%  そう言って机の上にあった参考書を読み出す真昼に、周はそっと息を吐いて、側に置いていたバッグの中から封筒を取り出す。


%  きっちりのり付けされたそれを丁寧に開けて中に入っていた便箋を取り出し、したためられていた文に目を通した。\\


%  簡潔にまとめると、会って話したいという旨と、連絡先が載っている。\\


% (……なんでまた俺に)\\


%  真昼の様子を見にきたのではなかったのか。何故、周という父親にとってはほぼ無関係の人間を呼び出すのか、全く分からない。\\


% 「……なんか、俺に会いたいそうで」


% 「娘ではなく周くんにですか。そうですか」\\


%  いっそうひんやりとした声になっていたので思わず真昼の頭を撫でると、くすぐったそうに目を細める。\\


% 「いえ、怒っているとかではなくて……純粋に、意味が分からないのです。何故周くんに会おうとするのか、理由が分かりません」


% 「……普通なら、娘に男が近寄っているから、とかだけど」


% 「あり得ませんね。今まで放置していたのに今更口出しするなんて」


% 「……これ、どうしたらいいと思う?」


% 「私は別に、会う事に制限はかけるつもりはないですよ」\\


%  本当に周に任せる気らしく、非常に淡白な返答がきた。\\


% 「ああ、心配は要らないと思いますよ。あの人は親としては失格ですが、何か脅しをかけるような人でもありませんので。……父の事をあまり知らない身で言うのもあれですけどね」


% 「……真昼」


% 「何を企んでいるのかは知りませんが、他人に害をなす人でもないのでそこは安心してもいいですよ。行くも行かないも、周くんの自由です」\\


%  そう言って周に体を預けるようにもたれてきた真昼に、周は「そっか」と小さく返して、もう一度手紙を眺めた。


