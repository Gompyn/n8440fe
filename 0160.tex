\subsection{天使大人的换装}

祭典开始前一个半小时,周和真昼做起了准备。

真昼和千岁一起拿着浴衣回到家里,而周一个人开始穿上浴衣。\\

穿浴衣也需要知识,但周不担心真昼。她能穿和服,浴衣肯定也轻轻松松吧。\\

问题是周这边。虽说志保子有灌输过相关知识,但周完全没有实践过,所以很担心自己穿着是否得体。\\

穿完后,周照了照镜子,姑且还像样,没有走形。\\

浴衣没有花纹,是深蓝的,而腰带是红豆色的,都很朴素。周不太喜欢花哨的东西,这样的选择很对他的胃口。\\

镜子里的自己配合上还算可以的身高,产生出有模有样的氛围。

不说好坏,周原本就有一副文静的面孔,穿上这种沉稳的衣服,从气质上来讲应该算是合适吧。\\

至于站在真昼旁边会不会显得逊色,就交给别人来判断了。

虽然他也在意别人的眼光和评价,但最后还是落在自己怎么想、真昼怎么想。\\

先穿好浴衣后,周坐在沙发上悠闲等待。\\

周知道女生打扮要花时间,况且他留好了充足的时间来准备,所以没有任何问题。

要穿浴衣的话,肯定会比平时花上更久,而且头发也要扎起来,得多花三成的时间来整理。

由于在此之上还要化妆,周坦率地觉得女生很厉害,对她们感到尊敬。\\

(虽然真昼什么都不做也很可爱,但是打扮之后,女生就会更亮眼,真厉害啊)\\

由于她努力想在男朋友前面变得更可爱,周感受着怜爱、欣慰以及难以言喻的幸福,悠闲地等着。这时,她大概是准备完成了,在门口发出了开锁的声音。\\

周很期待女友的打扮,便没有回头等她过来。伴随一声呢喃「周君」,周的肩膀上被轻拍了一下。

这时,他才终于回头——然后翘起了嘴角。\\

「可爱,挺合适的」

「……能、能判断得那么快吗」

「能的能的,看一眼就知道了」\\

真昼微微怀疑着说辞是不是事先准备好的,但没办法,周讲出来看到的感想就是这样。\\

周再次体会到志保子的眼光非常优秀。\\

或许考虑到真昼要和周在一起,真昼的浴衣是白底,上面有绣球的花纹,在沉稳中给人明朗的印象。

绣球有藏青,有浅紫,浓淡参差,显得成熟清秀。尽管花季稍过,却仍然非常和谐。\\

衣带是亮紫色的,衬托出设计简洁的浴衣。带扣上点缀着蜻蛉玉,给人清凉之感。\\

「平时也很可爱,今天更有种清秀中有成熟的感觉呢,又稳重又迷人。要说可爱也没错,不过更该说漂亮吧。嗯,很合适」

「是、是这样吗」\\

周认真讲完感想后,真昼有点难为情,心神不定地摆弄着侧边的头发。看到那副模样,周不由得笑了。\\

真昼扎起的头发用发簪固定着,每当她有动作,上面的银链都会摇来摇去。发簪上装饰着藏青色的天然石头以及设计与衣带上相同的蜻蛉玉,气质上和周所穿的浴衣莫名相似。\\

「昼儿昼儿,他本来就这样」

「我知道,早有体会」

「……这是在批评我吗」

「既在表扬又在批评,吧~?」

「什么跟什么啊」\\

周摸不着头脑而眯起眼睛,但千岁只是笑着,真昼又扭扭捏捏的很害羞,他没法去问这是什么意思。

不过,看真昼也没多少不满,应该不是什么坏事吧。\\

「……周、周君穿的也很合适」

「这样吗?谢谢。你这么一说我很开心」\\

周觉得应该还算穿的合适,但重要的是得到真昼保证了。虽然周觉得其中有女朋友的偏袒,但获得表扬还是令人高兴的。\\

周自认为坦率地接受了称赞,可不知为何,真昼的眼神里带着一点闹别扭的神色。\\

「……我做了什么吗?」

「大概是『只有自己害羞不公平』吧~」

「千、千岁」\\

听到解说后,真昼惊慌失措,那副模样如实表达出千岁说的是真的。

真昼似乎希望周也能害羞,但他并不会因为这点小事而难为情。虽然周心中有高兴、有害臊,但不至于羞耻得像真昼那样。\\

真昼显而易见地动摇后,千岁也快乐地笑道「真讨人喜欢~」贴到她身上摸了个遍。

周不知道该佩服千岁摸的动作巧妙,不弄乱头发、服装和化妆,还是该主张能疼爱真昼的只有自己。\\

看到真昼脸红起来更加害羞,周心想真昼那么可爱,看两人那么要好也不错吧,便简简单单放过了千岁,以温暖的眼神守望两人的嬉戏。
