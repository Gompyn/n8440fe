\subsection{天使大人照料病人}

因为发热,周很晚才想起来自家的现状——不如说是亲眼看到实际情况之后,周才后悔把真昼放进来。\\

周住的公寓是一房一厅,另外还有厨房和储藏室。\\

客厅面积挺大,有寝室,还有储藏室,这对独居生活而言已经相当奢侈了。因为父母还算富裕,考虑到安全和交通,所以最后选择了这里。\\

要求若是要独居的话得住在这里的是父母,所以周并不打算提什么意见。不过,周心里觉得,不用花那么多钱也没关系。一个人住这么大的房子,实在应付不过来。\\

先不说这些。周这个人是独自居住,并且不擅长整理收拾。\\

当然,别说客厅了,连寝室都是乱糟糟的。\\

「真是看不下去」\\

天使大人,或者说救世主大人,尽管外表可爱却开门见山地给周奉送了这么一句直白的话。

因为事实上确实看不下去,周也无法反驳。如果知道要让别人进来,周多多少少还会移开点东西。但事到如今说这些也晚了。\\

真昼光泽的嘴唇中发出一口叹气。即使如此她也没有回去,而是把周搬到了寝室。

在中途,两人还差点摔倒。把屋子弄得这么乱的周本人痛切地感受到,再不认真收拾或许就太不妙了。\\

「总之,我先出去一下,在我回来之前换好衣服。没问题吧」

「……你还会回来啊」

「把病人放在床上不管,我会睡不好觉的」\\

真昼似乎和周上次看到她那时候是一样的想法,所以周也不好多说什么。

在她离开房间之后,周就老老实实按真昼说的,换上了室内的便服。\\

「……到处都乱七八糟,不如说根本没地方落脚……这种地方你是怎么生活的啊……」\\

在换衣服的时候,传来了小小的困惑的声音,让周感到非常抱歉。\\

\vspace{2\baselineskip}

换好衣服躺下之后,周似乎就在不知不觉中睡着了。当他使劲睁开沉重的眼睛时,首先映入眼帘的是亚麻色的头发。\\

沿着头发往上方看去,只见真昼静静站着看着周。看来刚刚发生的一切都不是做梦。\\

「……现在几点」

「晚上七点了。你睡了有几个小时吧」\\

真昼淡淡地回答之后,配合着周坐起身体,把倒好在杯子里的运动饮料递了过来。

周心怀感激地接过杯子喝了一些,总算能够把视线转向周围了。\\

或许是因为睡了一觉,周感觉身体稍微好转了一点点。\\

然后周便注意到了自己脑袋上凉凉的。摸了摸,指尖上传来了像是布一样,有些硬硬的感觉。\\

周的头上贴着家里不可能有的冷敷贴。注意到这一点,周便抬头看向真昼。真昼则直截了当地回答说「从家里带来的」。

周的家里既没有冷敷贴,也没有运动饮料。所以,运动饮料也是她拿过来的吧。\\

「……谢谢你特意拿来了」

「不用谢」\\

这冷淡的回答让周只得苦笑。

她提出来照料应该只是由于罪恶感,而并不是想和周聊天。说到底,在几乎只是一面之交的男人家里两人独处,这种状态下想来也不可能亲近地说话。\\

「总之,我把桌子上的药拿过来了。这药最好不要空腹食用,你现在有食欲吗」

「嗯,还算有吧」

「这样吗。那我烧了点粥,你先喝吧」

「……诶,椎名亲手烧的?」

「除了我还能有谁。不想要的话我喝掉好了」

「啊不要我会喝的请给我喝吧」\\

周完全没想到真昼除了照料自己之外,还会为自己准备好粥,所以一瞬间有些惊慌失措。

说实话,真昼的料理水平是未知数。不过,周并没有听说过真昼在家庭科的课上失败的这类传闻,至少应该不会太糟糕吧。\\

周立刻就低头回答说要喝,让真昼露出了有些无语的眼神。不过,真昼还是点了头,并把床头柜上的温度计递了过来。\\

「我去把粥端来,你先量个体温」

「嗯」\\

周照着真昼说的,敞开了衬衫前面,拿出了温度计。这时,真昼慌忙撇开了脸。\\

「等我不在房间里了再量啊」\\

真昼的声音微微有些慌张。周往她那儿看过去,只见她的脸蛋有着淡淡的红晕。\\

周心里觉得,男人的胸板不像女孩子一样需要藏起来,所以对真昼的反应感到不可思议。不过,可能是对皮肤的颜色没有免疫力的关系,周只是敞开了前面就让她明显慌了起来。\\

真昼白色的脸颊染上了淡淡的蔷薇色,脸依旧朝着别处微微发着抖。不知是否是错觉,感觉她的耳朵也染上了颜色,可见真昼有多么害羞。\\

(……啊,感觉有点明白她周围的男生为什么会说可爱可爱的了)\\

周也觉得真昼确实是美少女,但并没有更多的想法。美丽、可爱,这是不假的,但也不过如此。

也许该说是人造品的美——真昼给周的印象就和艺术品差不多。\\

然而,现在的真昼露出了微微的害羞,她那慌张的样子更有人类的感觉,因此有种不可思议的可爱。\\

「……那你赶紧去把粥拿来不就好了?」

「不用你说我也会去的」\\

只是,两人的关系并没要好到周能老老实实夸她可爱的程度。要是说了出来,肯定会让她对周有奇怪的评价。于是,周把感想咽了下去。\\

在周没兴致地那么一说之后,真昼就啪嗒啪嗒地快步往房间外面走去。

她的动作多少有些慢,是因为动摇呢,还是因为房间太乱呢。恐怕是后者吧。\\

迷迷糊糊地目送她离开之后,周再次小小叹息了一声,心里想着为什么会变成这样。\\

(……大概是因为责任感和罪恶感吧)\\

一般来说,女生是不会跑进不怎么熟的男生家里照料病人的吧。要是被侵犯了事就大了。\\

真昼带着这个风险依然做出了这样的选择,看得出她内心非常愧疚吧。再加上周的态度明显对她没有兴趣,这说不定也是让她安心的原因。\\

不管怎么说,真昼是没有其他办法才来照料的,这一点应该不会有错吧。\\

「……我拿来了」\\

周用有些发热的脑子想着这些事情的时候,真昼有所顾虑地敲了敲门。

真昼似乎是担心着周衣服还没穿好而不打算立刻进来。这时,周才想起,把衣服弄松是为了量体温啊。\\

「体温还没量」

「请在我不在的时候量好啊……」

「抱歉,我走神了」\\

周老老实实道了歉,把温度计夹在腋下。没过多久就听到了有些闷闷的电子声。

他拿起温度计,只见屏幕上显示着38.3°C。虽然不至于去医院,但这个体温还是挺高的。\\

于是周整理好着装,然后告诉还不打算进来的真昼「已经好了」。真昼这才端着放着一锅粥的托盘小心地进来。

她看上去这么明显地放下心来,是因为周把衣服整理好了吧。\\

「多少度啊?」

「38.3°C。喝点药睡一觉就会好的」

「……药店的药都只是针对症状,不能消灭病毒本身。要好好休息,激活身体的免疫功能啊」\\

尽管被责备了,不过周知道真昼这是在担心,所以总觉得心里痒痒的。\\

说着「真拿你没办法」,真昼叹了口气,把锅连着托盘一起放到床头柜上,打开了锅盖。\\

锅里是放了梅干的粥。考虑到对胃的负担,粥比较稀,大概一份米七份水。

里面放了梅干,应该不是为了味道,而是因为据说这样对感冒中的人比较好吧。\\

锅上没有冒出热气,然而却传来了温暖的感觉。这种感觉大概是表明,这锅粥不是现做的,而是先做好后再故意放凉的吧。\\

真昼无视了盯着粥看的周,麻利地把粥盛到碗里。梅干细细地散在粥里面,里面的籽则都被细心地挑掉了,红色的果肉淡淡地混合到了白色里面去。\\

「喝吧。应该不烫了」

「嗯,Thank you」\\

周接过了粥,但只是握住勺子盯着粥看。真昼看着周的举动感到纳闷。\\

「……干什么,是想让我喂吗。那种服务我是不会答应的」

「我才没那么说……只是觉得原来你还会做饭啊」

「一个人独居这是肯定得会的」\\

对于还不能好好独立生活的周来说,刚刚那句话还是挺痛心的。\\

「……你在做饭之前,最好先把房间收拾收拾」

「您说得是」\\

真昼好像大概知道了周在想着什么,赶紧打了一剂预防针。周轻轻嘟囔着,舀了一勺粥放到嘴里来把这事糊弄过去。\\

舌头上粘稠的粥味,不出所料地充分体现出了米的原汁原味。盐放得很少。

不过,碎开的梅干带来柔和的酸味和咸味都非常入味,形成了绝妙的平衡。\\

周并不是特别喜欢吃咸梅干,不过却很喜欢这温和的酸味中带有微甜的感觉。如果身体健康的话,他想直接把这些梅干浇到米饭上做出茶泡饭。\\

「好吃」

「谢谢夸奖。不过只是煮粥,谁煮都差不多的」\\

真昼一副若无其事的表情回答说。不过她脸上还是微微一笑。

这和学校偶能见到的对外笑容不一样,是流露出安心的微笑。这让周下意识地凝视着她。\\

「……藤宫?」

「啊,没事」\\

柔和的笑容只露出了一瞬就很快消失了,让周感觉有些可惜。\\

尽管周心里这么想,但他并没有说出口,而是再次一口一口舀着粥吃试图蒙混过关。
