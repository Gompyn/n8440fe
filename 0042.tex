\subsection{天使大人的憧憬}

「请用」\\

就算是亲生父母也一样是客人,该招待也是当然的,不过真昼坚持说要自己端茶上来,于是周便拜托她了。

周还真没想到,真昼为了自己喝而拿来的茶具和红茶,竟能在这种地方派上用场。\\

周的父母坐在这平时周和真昼两个人坐的沙发上,露出了满脸的温和笑容。\\

「哎呀小真昼真是谢谢啦,你已经完全适应了呢」

「是、是的」

「这事原本应该得让周来做的哦?」\\

让周泡茶的话,恐怕只能泡出红茶的涩味,所以真昼才会亲自动手的。然而,这让志保子露出了略显无奈的表情。\\

「没有,只是我自己愿意的……」

「也是,要是周来泡的话,热水温度太随便了,也没办法」\\

虽然说得并没有错,但是被指出这些还是有点让人不爽的。

话虽如此,周也无法反驳,只能老老实实地闭上嘴,结果却被志保子笑嘻嘻地看着了。\\

「说起来啊周,开始好好用名字称呼小真昼了啊」\\

听到这突然的一言,周和真昼都僵住了身子。\\

因为这称呼已经很自然,所以周就把这事儿忘了。上次见母亲时,周还是用椎名这个姓氏称呼真昼的,而真昼叫周叫得也很别扭。

而现在,两人相互之间称呼得那么顺畅自然,就志保子那个性,肯定会胡思乱想的吧。\\

「……有什么关系」

「嗯,挺好挺好,关系亲密是件好事」\\

志保子故意没有进一步追问下去,只是眉开眼笑地观察着周这边。周则感觉到自己的脸上一阵抽搐。

说不定被开玩笑反而还更好一点。这种时候的志保子,脑袋里绝对是在快乐脑补着两个人这样那样的关系。\\

「志保子,别再逗周了」\\

不过,修斗这时踩下了刹车。\\

「志保子这习惯不好啦。别开周太多玩笑了」

「行咯,虽然很可惜但就算了吧」\\

只要修斗说的话志保子都会乖乖听,周作为被折腾的儿子真是对此感激不尽。\\

「话说回来,看到儿子和可爱的女孩子关系那么要好,果然还是很棒的吧」

「我倒是一直担心着志保子的坏习惯会不会失控哎」

「哎呀,修斗会阻止我的吧?」

「虽然我觉得既然有自知之明最好还是改掉,不过志保子这种地方我也喜欢所以没办法呢」

「哎哟……我说修斗你啦」\\

虽然说修斗是阻止了志保子,不过这次父母又开始微妙地形成了二人世界,周也不掩饰自己的叹息了。\\

修斗大体来说是个有常识的人,不过却会无意识间疼爱自己的老婆,时常会产生让其他人难以接近的气场。

幸好这个样子只会在家人前才会表现出来,在外是不会产生这么露骨的氛围的。然而,或许因为这里是周的家里,所以修斗就放松下来了吧。\\

长年不减的恩爱在儿子眼里算是表示夫妻和睦的好事,不过周还真是希望他们设身处地,为旁边看到这些场景的自己着想着想。\\

一旦变成那副模样,周也不想要进去打断了,于是就死了心坐到了餐厅拿来的椅子上,再次深深叹了一口气。

真昼也坐到了准备在旁边的椅子上,静静地看着周。\\

「……你爸妈关系真好啊」

「是啊。虽然在外面不是那个样子,不过在家里就是那种感觉了」

「是吗」\\

周苦笑着回答之后,真昼眯缝起眼睛看向志保子和修斗。\\

真昼的表情并没有表示出不快,相反地,是如同看到耀眼的东西时那样的感觉。

她的眼神中渗透出憧憬和艳羡,就好像看到了什么珍贵的东西一样。\\

看到真昼以虚幻渺茫的微笑望着两人,周情不自禁差点把手伸了过去——\\

「啊,周,怎么了嘛?」\\

接着,志保子似乎回到了现实世界,周听到了她的声音立刻把手收了回来。\\

「怎么了个什么啦。还不是你们俩进入二人世界让我们待不下去了嘛」

「哎哟羡慕了?」

「没有没有,不存在的。我是觉得这种事情给我在家里做啦」\\

似乎两人并没有注意到周差点去握住了真昼的手。真昼似乎也同样是没注意到,正因周说的话而露出着苦笑。\\

周不知道自己为什么会把手伸出去。

只是,周总觉得……不希望,让那样的真昼孤单一人。\\

真昼现在已经回到了平时的样子。周稍微变得放心了一些,同时为了不被察觉而回到了平时板着面孔的模样。\\

「所以,爸妈看到儿子的脸满意了么」

「看到真昼倒是挺满意了……」

「喂」

「有一半是开玩笑的啦。目的还没有完成呢」

「目的?」\\

周还以为志保子的目的是新年的走访和给真昼打个招呼,然而志保子似乎还有其他的目的。\\

「你们还没去新年参拜吧?」

「我准备人少一点之后再去」

「对吧?小真昼也还没去吧。发的消息里是这么说的」

「是的」

「就猜到是这样,所以咱把和服拿来了哟~」\\

看来志保子是想和真昼去新年参拜的样子。

事到如今,周终于明白了志保子为什么会满脸笑容地提着一大包行李过来。不知是今天第几次,周又叹了口气。\\

志保子喜欢可爱的东西,也喜欢给人穿衣打扮,肯定是不想放过这次机会的吧。

和服的话,光是周知道的范围里家里就有几件。他们似乎是把这些给带过来了。\\

「我的梦想就是给女儿穿上和服去新年参拜……小真昼的话我觉得肯定适合」

「妈你就是想要个换衣服的洋娃娃吧」

「没有的事哦?不过很大原因倒是想让真昼穿上呢」\\

志保子「毕竟肯定很适合」的自信满满的见解是正确的。

不如说,感觉没什么衣服会不适合真昼。\\

在周所见的范围里,男性化的服装、大小姐那样高雅的打扮、还有平时带着饰边和蕾丝的很少女的服装,真昼都穿过几次,每一种都很适合真昼。所谓美少女,大概是不择衣装的。

和服恐怕也会非常适合吧。\\

藤宫家里周是独生子,所以想给女儿打扮的志保子似乎是无法放过这个机会。\\

「……要是真昼愿意的话,就让她穿上过去呗」

「为什么说得好像周不去一样?」

「要是和真昼出门让学校里那帮子人知道就不好了吧」\\

如果只是父母和真昼的话,就算去新年参拜看上去也是一家人一起去,不会有问题。

而如果带上了周就有问题了。\\

外表不显眼的周和真昼一起参拜,如果给同年级的同学见到了,可以想象寒假过去之后将是哀声一片的地狱场景。

再怎么说,周也不觉得承担这个风险依旧还想去新年参拜。\\

「不被发现就可以了吗?」

「可以是可以啦不过正常来说肯定会……我说妈啊,不会是」

「哼哼,就是为了这种时候才拿来了这么多东西的哦?」

「哪种时候啊!?」\\

和服、衬衣、小饰品,周就觉得如果只是这些和服相关的东西的话行李不会那么多,结果看来是为了欺负周而带来了更多的行李。\\

「修斗也很来劲的」

「爸……」

「难得的机会,不是挺好的吗。我是觉得,既然是年度活动,可以的话最好还是一起去吧」\\

被这么一说,周就难以拒绝了。

志保子的提议也包含了修斗重视家庭的意向,周要是拒绝的话,会感觉有些不好意思。\\

「可是啊」

「没问题,相信妈吧。肯定会把周打扮得帅气到判若两人的!」

「这是在说现在的我很挫吧」

「和修斗长得那么像当然是不挫的,不过发型和给人的感觉都是土里土气的啦。这种的是叫不阳光吧」

「吵死了」\\

周自己也知道自己土里土气,但周是自愿打扮成这样的,不希望别人一一指出。\\

「要是打扮好的话明明还挺能看的,就是周嫌麻烦……」

「多管闲事」

「真是可惜。……我说小真昼啊,你也想看周整理好的打扮吧?」

「咦?」\\

突然提到真昼的话题,这让真昼肉眼可见地惊慌失措着。

尽管周希望志保子不要那么对真昼步步紧逼,然而志保子却是毫不客气。\\

「周要是打扮好的话,我觉得真昼也应该会对周刮目相看的。别看周这样,其实长得还挺不错的哦?周虽然性格不坦率,但是遗传了修斗的绅士风度,只要好好打理就真的是个好男人啦」

「呃,那个……是、是啊……?」

「不想一起去新年参拜吗?」

「想、想去是想去啦,可是」

「喂别出卖我啊」\\

不怕一万就怕万一,周是希望尽可能地拒绝的,而真昼却瞄了一眼吐槽的周。\\

「……周君不愿意的话,那就算了」\\

真昼发出了有些沮丧的声音微皱着眉头,让周突然感到一阵呼吸困难。\\

真昼本人似乎没打算表现出来,然而她明显是一副遗憾的样子。这副样子似乎并不是故意彰显出的,而是自然流露出来的。

真昼静静摇着长睫毛朝下看的样子,让周产生了强烈的罪恶感。\\

志保子丢来了好像在说「让小真昼伤心」的指责般的目光,而修斗的视线则好像在说「放弃才更快一点」。在两道视线下,周发出了唔唔的小声。

——这岂不就像是自己在欺负真昼一样了吗。\\

「……行吧」\\

看到那副表情,周不得不败下阵来。
