\documentclass{article}

\usepackage{luatexja}
\usepackage{luatexja-ruby}
\usepackage[no-math]{luatexja-fontspec}
\usepackage[a5paper]{geometry}
\usepackage{float}
\usepackage{titlesec}
\usepackage{setspace}
\usepackage[yyyymmdd]{datetime}
\usepackage{indentfirst}
% \usepackage{tipa}  % for \textsubdot
\usepackage[perpage]{footmisc}
\usepackage[unicode,hidelinks]{hyperref}
\usepackage[numbered]{bookmark}
\usepackage{luacode}

% 思源黑体
\newjfontfamily{\jpfont}[
  BoldFont=SourceHanSansJP-Bold,
  YokoFeatures={JFM=prop}
]{SourceHanSansJP-Regular}

\setmainjfont[
  BoldFont=SourceHanSansCN-Bold,
  YokoFeatures={JFM=prop}
]{SourceHanSansCN-Regular}
\renewcommand{\familydefault}{\sfdefault}

\ltjsetparameter{prebreakpenalty={`—,10000}}
\ltjsetparameter{jaxspmode={`,,inhibit}}
\ltjsetparameter{jaxspmode={`。,inhibit}}

\onehalfspacing
\titlespacing*{\subsection}{0pt}{8.9ex}{3.4ex}
\newcommand\subsectionbreak{\ifnum\value{subsection}>1\clearpage\fi}

\def\two@digits#1{\ifnum#1<10 0\fi\number#1}
\renewcommand{\thesubsection}{\two@digits{\value{subsection}}}
\renewcommand{\contentsname}{目录}
\renewcommand{\abstractname}{简介}
\renewcommand{\dateseparator}{.}

\counterwithout{subsection}{section}
\newcommand*\sectotoc[1]{
  \section*{#1}
  \addcontentsline{toc}{section}{#1}
}
\newcommand*\addchaps[2]{
  \luaexec{
    for i=#1,#2,1 do
      tex.sprint(string.format("\\setcounter{subsection}{\%d}", i-1))
      tex.sprint(string.format("\\input{\%04d}", i))
    end
  }
}
\newcommand*\psline{
  \par
  \kern20pt
  \hrule
  \kern1pt
  \hrule
  \kern10pt
}
