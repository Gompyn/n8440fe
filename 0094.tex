\subsection{关于考试的奖励}

「我说,真昼,下个周六树和门胁要在我家开学习会,没问题吧?」\\

吃完饭,周和真昼一起将盘子端去水槽。这时,周像是想起什么似的,突然向真昼问道。\\

为了这次的考试,周要集中精力学习,于是他顺便问了一下能不能让树和门胁一起参加。

顺带一提,树专心学习就能考得好,但因为不那么做所以成绩一般般。门胁则是样样精通的类型,若以上中下分级,大概会被分到上级;他不仅在外表和运动上都是出类拔萃,就连学习也毫不逊色,简直都想让人脱帽致敬了。\\

对于周来说,树他们在场也不会影响到自己学习,所以开学习会是没什么关系……但他开始担心过来做饭的真昼会怎么想。\\

「我没关系,把大家的那份做了就行了吧?」

「说的是啊,能这么做真是帮大忙了……但是没问题吗?」

「只是分量增加而已,没关系的……可以让我也参加吗?」

「真昼不介意就好……要不把千岁也叫过来?不过我不知道她有没有空,也不清楚她会不会认真学习就是了」\\

千岁说不上是认真的人。尽管她不是不会学习,但也绝对算不上脑子好。

至少周不太能想象出她认真学习的模样。\\

「那件事就不需要麻烦你了……我已经叫上她了」

「嗯?」

「千岁说『这次考得不够好的话爸爸会说我~』所以这个周末我本来就打算和千岁一起学习」

「他目标不会就是千岁吧?」\\

这个学习会是树提议的。周怀疑他是在得知千岁安排的基础之上才这样做。

周一边苦笑道「那些家伙」,一边用热水冲洗沾满油污的盘子。真昼也轻轻笑着把凉下来的剩菜装进餐盒里。\\

「也许吧。不管是不是这样,学习会应该都会很热闹吧」

「真昼不介意会吵闹吗?」

「我没关系,而且我平时就有学习,所以并不会那么着急」\\

周知道,真昼之所以能说出这种从容不迫的发言,是由于她平时就有在努力。因此,他也没什么特别的感想。

只不过,周倒是挺在意真昼到底是怎样才能如此高效地学习。\\

「我说真昼,待会可以让我看看你的笔记本吗?」

「我是没有关系。不过周君的笔记也做得很好啊」

「嗯,我整理得还算行吧,但我还是想看年级第一的笔记」

「其实也没什么值得期待的」\\

真昼扑哧一笑,将剩菜放进冰箱。

放在冰箱里的晚餐就是周明天的早餐,因此周在洗着碗筷的同时还在内心拜了拜真昼。不仅是晚餐,连早饭都是吃真昼亲手烹制的食物,这让周相信自己每天都过着充实健康的饮食生活。\\

「周君,这次的考试干劲十足啊」

「嗯,这也是增强自信的一环。既然要做,我就想全力以赴」

「这样子啊……那要我再给你一些动力吗?」

「动力?」

「如果周君进了年级前十,我就给周君膝枕、掏耳朵。上次周君睡得那么熟,看样子是挺喜欢的吧」\\

「虽然不知道算不算是奖励啦」真昼笑着加了一句。周则是洗着盘子,在心里喃喃自语「这可是很棒的奖励啊」。\\

周心想「这说的是什么话」,却舍不得膝枕的魅力。就算周想要拒绝,真昼却在旁边有点寂寞地说道「如果不愿意的话就算了吧」,导致他条件反射般回答说「要是我做到了,你可要兑现诺言」。\\

尽管周有点嫌弃过分忠于欲望的自己,但真昼却略有害羞地笑着说「那就这么约好了」,于是周就轻易地败给欲望,接受了这件事。
