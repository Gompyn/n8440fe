% \subsection{150 服のゆくえ}
\subsection{150 服のゆくえ}

% 服を買って店をあとにした周達は、ショッピングモールを目的もなくぶらついていた。\\


% 県でも随一の広さを誇るこのショッピングモールは歩いているだけで案外楽しいので苦ではないのだが、視線を集めやすいので何とも言えない気持ちにもなる。\\


% 贔屓目抜きに両親は見目整っているし、真昼は言わずもがな。そんな人達が固まっているので、人目を引くのは仕方ないだろう。\\


% 真昼は最早慣れきっているので、気にした様子はない。周の腕に寄り添っている。\\


% ただ、視線が集まる事には慣れているが、周の腕に自らの腕を絡めて歩くのは恥ずかしいのか、ほんのりと頬が上気している。\\


% こちらはこちらで柔らかいものが当たっているので正直平静ではいられないのだが、表に出せば志保子にからかわれるのが目に見えているので顔に出ないように注意する羽目になっていた。\\


% 真昼の買った服の入った袋を握って意識を逸らすが、そうすれば「どうしてこちらを見ないのか」と言わんばかりにぎゅっとくっつかれるので、非常にやりにくい。\\


% 「真昼、あのさあ」


% 「はい」


% 「……あー、いやその」


% 「何ですか?」


% 「……そういえば、ゴールデンウィークに買った服って着ないんだな」\\


% 胸が当たっている、と指摘するか悩んだが、たまに真昼は小悪魔的に「当ててるんです」と言う事もあるので、どうしようか悩んだ挙げ句別の話題を持ち出す事にした。\\


% 今日の真昼の服装はお嬢様らしいデザインの清楚系ワンピースではあるが、前買ったオフショルダーのワンピースではない。着て見せるとは言っていたが結局見ていないのでどうしたのかと思ったのだ。\\


% ゴールデンウィーク、という言葉に目を瞬かせた真昼は、そのあと淡くはにかむ。\\


% 「……二人きりでデートする時に見せたいなって思ったので」


% 「……そ、そうか」


% 「連れていってくれるのでしょう?」\\


% ぴと、と寄り添いながら首を傾げた真昼が無性に愛おしくて、周はそっと絡み付いている真昼の腕の先、掌をゆっくりと握る。\\


% 「……そうだな、二人で行こうか。これは、家族のおでかけだからな。デートとは別だもんな」


% 「……は、はい」


% 「どこ行きたい?」


% 「周くんと一緒なら、どこへでも」


% 「そういう事言われると、どこにも行きたくなくなるなあ。おめかししてくれるのはいいんだけど、人に見せたくないし」


% 「……そういうのはおうちデートというらしいですよ。おうちでも、全然いいです。数日天候が崩れるかもしれないらしいですし」\\


% そういえば台風が発生していて徐々に近づいてきているのか、ニュースでの週間予報は雲行きが怪しかった。


% 直撃する訳ではないが、余波が飛んでくるのでまあ雨は降るだろう。


% 家に帰る頃には通りすぎているだろうが、折角の帰省なのだからいい天気でいてほしいものだ。\\


% 台風の事を考えたら、ひょっとすればおでかけは出来ないかもなあ、と思ったが、真昼は二人で過ごす事に比重を置いているらしく外出という行為そのものはあまり拘っていないようだ。


% 帰ったら天気を調べておこう、と決めつつ、真昼の手を改めて握りしめる。\\


% 「俺も真昼と過ごせたらどっちでもいいかな。また天気見て日程決めような」


% 「はい」


% 「……後ろでいちゃいちゃしてると思ったらデートの約束を取り付けてたのね」


% 「残念、元からする予定だったから」\\


% 前を歩いていた志保子が悪戯っぽい声でからかってくるのでしれっと反論すれば、両親が前で小さく笑う。


% ただ、からかうというよりは微笑ましそうな気配で、それ以上追求する事はなく前を向いたので、周は小さく鼻を鳴らして真昼の手を引いた。

