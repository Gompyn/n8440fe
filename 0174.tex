% \subsection{174 夏休み明けの朝}
\subsection{174 夏休み明けの朝}

% 新学期の朝、起きて無意識に隣を見て、誰も居ない事に周は少しだけ残念だと思った。\\


% 昨日泊まる筈だった真昼は周の言葉にキャパシティオーバーを引き起こしたのか、恥ずかしがって夕食後には帰ってしまった。\\


% 流石にこのタイミングで手出しをするつもりはなかったのだが、言った言葉が言葉だったので意識してしまったらしい。\\


% 嫌ではなく心を落ち着かせるために帰る、と言って帰宅した真昼に、最初の慰めと不安の解消という目的は達成しているのでそのまま見送ったのだ。\\


% あの華奢な体を抱き締めて寝たらさぞ幸せだったんだろうな、と思いながら起き上がって着替えていると、玄関の方からドアの蝶番が軋む音がした。\\


% 夏休み明け初めての登校日なので、いつもより早めにやってきたのだろう。\\


% 制服に着替えて自室から出れば、丁度エプロンを身に付けている真昼がキッチンの入り口に居た。\\


% 「あ……お、おはようございます」\\


% やや躊躇いがちに挨拶してきた真昼に、周はひっそりと笑った。\\


% 明らかにこちらを意識しているのが分かる。\\


% 「おはよ。ちゃんと眠れたか?」


% 「……一応。周くんのせいで別件で悩みましたけどね」


% 「嫌だったか?」


% 「……分かってる癖にそういう事聞くのやめてください」\\


% 顔を赤くした真昼にぺし、と側にあったキッチンミトンではたかれたので、周は悪びれる事なく笑って、一旦身支度を整えるために洗面所に向かった。\\


% \\


% 「……何でこちらを見るのですか」


% 「いや、照れてる真昼が可愛いと思って」


% 「周くんがたらしに進化していてよくないと思います」


% 「真昼限定だから問題ない」\\


% 朝食の時に未だに照れている真昼を眺めていたら、真昼が微妙に拗ねる、というよりは照れ隠しに拗ねた振りをしていた。


% 昨日の事が響いている真昼はこちらを意識している素振りをちょくちょく見せるので、それが可愛くてつい笑ってしまう。\\


% 余裕を見せる周に真昼は「ばか」と随分と罵声にしては可愛らしい言葉を周に向け、だし巻き玉子を口に放り込む。\\


% 微妙に唇が山を築いているのは、周の態度が原因だろう。\\


% むくれつつも上品な仕草で朝食をとっている真昼が微笑ましくて穏やかな眼差しで眺めていれば、気付いたらしくじと目で見られた。\\


% 「……何ですか」


% 「いーや、幸せだなあと。一緒に美味しい朝ご飯をとる事が出来て」


% 「……それは私も幸せですけど、今そういう事を言うのはわざとですよね」


% 「ご想像にお任せしよう」\\


% そう言ったら真昼が机の下でつつくように足を蹴ってきたので、羞恥に顔を染めながらご飯を食べる真昼を眺め、周もこれ以上刺激をすまいと静かにご飯を食べるのであった。

