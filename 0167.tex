% \subsection{167 天使様と食いしん坊}
\subsection{167 天使様と食いしん坊}

% 「ふいー、食べた食べたー」


% 「どこにあの量が入ったんだ……」\\


% 屋台を粗方回り終え、千歳はお腹をさすりながら満足そうに頬を緩めていた。


% 腹部は屋台を回る前よりやや膨らんでいるように見えるがそれでも細く、よくあの量が入ったなと感心すればいいのか呆れればいいかが悩み所だった。\\


% 「んふー、こういうお祭りのご飯は格別ですなあ」


% 「まあお前が満足してるならいいんだけどさ……食べ過ぎには気を付けろよ」


% 「普段はこんなに食べませーん。ちゃんと調整してますー」\\


% スレンダーな体型を維持している千歳の言う事なので信じるしかないが、それにしても食べすぎな気がしなくもない。ただ本人は納得しているようなので、周がとやかくいう事でもないだろう。\\


% 「そういう周は足りるの? 私からすれば全然食べてないけど」


% 「ん……俺は家でちょっと食べるつもりだったし。真昼が出汁冷やしてるからレトルトの飯で冷やし出汁茶漬けでもしようかと」


% 「なにそれ美味しそう」


% 「まだ食う気力あるのかよ……」\\


% 屋台の品もいいが一日の〆は真昼の料理がよかったので家で真昼の作り置きの出汁を使って茶漬けにしようと思ってあまり食べずにいたのだが、まさか千歳がまだ食欲を余らせているとは思わなかった。\\


% 千歳の食欲に苦笑している真昼は「また今度にしてください」と窘めている。今日見ただけでも焼きそばや唐揚げ、フランクフルトに真昼の買ったたこ焼き一粒やチョコバナナ、かき氷と男子でもお腹が満たされる程度に食べているので、胃の心配をしているのだろう。\\


% どこに入ったんだろうか、と細い腰を見ながら考えていたら、視線に気付いたらしい千歳が「やんえっち」と体をくねらせたので、白けた眼差しを返しておいた。\\


% 「まあ千歳の胃の容量は今後要観察でいいとして」


% 「わおつれない」


% 「どうする? もう帰るか?」\\


% ある程度遊び回ったし、夏場で日が暮れるのが遅いとはいえ既に空は闇色。もうすぐ二十時半になるので、家のある区域から離れた周と真昼の移動時間も考えてそろそろ解散にするのが無難だろう。


% 千歳も樹が居るとはいえあまり遅い時間に出歩かせるのもよくない。\\


% 「んー、帰るのは構わないけど私まひるんち泊まるよ?」


% 「は?」


% 「事前にまひるんちに荷物運んでおいたしー、ちゃんとこっちは前から許可取ってたよ?」\\


% ねー、と真昼に笑いかけた千歳に、真昼も苦笑しつつ頷く。


% ちなみに嫌がるような表情ではないので、周も心配はしていないが、出来れば先に言ってほしいところでいる。周が食材の買い出しをするので、三人なら三人分の食材の用意が必要なのだ。\\


% にまーっと笑う千歳に「オレも周に頼んどけばよかった」と樹は惜しそうにしている。彼だけ一人で帰るのは可哀想だと思いつつ、着替えがないのでどうしようもない。\\


% 「……まあ、真昼がいいってんならいいけど」


% 「おやぁ周くん、まひるんを取られてご機嫌ななめー?」


% 「女に妬いてどうするんだよ。真昼は俺のものって分かってるから別にいい」\\


% どちらかといえば真昼が千歳にべたべたされるのが嫌というより、同性だから気軽に家に出入り出来るのが羨ましいといった感じだ。


% 今度真昼の家にあがらせてもらう約束はしているが、こちらにも覚悟が要るのであっさりと入れる千歳が羨ましかった。\\


% なので今更千歳に妬きはしない、と肩を竦めた周だったが、真昼が頬を赤らめて千歳の方にすすすと逃げていく。\\


% 「……千歳さん、これですよ。周くん最近こういう風になってきたんですよ……」


% 「あやー、これはまひるんも大変ですなあ」


% 「なんだよその顔」


% 「別にー?」\\


% ねーまひるん、と先程真昼に同意を求めたのとはまた別の悪戯っぽい千歳の笑みに、真昼はこくこくと無言で頷いて千歳にくっついて恥ずかしそうにこちらを窺うのだった。

