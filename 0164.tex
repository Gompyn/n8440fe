% \subsection{164 天使様とかき氷}
\subsection{164 天使様とかき氷}

% 「次はかき氷食べよー!」\\


% ぶらつくのを再開した四人だが、千歳の発言に再度足を止める事になった。かき氷の屋台は通りすぎた。恐らく進んだ先にもまだあるのだろうが、どこにあるのかは分からないので少し戻った方が早い。\\


% 「どんな胃袋してるんだよほんと……」


% 「こんな胃袋だよー」\\


% ぽんとお腹を叩いているが、真昼に負けず劣らずの細さが分かるだけだ。このお腹に焼そばと唐揚げとイカ焼きが格納されているのだから驚きである。


% どこに仕舞われてるんだ……と真顔になってお腹を見ていたら、真昼も同じ事を思ったのか苦笑を浮かべていた。\\


% 「千歳さん太りませんよね。すごくスレンダーで羨ましいです」


% 「健康的な細さだよな。引き締まってるし」


% 「えへー、もっと褒めたまえ」


% 「ほんとちぃは細いんだよなあ……抱っこした時とかすごく細いし」\\


% よくくっついている樹だから、千歳の細さもよく分かっているのだろう。樹は特段太いという訳ではなく中肉中背なのに、くっついていると千歳の細さが目立つのだからかなり細い。


% それでいて筋肉がうっすらと浮かびつつごつくない絶妙な体つきをしているのだから、千歳の努力が窺える。\\


% 「よく食べるのに太らないんだよなあ」


% 「代謝いいもん」


% 「まあそれにちぃは体質的にも太りにくいからなあ。その分他のところにつかないんだけど」


% 「……いっくん、こっちにオイデ」\\


% 口を滑らせたな、と一瞬で悟ったのは、千歳がにこやかな笑顔で抑揚のない声をあげたからだろう。


% 千歳が地味に気にしている部位の事に触れたのだから、当然怒る。むしろ彼氏だからこそ余計に怒っている気がする。\\


% 「ごめん失言だったから脛蹴るのやめてください」


% 「毎度言ってるけどいっくんは一言余計だなー? 向こうで仲良く話そ?」\\


% にこにこと笑いながら樹の腕にくっついてひっぱる千歳に、御愁傷様と口には出さず樹に送っておく。\\


% 「雉も鳴かずば撃たれまい……」


% 「何か言った?」


% 「いーやなんでも」\\


% こちらに飛び火するのは勘弁なのでさらっと否定して、隣で困っている真昼に樹の救援要請をスルーすべくわざとらしく微笑みかける。\\


% 「真昼はかき氷何食べる?」


% 「え……い、いちご……?」


% 「ん。じゃあ買いに行こうか。千歳ー、先にかき氷買ってくるからそこで仲良くしてろー」


% 「はーい」\\


% 樹を威圧しつつも笑顔で振り返って返事する千歳に小さく笑って、周は真昼の手を引いて一度道を戻る事にした。\\


% \\


% 二人がかき氷を買って戻ってきても、千歳のお説教は終わっていなかった。


% 道から少し外れたところで仲良く話し合いをしている二人を遠目に見て肩を竦めた周は、周の腕にくっつきながらなんとも言えなそうに苦笑いを浮かべている真昼を見る。\\


% 「……まだやってるんだよなあ」


% 「仲いいですよねえ」


% 「まああいつらなりのいちゃつき方だよなあ。若干千歳が怒ってるけど」


% 「あ、あはは……」\\


% 本気で怒っている訳ではないのも分かっているので止めたりはせず、手にしていたかき氷のカップを真昼に手渡す。\\


% 「ほら真昼」


% 「ありがとうございます。周くんは……なんか渋いですね」


% 「ほんとは宇治金時がいいんだけど流石に屋台にはなかった」\\


% ちなみに周は抹茶を選んだ。


% あったならば宇治金時にしたのだが、流石に屋台にあんこと白玉を求めるのはきついものがあるので致し方ない妥協である。\\


% 「周くんそういう甘いのは食べるのですね。あんまり食べようとしないですけど」


% 「別に甘いもの嫌いじゃないぞ、好んで食べないだけで。あんこは好き。特につぶあん」\\


% 甘いものは自分から食べないだけで出されたら食べる。自ら食べようとするのはカスタード系のものくらいだろう。それもあまり食べないので、好きというイメージはまずつかない。


% あんこが好きなのは抹茶や緑茶に合うからである。苦いものに甘いものは互いを引き立てあってとても合うので、実は好きだったりする。\\


% 「そうなんですか。……餡を炊くのは大変ですから何か作るのも苦労しますね」


% 「あんこを炊くところから考え始める真昼がすげえよ。市販のやつでいいだろ……」\\


% 普通小豆を炊いていくところから始めようという発想はないだろう。市販でもあんこは袋づめされて売られているのだから、手間隙と時間を考えたらそちらを選ぶ人間の方が圧倒的に多い。\\


% ただ、真昼は手作りの方が先に来るようだ。\\


% 「好きな人には美味しいものを食べさせたい心なのです。市販のだと中々甘さは調整出来ませんし、粒の感触が残らないのが多いので」\\


% 周くんには美味しそうに食べてほしいなんて健気な事を言って微笑む真昼に、周も申し訳なさやら愛されてる実感に幸福を感じるやらで、頬が緩めばいいのか引き締まればいいのか分からなかった。\\


% 「……じゃあ抹茶プリンにあんこ添えたやつたべたい。あとどらやき」


% 「ふふ、はぁい。お任せあれですよ」\\


% 周くんのためなら何でも作りますよ、と真昼が言えば過言でもなさそうな言葉を口にしてかき氷を食べた真昼に、周は何とも言えない照れ臭さを感じて誤魔化すように自分のかき氷を口に運んだ。

