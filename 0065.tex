\subsection{65 天使大人的独白与眼泪}

% 真昼を家に招いて、ソファに座らせる。
周让真昼进了自己家,坐在沙发上。

% 弱々しい笑みを浮かべる真昼は風に吹かれれば溶けて消えそうで、真昼の手を握ったまま腰を下ろした周は包み込むように手首から掌に握る場所を移動させた。
露出孱弱笑容的真昼,看上去十分的弱不禁风。握着真昼的手的周,似是要将那手包住一般,把握着的地方从掌尖移到了掌心。

% 包み込むように握ると、へにゃりと眉尻が下がった。\\
周包着一般握住了真昼的手,真昼则无力地露出了沮丧而不安的神情。\\

% 「……つまらない話ですけど、聞いてくれますか」\\
「……虽然只是些无聊的事情,可以听一听我说吗」\\

% 真昼の方からそう切り出したのは、周の部屋について十分ほど経った頃だった。\\
在周家里坐了许久,真昼终于发话道。\\

% 「私の両親は、愛しあって結婚した訳じゃないです。細かい事情は伏せますけど、家庭の事情と利害の一致で結婚しただけでした」\\
「我的父母,并不是自由恋爱而结婚的。具体的事情暂时不提,但是他俩是因为家族原因和利益关系一致才结婚的」\\

% 真昼は静かに話しているが、現代日本ではあまり見られないような結婚理由だろう。
虽然真昼说得很平淡,但这样的结婚理由,在如今的日本社会已经十分少见了。

% 普通ならば好きで結婚するのであり、利害の一致で結婚するというのはあり得なくもないがもう少し昔の話だと思っていた。\\
一般来说要论结婚的话都是因为互相喜欢,因为利益关系一致所以结婚这样的理由,虽然不会让人觉得诧异,但还是有种在说过去的事情的感觉。\\

% 彼女は恐らく上流階級の人間だろうから、その親は当然上流階級の人間。そういう理由でするのもなきにしもあらずなのだろうが……それでも、周には信じがたいものだった。\\
真昼她看上去像是个上流社会的孩子,相比父母也是上流社会的人吧。这么一说这样的理由倒也不能说完全不可能……但即便如此,周还是很难接受。\\

% 「だから……本当は、子供なんて作るつもりはなかったみたいです。ただ、一夜の過ちで出来てしまった。生んでしまったから、仕方なく金銭的に養ってるだけ。私を育てるつもりなんてなかったんでしょうね」
「所以……本来,他们似乎是不打算要孩子的。可是,只因一夜的冲动结果就怀上了。他们抱着的,只是已经生下来了,没办法只好出钱供着的想法而已。感觉是不想自己来带我的样子呢」

% 「育てるつもりがなかったって、」
「不想自己带孩子这——」

% 「……あの人たちは、滅多に帰ってきませんでした。帰ってきても宿泊施設として使ってるだけですから」\\
「……他们俩基本上是不怎么回家的。就算回家他们也只是把家当成旅馆住」\\

% 幼い頃からあんまり両親の顔とか見てないんですよね、と小さくこぼした真昼は、憔悴しているようにも見える。\\
从小时候起,我就一直没怎么见过父母的脸呢——真昼如此小声叹息道,她的脸上写满了忧愁。\\

% 「親らしい事はされた覚えがありません。私の育ての親は実質ハウスキーパーの人です。二人とも、外に愛人作ってそっちにかかりきり。私にはお金だけ渡して放ってるんですよ。私は要らないんですって。どれだけ頑張っても、いい子でいても、見てくれませんでしたから」\\
「我不记得他们做过什么像是父母该做的事情。把我带大的人实际上上是家里的管家和保姆们。他们两个,都在外面另有情人,平常也一直在那边。我的话除了给钱以外,总是被被丢在一边。我只是个没人要的孩子。因为就算我再怎么努力,再怎么做一个好孩子,他们都不会来看我」\\

% そこで、真昼がどうしていい子(天使様)として振る舞うのか、ようやく真に理解した。\\
说到这里,周总算理解了,真昼在学校装出一副好孩子{ruby:天使大人}样子的真正原因。\\

% 真昼は、両親に少しでも自分の事を見てもらいたかったのだ。
真昼曾渴求父母关心她, 哪怕只是一下。

% いい子にしていれば自分に目を留めてくれるかもしれない、褒めてくれるかもしれない――そんな淡い期待を抱いて振る舞い続けて、止め時を失って今に至った。\\
如果自己表现成好孩子的话,他们或许就会关心自己,夸奖自己——怀着这样淡淡的期待,真昼一致保持着这样的举止,错过了停下来的机会直到现在。\\

% 今でも止めないのは、本当に少しの可能性にかけてなのか、それとも内側にある自分に触れて欲しくなくて仮面を被らざるをえなくなったのか。
如今依旧保持着这副举止,不知是因为还在念想着那些微的可能性,还是不愿触及自己的内心而为自己带上的面具。

% どちらかは分からなかったが、少なくとも望んで着けている訳ではないだろう。\\
虽然不知具体为何,但周还是明白,真昼并不是真心想要这样。\\

% 「結局、私なんて見てくれないんです。綺麗に育っても、勉強が出来ても、運動が出来ても、家事が出来ても、あの人達は一度も私を見てくれた事はないのです。……がんばっても無駄なのにがんばってしまった私は、きっと馬鹿なのでしょう」\\
「到头来,我还是那个没人关心的孩子。就算长得多么漂亮,成绩多么优秀,运动多么擅长,家务多么能干,那两个人也从来不曾关心过我。……明明努力也无济于事却仍然这么努力的我,看起来肯定是个笨蛋吧」\\

% 報われないのにね、と。
明明知道不会有结果——

% 諦念で満たされた嘆きに、胸が締め付けられた。\\
这充满了无奈的叹息,令周不禁感到心头闷闷的。\\

% 「私が居るから、あの人達は離婚出来ない。どちらも引き取りたがらないんです。愛人の家族に気を使わせる。祖父母には期待出来ない。だから、私が大学を出るまで待ってるんですよ。一人立ちさえしてしまえば、あとはほとんど関係ないですし」
「因为我,那两个人没法离婚。因为那两个人都不想当我的监护人。情人的家里也很介意我。爷爷奶奶他们也指望不上。所以我一直在等大学毕业。等我能够自立了之后,就可以跟他们一刀两断了」

% 「それは……」
「这……」

% 「……要らない子って、面と向かって言われた時は……流石にショックでしたね。思わず雨の中ブランコを漕ぐぐらいには自暴自棄になりました」\\
「……被当面直说是没人要的孩子的时候……再怎么说我还是惊呆了呢。然后就不自觉地在雨里荡着秋千自暴自弃了起来」\\

% その言葉に、数ヶ月越しにあの時どうして真昼が雨の中公園に居たのか、理解した。\\
听了这话,周总算明白了几个月前,真昼为什么要在公园里淋雨了。\\

% あれは、親に心ない言葉を突きつけられて、傷付いてさまよってたどり着いたところだったのだ。
那时的真昼,被父母的无心之语伤到,伤心地彷徨着,走到了公园里。

% 居場所がない、そう認識したからこそ、あんな――迷子のような、幼くて不安げな表情をしていたのだろう。\\
因为觉得自己无家可归所以才露出了那样的——如同迷路的孩子般,幼小而不安的表情吧。\\

% 誰にも助けを求められず、突き付けられた言葉を飲み込みきれず、ただどうしていいのか分からなくて、あの場所にたどり着いて一人佇んでいた。\\
既没有向任何人寻求帮助,也无法接受这伤心的话语,只是不知如何是好,独自一人失神地走到了那儿,只身伫立。\\

% それを想像したところで、口の中に僅かに鉄の味が広がった。
周正想像着的时候,口中泛起了一股淡淡的铁腥味。

% どうやら無意識の内に唇を噛みきっていたらしく、小さな痛みと独特の風味が口の中にある。あまりに理不尽な事に、知らず知らず怒りが溜まっていたのだろう。\\
看来是无意识间把嘴唇咬破了吧,微微的痛楚与那独特的味道在嘴里泛开。大概是因为这过于不讲道理的现实,周的心里也不自觉地蓄起了愤怒吧。\\

% 「……困るなら、産まなければよかったのにね」\\
「……要是嫌麻烦的话,就别把我生下来啊」\\

% 本当に小さな囁きは、聞いているだけで胸に杭を打ち込んだかのように痛みを突きつけて、全ての動きを止めさせた。\\
这声十分微弱的低吟,在周听来却如同木桩钉入胸前般痛苦,令周不得动弹。\\

% ここまで、真昼に言わせている真昼の実の両親に頭が真っ白になるくらいの怒りを覚えてしまう。
话听到这里,对真昼口中的亲生父母,周心中充斥着甚至能令头脑发白的怒火。

% 両親からの愛情を一つも受けてこなかったからこそ、こんなにも繊細でか弱く育ってしまったのだ。表面上強く振る舞って、内側で泣き続けた結果、真昼は誰にも助けを求められなくなった。
正因为从未受到过来自父母的爱情的滋养,真昼才变得如此纤细而弱小。表面上的逞强行径,内心里不住的泪水,让真昼无法向任何人发出求助。

%  


% いい子の仮面を剥ぎ取ってしまえば、ささやかな風でも崩れて消えそうなくらいに儚い姿が現れる。\\
取下了那好孩子的假面后,真昼现出的姿态是那么的弱不禁风,脆弱易逝。\\

% (どうしてここまで追い詰める事が出来るのか)\\
(为什么要把她逼到这样的地步啊)\\

% 声を荒げて問いたかったが、真昼を見捨てた本人達は、ここには居ない。
周想要大声质问,然而舍弃了真昼的那俩当事人却不在此处。

% それに、どうしていいのか分からない。\\
何况,周也不知道该怎么办。\\

% あまりの家庭環境のひどさに憤っているが、周は真昼とは他人だ。
就算对真昼的家庭环境再怎么感到气愤,周和真昼也不是一家人。

% 真昼の家庭事情に他人が首を突っ込んでいいとは思わない。余計に状況を悪化させる可能性もある。無闇に口出しして、更に真昼が傷付く可能性を考えたら、周には何も出来やしないのだ。\\
真昼家里的事情周一个外人指手画脚也不大好。甚至还有让事态继续恶化的可能。考虑到随意地说话很可能反而会伤到真昼,周便什么也没有做。\\

% ただ、このまま放っておいたら空気に溶けるように消えてしまいそうで――周は、側にあったブランケットを真昼の頭からかける。\\
只不过,要是就这么放着现在这样子的

% 顔まで影が差すように隠して、それから戸惑う真昼を腕の中に収めた。\\


% 初めて自ら抱き締めた体は、とても華奢で頼りない。少しでも無理に力を込めてしまえば、容易く折れそうなほど。


% 誰にも寄りかからずに耐えてきた体をしっかりと抱き寄せて、周は真昼を包み込む。\\


% 「え、あ、あまねくん……?」


% 「……何でさ、お前がこういう性格に育ったか、理由分かった気がする」


% 「可愛げないってところですか」


% 「ちげえよ。……我慢強くて、他人に弱いところを見せたくないってところだ」\\


% 我慢せざるを得なかったのだ。一度弱音を吐き出してしまえば、確実に折れてしまうから。\\


% お手伝いさんは真昼を大切にしてくれていたようだが、それでもあくまで雇われていた他人で、真昼を助けてくれる人ではなかった。\\


% 誰にも助けを求められない状況で、彼女は一人耐え続けたから、こんなにも自分を偽るのがうまくなってしまったのだろう。\\


% 「……別にさ、俺はお前の家庭に口出しするつもりはないよ。他人の家庭に首突っ込む訳にいかないし」\\


% 周は、他人だ。家族というデリケートなものに触れる訳にはいかない。\\


% けれど、それと真昼を支えないという事は同意義ではないのだ。\\


% 「……見て見ぬ振りしてやる。泣くなら泣けよ、んなひどい面してるのに我慢したって、息がつまるだけだろ」\\


% 本当は、泣かせたくはない。\\


% けれど、このまま溜め続けていれば、いつか彼女は壊れてしまう。\\


% だから、泣いてほしかった。我慢したもの全て吐き出してほしかった。\\


% 苦しいならば苦しいと言ってほしい、寂しいなら寂しいと言ってほしい。そうしたら、周は彼女の側に居て聞き届けるのだから。\\


% 彼女の置かれた状況はどうしようもなくても、周は真昼の苦しみを受け止めるくらい、出来るのだ。\\


% おこがましいとかそんな事もちらりと頭の隅を掠めたが、真昼が周の腕の中でもぞりと動き、周の胸に自ら顔を埋めたので、それも全て消えていった。\\


% 「……ないしょにしてくれますか」


% 「見てないから知らん」


% 「……じゃあ、ちょっとだけ……貸してください」\\


% 震える声で小さく呟いた彼女に周は返事をせず、ただ頭からかけたブランケットをもう一度深くかけさせて、頼りない背中をしっかりと抱き締めた。\\


% やがて、小さな嗚咽が聞こえ始める。


% 大きくはない、けれど確かに聞こえる泣き声は、真昼から放たれたもの。\\


% いつでも嘆かずに一人で耐えていた真昼が初めて周に求めた『支えて』という願いに、周も少しだけ泣きそうになりながら真昼の小さな背中を抱き締めた。\\


% \\


% 「……見てるじゃないですか」\\


% 彼女は長くは泣かなかった。


% 時間は数えていないが、十分あるかないか程度。\\


% 十六年分の苦しみを吐き出してくれてもよかったのだが、あまり泣きすぎても疲れてしまうので体が強制的に止めたのかもしれない。精神疲労に加えて肉体疲労まで得てしまったら、恐らく脳が強制的に休眠モードに移行するだろうから。\\


% 顔を上げた真昼の瞳は濡れていたが、少しだけ元気を取り戻したのか周を見る瞳はしっかりとしたものだ。\\


% 「俺の胸にもたれてたんだから仕方ないだろ。泣くところまでは見ないようにしてやった」\\


% いつの間にかずり落ちたブランケットを引っ張って見せれば、小さな笑みが浮かんだ。\\


% 「……あまねくん」


% 「なんだよ」


% 「……ありがとうございます」


% 「何の事だか分からん」\\


% こっちが好きでやってるんだから感謝される覚えはない、とそっぽを向けば、真昼はまた周の胸に顔を埋めた。\\


% 「もう少しだけ、貸してください」


% 「……おう」\\


% この状態の真昼を突き放せる訳もない。それに、支えてやりたかった。\\


% 平静を装いながら小さな体を抱き締め直して、ゆっくりと頭を撫でる。


% 誰も真昼を褒めないのなら、周が褒めてやればいいのだ。


% よく頑張った、もう自分の前で無理に頑張る必要はないんだ、という気持ちを込めて優しく掌で撫でていると、真昼も落ち着いてきたのか不要な力が抜けた表情で周を見上げる。\\


% ただ、それでも色々と不安や考え事があるのか、表情が明るくなったという訳ではない。\\


% 「……どうしたらいいんでしょうね、これから」\\


% 小さく呟いた真昼は、周の瞳を見ながら困ったように微笑んだ。\\


% 「頑張っても、見てくれないんですから。他の人だってそうです、天使様なんてもてはやされても、私が必要とされている訳じゃないんですよね。天使のようにふるまう椎名真昼が好まれていて、必要とされていて……本来の私は必要とされてないんです。自分でそう仕向けたのに苦しむなんて馬鹿らしい話ですけど」\\


% 自分で自分の首を絞めてるんですから、と苦笑して、きゅっと周の胸元の布を掴む。\\


% 「ほんとの私は、可愛げとかないし、臆病で自分勝手だし、性格悪いし、口悪いし……好かれる要素なんて、ないですもの」


% 「俺は割と好きだぞ」\\


% 思わず、本音が口からこぼれた。\\


% ぱちりと瞬きを繰り返す真昼を見つめて、続ける。\\


% 「まあ、可愛げない時はもちろんあるけどさ、それ以上に可愛いとか守ってあげたいとかそういうの思うし、お前のはっきりした物言いは好ましいと思ってるよ。あと、性格悪いならそんな事で悩まねえよ」\\


% 後ろ向きすぎだ、と真昼のおでこを軽く弾くと、どこか呆けたように真昼が表情から負の色を抜く。\\


% 周としては、なんで真昼がそこまで自分を悪し様に言うのかちっとも理解出来なかった。


% 誰がどう見たって、彼女は努力家で心優しい少女だろう。多少言動が明け透けなところはあるが、指摘は正確だし人を想っての発言ばかりだ。


% 臆病と言ったが、別に悪い事でもない。傷付きすぎて、これ以上自分が傷を負うのが嫌で守りの態勢に入っているだけだろう。\\


% あと、可愛げがないなら周は真昼にしょっちゅう悶える羽目に陥っていない。


% むしろ素の時の方が可愛い事を自覚してほしいくらいである。\\


% 「そんな卑下すんなよ、お前の素見てもそれが好きってやつがここに居るだろ」\\


% 愛されない、と思い込んでいるからこそ自分自身に自信がないのだろうが、好ましく思っている人間なんて周だけでなくて周の周囲にも居るのだから、思い込みも甚だしい。


% 千歳なんて素の真昼の方が可愛いとべたべたしているのだ。あれはどう考えてもうわべだけなんてあり得ない。\\


% 真昼のカラメル色の瞳をじっと見つめて言い聞かせたのだが、真昼は視線をそらし始めた。


% それどころか、目元のほんのりとした赤色に負けじと頬まで赤くなっている。\\


% すぐに薔薇色と言っていいくらいに色付いていて、これは羞恥からくるものではないかと気付いた時には真昼は縮こまって瞳がこれでもかと忙しなく泳いでいた。\\


% 真昼の様子から自分でもかなり際どい発言をしていたのだと気付かされて、周まで顔が赤くなる。\\


% 「い、いや、千歳達もそう思ってるから! 決して、他意があった訳じゃなくてだな! 俺だけじゃない、母さん達も、千歳や樹も、お前の天使様でないところ見て気に入ってつきあってるんだから! お前は自分が思うより、ずっと……その、好ましい人柄なやつだと思うよ」\\


% 慌てて自分の発言を説明していると、真昼もようやく視線が周を捉える。


% ただ、一瞬でも勘違いした事には変わりないのか真っ赤な顔で震えているので、相当恥ずかしい思いをさせたらしい。周もかなり恥ずかしい思いをしたのだが、言われる身では更に恥ずかしいのかもしれない。\\


% 「その、頑張りきれなくなったり親とかがどうしてもいやになったらうちに避難とかしてくれていいし。母さん達は事情知ったら匿うくらいするから。あれだ、療養みたいな感じでもいいからさ」


% 「……うん」


% 「母さん達は真昼の事気に入ってるからさ、ずっと居たっていいって言ってくれると思うしさ……むしろ真昼が幸せになるまで離してくれないと思う。俺達にはお前が親とどうするかなんて決められないけどさ、お前が踏ん切りつくまでいくらでも甘えさせるというか、支えるから」


% 「うん……」\\


% 一生懸命に誤解されないように説明していたら、また真昼が涙をこぼした。\\


% 「な、何でまた泣くんだよ」


% 「恵まれてるなあって……」


% 「むしろ恵まれなさすぎだからもう少しわがまま言ってもいいんだぞ」\\


% 金銭面では恵まれているのかもしれないが、それ以外を彼女は奪われていたのだ。与えられるべき愛情を一つも受けずに、よくもここまでひねくれず育ったと感心するほどだ。


% そんな真昼には、誰かに甘えてもいいのだ。わがままだって言えばいい。誰も聞き入れなかった分、少しでも取り返してやれたらと思う。\\


% 「……じゃあ、お願いしてもいいですか?」


% 「何だ?」\\


% 俺に叶えられるなら、という前提を付け足すと、真昼は小さく笑って「周くんにしか出来ない事です」と囁く。\\


% 「……もっと、見ていてください」


% 「お前の頑張りはちゃんと見てるし、目を離したらどっか飛んでいきそうだから見てるよ」


% 「……捕まえておいてください」


% 「手でも握っておくよ」\\


% これだけか? と真昼の顔を覗き込むと、真昼はしばらく周を見つめて、それからはにかんでみせた。\\


% 「今日のところは、全身で捕まえておいてください」\\


% そう言って周の背中に手を回して胸に顔を埋めた真昼に、周は一瞬どきりとしたものの不埒な思いを抱いては駄目だろうと飲み込んで、華奢な体を改めて包み込んだ。

